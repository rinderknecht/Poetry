%%%-*-latex-*-

\documentclass[a4paper,12pt]{article}

\usepackage[T1]{fontenc}
\usepackage[utf8]{inputenc}
\usepackage[english]{babel}
\usepackage[charter]{mathdesign}
\usepackage{verse}
\usepackage{url}

\title{Haikus}
\author{Christian Rinderknecht\\
{\small \url{rinderknecht@free.fr}}}
\date{\today}

\begin{document}

\begin{verse}
  Winter's evening falls \\
  under the full moon walking \\
  the plane's tight rope
\end{verse}

\begin{verse}
  Hundreds of swallows \\
  sing their hearts out on the staff \\
  of electric wires
\end{verse}

\begin{verse}
  Tired I keep searching \\
  lost footprints under the snow \\
  ---~a little boy afar
\end{verse}

\begin{verse}
  Full moon of summer \\
  resting on the mountain's top \\
  ---~breath taken away
\end{verse}

\begin{verse}
  The melting stream \\
  carries faraway echos \\
  ---~I soak my sleeves
\end{verse}

\begin{verse}
  Winter night dreaming \\
  the glow of desired storms \\
  ---~wet wood in the hearth
\end{verse}


\end{document}

\begin{verse}
le soir d'hiver tombe \\
sous la lune funambule \\
l'avion tend un fil
\end{verse}

\begin{verse}
mouche même toi \\
tu me déchiffres à tatons \\
parcourant ma page
\end{verse}

\begin{verse}
lune d'été ronde \\
posée au sommet du mont \\
le souffle coupé
\end{verse}

\begin{verse}
assis immobile \\
dans le reflet insondable \\
carpe à contre-courant
\end{verse}

\begin{verse}
son doigt prend mon doigt \\
viens contre mon c{\oe}ur petit \\
liseron grimpant
\end{verse}

\begin{verse}
fiévreux dans son lit \\
il rêve du grand large \\
le ru s'assèchant
\end{verse}

\begin{verse}
haïku mais \\
tu ris de moi aux éclats \\
de ton vers brisé
\end{verse}

\begin{verse}
toc toc il cogne \\
dans le bois le pic-épeiche \\
aux portes de l'hiver
\end{verse}

\begin{verse}
le pauvre escargot \\
jusqu'où trainera-t-il sa \\
langue d'affamé?
\end{verse}

\begin{verse}
ah qui osera \\
lui passer la muselière \\
morsures du froid
\end{verse}

\begin{verse}
luciole tombée \\
amoureuse et perdue \\
je suis ton ami
\end{verse}

\begin{verse}
la bise à l'aurore \\
effleure encore les pensées \\
---~ploc! première goutte
\end{verse}

\begin{verse}
déluge automnal \\
charrie deux pauvres lombrics \\
Noé est au sec
\end{verse}

\begin{verse}
soir tombant d'été \\
un canard esseulé tire \\
un long V en feu
\end{verse}

\begin{verse}
Dame Souricette \\
grignote entre deux traverses \\
---~longs quais en été
\end{verse}

\begin{verse}
rendez-vous d'automne \\
scarabées en réunion \\
---~colloque amoureux
\end{verse}

\begin{verse}
le vent de juillet \\
porte la nue infinie \\
---~traîne de mariée
\end{verse}

\begin{verse}
ni vu ni connu \\
je fauche des chrysanthèmes \\
---~soir de la Toussaint
\end{verse}

\begin{verse}
sur le champ de neige \\
les corbeaux affamés comptent \\
les mottes de terre
\end{verse}

\begin{verse}
flocons et soir tombent \\
coiffant leurs bonnets de nuit \\
les clochers s'endorment
\end{verse}

\begin{verse}
flocons s'égrenant \\
dans le bois un cache-cache \\
seul avec les arbres
\end{verse}

\begin{verse}
Mardi Gras foufou \\
même la neige maquille \\
poteaux en bouleaux
\end{verse}

\begin{verse}
j'ai dû oublier \\
mes oreilles à la maison \\
---~quel froid de canard
\end{verse}

\begin{verse}
araignée du soir \\
verticale équilibriste \\
tends-moi la perche
\end{verse}

\begin{verse}
Ombrelle à fourmis \\
mais il rêve du grand bleu \\
le perce neige
\end{verse}
