%%%-*-latex-*-

\documentclass[a4paper,12pt]{article}

\usepackage[T1]{fontenc}
\usepackage[utf8]{inputenc}
\usepackage[english]{babel}
\usepackage[charter]{mathdesign}
\usepackage{verse}
\usepackage{url}

\title{Haikus}
\author{Christian Rinderknecht\\
{\small \url{rinderknecht@free.fr}}}
\date{\today}

\begin{document}

\begin{verse}
  the melting stream \\
  carries faraway echos \\
  --- I soak my sleeves
\end{verse}

%% \begin{verse}
%% le ruisseau de fonte \\
%% charrie de lointains �chos \\
%% j'y trempe mes manches
%% \end{verse}

%% \begin{verse}
%% las je vais cherchant \\
%% mes pas perdus sous la neige \\
%% un enfant au loin
%% \end{verse}

%% \begin{verse}
%% le soir d'hiver tombe \\
%% sous la lune funambule \\
%% l'avion tend un fil
%% \end{verse}

%% \begin{verse}
%% cent hirondelles \\
%% tue-t�te sur la port�e \\
%% des fils �lectriques
%% \end{verse}

%% \begin{verse}
%% mouche m�me toi \\
%% tu me d�chiffres � tatons \\
%% parcourant ma page
%% \end{verse}

%% \begin{verse}
%% lune d'�t� ronde \\
%% pos�e au sommet du mont \\
%% le souffle coup�
%% \end{verse}

%% \begin{verse}
%% assis immobile \\
%% dans le reflet insondable \\
%% carpe � contre-courant
%% \end{verse}

%% \begin{verse}
%% son doigt prend mon doigt \\
%% viens contre mon c{\oe}ur petit \\
%% liseron grimpant
%% \end{verse}

%% \begin{verse}
%% fi�vreux dans son lit \\
%% il r�ve du grand large \\
%% le ru s'ass�chant
%% \end{verse}

%% \begin{verse}
%% ha�ku mais \\
%% tu ris de moi aux �clats \\
%% de ton vers bris�
%% \end{verse}

%% %% \begin{verse}
%% %% toc toc il cogne \\
%% %% dans le bois le pic-�peiche \\
%% %% aux portes de l'hiver
%% %% \end{verse}

%% %% \begin{verse}
%% %% le pauvre escargot \\
%% %% jusqu'o� trainera-t-il sa \\
%% %% langue d'affam�?
%% %% \end{verse}

%% \begin{verse}
%% ah qui osera \\
%% lui passer la museli�re \\
%% morsures du froid
%% \end{verse}

%% %% \begin{verse}
%% %% lui tenant la queue \\
%% %% elle lui renifle le cul \\
%% %% chalande et melon
%% %% \end{verse}

%% %% \begin{verse}
%% %% luciole tomb�e \\
%% %% amoureuse et perdue \\
%% %% je suis ton ami
%% %% \end{verse}

%% %% \begin{verse}
%% %% la bise � l'aurore \\
%% %% effleure encore les pens�es \\
%% %% ploc premi�re goutte
%% %% \end{verse}

%% %% \begin{verse}
%% %% petit h�risson \\
%% %% un {\oe}il curieux te trahit \\
%% %% tu n'es pas f�ch�
%% %% \end{verse}

%% %% \begin{verse}
%% %% d�luge automnal \\
%% %% charrie deux pauvres lombrics \\
%% %% No� est au sec
%% %% \end{verse}

%% %% \begin{verse}
%% %% araign�e g�ante \\
%% %% entre moi et la lune \\
%% %% qui eut le plus peur?
%% %% \end{verse}


%% %% \begin{verse}
%% %% soir tombant d'�t� \\
%% %% un canard esseul� tire \\
%% %% un long V en feu
%% %% \end{verse}

%% %% \begin{verse}
%% %% Dame Souricette \\
%% %% grignote entre deux traverses \\
%% %% longs quais en �t�
%% %% \end{verse}

%% %% \begin{verse}
%% %% rendez-vous d'automne \\
%% %% scarab�es en r�union: \\
%% %% colloque amoureux?
%% %% \end{verse}

%% \begin{verse}
%% le vent de juillet \\
%% porte la nue infinie \\
%% tra�ne de mari�e
%% \end{verse}

%% %% \begin{verse}
%% %% ni vu ni connu \\
%% %% je fauche des chrysanth�mes \\
%% %% soir de la Toussaint
%% %% \end{verse}

%% %% \begin{verse}
%% %% sur le champ de neige \\
%% %% les corbeaux affam�s comptent \\
%% %% les mottes de terre
%% %% \end{verse}

%% %% \begin{verse}
%% %% flocons et soir tombent \\
%% %% coiffant leurs bonnets de nuit \\
%% %% les clochers s'endorment
%% %% \end{verse}

%% %% \begin{verse}
%% %% flocons s'�grenant \\
%% %% dans le bois un cache-cache \\
%% %% seul avec les arbres
%% %% \end{verse}

%% %% \begin{verse}
%% %% Mardi Gras foufou \\
%% %% m�me la neige maquille \\
%% %% poteaux en bouleaux
%% %% \end{verse}

%% %% \begin{verse}
%% %% j'ai d� oublier \\
%% %% mes oreilles � la maison \\
%% %% quel froid de canard
%% %% \end{verse}

%% \begin{verse}
%% araign�e du soir \\
%% verticale �quilibriste \\
%% tends-moi la perche
%% \end{verse}

\end{document}
