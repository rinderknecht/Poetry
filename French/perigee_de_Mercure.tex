\documentclass[twocolumn,11pt]{article}

\usepackage[french]{babel}
\usepackage[T1]{fontenc}
\usepackage[utf8]{inputenc}
\usepackage[charter]{mathdesign}

\setlength\columnsep{0em}

\title{PÉRIGÉE DE MERCURE}
\author{}
\date{}

\begin{document}

\maketitle

\thispagestyle{empty}

\begin{verse}
%% Valéry avait sa corbeille et ses abeilles,\\
%% il voyait la mer toujours recommencée;\\
%% Neruda avait le rire de M pour réveil,\\
%% il chantait le ciel et la terre ensemencée.\\
%% \ \\

%% Valéry et Neruda avaient leur muse,\\
%% que j'aimerais leur prendre par ruse!\\
%% \ \\

%% Cette obscurité n'est que la nuit qui tombe\\
%% et aucun feu follet n'éclaire ces tombes;\\
%% ce n'est pas ta nuisette entre mes doigts\\
%% dans la brume qui flotte dans le bois.\\
%% Dans le vent filant sous les nuages,\\
%% ce n'est pas toi glissant sous les draps,\\
%% et cette rumeur qui gronde avec l'orage\\
%% n'est pas moi mordant tes bras trop sages.\\
%% \ \\

Mes yeux errent dans le firmament,\\
ce même ciel qui sépare les amants,\\
et je songe avec gravité à un détour,\\
comme un voleur ailé\\
à une course sans retour.\\
\ \\

Qui de nous deux s'en est allé?\\
Je baisse les yeux vers la terre.\\
Qui de nous deux ira vers l'autre\\
après cette saison en enfer? \\
\ \\

Quand le vif-argent descendra,\\
tu avaleras des larmes de rage\\
de ne pouvoir être délivrée, \\
et ton encre renversée sur la page\\
sera le soir sur le champ givré!\\
\ \\

Alors je file à toute allure\\
vers ce point minuscule...\\
Déjà, du bout de ma langue, je creuse\\
de longs sillons dans ta chevelure,\\
je suis le laboureur du crépuscule,\\
l'alchimiste à son {\oe}uvre au noir,\\
et j'avive le feu sous le creuset\\
jusqu'à fendre la terre cuite.\\
%\ \\

\newpage

Des monts se soulèvent, \\
des sources jaillissent, \\
des continents se fracassent!\\
\ \\

Étourdi au bord du gouffre\\
(est-ce le sel ou le soufre?),\\
je m'éclipse dans une crevasse\\
et tu soupires comme un soufflet\\
sur la fournaise renouvelée,\\
sur l'origine du monde.\\
\ \\

Nos lèvres deviennent une\\
et je goûte la chair vermillon\\
d'une grenade au clair de lune.\\
\ \\

Tu imposes tes mains\\
et me sacres témoin\\
des noces mystiques\\
de l'eau et du feu;\\
à tâtons,\\
je lis de ce livre muet\\
la géographie céleste\\
de tes grains de beauté,\\
un zodiaque secret\\
sur ton corps d'ambre lisse\\
où je meurs et renais.
\end{verse}

\end{document}
