\documentclass[12pt,a4paper]{article}

\usepackage[T1]{fontenc}
\usepackage[utf8]{inputenc}
\usepackage[french]{babel}
\usepackage[charter]{mathdesign}
\usepackage{verse}
\usepackage{url}

\title{Poèmes}
\author{Christian Rinderknecht\\
{\small \url{rinderknecht@free.fr}}}
\date{\today}

\begin{document}

\maketitle

\thispagestyle{empty}

\newpage\leavevmode\thispagestyle{empty}\newpage

%----------------------------------------------------------------------

\newpage

\poemtitle{}

\settowidth{\versewidth}{d'agripper le nappe brûlante}

\bigskip

\begin{verse}[\versewidth]
  Comme une tasse de thé \\
  qui déborde \\

  elle continue d'aller \\
  où je ne l'attends pas

  au-delà de moi \\
  dessinant des silhouettes \\
  parfumées \\
  sur le matin blanc

  et je ne peux m'empêcher \\
  d'agripper le nappe brûlante

  encore et encore

  je drape mon cœur \\
  en fantôme \\
  et je continue d'aller \\
  où l'on ne m'attend pas

  hantant les tasses de thé \\
  d'un visage tremblant
\end{verse}

%----------------------------------------------------------------------

\newpage

\poemtitle{}

\settowidth{\versewidth}{J'aimerais que nous soyons eux}

\bigskip

\begin{verse}[\versewidth]
  Sans réfléchir \\
  tu me tends un second bocal

  Sans réfléchir \\
  je l'ouvre et tu souris \\
  de profil \\
  de cette intimité \\
  comme une épouse

  J'aimerais que nous soyons eux \\

  cachés en pleine vue

  des délices secrètes \\
  qui peuvent toucher \\
  et être ouvertes \\
  sans réfléchir
\end{verse}

%----------------------------------------------------------------------

\newpage

\poemtitle{}

\settowidth{\versewidth}{reflètant la lumière du jour}

\bigskip

\begin{verse}[\versewidth]
  Parmi les invités \\
  rayonnant et riant \\
  aux blagues brillantes

  nous partageons un verre \\
  avec deux glaçons

  Côte à côte \\
  complices et beaux

  reflétant la lumière du jour

  les icebergs fatals \\
  flottent toujours \\
  dans un monde de silence

  donnent toujours \\
  le change aux soleils \\
  quatre-vingt dix pour-cents \\
  sous la surface
\end{verse}

%----------------------------------------------------------------------

\newpage

\poemtitle{}

\settowidth{\versewidth}{invisible et pourtant si proche}

\bigskip

\begin{verse}[\versewidth]
  Ses bras ballant \\
  abandonnés au vent \\
  comme des lianes

  battent gentiment \\
  au rythme de ses pas

  oublieux du chœur \\
  de la forêt autour de nous \\
  qui murmure

  qui fixe les pendules divins \\
  attendant un signe

  comme moi

  Reconnais le cerf sauvage \\
  invisible et pourtant si proche

  Entends sa poitrine profonde \\
  tambouriner obscurément \\
  une chanson sauvage

  sur toi
\end{verse}

%----------------------------------------------------------------------

\newpage

\poemtitle{}

\settowidth{\versewidth}{--- Faisons du pain perdu!}

\bigskip

\begin{verse}[\versewidth]
  Quel gâchis \\
  je me dis

  ce pain entier \\
  rassi \\
  que tu as oublié \\
  au fond du placard

  Je me souviens \\
  de la miche \\
  chaude et tendre \\
  comme la promesse \\
  d'un amour de jeunesse

  et

  --- Faisons du pain perdu! \\
  me dis-tu
\end{verse}

%----------------------------------------------------------------------

\newpage

\poemtitle{}

\settowidth{\versewidth}{je glisse mes mots}

\bigskip

\begin{verse}[\versewidth]
  Quand je te donne \\
  la réplique

  parfois \\
  je glisse mes mots \\
  et tu veux \\
  glisser aussi

  avec moi

  une page vierge
\end{verse}

%----------------------------------------------------------------------

\newpage

\poemtitle{}

\settowidth{\versewidth}{d'une lumière chaude}

\bigskip

\begin{verse}[\versewidth]
  Au loin

  les bouleaux \\
  baignés \\
  d'une lumière chaude

  Au près

  tes doigts blancs \\
  peignant \\
  ta chevelure dorée

  Assoiffé

  tu guides mes mains \\
  vers l'eau claire \\
  de la fontaine

  Aveuglé

  de tes lèvres \\
  la goulée \\
  éclipse le soleil
\end{verse}

%----------------------------------------------------------------------

\newpage

\poemtitle{}

\settowidth{\versewidth}{Mais les métaphores murmuraient}

\bigskip

\begin{verse}[\versewidth]
  Tu te cachais dans mes livres \\
  dans les coins et recoins

  Je te retrouvais dans Neruda \\
  tamisant mes pages préférées \\
  pour une pépite de mon âme

  Mais les métaphores murmuraient \\
  comme je ne le pouvais pas \\
  comme je ne l'osais pas

  jusqu'à ce que je trouve \\
  une boucle de tes cheveux \\
  dans une marge

  Soudainement

  d'une chiquenaude \\
  le ressort de bronze \\
  fait palpiter mon cœur \\
  comme une montre cassée

  Penché sur ces pages \\
  sur le coin d'une table \\
  les mots fondent au blanc \\
  comme un film surexposé \\
  au halo de tes cheveux blonds

  et il ne reste \\
  que le cœur sans rythme \\
  d'un aveugle
\end{verse}

%----------------------------------------------------------------------

\newpage

\poemtitle{}

\settowidth{\versewidth}{sur ton corps d'ambre lisse}

\bigskip

\begin{verse}[\versewidth]
  Cette nuit \\
  tu ouvres pour moi \\
  un livre d'alchimie

  Près de la lampe à huile \\
  palpitante \\
  le livre muet parle \\
  en creusets et symboles

  de roses sauvages écloses \\
  sur la tombe des amants

  d'une grenade fendue \\
  sous une pleine lune

  de soufflets soupirant \\
  sur une fournaise renouvelée

  de l'union mystique \\
  de l'eau et du feu

  de la géographie céleste \\
  de tes grains de beauté

  un zodiaque secret \\
  pour tous les sens \\
  sur ton corps d'ambre lisse \\
  où je meurs et renais
\end{verse}

%----------------------------------------------------------------------

\newpage

\poemtitle{}

\settowidth{\versewidth}{mon haleine faisant descendre un frisson}

\bigskip

\begin{verse}[\versewidth]
  Te souviens-tu comme on jouait \\
  à l'hiver en été?

  Tu fendais ma chevelure \\
  du bout de ta langue \\
  et le champ frissonnait \\
  sous le soc

  Je m'asseyais derrière toi \\
  et te serrais comme un manteau \\
  mon haleine se fondant en un frisson \\
  le long de ta colonne vertébrale

  Je faisais semblant de glisser mes mains \\
  comme des luges sur ton buste \\
  et ton rire nerveux \\
  faisait fondre mon cœur

  Aussi épuisée que le jour \\
  tu t'affalais enfin sur moi \\
  et nos souffles comme des geysers \\
  montaient vers la nuit étoilée

  tournant doucement \\
  autour de nos nez froids \\
  l'un contre l'autre \\
  dans l'équateur de mes bras \\
  autour de toi
\end{verse}

%----------------------------------------------------------------------

\newpage

\poemtitle{}

\settowidth{\versewidth}{et se montraient l'un l'autre}

\bigskip

\begin{verse}[\versewidth]
  Une fois à l'aube \\
  de la fenêtre du chalet

  nous avons surpris \\
  le monde transi \\
  fixer du regard \\
  son reflet \\
  dans la rosée

  En un clin d'œil \\
  les ronds miroirs \\
  d'eau irisée \\
  avaient formé \\
  un kaléidoscope illimité \\
  et se montraient l'un l'autre

  tes yeux de rêve \\
  et vision immense

  et s'émerveillaient
\end{verse}

%----------------------------------------------------------------------

\newpage

\poemtitle{}

\settowidth{\versewidth}{l'ombre poursuit les murmures}

\bigskip

\begin{verse}[\versewidth]
  Déjà l'heure avide
  engloutit \\
  l'orée du monde

  Vers l'horizon \\
  l'ombre poursuit les murmures \\
  égarés parmi les tombes\\
  le long des murs

  De son crâne \\
  le poète voit partout \\
  l'encre qui l'inonde

  Tu surgis pourtant \\
  dans l'immobile seconde \\
  illuminant de mon caveau \\
  l'embrasure

  Ton visage brillant \\
  d'en-haut me sourit \\
  simple et élégant \\
  comme une lame

  Dans une sarabande \\
  tu sembles dire \\
  rejoins-moi \\
  dans les champs bleus

  et laisse ton cœur \\
  lentement battre le tambour \\
  d'un autre jour

  laisse cette longue nuit \\
  te surprendre \\
  d'une seconde pleine lune
\end{verse}

%----------------------------------------------------------------------

\newpage

\poemtitle{}

\settowidth{\versewidth}{le firmament en soleils innombrables}

\bigskip

\begin{verse}[\versewidth]
  Le jour mourant rétrécit \\
  entre les nuages \\
  et la pluie commence à dessiner \\
  des miroirs dans la boue

  jusqu'à ce que les ténèbres \\
  noient l'allumette de l'éloquence \\
  et des aiguilles percent \\
  le firmament \\
  en étoiles innombrables

  Elles clignent à des lieues \\
  comme tu le faisais

  quand tu t'étirais sur moi \\
  comme une constellation \\
  et m'embrassais distraitement

  Maintenant à genoux \\
  je cherche toujours \\
  à atteindre les étoiles \\
  mais la goulée \\
  n'est que de l'eau sale
\end{verse}

%----------------------------------------------------------------------

\newpage

\poemtitle{}

\settowidth{\versewidth}{qui sondent silencieusement ta nuit}

\bigskip

\begin{verse}[\versewidth]
  Silencieusement à contre-jour \\
  tu es cadrée à la perfection

  Je me fiche de ce qu'ils pensent \\
  au sujet de l'optique de l'amour \\
  au sujet du cliché idéal \\
  du bonheur

  Comme une forêt \\
  adossée au crépuscule

  tu es l'hôte \\
  d'une multitude de vies \\
  qui murmurent indistinctement \\
  comme une seule \\
  et soudain se taisent \\
  quand quelqu'un rit

  Lentement j'ose une main \\
  sur tes longs cheveux noirs \\
  et je touche une ombre \\
  dans l'ombre

  Mais tu es surprise par mes yeux \\
  comme ceux d'un acteur de film muet \\
  en gros plan

  Prise dans mes phares \\
  cette âme magnifique \\
  hésite avant de disparaître \\
  prestement

  me laissant bouche bée \\

  les pupilles dilatées \\
  sondant silencieusement ta nuit
\end{verse}

\end{document}
