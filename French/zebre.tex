%-*-latex-*-

\documentclass[twocolumn,11pt,a4paper]{article}

\usepackage[T1]{fontenc}
\usepackage[utf8]{inputenc}
\usepackage[francais]{babel}
\usepackage[charter]{mathdesign}
%\usepackage{verse}

\setlength\columnsep{0em}

\thispagestyle{empty}

\title{À mon petit zèbre}
\author{}
\date{}

%\settowidth{\versewidth}{ses teintes au prisme dérobées,}

\begin{document}

\maketitle

\begin{verse}%[\versewidth]
Au carnaval des animaux\\
défilent des humains\\
poussant des ah! et des oh!\\
levant au ciel leurs mains.\\
\ \\

Tu n'en crois pas tes cinq sens,\\
parmi les rues de mèche\\
semées de roses fraîches,\\
c'est le cortège de Saint-Saëns!\\
\ \\

Voici Jeanne la Papesse,\\
bénissant d'une cravache\\
la joyeuse kermesse\\
sur le dos d'une vache;\\
\ \\

alors qu'Arlequin révèle\\
ses teintes au prisme dérobées,\\
culbute Polichinelle\\
avec sa bosse enrobée,\\
\ \\

Colombine suit un miroir,\\
Pierrot sort d'un tiroir\\
la lune ronde qu'il aime,\\
qu'une pantomime fait poème.\\
\ \\

Tu crains d'être fou à lier\\
quand un curieux chevalier\\
sur sa verte monture\\
conjure le roi Arthur!\\
\ \\

De la fin de cette saga\\
tu crois rêver ce couagga;\\
toi seul tu l'envies\\
d'être à moitié toi,\\
même pas en vie,\\
ou à moitié, comme toi.\\

\newpage

Au carnaval des animaux\\
tous portent des masques,\\
dansent des bergamasques\\
et oublient leurs maux.\\
\ \\

Tu te laisses caresser,\\
tu fais bonne figure,\\
comme si bien dressé,\\
mais tu sais l'augure:\\
\ \\

Aux jours et aux nuits tu dérobes\\
la pluie et les ténèbres\\
qui font les cortèges funèbres\\
et tu les portes sur ta robe\\
--- Ah! Mon petit zèbre!
\end{verse}

\end{document}

