%%%-*-latex-*-

\documentclass[11pt,a4paper]{article}

\usepackage[T1]{fontenc}
\usepackage[utf8]{inputenc}
\usepackage[spanish,french]{babel}
\usepackage[charter]{mathdesign}
\usepackage{verse}

\begin{document}

\thispagestyle{empty}

\poemtitle{PIERROT TOURMENTÉ}

\settowidth{\versewidth}{dans la pâleur de tes tulles blancs... dans la ronde!}

\bigskip

\begin{verse}[\versewidth]
Déjà l'heure avide engloutit l'orée du monde; \\
vers l'horizon l'ombre poursuit les murmures \\
égarés parmi les tombes, le long des murs... \\
--- Le poète voit partout l'encre qui l'inonde! \\ \

Vers l'horizon l'ombre poursuit les murmures; \\
tu surgis pourtant dans l'immobile seconde \\
(le poète voit partout l'encre qui l'inonde), \\
illuminant de mon être l'embrasure! \\ \

Tu surgis pourtant dans l'immobile seconde, \\
ô visage rond fardé d'argent et d'azur, \\
illuminant de mon être l'embrasure \\
dans la pâleur de tes tulles blancs... dans la ronde! \\ \

Haut visage rond fardé d'argent et d'azur \\
tu es, simple et élégante comme l'aronde \\
dans la pâleur de tes tulles blancs; dans la ronde \\
gaie elle m'entraîne en battant la mesure. \\ \

Tu es simple et élégante comme l'aronde; \\
parmi les épis bleus, hors de la masure \\
gaie, elle m'entraîne en battant la mesure... \\
--- Vers quelle face cachée, quelle bête immonde?
\end{verse}

\end{document}
