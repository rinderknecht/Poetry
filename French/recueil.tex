%-*-latex-*-

\documentclass[11pt,a4paper]{book}

\usepackage[T1]{fontenc}
\usepackage[utf8]{inputenc}
\usepackage[francais]{babel}
\usepackage[charter]{mathdesign}
\usepackage{verse}
\usepackage{url}

\begin{document}

\thispagestyle{empty}

\vspace*{70mm}

\begin{center}
{\Huge\textbf{L'archipel des comètes}} \\
\bigskip
\textbf{Christian Rinderknecht}
\end{center}

\bigskip\bigskip\bigskip
\hfill Pour Élise Camier

\newpage

\poemtitle{La comète}

\settowidth{\versewidth}{Parfois l'île vient au marin}

\bigskip

\begin{verse}[\versewidth]
  Tu me dis que tu es revenue \\
  vers moi

  mais je suis la comète \\
  eccentrique

  De mon archipel \\
  de solitude \\
  loin de ton orbite \\
  circulaire \\
  j'ai senti ton rappel \\
  et fui

  à la vitesse de libération \\
  celle de la lumière \\
  de tes yeux clairs

  Parfois l'île vient au marin
\end{verse}

\newpage

\poemtitle{La fontaine}

\settowidth{\versewidth}{fait pâlir le soleil}

\bigskip

\begin{verse}[\versewidth]
  Assoiffé

  tu guides mes mains \\
  vers l'eau claire \\
  de la fontaine

  Aveuglé

  de tes lèvres \\
  la goulée \\
  éclipse le soleil
\end{verse}

\newpage

\poemtitle{Chevelure}

\settowidth{\versewidth}{d'une lumière chaude}

\bigskip

\begin{verse}[\versewidth]
  Au loin

  les bouleaux \\
  baignés \\
  d'une lumière chaude

  Au près

  tes doigts blancs \\
  peignant \\
  ta chevelure dorée
\end{verse}

\newpage

\poemtitle{Les bocaux}

\settowidth{\versewidth}{de cette intimité}

\bigskip

\begin{verse}[\versewidth]
  Tu me tends \\
  un second bocal

  Sans réfléchir \\
  je l'ouvre \\
  et tu souris \\
  de profil \\
  de cette intimité

  comme des mariés

  nous sommes \\
  comme ces bocaux \\
  à ouvrir

  sans réfléchir
\end{verse}

\newpage

\poemtitle{Ad libitum}

\settowidth{\versewidth}{je glisse mes mots}

\bigskip

\begin{verse}[\versewidth]
  Quand je te donne \\
  la réplique

  parfois \\
  je glisse mes mots \\
  et tu veux \\
  glisser aussi

  avec moi

  une page vierge
\end{verse}

\newpage

\poemtitle{Dualité}

\settowidth{\versewidth}{d'onde et de particules}

\bigskip

\begin{verse}[\versewidth]
  Espiègle \\
  tu me parles \\
  d'ondes et de particules

  éclectiques

  d'avance de phase \\
  et de décharges

  électriques

  Moi je ne suis que \\
  gravité

  faible \\
  à la portée \\
  de tes bras
\end{verse}

\newpage

\poemtitle{Mauvais temps}

\settowidth{\versewidth}{où je t'ai réchauffée}

\bigskip

\begin{verse}[\versewidth]
  Dans cette valse \\
  de l'oubli \\
  et des regrets

  tu fais un pas \\
  et tu oublies

  j'en fais un autre \\
  et je regrette

  L'un contre l'autre \\
  cherchons \\
  le troisième temps

  comme cet après-midi \\
  de mauvais temps \\
  où je t'ai réchauffée \\
  dans mes bras
\end{verse}

\newpage

\poemtitle{Les regrets}

\settowidth{\versewidth}{--- Faisons du pain perdu!}

\bigskip

\begin{verse}[\versewidth]
  Quel gâchis

  je me dis

  ce pain entier \\
  rassi \\
  que tu as oublié \\
  au fond du placard

  Je me souviens \\
  de la miche \\
  chaude et tendre \\
  comme la promesse \\
  d'un amour de jeunesse

  et

  --- Faisons du pain perdu! \\
  me dis-tu
\end{verse}

\newpage

\poemtitle{La note}

\settowidth{\versewidth}{planter quelques clous}

\bigskip

\begin{verse}[\versewidth]
  Sur la table de la cuisine \\
  une note

  « Je vois \\
  que tu as \\
  aiguisé les couteaux \\
  et réparé la table \\
  bancale

  « Tu vois \\
  il fallait \\
  planter quelques clous \\
  dans le bois tendre \\
  pour ne plus se couper

  « Et je t'aime. »

\end{verse}

\newpage

\poemtitle{Oui mais non}

\settowidth{\versewidth}{même au bout d'un fusil}

\bigskip

\begin{verse}[\versewidth]
  Oui

  tu n'as jamais été \\
  aussi belle \\
  que sur tes photos \\
  de mariage

  Mais non

  même au bout d'un fusil \\
  je ne serais venu

  Ajouter du plomb \\
  au plomb \\
  dans le cœur \\
  ne n'aurait pas alourdi \\
  autant que de venir

  --- non ---

  et de te voir briller \\
  dans cette lumière

  --- non ---

  et de te laisser partir \\
  dans ce fondu au blanc

  en disant oui
\end{verse}

\newpage

\poemtitle{Grand bleu}

\settowidth{\versewidth}{où je cherche mon reflet}

\bigskip

\begin{verse}[\versewidth]
  Ces yeux clairs

  plus clairs ce matin \\
  d'avoir aimé un autre

  ciel plus bleu

  dont le soleil levant \\
  a laissé deux mares \\
  où je cherche mon reflet \\
  sous la rosée
\end{verse}

\newpage

\poemtitle{8/9}

\settowidth{\versewidth}{Ils ne sont pas différents}

\bigskip

\begin{verse}[\versewidth]
  Parmi les invités riant \\
  nous partageons un verre \\
  avec deux glaçons

  accolés

  Ces icebergs fatals \\
  complices et beaux \\
  flottent \\
  dans un monde de silence

  donnant aux soleils \\
  le change \\
  au neuvième
\end{verse}

\newpage

\poemtitle{Le cimetière des étoiles}

\settowidth{\versewidth}{Quand le ciel, comme une mer renversée,}

\bigskip

\begin{verse}[\versewidth]
  Quand le ciel, comme une mer renversée, \\
  resplendit de l'éclat de tes yeux, \\
  je m'élève et plonge dans la nuée \\
  à la recherche de perles bleues.

  Pris de vertige dans la lame qui passe, \\
  le pêcheur lesté d'une pierre \\
  désespérément coule une brasse \\
  comme une rythmique prière.

  Au cimetière des étoiles \\
  s'irisent deux blancs coquillages, \\
  lunes sous un linceul de voiles;

  dévoilant ses iris ultra-marins, \\
  l'Abysse reconnaît le marin \\
  et lui promet d'autres rivages.
\end{verse}

\newpage

\poemtitle{Les vers blancs}

\settowidth{\versewidth}{l'haleine coupée à chaque descente,}

\bigskip

\begin{verse}
  Des luges dévalent une colline, \\
  chaque bosse enfilant une note \\
  aux tresses de leurs lignes de vie, \\
  aux colliers des éclats de rire. \\
  \ \\
  Comme elles, je trace \\
  sur des pages immaculées \\
  des parallèles invisibles \\
  qui conjurent ta silhouette. \\
  \ \\
  À cache-cache dans la brume, \\
  tes mains diaphanes sur mes yeux, \\
  ton souffle haletant sur ma nuque \\
  fait descendre un long frisson \\
  qui se mêle à celui de l'hiver. \\
  \ \\
  Je caresse les pages blanches \\
  d'un livre ouvert à l'invisible, \\
  %  invoquant l'empreinte de tes doigts \\
  invoquant ton visage pâle \\
  sous mes doigts qui frémissent \\
  de douces collines familières, \\
  nues sous le manteau d'hermine, \\
  l'haleine coupée à chaque descente, \\
  si impatient à chaque montée! \\
  \ \\
  %% Épuisés, allongés sur la neige, \\
  %% nous regardons passer les nuages \\
  %% et je m'abreuve au cours calme de tes mots \\
  %% avant qu'il ne plonge dans l'oubli \\
  %% comme une cataracte gelée.


  Fantôme bien-aimé, \\
  était-ce un mot trop pur, \\
  un c{\oe}ur trop chaud \\
  qui te fit évaporer? \\
  \ \\
  Tu ne laissas, \\
  dans une marge \\
  du grimoire blanc, \\
  qu'un cheveu doré. \\
  \ \\
  D'une chiquenaude, \\
  l'infime ressort \\
  fait palpiter mon c{\oe}ur \\
  comme une montre. \\% à rebours... \\
  \ \\
  Penchés sur ces pages \\
  au coin d'une table \\
  nous admirions... quoi? \\
  \ \\
  Le sourire en efface le souvenir, \\
  comme une pellicule surexposée, \\
  et seuls restent ces vers en braille \\
  où je cherche à tâtons \\
  tes pas vers le paradis blanc.
\end{verse}

\newpage

\poemtitle{Mille et une voix}

\settowidth{\versewidth}{qui te transfigurent,}

\bigskip

\begin{verse}[\versewidth]
\emph{Lui}

De ces âmes en toi \\
qui te transfigurent, \\
une seule voit, \\
mille autres murmurent.

\emph{La Raison}

--- Éphémères voix, \\
dehors rien ne dure. \\
Silence! Je vois \\
dans un monde dur!

Que veut le désir, \\
une avide flamme \\
qui ne peut saisir, \\
sinon ruiner l'âme?

\emph{Le Désir}

--- Notre feu est doux, \\
il chauffe le cœur \\
et rosit la joue \\
comme une liqueur.

Que craint la raison, \\
qui voit chaque pas \\
mais qui n'y croit pas, \\
sinon l'oraison?

\emph{Lui}

Mille et un et moi, \\
qui pensent et prient \\
mille et un émois \\
d'un sésame épris.
\end{verse}

\newpage

\poemtitle{Un feu nouveau}

\settowidth{\versewidth}{aux pieds froids sur des mosaïques.}

\bigskip

\begin{verse}[\versewidth]
  Loin du regard impie des laïcs, \\
  les nuits je luis d'un éclat égal, \\
  servi par de lasses vestales \\
  aux pieds froids sur des mosaïques.

  Une parque de son piedestal \\
  de marbre descend et me sourit: \\
  «~Petit feu du devoir ancestral, \\
  oublie le souffle qui te nourrit;

  «~Aucune tendre bouche de chair \\
  n'est aussi pure que de pierre! \\
  Oublie des vierges les tuniques,

  «~Laisse mes mains dures te ravir \\
  jusqu'à la forge vulcanique, \\
  ensemble allons ce volcan gravir!~»
\end{verse}

\end{document}
