\documentclass[twocolumn,11pt]{article}

\usepackage[francais]{babel}
\usepackage[T1]{fontenc}
\usepackage[utf8]{inputenc}
\usepackage[charter]{mathdesign}

\setlength\columnsep{0em}

\title{LES VERS BLANCS}
\author{}
\date{}

\begin{document}

\maketitle

\thispagestyle{empty}

\begin{verse}
  Des luges dévalent une colline, \\
  chaque bosse enfilant une note \\
  aux tresses de leurs lignes de vie, \\
  aux colliers des éclats de rire. \\
  \ \\
  Comme elles, je trace \\
  sur des pages immaculées \\
  des parallèles invisibles \\
  qui conjurent ta silhouette. \\
  \ \\
  À cache-cache dans la brume, \\
  tes mains diaphanes sur mes yeux, \\
  ton souffle haletant sur ma nuque \\
  fait descendre un long frisson \\
  qui se mêle à celui de l'hiver. \\
  \ \\
  Je caresse les pages blanches \\
  d'un livre ouvert à l'invisible, \\
  %  invoquant l'empreinte de tes doigts \\
  invoquant ton visage pâle \\
  sous mes doigts qui frémissent \\
  de douces collines familières, \\
  nues sous le manteau d'hermine, \\
  l'haleine coupée à chaque descente, \\
  si impatient à chaque montée! \\
  \ \\
  Épuisés, allongés sur la neige, \\
  nous regardons passer les nuages \\
  et je m'abreuve au cours calme de tes mots \\
  avant qu'ils ne plongent dans l'oubli \\
  comme une cataracte gelée.

  \newpage

  Fantôme bien-aimé, \\
  était-ce un mot trop pur, \\
  un c{\oe}ur trop chaud \\
  qui te fit évaporer? \\
  \ \\
  Tu ne laissas, \\
  dans une marge \\
  du grimoire blanc, \\
  qu'un cheveu doré. \\
  \ \\
  D'une chiquenaude, \\
  l'infime ressort \\
  fait palpiter mon c{\oe}ur \\
  comme une montre. \\% à rebours... \\
  \ \\
  Penchés sur ces pages \\
  au coin d'une table \\
  nous admirions... quoi? \\
  \ \\
  Le sourire en efface le souvenir, \\
  comme une pellicule surexposée, \\
  et seuls restent ces vers en braille \\
  où je cherche à tâtons \\
  tes pas vers le paradis blanc.
\end{verse}

\end{document}
