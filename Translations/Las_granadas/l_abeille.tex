%-*-latex-*-

\addcontentsline{toc}{subsection}{La abeja (\emph{L{}'abeille})}

\selectlanguage{spanish}

\poemtitle*{La abeja}

\settowidth{\versewidth}{Sobre quién Pasión muere tan lacia,}

\begin{flushright}
{\scriptsize \emph{A Francis de Miomandre.}}
\end{flushright}

%\bigskip

\begin{verse}[\versewidth]
  Tan fina sea tu punta, hasta \\
  mortal, rubia abeja salvaje, \\
  yo sólo a mi tierna canasta \\
  le eché un sueño de encaje.

  Pica del seno la calabaza ya, \\
  sobre quién Pasión muere tan lacia, \\
  ¡qué un poco de mí bermejo vaya \\
  por la carne curva y reacia!

  Un fugaz tormento que se siente: \\
  ¡dolor vivo y bien terminado \\
  vale más que suplicio durmiente!

  Sé mi sentido iluminado \\
  por la punta que alerta pone \\
  ¡sin quién Pasión muere o transpone!
\end{verse}

\newpage

\selectlanguage{french}

\poemtitle*{\emph{L{}'abeille}}

\settowidth{\versewidth}{J'ai grand besoin d'un prompt tourment:}

\begin{flushright}
{\scriptsize \emph{À Francis de Miomandre.}}
\end{flushright}

{\itshape
\begin{verse}[\versewidth]
  Quelle, et si fine, et si mortelle, \\
  que soit ta pointe, blonde abeille, \\
  je n'ai, sur ma tendre corbeille, \\
  jeté qu'un songe de dentelle.

  Pique du sein la gourde belle, \\
  sur qui l'Amour meurt ou sommeille, \\
  qu'un peu de moi-même vermeille \\
  vienne à la chair ronde et rebelle!

  J'ai grand besoin d'un prompt tourment: \\
  un mal vif et bien terminé \\
  vaut mieux qu'un supplice dormant!

  Soit donc mon sens illuminé \\
  par cette infime alerte d'or \\
  sans qui l'Amour meurt ou s'endort!
\end{verse}
}
