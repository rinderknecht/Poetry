\vspace*{-15mm}
\centeredornament
\vspace*{2mm}

\selectlanguage{french}

\begin{center}
  \textbf{\large Conscience}
\end{center}

\addcontentsline{toc}{section}{Conscience}

\poemtitle*{I}

\addcontentsline{toc}{subsection}{I. \ \emph{L'aube détache le ciel de la terre}}

\begin{verse}
  L'aube détache le ciel de la terre \\
  et, au son de sa voix claire et douce, \\
  scarabées et enfants pirouettent \\
  en entrant dans la lumière du jour; \\
  l'air n'est pas humide, la brillante légèreté flotte!\hspace*{-4mm} \\
  Avec la nuit, elles se posèrent sur les arbres \\
  comme de petits papillons, les feuilles. \\
\end{verse}

\bigskip

\poemtitle*{II}

\addcontentsline{toc}{subsection}{II. \ \emph{J'ai vu des tableaux barbouillés de bleu}}

\selectlanguage{french}

\begin{verse}
  J'ai vu des tableaux barbouillés de bleu, \\
  rouge et jaune dans mes rêves \\
  et je sentis que tout était en ordre --- \\
  pas un seul grain de poussière qui virevolte. \\
  Maintenant estompé, mon rêve descend dans mes membres, \\
  et le monde de fer est l'ordre. \\
  Avec le jour, une lune point en moi et, \\
  à la nuit tombée, un soleil brille ci-dedans. \\
\end{verse}

\newpage

\vspace*{-15mm}
\centeredornament
\vspace*{2mm}

\selectlanguage{hungarian}

\begin{center}
  \textbf{\large Eszmélet}
\end{center}

\poemtitle*{I}

\begin{verse}
  Földtől eloldja az eget \\
  a hajnal s tiszta, lágy szavára \\
  a bogarak, a gyerekek \\
  kipörögnek a napvilágra; \\
  a levegőben semmi pára, \\
  a csilló könnyűség lebeg! \\
  Az éjjel rászálltak a fákra, \\
  mint kis lepkék, a levelek.
\end{verse}

\bigskip

\poemtitle*{II}

\begin{verse}
  Kék, piros, sárga, összekent \\
  képeket láttam álmaimban \\
  és úgy éreztem, ez a rend --- \\
  egy szalló porszem el nem hibbant. \\
  Most homályként száll tagjaimban \\
  álmom s a vas világ a rend. \\
  Nappal hold kél bennem s ha kinn van \\
  az éj --- egy nap süt idebent.
\end{verse}

\newpage

\vspace*{-15mm}
\centeredornament

\selectlanguage{french}

\poemtitle*{III}

\addcontentsline{toc}{subsection}{III. \ \emph{Je suis maigre}}

\begin{verse}
  Je suis maigre, \\
  parfois je ne mange que du pain; \\
  entouré par ces âmes oisives et bavardes, \\
  je cherche en vain plus de certitude, \\
  comme le dé. \\
  Aucun rôti de palette ne trouve ma bouche \\
  quand j'étreins un enfant sur mon cœur \\
  --- aussi fûté soit-il, \\
  le chat ne peut attrapper d'un coup \\
  la souris dehors et la souris dedans. \\
\end{verse}

\bigskip

\poemtitle*{IV}

\addcontentsline{toc}{subsection}{IV. \ \emph{Tout comme un tas de bûches}}

\begin{verse}
  Tout comme un tas de bûches, \\
  le monde gît en vrac; \\
  chaque chose presse, pèse, \\
  s'arrime à l'autre, \\
  et ainsi tout est déterminé. \\
  Seul ce qui n'est pas a un arbrisseau, \\
  seul ce qui sera peut fleurir; \\
  ce qui est tombera en pièces. \\
\end{verse}

\newpage

\vspace*{-15mm}
\centeredornament

\selectlanguage{hungarian}

\poemtitle*{III}

\begin{verse}
  Sovány vagyok, csak kenyeret \\
  eszem néha, e léha, locska \\
  lelkek közt ingyen keresek \\
  bizonyosabbat, mint a kocska. \\
  Nem dörgölődzik sült lopcska \\
  számhoz s szivemhez kisgyerek --- \\
  ügyeskedhet, nem fog a macska \\
  egyszerre kint s bent egeret.
\end{verse}

\bigskip

\poemtitle*{IV}

\begin{verse}
  Akár egy halom hasított fa, \\
  hever egymáson a világ, \\
  szorítjanyomja, összefogja \\
  egyík dolog a másikát \\
  s így mindenik determinált. \\
  Csak ami nincs, annak van bokra, \\
  csak ami lesz, az a virág, \\
  ami van, széthull darabokra.
\end{verse}

\newpage

\vspace*{-15mm}
\centeredornament

\selectlanguage{french}

\poemtitle*{V}

\addcontentsline{toc}{subsection}{V. \ \emph{À la gare de fret}}

\begin{verse}
  À la gare de fret, \\
  je m'étalai derrière le pied de l'arbre, \\
  comme une masse de silence; \\
  une herbe grise atteignit ma bouche, \\
  crue, étrange-sucrée. \\
  Faisant le mort, je regardais le garde \\
  --- ressentant quoi? --- \\
  et son ombre qui, sur les wagons silencieux, \\
  s'entêtait à bondir sur les charbons reluisants,\hspace*{-4mm} \\
  couverts de rosée. \\
\end{verse}

\bigskip

\poemtitle*{VI}

\addcontentsline{toc}{subsection}{VI. \ \emph{Voici le tourment intérieur}}

\begin{verse}
  Voici le tourment intérieur, \\
  pourtant l'explication gît à l'extérieur. \\
  Ta blessure est le monde \\
  --- en feu, s'échauffant --- \\
  et tu sens ton âme, la fièvre. \\
  Tu es captif tant que ton cœur se révolte \\
  --- ainsi tu seras libre s'il se complait \\
  à ne pas bâtir pour toi une maison \\
  où un propriétaire vient demeurer. \\
\end{verse}

\newpage

\vspace*{-15mm}
\centeredornament

\selectlanguage{hungarian}

\poemtitle*{V}

\begin{verse}
  A teherpályaudvaron \\
  úgy lapultam a fa tövéhez, \\
  mint egy darab csönd; szürke gyom \\
  ért számhoz, nyers, különös-édes. \\
  Holtan lestem az őrt, mit érez, \\
  s a hallgatag vagónokon \\
  árnyát, mely ráugrott a fényes, \\
  harmatos szénre konokon.
\end{verse}

\bigskip

\poemtitle*{VI}

\begin{verse}
  Im itt a szenvedés belül, \\
  ám ott kívül a magyarázat. \\
  Sebed a világ --- ég, hevül \\
  s te lelkedet érzed, a lázat. \\
  Rab vagy, amíg a szíved lázad --- \\
  úgy szabadulsz, ha kényedül \\
  nem raksz magadnak olyan házat, \\
  melybe háziúr települ.
\end{verse}

\newpage

\vspace*{-15mm}
\centeredornament

\selectlanguage{french}

\poemtitle*{VII}

\addcontentsline{toc}{subsection}{VII. \ \emph{Par dessous le soir}}

\begin{verse}
  Par dessous le soir, \\
  j'ai levé les yeux aux rouages des cieux: \\
  des fils scintillants de la chance \\
  le métier du passé avait tissé la loi; \\
  par dessous la vapeur de mes rêves, \\
  j'ai regardé à nouveau dans les cieux \\
  et j'ai vu la toile de la loi \\
  toujours se déchirer quelque part. \\
\end{verse}

\bigskip

\poemtitle*{VIII}

\addcontentsline{toc}{subsection}{VIII. \ \emph{Le silence écoutait attentivement}}

\begin{verse}
  Le silence écoutait attentivement \\
  --- Une heure sonna. \\
  Tu pourrais revisiter ton enfance; \\
  entre les murs de ciment humide \\
  tu pourrais imaginer un peu de liberté \\
  --- me dis-je. Et dès que je fus sur pied, \\
  les étoiles, la Grande Ourse \\
  scintillaient au-dessus, \\
  comme les grilles en haut dans ma cellule. \\
\end{verse}

\newpage

\vspace*{-15mm}
\centeredornament

\selectlanguage{hungarian}

\poemtitle*{VII}

\begin{verse}
  Én fölnéztem az est alól \\
  az eget fogaskerekére --- \\
  csilló véletlen szálaiból \\
  törvényt szőtt a mult szövőszéke \\
  és megint fölnéztem az égre \\
  álmaim gőzei alól \\
  s láttam, a törvény szövedéke \\
  mindig fölfeslik valahol.
\end{verse}

\bigskip

\poemtitle*{VIII}

\begin{verse}
  Fülelt a csend --- egyet ütött \\
  Fölkereshetnéd ifjúságod; \\
  nyirkos cementfalak között \\
  képzelhetsz egy kis sabadságot --- \\
  gondoltam. S hát hát amint fölállok \\
  a csillagok, a Göncölök \\
  úgy fénylenek fönt, mint a rácsok \\
  a hallgatag cella fölött.
\end{verse}

\newpage

\vspace*{-15mm}
\centeredornament

\selectlanguage{french}

\poemtitle*{IX}

\addcontentsline{toc}{subsection}{IX. \ \emph{J'ai entendu le fer sangloter}}

\begin{verse}
  J'ai entendu le fer sangloter, \\
  j'ai entendu la pluie rire. \\
  Je vis que le passé était craquelé \\
  et que seuls les souvenirs peuvent s'oublier, \\
  et comment je ne peux qu'aimer, \\
  pliant sous mes fardeaux --- \\
  pourquoi devrais-je aussi forger une arme \\
  de toi, for intérieur doré! \\
\end{verse}

\bigskip

\poemtitle*{X}

\addcontentsline{toc}{subsection}{X. \ \emph{Il est un homme accompli}}

\begin{verse}
  Il est un homme accompli celui qui n'a \\
  ni mère ni père en son cœur, \\
  celui qui sait qu'il reçoit la vie \\
  tel un supplément à la mort, et la rendra \\
  à tout moment comme un objet trouvé \\
  --- par conséquent il la garde, \\
  celui qui n'est ni un dieu ni un prêtre, \\
  ni pour lui-même ni pour autrui.
\end{verse}

\newpage

\vspace*{-15mm}
\centeredornament

\selectlanguage{hungarian}

\poemtitle*{IX}

\begin{verse}
  Hallottam sírni a vasat, \\
  hallottam az esőt nevetni. \\
  Láttam, hogy a mult meghasadt \\
  s csak képzetet lehet feledni; \\
  s hogy nem tudok mást, mint szeretni, \\
  görnyedve terheim alatt --- \\
  minek is kell fegyvert veretni \\
  belőled, arany öntudat!
\end{verse}

\bigskip

\poemtitle*{X}

\begin{verse}
  Az meglett ember, akinek \\
  szívében nincs se anyja, apja, \\
  ki tudja, hogy az életet \\
  halálra ráadásul kapja \\
  s mint talált tárgyat visszaadja \\
  bármikor --- ezért őrzi meg, \\
  ki nem istene és nem papja \\
  se magának, sem senkinek.
\end{verse}

\newpage

\vspace*{-15mm}
\centeredornament

\selectlanguage{french}

\poemtitle*{XI}

\addcontentsline{toc}{subsection}{XI. \ \emph{J'ai vu le bonheur}}

\begin{verse}
  J'ai vu le bonheur; il était doux, brillant \\
  et un quintal et demi. \\
  Sur la mauvaise herbe de la cour de ferme \\
  son sourire courbé se balançait. \\
  Il s'affala dans la marre tendre et tiède, \\
  plissa les yeux, puis me grogna une fois --- \\
  Jusqu'à ce jour, je vois avec quelle hésitation \\
  la lumière du jour s'amusait avec ses soies.
\end{verse}

\bigskip

\poemtitle*{XII}

\addcontentsline{toc}{subsection}{XII. \ \emph{Je vis près du chemin de fer}}

\begin{verse}
  Je vis près du chemin de fer. \\
  Nombreux sont les trains \\
  qui viennent et vont, et j'observe de loin \\
  comment les fenêtres illuminées défilent \\
  dans l'obscurité vacillante et peluchée. \\
  Ainsi les jours luisants se pressent dans la nuit éternelle, \\
  et je me tiens dans la lueur des compartiments,\hspace*{-4mm} \\
  je m'accoude et garde le silence.
\end{verse}

\newpage

\vspace*{-15mm}
\centeredornament

\selectlanguage{hungarian}

\poemtitle*{XI}

\begin{verse}
  Láttam a boldogságot én, \\
  lágy volt, szőke és másfél mázsa. \\
  Az udvar szigorú gyöpén \\
  imbolygott göndör mosolygása. \\
  Ledőlt a puha, langy tócsába, \\
  hunyorgott, röffent még felém --- \\
  ma is látom, mily tétovázva \\
  babrált pihéi közt a fény.
\end{verse}

\bigskip

\poemtitle*{XII}

\begin{verse}
  Vasútnál lakom. Erre sok \\
  vonat jön-megy és el-elnézem, \\
  hogy' szállnak fényes ablakok \\
  a lengedező szösz-sötétben. \\
  Így iramlanak örök éjben \\
  kivilágított nappalok \\
  s én állok minden fülke-fényben, \\
  én könyöklök és hallgatok.
\end{verse}
