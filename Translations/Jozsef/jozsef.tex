%-*-latex-*-

\documentclass[a4paper,12pt,twoside,final]{book}

% Page geometry for L'Harmattan
%
%% \usepackage[bindingoffset=0cm,
%%             a4paper,
%%             left=55mm,
%%             top=6cm,
%%             textheight=185mm,
%%             textwidth=10cm,
%%             footskip=13mm,
%%             includefoot,
%%             includehead,
%%             dvips]{geometry}% showframe, showcrop

\usepackage[T1]{fontenc}
%\usepackage[utf8]{inputenc} % Default nowadays
\usepackage[main=english,hungarian,french,spanish]{babel}

% https://tex.stackexchange.com/questions/642987/pdflatex-stops-when-both-german-and-hungarian-languages-are-set-in-babel

\AddToHook{begindocument}[test]{\catcode\string``=12 }
\DeclareHookRule{begindocument}{test}{before}{babel}

\usepackage{ebgaramond}
\usepackage{verse}
\poemlines{1} % Number each verse

\usepackage{url}
\usepackage{hyphenat}

\title{Collected Poems}
\author{Attila József}

\newcommand{\myfoot}[1]{\footnote{\selectlanguage{english} {#1}}}

\begin{document}

\maketitle

%% =====================================================================

\chapter{\oldstylenums{1925}}

\selectlanguage{english}

\settowidth{\versewidth}{and now the tender light of my bread shines purer.}

\begin{verse}[\versewidth]
  You are so silly. \\
  You run like the morning wind, \\
  sometime you'll be knocked down by a car. \\
  By the way, I scrubbed clean my little table, \\
  and now the tender light of my bread shines purer. \\
  Well, come back; if you want \\
  I'll buy a blanket for my iron bed. \\
  A plain, grey blanket. \\
  It will suit my Poverty, who loves you, \\
  and the Lord loves you also very much, \\
  and He also loves me. \\
  The Lord never comes in a blaze of glory, \\
  He doesn't want to ruin my eyes, \\
  which long to see you, \\
  and will look at you very beautifully. \\
  When you come back, I will kiss you carefully, \\
  I won't rip off your coat. \\
  I will tell you all the many jokes, \\
  because I devised many since then, \\
  and how merry you'll be and will blush, \\
  how you'll look down to the ground, \\
  and we will crack up, \\
  and we will be heard by our neighbours \\
  and even by the taciturn, austere day labourers \\
  who, amidst their tired, shattered dreams, \\
  will then begin to smile too. \\
\end{verse}

Autumn

\newpage

\noindent \textbf{French}

\selectlanguage{french}

\settowidth{\versewidth}{et maintenant la lumière tendre de mon pain est plus pure.}

\begin{verse}[\versewidth]
  Ne sois pas si bête. \\
  Tu cours comme le vent matinal, \\
  un jour tu seras renversée par une auto. \\
  D'ailleurs, j'ai récuré ma petite table, \\
  et maintenant la lumière suave de mon pain est plus pure. \\
  Hé bien, reviens; si tu veux \\
  j'achèterai une couverture pour mon lit de fer. \\
  Une couverture ordinaire, grise. \\
  Elle conviendra à ma Pauvreté, qui t'aime, \\
  et le Seigneur t'aime aussi beaucoup \\
  et Il m'aime aussi. \\
  Le Seigneur ne vient jamais dans toute sa splendeur, \\
  Il ne veut pas abîmer mes yeux, \\
  qui ont hâte de te voir \\
  et qui te regarderont avec beauté. \\
  Quand tu reviendras, je t'embrasserai doucement, \\
  je n'arracherai pas ton manteau. \\
  Je te raconterai toutes les nouvelles blagues, \\
  parce que j'en ai inventé beaucoup depuis, \\
  et que tu seras gaie et rougieras, \\
  et tu baisseras les yeux vers le sol, \\
  et nous éclaterons de rire, \\
  et nos voisins nous entendrons \\
  et jusqu'aux journaliers taciturnes et austères \\
  qui, parmi leurs rêves fatigués et brisés, \\
  esquisseront un sourire aussi. \\
\end{verse}

Automne

\newpage

\selectlanguage{english}

\noindent \textbf{Spanish}

\selectlanguage{spanish}

\settowidth{\versewidth}{y ahora la luz suave de mi pan brilla más pura.}

\begin{verse}[\versewidth]
  No seas tonta. \\
  Corres como el viento de la madrugada, \\
  algún día te atropellará un coche. \\
  De hecho, fregué my mesita, \\
  y ahora la luz suave de mi pan brilla más pura. \\
  Pues regresa; si quieres \\
  compraré una manta para mi cama de hierro. \\
  Una manta ordinaria, gris. \\
  Le caerá bien a mi Pobreza, que te quiere, \\
  y el Señor también te ama, \\
  y me ama también. \\
  El Señor nunca viene con gran esplendor, \\
  no quiere arruinar mis ojos, \\
  que anhelan verte, \\
  y que te mirarán con belleza. \\
  Cuando regreses, te besaré con cuidado, \\
  no te arrancaré el abrigo. \\
  Te contaré todos los nuevos chistes, \\
  porque hallé muchos desde entonces, \\
  y qué alegre te pondrás y te sonrojarás, \\
  y cómo mirarás al suelo, \\
  y nos reiremos a carcajadas, \\
  y nos oirán nuestros vecinos \\
  y hasta los taciturnos y austeros jornaleros \\
  que, entre sus sueños cansados y quebrados, \\
  empezarán a sonreirse también. \\
\end{verse}

Otoño

\newpage

\selectlanguage{english}

\noindent \textbf{Original Hungarian}

\selectlanguage{hungarian}

%% Olyan bolond vagy
%% szaladsz
%% akár a reggeli szél.
%% Még elüt valamelyik autó.
%% Pedig lesikáltam kis asztalomat
%% és most
%% tisztábban világít kenyerem enyhe fénye.
%% No gyere vissza, ha akarod
%% veszek takarót vaságyamra.
%% Egyszerű, szürke takarót.
%% Illik az
%% szegénységemhez, aki szeret téged
%% és az Úr is szereti nagyon
%% és engem is szeret az Úr
%% nem jön soha nagy fényességgel
%% Nem akarja, hogy elromoljanak
%% szemeim, akik
%% nagyon kívánnak látni téged.
%% És nagyon szépen néznek majd terád
%% ha visszajössz
%% vigyázva foglak megcsókolni,
%% nem tépem le rólad a kabátot
%% és elmondom mind a sok tréfát,
%% mert sokat kieszeltem azóta,
%% hogy te is örülj,
%% majd elpirulsz,
%% lenézel a földre és kacagunk
%% hangosan, hogy behallatszik szomszédunkba
%% a szótlan, komoly napszámosokhoz is behallik
%% és fáradt, összetört
%% álmukban majd elmosolyodnak ők is.

\settowidth{\versewidth}{tisztábban világít kenyerem enyhe fénye.}

\begin{verse}[\versewidth]
  Olyan\myfoot{arch. \emph{oly} `such' + deadj. adv.-forming \emph{-an}
  $\rightarrow$ `such a/an'} bolond\myfoot{adj. `silly/foolish'}
  vagy\myfoot{vb. \emph{van} `to be' + 2nd pers. sg. ind. pres. indef.} \\
  szaladsz\myfoot{vb. \emph{szalad} `to run' + 2nd
    pers. sg. ind. pres. indef. \emph{-sz}} \\
  akár\myfoot{lit. `just like'} a reggeli\myfoot{\emph{reggel}
    `morning' + adj.-forming \emph{-i}} szél.\myfoot{`wind'} \\
  Még\myfoot{adv. `moreover/later/sometime'}
  elüt\myfoot{vb. pref. \emph{el-} `away' + vb. \emph{üt} `to hit' +
    3rd pers. sg. ind. pres. indef.}
  valamelyik\myfoot{pron. pref. \emph{vala} `some-' +
    inter. \emph{myelik} `which/what' $\rightarrow$ `one/any of'}
  autó.\myfoot{`car'} \\
  Pedig\myfoot{conj. expressing contrast `whereas/but/although' or
    addition `and (by contrast)'}
  lesikáltam\myfoot{vb. pref. \emph{le-} completion +
    vb. \emph{sikál} `to scrub/scour (out)' + 1st
    pers. sg. ind. past.def. \emph{-am}}
  kis\myfoot{adj. `small'} asztalomat\myfoot{\emph{asztal} `table' +
    1st pers. sg. single poss. \emph{-(o)m} + acc. \emph{-(a)t}} \\
  és most\myfoot{adv. `now'} \\
  tisztábban\myfoot{adj. \emph{tiszta} `clean/clear' + comp. \emph{-bb} +
    deadj. adv.-forming \emph{-am} $\rightarrow$ `more
    clearly/cleanly'} világít\myfoot{\emph{világ}
    `light (arch.)/world' +
    caus. vb.-forming \emph{-ít} `to make smth ...-like' + 1st
    pers. sg. ind. pres. indef. `to make shine'}
  kenyerem\myfoot{\emph{kenyér} `bread' + 1st pers. sg. single
    poss. \emph{-em}}
  enyhe\myfoot{\emph{enyh} `light/mild/tender/faint' + 3rd
    pers. sg. single poss. \emph{-e} `of'} fénye.\myfoot{\emph{fény}
    `light' + 3rd pers. sg. single poss. \emph{-e}} \\
  No\myfoot{interj. `well/now'} gyere\myfoot{vb. \emph{jön} `to come' + 2nd
    pers. sg. ind. pres. indef.} vissza,\myfoot{adv. `back[wards]'}
  ha\myfoot{conf. `if' + cond. or `when/once'}
  akarod\myfoot{vb. \emph{akar} `to want' + 2nd
    pers. sg. ind. pres. def. \emph{-od}} \\
  veszek\myfoot{vb. \emph{vesz} `to take/grab/buy/get/consider/put on'
  + 1st pers. sg. ind. pres. indef. \emph{-ek}}
  takarót\myfoot{vb. \emph{takar} `to cover (protect)' +
    pres. part. \emph{-ó} $\rightarrow$
    `cover/blanket' + acc. \emph{-t}} vaságyamra.\myfoot{\emph{vas}
    `iron'+ \emph{ágy} `bed' + 1st pers. sg. single poss. \emph{-am} +
  all. \emph{-ra} `onto/on'} \\
  Egyszerű,\myfoot{\emph{egy} `one' + \emph{-szerű} `-like'
    $\rightarrow$ `plain/simple/unadorned/humble (< 1945)'}
  szürke\myfoot{adj. `grey'} takarót. \\
  Illik\myfoot{vb. `to suit/match/fit' + 3rd
    pers. sg. ind. pres. indef.} az \\
  szegénységemhez,\myfoot{\emph{szegény} `poor' + noun-forming
  \emph{-ség} (state of being) + 1st
  pers. sg. single poss. \emph{-em} + all. \emph{-hez} `to'}
  aki\myfoot{conj. `who'} szeret\myfoot{vb. `to love'+ 3rd
    pers. sg. ind. pres. indef.} téged\myfoot{pron. \emph{te} `you'
    sg. + acc.} \\
  és az Úr\myfoot{`Lord' (the Christian god)} is
  szereti\myfoot{vb. \emph{szeret} `to love' + 3rd
    pers. sg. ind. pres. indef. \emph{-i}}
  nagyon\myfoot{adj. \emph{nagy} `great' + adv.-forming \emph{-on}
    $\rightarrow$ `very (to a high degree)'} \\
  és engem\myfoot{pron. \emph{én} `I' + acc. $\rightarrow$ `me'} is
  szeret\myfoot{vb. `to love'+ 3rd pers. sg. ind. pres. indef.} az Úr \\
  nem jön\myfoot{vb. `to come' + 3rd pers. sg. ind. pres. indef.}
  soha\myfoot{adv. `never'} nagy\myfoot{adj. `great'}
  fényességgel\myfoot{\emph{fény} `light' + adj.-forming \emph{-es}
    `with/having' $\rightarrow$ `bright' + noun-forming
  \emph{-ség} (state of being) $\rightarrow$ `brightness' +
  instr. \emph{-vel} `with' (\emph{-gel} after consonant
  \emph{g}) $\rightarrow$ `with brightness'} \\
  Nem akarja,\myfoot{vb. `to want' + 3rd
    pers. sg. ind. pres. indef. \emph{-ja}} hogy
  elromoljanak\myfoot{perfect. vb. pref. \emph{el-} +
    vb. \emph{romlik} `to break down/deteriorate' + 3rd
    pers. pl. subj. indef. \emph{-janak}} \\
  szemeim,\myfoot{\emph{szem} `eye' + 1st
    pers. sg. mult. poss. \emph{-eim}} akik\myfoot{pron. `who' +
    pl. \emph{-k}} \\
  nagyon\myfoot{adj. \emph{nagy} `great' + adv.-forming \emph{-on}
    $\rightarrow$ `very (to a high degree)'}
  kívánnak\myfoot{vb. \emph{kíván} `to wish' + 3rd
    pers. pl. ind. pres. indef. \emph{-nak}}
  látni\myfoot{vb. \emph{lát} `to see' + inf. \emph{-ni}}
  téged.\myfoot{pron. \emph{te} `you' sg. + acc.} \\
  És nagyon\myfoot{adj. \emph{nagy} `great' + adv.-forming \emph{-on}
    $\rightarrow$ `very (to a high degree)'} szépen\myfoot{adj. \emph{szép}
  `beautiful'} néznek\myfoot{vb. \emph{néznek} `to look at smth' + 3rd
  pers. pl. ind. pres. indef. \emph{-nek}}
  majd\myfoot{adv. `sometime/later'} terád\myfoot{pron. sg. \emph{te}
    `you' (here emph.) + pron. \emph{rá} + 2nd pers. sg. `on/of you'} \\
  ha\myfoot{conf. `if' + cond. or `when/once'}
  visszajössz\myfoot{vb. pref. \emph{vissza-} `back-' + vb. \emph{jön}
  `to come'+ 2nd pers. sg. ind. pres. indef. \emph{-sz}} \\
  vigyázva\myfoot{vb. \emph{vigyáz} `to protect/pay attention' +
  adv. part. \emph{-va} result of action or simultaneous actions}
  foglak\myfoot{vb. aux. \emph{fog} `to will' + 1st
    pers. sg. ind. pres. 2nd obj. \emph{-lak} `you'}
  megcsókolni,\myfoot{perfect. vb. pref. \emph{meg-} +
    vb. \emph{csókol} `to kiss' + inf. \emph{-ni}} \\
  nem tépem\myfoot{vb. \emph{tép} `to tear/rend' + 1st
    pers. sg. ind. pres. def. \emph{-em}} le\myfoot{comp. `down'}
  rólad\myfoot{pron. \emph{róla} `about/off her' + 2nd pers. sg.}
  a kabátot\myfoot{\emph{kabát} `coat' + acc. \emph{-(o)t}} \\
  és elmondom\myfoot{vb. \emph{elmond} `to tell' + 1st
    pers. sg. ind. pres. def. \emph{-om}}
  mind\myfoot{pron. sg. `all/each of'} a sok\myfoot{adj. `many'}
  tréfát,\myfoot{\emph{tréfá} `joke' + acc. \emph{-(a)t}} \\
  mert\myfoot{conj. `because'} sokat\myfoot{\emph{sok} `many' +
    acc. \emph{-at}} kieszeltem\myfoot{vb. \emph{kieszel} `to devise'
    + 1st pers. sg. past. \emph{-tem}} azóta,\myfoot{art. \emph{az}
    `that' + \emph{óta} `since' $\rightarrow$ `since then'} \\
  hogy te\myfoot{pron. sg. `you'} is\myfoot{doublet of \emph{és}
    `and', adv. `also'} örülj,\myfoot{vb. \emph{örül} `to rejoice' +
    2nd pers. sg. subj. indef. \emph{-j}} \\
  majd\myfoot{adv. `sometime/later'}
  elpirulsz,\myfoot{perfect. vb. pref. \emph{el-} + vb. \emph{pirul}
    `to blush' + 2nd pers. sg. ind. pres. indef. \emph{-sz}} \\
  lenézel\myfoot{vb. `to look down'+ 2nd
    pers. sg. ind. pres. indef. \emph{-el}} a
  földre\myfoot{\emph{föld} `earth/ground' + subl. \emph{-re} `onto'}
  és kacagunk\myfoot{vb. \emph{kacag} `to laugh (a lot)' + 1st
    pers. pl. ind. pres. indef. \emph{-unk}} \\
  hangosan,\myfoot{adv. `loudly'} hogy
  behallatszik\myfoot{vb. pref. \emph{be-} `inwardly' +
    vb. \emph{hall} `to hear' + caus. \emph{-at}
    + vb.-forming \emph{-szik} $\rightarrow$ `to cause to sound in'}
  szomszédunkba\myfoot{\emph{szomszéd} `neighbour' + 1st
    pers. pl. single poss. \emph{-unk} + illative \emph{-ba} `into'} \\
  a szótlan,\myfoot{\emph{szó} `word/voice' + privative adj.-forming
    \emph{-tlan} `-less'} komoly\myfoot{adj. `severe/austere'}
  napszámosokhoz\myfoot{\emph{napszámos} `day labourer' +
    pl. \emph{-(o)k} + allat. `\emph{-hoz} `towards'} is
  behallik\myfoot{syn. \emph{behallatszik}} \\
  és fáradt,\myfoot{vb. \emph{fárad} `to tire' +
    past. part. \emph{-t}} összetört\myfoot{\emph{össze} `together' +
    vb. \emph{tör} `to break' + past. part. \emph{-t}} \\
  álmukban\myfoot{\emph{álom} `dream'+ 3rd pers. pl. single
    poss. \emph{-uk} + innes. \emph{-ban} `in'}
  majd\myfoot{adv. `sometime/later'}
  elmosolyodnak\myfoot{vb. inf. \emph{elmosolyodni}
  `to start to smile' + 3rd pers. ind. pres. indef.}
  ők\myfoot{pron. `they'} is.\myfoot{doublet of \emph{és} `and',
    adv. `also'} \\
\end{verse}

Ősze

\bigskip

\selectlanguage{english}

% Note: The edition by \emph{Szépirodalmi könyvkiadó}, page~169, misses
% two verses and a conjunction.

%% =====================================================================

\chapter{\oldstylenums{1934}}

\selectlanguage{english}

\section{Consciousness (\emph{Eszmélet})}

\poemtitle*{I}

\settowidth{\versewidth}{Overnight they alighted on the trees}

\begin{verse}[\versewidth]
  Dawn unbinds sky from earth \\
  and upon its clear, soft word \\
  beetles and children \\
  spin forth into the daylight; \\
  there is no moisture in the air, \\
  the bright levity floats! \\
  Overnight they alighted on the trees \\
  like small butterflies, the leaves. \\
\end{verse}

\noindent \textbf{French}

\selectlanguage{french}

\settowidth{\versewidth}{Avec la nuit, elles se posèrent sur les arbres}

\begin{verse}[\versewidth]
  L'aube détache le ciel de la terre \\
  et, à sa voix claire et douce, \\
  scarabées et enfants virevoltent \\
  en entrant sur la scène du jour; \\
  l'air n'est pas humide, \\
  la brillante légèreté flotte! \\
  Avec la nuit, elles se posèrent sur les arbres \\
  comme de petits papillons, les feuilles. \\
\end{verse}

\newpage

\selectlanguage{english}

\noindent \textbf{Spanish}

\selectlanguage{spanish}

\settowidth{\versewidth}{Con la noche se posaron en los árboles}

\begin{verse}[\versewidth]
  El amanecer desata el cielo de la tierra \\
  y, al son de su voz clara y suave, \\
  escarabajos y niños entran \\
  y giran en la luz del día; \\
  el aire no es húmedo, \\
  ¡la brillante levedad flota! \\
  Con la noche se posaron en los árboles \\
  como pequeñas mariposas, las ojas. \\
\end{verse}

\selectlanguage{english}

\noindent \textbf{Original Hungarian}

\selectlanguage{hungarian}

%% Földtől eloldja az eget
%% a hajnal s tiszta, lágy szavára
%% a bogarak, a gyerekek
%% kipörögnek a napvilágra;
%% a levegőben semmi pára,
%% a csilló könnyűség lebeg!
%% Az éjjel rászálltak a fákra,
%% mint kis lepkék, a levelek.

\settowidth{\versewidth}{a hajnal s tiszta, lágy szavára}

\begin{verse}[\versewidth]
  Földtől\myfoot{\emph{föld} `earth' +
  abl. \emph{-től} `without/from'}
  eloldja\myfoot{vb. pref. \emph{el-} `away' + vb. \emph{old} `to
  unbind' + 3rd pers. sg. ind. pres. def. \emph{-ja}} az
  eget\myfoot{\emph{ég} `sky' + acc. \emph{-et} $\rightarrow$
  \emph{eget}} \\
  a hajnal\myfoot{`dawn'} s\myfoot{poetic abbrv. \emph{es} `and'
  (does not count as a syllable)} tiszta\myfoot{adj. `bright'},
  lágy\myfoot{adj. `soft'} szavára\myfoot{\emph{szó} `voice',
  `word' + 3rd pers. sg. single poss. \emph{-a} $\rightarrow$
  \emph{szava} + subl. \emph{-ra} `by' $\rightarrow$
  \emph{szavára}} \\

  a bogarak,\myfoot{\emph{bogár} `beetle' + pl. \emph{-(a)k}
  $\rightarrow$ \emph{bogarak}} a gyerekek \\
  kipörögnek\myfoot{vb. \emph{kipörög} `to spin' + 3rd
  pers. pl. ind. pres. indef. \emph{-nek}}
  a napvilágra\myfoot{\emph{nap} `sun/day' + \emph{világ}
  `world' + subl. \emph{-ra} `onto'}; \\
  a levegőben\myfoot{\emph{levegő} `air' +
  innes. \emph{-ben} `in'} semmi pára\myfoot{`moisture/mist'}, \\
  a csilló\myfoot{vb. \emph{csillog} `to glitter' +
  adj.-forming \emph{-o} $\rightarrow$
  \emph{csilló}; usually \emph{csillogó}}
  könnyűség\myfoot{\emph{könnyű} `simplicity/easiness' + noun-forming
  \emph{-ség} (state of being)}
  lebeg!\myfoot{vb. \emph{lebeg} `to float (in the air)', 3rd
  pers. sg. ind. pres. indef.} \\
  Az éjjel\myfoot{poetic \emph{éj} `night' + instr. \emph{-vel}
  `with' $\rightarrow$ \emph{éjjel}} rászálltak\myfoot{pointed
  action \emph{rá-} `on' + vb. \emph{száll} `to fly/land/to come down'
  + 3rd pers. sg. past indef. \emph{-tak}} a fákra,\myfoot{\emph{fa} `tree' +
  pl. \emph{-k} + subl. \emph{-ra} `onto' $\rightarrow$ \emph{fákra}} \\
  mint\myfoot{`like'} kis\myfoot{`small'} lepkék,\myfoot{\emph{lepke}
  `butterfly' + pl. \emph{-k} $\rightarrow$ \emph{lepkék}} a
  levelek.\myfoot{\emph{level} `leaf' + pl. \emph{-(e)k}} \\
\end{verse}

\selectlanguage{english}

\noindent Chiastic structure: \\
1-4: e-e(8)/á-[r]a(9)/e-e(8)/á-[r]a(9) \\
5-8: á-[r]a(9)/e-e(8)/á-[r]a(9)/e-e(8)

\newpage

\poemtitle*{II}

\selectlanguage{english}

\settowidth{\versewidth}{Now my dream comes down as a blur into my members}

\begin{verse}[\versewidth]
  I saw paintings daubed with blue, \\
  red, yellow in my dreams \\
  and felt that everything was just so ---\\
  not one speck of dust flying madly about. \\
  Now my dream comes down as a blur into my members, \\
  and the iron world is the order. \\
  During the day a moon rises inwardly and, \\
  when the night is out, a sun shines herewithin. \\
\end{verse}

\bigskip

\noindent \textbf{Cursory remark}

\medskip

Verses 1-4 are about the night, verses 5-7 transition to the day, and
verse~8 brings us back to the night --- this cyclic structure
supporting the mismatched, recurring alternance of days and
nights.

\bigskip

\noindent \textbf{French}

\selectlanguage{french}

\settowidth{\versewidth}{Maintenant mon rêve descend, estompé, dans mes membres}

\begin{verse}[\versewidth]
  J'ai vu des tableaux barbouillés de bleu, \\
  rouge et jaune dans mes rêves \\
  et je sentis que tout était en ordre --- \\
  pas un seul grain de poussière qui vole folement. \\
  Maintenant mon rêve descend, estompé, dans mes membres, \\
  et le monde de fer est l'ordre. \\
  Avec le jour, une lune point en moi et, \\
  à la nuit tombée, un soleil brille ci-dedans. \\
\end{verse}

\bigskip

\noindent \textbf{Spanish}

\selectlanguage{spanish}

\settowidth{\versewidth}{Ahora mi sueño se pone, esfumado, en mis miembros,}

\begin{verse}[\versewidth]
  Vi cuadros embadurnados con azul, \\
  rojo y amarillo en mis sueños \\
  y sentí que todo estaba en orden --- \\
  ni una sola mota de polvo que vuela locamente. \\
  Ahora mi sueño se pone, esfumado, en mis miembros, \\
  y el mundo de hierro es el orden. \\
  Con el día, una luna sale en mí y, \\
  al llegar la noche, un sol brilla aquí dentro. \\
\end{verse}

\newpage

\selectlanguage{english}

\noindent \textbf{Original Hungarian}

\selectlanguage{hungarian}

%% Kék, piros, sárga, összekent
%% képeket láttam álmaimban
%% és úgy éreztem, ez a rend ---
%% egy szalló porszem el nem hibbant.
%% Most homályként száll tagjaimban
%% álmom s a vas világ a rend.
%% Nappal hold kél bennem s ha kinn van
%% az éj --- egy nap süt idebent.

\settowidth{\versewidth}{Nappal hold kél bennem s ha kinn van}

\begin{verse}[\versewidth]
  Kék, piros, sárga, összekent\myfoot{\emph{össze} `together' +
  \emph{ken} `smear' $\rightarrow$ \emph{összeken} + 3rd
  pers. sg. past indef. \emph{-t}} \\
  képeket\myfoot{\emph{kép} `picture' + pl. \emph{-(e)k} +
  acc. \emph{-(e)t}} láttam\myfoot{vb. \emph{lát} `to see' +
  short base \emph{-t} + 1st pers. sg. past def. \emph{-tam}}
  álmaimban\myfoot{\emph{álom} `dream' +
  1st pers. sg. poss. \emph{-a} + poss. pl. \emph{-i} + 1st
  pers. sg. pers. \emph{-m} + innes. \emph{-ban} `in' $\rightarrow$
  \emph{álmaimban}} \\
  és úgy éreztem,\myfoot{vb. \emph{érez} `to feel', 1st
  pers. sg. past def.} ez a rend\myfoot{`order'} --- \\
  egy szalló\myfoot{vb. \emph{száll} `to fly' +
  pres. part. \emph{-ó}} porszem\myfoot{`a speck of dust'}
  el\myfoot{\emph{el nem} + past. part.; the verbal prefix \emph{el}
  emphasises the completion of the action} nem
  hibbant.\myfoot{vb. \emph{hibban}
  `to become imbalanced', arch. + past part. \emph{-t}} \\
  Most homályként\myfoot{\emph{homály} `blur' +
  essive formal \emph{-ként} `as'} száll\myfoot{vb. `to land/to come
  down', 3rd pers. sg. ind. pres.} tagjaimban\myfoot{\emph{tag}
  `member' + 1st pers. sg. poss. \emph{-(j)a} +
  poss. pl. \emph{-i} + 1st pers. sg. pers. \emph{-m} +
  innes. \emph{-ban} `in' $\rightarrow$ \emph{tagjaimban}} \\
  álmom s a vas\myfoot{`iron'} világ a rend. \\
  Nappal\myfoot{\emph{nap} `day' + instr. \emph{-val}
  `with' $\rightarrow$ \emph{nappal}} hold\myfoot{`moon'}
  kél\myfoot{vb. `to rise', 3rd pers. sg. ind. pres. def.,
  arch. alt. \emph{kel}} bennem\myfoot{pron. \emph{benne} `inside
  smth/smb' + 1st pers. sg. pers. \emph{-m}} s ha\myfoot{`when'}
  kinn\myfoot{`outside'} van  \\
  az éj\myfoot{`night'} --- egy nap süt\myfoot{vb. `to
  roast/shine', 3rd pers. sg. ind. pres. indef.}
  idebent.\myfoot{adv. \emph{ide} `here' + \emph{bent} `within'} \\
\end{verse}

\selectlanguage{english}

\noindent Chiastic structure: \\
1-4: e-ent(8)/ai-an(9)/a-rend(8)/i-an(9) \\
5-8: i-an(9)/a-rend(8)/ai-an(9)/e-ent(8)
%% 1-4: e-ent(8)/aim-ban(9)/a-rend(8)/i-an(9) \\
%% 5-8: aim-ban(9)/a-rend(8)/[a]i-an(9)/e-ent(8)

\newpage

\selectlanguage{english}

\poemtitle*{III}

\settowidth{\versewidth}{however clever, the cat cannot catch at once}

\begin{verse}[\versewidth]
  I am thin, at times I only eat bread; \\
  amidst those idle and blithering souls, \\
  I seek in vain more certainty, like the die. \\
  No chuck roast reaches my mouth \\
  while I cuddle a small child to my heart --- \\
  however clever, the cat cannot catch at once \\
  the mouse outdoors and the one indoors. \\
\end{verse}

\bigskip

\noindent \textbf{Cursory remark}

\medskip

J cannot pursue his art and, at the same time, keep a job that would
put real food on his table. Cuddling a small child to his heart is his
metaphor for the creation of poetry. J is the cat in verses 6-7.

\bigskip

\noindent \textbf{French}

\selectlanguage{french}

\settowidth{\versewidth}{aussi fûté soit-il, le chat ne peut attrapper d'un coup}

\begin{verse}[\versewidth]
  Je suis maigre, parfois je ne mange que du pain; \\
  entouré par ces âmes oisives et bavardes, \\
  je cherche en vain plus de certitude, comme le dé. \\
  Aucun rôti de palette ne trouve ma bouche \\
  quand j'étreins un enfant sur mon cœur --- \\
  aussi fûté soit-il, le chat ne peut attrapper d'un coup \\
  la souris dehors et celle dedans. \\
\end{verse}

\bigskip

\noindent \textbf{Spanish}

\selectlanguage{spanish}

\settowidth{\versewidth}{por muy astuto que sea, el gato no puede pillar a la vez}

\begin{verse}[\versewidth]
  Estoy flaco, a veces sólo como pan; \\
  rodeado por esas almas ociosas y facundas, \\
  busco en vano más certidumbre, como el dado. \\
  Ninguna paleta asada halla mi boca \\
  cuando ciño a un niño sobre mi corazón --- \\
  por muy astuto que sea, el gato no puede pillar a la vez \\
  el ratón de afuera y el de adentro. \\
\end{verse}

\newpage

\selectlanguage{english}

\noindent \textbf{Original Hungarian}

%% Sóvany vagyok, csak kenyeret
%% eszem néha, e léha, locska
%% lelkek közt ingyen keresek
%% bizonyosabbat, mint a kocska.
%% Nem dörgölődzik sült lopcska
%% számhoz s szivemhez kisgyerek ---
%% ügyeskedhet, nem fog a macska
%% egyszerre kint s bent egeret.

\settowidth{\versewidth}{számhoz s szivemhez kisgyerek ---}

\begin{verse}[\versewidth]
  Sóvany\myfoot{`slim/thin', fig. `poor'} vagyok, csak
  kenyeret\myfoot{\emph{kenyér} `bread' + acc. \emph{-(e)t}
  $\rightarrow$ \emph{kenyeret}} \\
  eszem\myfoot{vb. \emph{esz} `to eat', rare without verbal
  pref., inf. \emph{enni} + 1st pers. sg. ind. pres. def. \emph{-em}
  $\rightarrow$ \emph{eszem}} néha,\myfoot{adv. `sometimes'} e\myfoot{poetic
  abbrv. \emph{eme/ez} `this'} léha,\myfoot{`improvident/idler'}
  locska\myfoot{adj. `blethering'} \\
  lelkek\myfoot{\emph{lélek} `soul' +
  pl. \emph{-(e)k} $\rightarrow$ \emph{lelkek}}
  közt\myfoot{post. `between/amid' from \emph{között}: \emph{köz}
  `interval' + loc. \emph{-ött}} ingyen\myfoot{fig. arch. `in
  vain'; usually `free of charge'} keresek\myfoot{vb.
  \emph{keres} `to seek' + 1st pers. sg. ind. pres. indef.} \\
  bizonyosabbat,\myfoot{adj. \emph{bizonyos} `certain' +
  comp. \emph{-abb} `more' + acc. \emph{-t}} mint\myfoot{`like'} a
  kocska.\myfoot{`a die'} \\
  Nem dörgölődzik\myfoot{vb. `to press against', 3rd
  pers. sg. ind. pres. indef.} sült\myfoot{vb. \emph{sül} `to
  roast'  + past part. \emph{-t}} lopcska\myfoot{`chuck'
  (shoulder meat cut, probably pork)} \\
  számhoz\myfoot{\emph{száj} `mouth' + 1st
  pers. sg. single poss. \emph{-m} $\rightarrow$ \emph{szám} +
  allat. \emph{-hoz} `towards' $\rightarrow$ \emph{számhoz}} s
  szivemhez\myfoot{\emph{szivem} `heart' +
  allat. \emph{-hez} `towards'} kisgyerek\myfoot{\emph{kis} `small' +
  \emph{gyerek} `child'} --- \\
  ügyeskedhet,\myfoot{vb. \emph{ügyesked} `to be clever' +
  3rd pers. sg. ind. pres. potential \emph{-het}} nem fog\myfoot{vb.
  `to hold', 3rd pers. sg. pers. indef.} a macska\myfoot{`cat'} \\
  egyszerre\myfoot{adv. \emph{egyszer} `at once' +
  abl. \emph{-re} `onto'} kint\myfoot{adv. `outside'} s
  bent\myfoot{adv. `inside'} egeret.\myfoot{\emph{egér}
  `mouse' + acc. \emph{-(e)t} $\rightarrow$ \emph{egeret}} \\
\end{verse}

\selectlanguage{english}

\noindent Chiastic structure: \\
1-4: e-ret(8)/o-cska(9)/e-ek(8)/o-cska(9) \\
5-8: o-cska(8)/e-ek(8)/a-cska(8)/e-ret(8) ---~same length

\newpage

\selectlanguage{english}

\poemtitle*{IV}

\settowidth{\versewidth}{and so everything is determined.}

\begin{verse}[\versewidth]
  Just like a pile of logs, \\
  the world lies in a jumble; \\
  each thing presses, weighs on, \\
  holds fast to the next, \\
  and so everything is determined. \\
  Only what is not has a shrub, \\
  only that will be that flower; \\
  what is falls to pieces. \\
\end{verse}

\bigskip

\noindent \textbf{Cursory remark}

\medskip

`To be or not to be' is usually construed as the dichotomy between
life and death. J tells us that not-to-be is potential beauty, whereas
to-be is actual death.

\bigskip

\noindent \textbf{French}

\selectlanguage{french}

\settowidth{\versewidth}{Seul ce qui n'est pas relève de l'arbuste,}

\begin{verse}[\versewidth]
  Tout comme un tas de bûches, \\
  le monde gît en vrac; \\
  chaque chose presse, pèse, \\
  s'arrime à l'autre, \\
  et ainsi tout est déterminé. \\
  Seul ce qui n'est pas a un arbuste, \\
  seul cela sera cette fleur; \\
  ce qui est tombera en pièces. \\
\end{verse}

\bigskip

\noindent \textbf{Spanish}

\selectlanguage{spanish}

\settowidth{\versewidth}{Sólo lo que no es tiene un arbusto,}

\begin{verse}[\versewidth]
  Así como un montón de leña, \\
  el mundo yace a granel; \\
  cada cosa aprieta, pesa, \\
  se arrima a la próxima, \\
  y así todo está determinado. \\
  Sólo lo que no es tiene un arbusto, \\
  sólo eso será esta flor; \\
  lo que es caera a pedazos. \\
\end{verse}

\newpage

\selectlanguage{english}

\noindent \textbf{Original Hungarian}

\selectlanguage{hungarian}

%% Akár egy halom hasított fa,
%% hever egymáson a világ,
%% szorítjanyomja, összefogja
%% egyík dolog a másikát
%% s így mindenik determinált.
%% Csak ami nincs, annak van bokra,
%% csak ami lesz, az a virág,
%% ami van, széthull darabokra.

\settowidth{\versewidth}{Csak ami nincs, annak van bokra,}

\begin{verse}[\versewidth]
  Akár\myfoot{conj. literary `just like'} egy
  halom\myfoot{`pile/heap'} hasított\myfoot{vb.
  \emph{hasít} `to cleave/split' +
  past. part. \emph{-ott} } fa,\myfoot{`tree'} \\
  hever\myfoot{vb. `to lie', 3rd pers. sg. ind. pres. indef.}
  egymáson\myfoot{pron. \emph{egymás} `each other/one
  another' + super. \emph{-on} `on'} a világ,\myfoot{`world'} \\
  szorítja,\myfoot{vb. \emph{szorít} `to press/constrict', inf. \emph{szorítani} + 3rd
  pers. sg. ind. pres. def. \emph{-ja}} nyomja,\myfoot{vb.
  \emph{nyom} `to weigh' + 3rd pers. sg. ind. pres. def. \emph{-ja}}
  összefogja\myfoot{\emph{össze} `together' + vb. \emph{fog}
  `to hold' + 3rd pers. sg. ind. pres. def. \emph{-ja}} \\
  egyik\myfoot{\emph{egy} `one' + 3rd
  pers. pl. mult. poss. \emph{-ik} `of' } dolog\myfoot{`thing'} a
  másikát\myfoot{adj. \emph{más} `other' + noun-forming \emph{-ik}
  $\rightarrow$ `another' + 3rd pers. sg. single poss. \emph{-a} +
  acc. \emph{-t} $\rightarrow$ \emph{-át}} \\
  s így\myfoot{`just so'} mindenik\myfoot{adv. \emph{minden}
  `everyone/everything'} determinált.\myfoot{Latin \emph{determinare} +
  vb.-forming \emph{-ál} `to determine' + past part. \emph{-t}} \\
  Csak\myfoot{adv. `only'} ami\myfoot{rel. pron. `which/that'} nincs,
  annak\myfoot{art. \emph{az} `that' + attr. poss. \emph{-nak}
  $\rightarrow$ \emph{annak}}
  van bokra,\myfoot{\emph{bokor} `shrub/bush' + 3rd
  pers. sg. single poss. \emph{-a}} \\
  csak\myfoot{adv. `only'} ami lesz,\myfoot{vb. `to
  become', 3rd pers. sg. ind. fut. indef.} az a virág, \\
  ami van, széthull\myfoot{vb. `to fall apart',
  3rd pers. sg. ind. pres. indef.}
  darabokra.\myfoot{adv. \emph{darab} `piece/chunk' +
  pl. \emph{-ok} + subl. \emph{-ra} `into'} \\
\end{verse}

\selectlanguage{english}

\noindent Chiastic structure: \\
1-4: o-a(9)/i-á(8)/o-a(9)/i-á(8) \\
5-8: i-á(8)/o-a(9)/i-á(8)/o-a(9)

\newpage

\selectlanguage{english}

\poemtitle*{V}

\settowidth{\versewidth}{stubbornly leaping at the bright, dewy coal.}

\begin{verse}[\versewidth]
  At the freight train station, \\
  I lay down just so behind the tree's butt, \\
  like a chunk of silence; \\
  a grey weed reached my mouth, \\
  raw, strange-sweet. \\
  Playing dead, I watched the guard, sensing what, \\
  and his shadow on the quiet wagons \\
  stubbornly leaping at the bright, dewy coal. \\
\end{verse}

\bigskip

\noindent \textbf{Cursory remark}

\medskip

This is likely a memory from J's childhood, when his family was so
poor that he would occasionally steal lumps of coal from the wagons at
the freight train station nearby his home.

\bigskip

\noindent \textbf{French}

\selectlanguage{french}

\settowidth{\versewidth}{s'entêtait à bondir sur les charbons reluisants, couverts de rosée.}

\begin{verse}[\versewidth]
  À la gare de fret, \\
  je m'étalai ainsi derrière le pied de l'arbre, \\
  comme une masse de silence; \\
  une herbe grise atteignit ma bouche, \\
  crue, étrange-sucrée. \\
  Faisant le mort, je regardais le garde, ressentant quoi, \\
  et son ombre qui, sur les wagons silencieux, \\
  s'entêtait à bondir sur les charbons reluisants, couverts de rosée. \\
\end{verse}

\bigskip

\noindent \textbf{Spanish}

\selectlanguage{spanish}

\settowidth{\versewidth}{que se empeñaba a saltar sobre los relucientes, rociados carbones.}

\begin{verse}[\versewidth]
  A la estación de tren de carga, \\
  me tumbé así detrás del pie del árbol, \\
  como una masa de silencio; \\
  una hierba gris alcanzó mi boca, \\
  cruda, extraña-dulce. \\
  Haciendome el muerto, miraba al guardia, que sentía qué, \\
  y, sobre los quietos vagones, a su sombra \\
  que se empeñaba a saltar sobre los relucientes, rociados carbones. \\
\end{verse}

\newpage

\selectlanguage{english}

\noindent \textbf{Original Hungarian}

\selectlanguage{hungarian}

%% A teherpályaudvaron
%% úgy lapultam a fa tövéhez,
%% mint egy darab csönd; szürke gyom
%% ért számhoz, nyers, különös-édes.
%% Holtan lestem az őrt, mit érez,
%% s a hallgatag vagónokon
%% árnyát, mely ráugrott a fényes,
%% harmatos szénre konokon.

\settowidth{\versewidth}{Csak ami nincs, annak van bokra,}

\begin{verse}[\versewidth]
  A teherpályaudvaron\myfoot{\emph{teher} `freight' + \emph{pálya}
  `track' + \emph{udvar} `courtyard' + super. \emph{-on} `on'} \\
  úgy\myfoot{adv. `so (in this/that manner)'}
  lapultam\myfoot{vb. \emph{lapul} `to lie low'
  + 1st pers. sg. past indef. \emph{-tam}} a
  fa\myfoot{`tree'} tövéhez,\myfoot{\emph{tö} `base of the stem
  of a plant' + 3rd pers. sg. single poss. \emph{-e} `of'
  $\rightarrow$ \emph{töve} + allat. \emph{-hez} `towards'
  $\rightarrow$ \emph{tövéhez}} \\
  mint\myfoot{`like'} egy darab\myfoot{`piece'}
  csönd\myfoot{`silence', alt. \emph{csend}};
  szürke\myfoot{adj. `grey'} gyom\myfoot{`weed'} \\
  ért\myfoot{vb. \emph{ért} `to reach', arch. `to touch',
  fig. `to understand', 3rd pers. sg. past indef.}
  számhoz,\myfoot{\emph{száj} `mouth' + 1st
  pers. sg. single poss. \emph{-m} $\rightarrow$ \emph{szám} +
  allat. `towards' \emph{-hoz} $\rightarrow$ \emph{számhoz}}
  nyers,\myfoot{adj. `raw'}
  különös\myfoot{`strange'}-édes.\myfoot{`sweet'} \\
  Holtan\myfoot{\emph{holt} `dead' + adv.-forming \emph{-an}}
  lestem\myfoot{vb. \emph{les} `to spy' + 1st
  pers. sg. past indef. \emph{-tem}} az őrt,\myfoot{\emph{őr} `guard' +
  acc. \emph{-t}} mit\myfoot{inter. `what'}
  érez,\myfoot{vb. `to feel/smell/taste',
  3rd pers. sg. ind. pres. indef.} \\
  s a hallgatag\myfoot{vb. \emph{hallgat} `to remain silent/to
  listen' + adj.-forming \emph{-ag}} vagónokon\myfoot{\emph{vagón}
  `wagon' + pl. \emph{-ok} + super. \emph{-on} `on'} \\
  árnyát,\myfoot{\emph{árny} `shadow/ghost' + 3rd pers. sg. single
  poss. \emph{-a} + acc. \emph{-t} $\rightarrow$ \emph{-át}}
  mely\myfoot{literary, `which/that'} ráugrott\myfoot{pointed
  action \emph{ra-} `at' + vb. \emph{ugrik} `to leap/jump' + 3rd
  pers. sg. ind. past. indef. \emph{-(o)tt} $\rightarrow$ \emph{ugrott}} a
  fényes,\myfoot{\emph{fény} `light/glitter' + adj.-forming \emph{-es}
  $\rightarrow$ `bright'} \\
  harmatos\myfoot{adj. `dewy'} szénre\myfoot{\emph{szén}
  `coal' + subl. \emph{-re} `onto'}
  konokon.\myfoot{adj. \emph{konok} `stubborn' +
  super. \emph{-on} `on (that manner)'} \\
\end{verse}

\selectlanguage{english}

\noindent Chiastic structure: \\
1-4: a-o(8)/é-e(9)/e-o(8)/é-e(9) \\
5-8: é-e(9)/o-o(8)/é-e(9)/o-o(8)

\newpage

\selectlanguage{english}

\poemtitle*{VI}

\settowidth{\versewidth}{Your wound is the world --- on fire, burning up ---}

\begin{verse}[\versewidth]
  Behold the anguish inside, \\
  yet the explanation lies outside. \\
  Your wound is the world --- on fire, burning up --- \\
  and you feel your soul, the fever. \\
  You are captive as long as your heart rebels --- \\
  so you will be free if it indulges \\
  not building for yourself a house \\
  where a landlord settles in. \\
\end{verse}

\bigskip

\noindent \textbf{Cursory remark}

\medskip

We are not captives of the world, but of our own heart, which invited
the world (the landlord) in for us to contend with. The inner struggle
is felt as anguish and yields a feverish soul.

\bigskip

\noindent \textbf{French}

\selectlanguage{french}

\settowidth{\versewidth}{Ta blessure est le monde --- en feu, s'échauffant ---}

\begin{verse}[\versewidth]
  Voici le tourment intérieur, \\
  pourtant l'explication se trouve à l'extérieur. \\
  Ta blessure est le monde --- en feu, s'échauffant --- \\
  et tu sens ton âme, la fièvre. \\
  Tu es captif tant que ton cœur se révolte --- \\
  ainsi tu seras libre s'il se complait \\
  à ne pas bâtir pour toi une maison \\
  où un propriétaire vient demeurer. \\
\end{verse}

\bigskip

\noindent \textbf{Spanish}

\selectlanguage{spanish}

\settowidth{\versewidth}{Tu herida es el mundo --- en llamas, ardiendo ---}

\begin{verse}[\versewidth]
  He aquí el tormento interior, \\
  aunque la explicación yace afuera. \\
  Tu herida es el mundo --- en llamas, caldeando --- \\
  y tu sientes tú alma, la fiebre. \\
  Quedas cautivo mientras tu corazón se rebela --- \\
  así estarás libre si él deja de complacerse \\
  en construir par ti una casa \\
  donde un dueño se asenta. \\
\end{verse}

\newpage

\selectlanguage{english}

\noindent \textbf{Original Hungarian}

\selectlanguage{hungarian}

%% Im itt a szenvedés belül,
%% ám ott kívül a magyarázat.
%% Sebed a világ --- ég, hevül
%% s te lelkedet érzed, a lázat.
%% Rab vagy, amíg a szíved lázad ---
%% úgy szabadulsz, ha kényedül
%% nem raksz magadnak olyan házat,
%% melybe háziúr települ.

\settowidth{\versewidth}{nem raksz magadnak olyan házat,}

\begin{verse}[\versewidth]
  Im\myfoot{literary, abbrv. \emph{íme} `here is/behold'}
  itt a szenvedés\myfoot{vb. \emph{szenved} `to suffer'
  + noun-forming \emph{-és}} belül,\myfoot{`inside'} \\
  ám\myfoot{poetic, interj. contrad. `but'} ott
  kívül\myfoot{adv. `outdoor/outside'} a
  magyarázat.\myfoot{vb. \emph{magyaráz} `to explain' +
  noun-forming \emph{-at}} \\
  Sebed\myfoot{\emph{seb} `wound' + 2nd
  pers. sg. single poss. \emph{-ed}} a világ --- ég,\myfoot{vb.
  `to be on fire', 3rd pers. sg. ind. pres. indef.}
  hevül\myfoot{vb. `to grow hot', 3rd
  pers. sg. ind. pres. indef.} \\
  s te\myfoot{inform. sg. `you'} lelkedet\myfoot{\emph{lélek} `soul' + 2nd
  pers. sg. single poss. \emph{-ed} $\rightarrow$ \emph{lelked} +
  acc. \emph{-(e)t}} érzed,\myfoot{vb. \emph{érez} `to feel', 2nd
  pers. sg. ind. pres. def.} a lázat.\myfoot{\emph{láz} `fever' +
  acc. \emph{-(a)t}} \\
  Rab\myfoot{`captive'} vagy, amíg\myfoot{adv. `as long as';
  `until' if followed by negated verb} a
  szíved\myfoot{\emph{szíved} `heart' + 2nd pers. sg. single
  poss. \emph{-ed}} lázad\myfoot{vb. \emph{to rebel}, 3rd
  pers. sg. ind. pres. indef.} --- \\
  úgy\myfoot{\emph{úgy ... ha ...} `then ... if ...'}
  szabadulsz,\myfoot{\emph{szabad} `free' + vb.-forming \emph{-ul}
  + 2nd pers. sg. ind. pres. indef. \emph{-sz}} ha\myfoot{`if'}
  kényedül\myfoot{literary, vb. `to indulge', 3rd
  pers. sg. ind. pres. indef.} \\
  nem raksz\myfoot{vb. \emph{rak} `to set up/to build', 2rd
  pers. sg. ind. pres. indef. \emph{-sz}}
  magadnak\myfoot{\emph{maga} `oneself' + 2nd pers. sg. single
  poss. \emph{-d} + dat. \emph{-nak}}
  olyan\myfoot{pron. `such/that kind of thing'}
  házat,\myfoot{\emph{ház} `house' + acc. \emph{-(a)t}} \\
  melybe\myfoot{literary, arch. \emph{mely}
   which' + illative \emph{-be} `into (the inside of)'}
  háziúr\myfoot{\emph{ház} `house' + adj.-forming `of the'
  \emph{-i} + \emph{úr} `master'} települ.\myfoot{vb. `to
  settle (somewhere)', 3rd pers. sg. ind. pres. indef.} \\
\end{verse}

\selectlanguage{english}

\noindent Chiastic structure: \\
1-4: e-ül(8)/á-at(9)/e-ül(8)/á-at(9) \\
5-8: á-ad(9)/e-ül(8)/á-at(9)/e-ül(8)

\newpage

\selectlanguage{english}

\poemtitle*{VII}

\settowidth{\versewidth}{from under the steam of my dreams,}

\begin{verse}[\versewidth]
  I looked up from beneath the evening \\
  to the cogwheels of the heavens --- \\
  from the glittering threads of chance \\
  the loom of the past wove the law; \\
  and I looked up again onto the sky \\
  from beneath the steam of my dreams, \\
  and I saw the fabric of the law \\
  always tearing up somewhere. \\
\end{verse}

\bigskip

\noindent \textbf{Cursory remark}

\medskip

Whereas verses 1-2 echo the ancient mechanical understanding of the
firmament as concentric spheres, and verses 2-4 hearken back at the
mythological Fates (Parc\ae), verses 5-8 evoke the image of an
aerostat, whose aeronaut is J, where J's dreams are the steam that
tries to lift him up, and the law is the envelope, whose tears are J's
constant misfortune. (The metaphor is a little off, as balloons use
hot air, not steam.)

\bigskip

\noindent \textbf{French}

\selectlanguage{french}

\settowidth{\versewidth}{j'ai levé les yeux aux rouages des cieux ---}

\begin{verse}[\versewidth]
  Par dessous le soir, \\
  j'ai levé les yeux aux rouages des cieux --- \\
  des fils scintillants de la chance \\
  le métier du passé avait tissé la loi; \\
  par dessous la vapeur de mes rêves, \\
  j'ai regardé à nouveau dans les cieux \\
  et j'ai vu le tissu de la loi \\
  toujours se déchirer quelque part. \\
\end{verse}

\newpage

\selectlanguage{english}

\noindent \textbf{Original Hungarian}

\selectlanguage{hungarian}

%% Én fölnéztem az est alól
%% az eget fogaskerekére ---
%% csilló véletlen szálaiból
%% törvényt szőtt a mult szövőszéke
%% és megint fölnéztem az égre
%% álmaim gőzei alól
%% s láttam, a törvény szövedéke
%% mindig fölfeslik valahol.

\settowidth{\versewidth}{törvényt szőtt a mult szövőszéke}

\begin{verse}[\versewidth]
  Én fölnéztem\myfoot{vb. pref. \emph{föl-} `upwards'+ vb.
  \emph{néz} `to look' + 1st pers. sg. past. def. \emph{-tem}} az
  est\myfoot{arch. `(in the) evening'} alól\myfoot{\emph{al}
  `lower part' + abl. \emph{-l} `from'} \\
  az eget\myfoot{\emph{ég} `sky' + acc. \emph{-(e)t}}
  fogaskerekére\myfoot{\emph{fogas} `tooth' +
  \emph{kerék} `wheel': `cogwheel' + 3rd pers. sg. single
  poss. \emph{-e} + subl. \emph{-re} `towards'} --- \\
  csilló\myfoot{vb. \emph{csillog} `to glitter' +
  adj.-forming \emph{-o} $\rightarrow$
  \emph{csilló}; usually \emph{csillogó}}
  véletlen\myfoot{`chance'} szálaiból\myfoot{\emph{szál}
  `thread' + 3rd pers. sg. poss. \emph{-a} + poss. pl. \emph{-i} +
  elative \emph{-ból} `from/out of'} \\
  törvényt\myfoot{\emph{törveny} `law/principle' + acc. \emph{-t}}
  szőtt\myfoot{vb. \emph{sző} `to weave', past part.} a
  mult\myfoot{\textbf{\emph{múlt}}? rare, past part. `past',
  inf. \emph{múlik}} szövőszéke\myfoot{\emph{szövő} `weaver' +
  \emph{szék} `chair': `loom' + 3rd pers. sg. single poss. \emph{-e}} \\
  és megint\myfoot{\emph{meg} `again' (short action) +
  adv.-forming \emph{-int}} fölnéztem az égre\myfoot{\emph{ég}
  `sky' + subl. sg. \emph{-re} `onto'} \\
  álmaim\myfoot{\emph{álom} `dream' +
  1st pers. sg. poss. \emph{-a} + poss. pl. \emph{-i} + 1st
  pers. sg. pers. \emph{-m}} gőzei\myfoot{\emph{gőz} `steam,
  vapour' + 3rd pers. sg. poss. \emph{-e} + pl. poss. \emph{-i}} alól \\
  s láttam,\myfoot{vb. \emph{lát} `to see' +
  short base \emph{-t}, 1st pers. sg. past def. \emph{-tam}} a törvény
  szövedéke\myfoot{\emph{szövedék} `fabric/cloth/web' + 3rd
  pers. sg. single poss. \emph{-e}} \\
  mindig\myfoot{\emph{mind} `always' (repetition) + term. `until'
  \emph{-ig}} fölfeslik\myfoot{vb. pref. \emph{föl-} `upwards'+
  arch., poetic, vb.
  \emph{feslik} `to come unstitched', 3rd pers. sg. ind. pres. indef.}
  valahol.\myfoot{`somewhere'} \\
\end{verse}

\selectlanguage{english}

\noindent Chiastic structure: \\
1-4: a-ól(8)/é-e(9)/a[i]-ól(9)/é-e(9) \\
5-8: é-e(9)/[i]a-ól(8)/é-e(9)/a-ol(8)

\newpage

\selectlanguage{english}

\poemtitle*{VIII}

\settowidth{\versewidth}{The silence listened intently --- One struck.}

\begin{verse}[\versewidth]
  The silence listened intently --- One struck. \\
  You could visit your childhood; \\
  between the damp cement walls \\
  you could imagine a bit of freedom --- \\
  I thought. And as soon as I stood up, \\
  the stars, the Great Bear \\
  indeed were glittering above, \\
  like the grilles up in the silent cell. \\
\end{verse}

\bigskip

\noindent \textbf{Cursory remark}

\medskip

This is likely a memory of J's imprisonment in \oldstylenums{1931}.

\bigskip

\noindent \textbf{French}

\selectlanguage{french}

\settowidth{\versewidth}{Le silence écoutait attentivement --- Une heure sonna.}

\begin{verse}[\versewidth]
  Le silence écoutait attentivement --- Une heure sonna. \\
  Tu pourrais visiter ton enfance; \\
  entre les murs de ciment humide \\
  tu pourrais imaginer un peu de liberté --- \\
  me dis-je. Et dès que je fus sur pied, \\
  les étoiles, la Grande Ourse \\
  scintillaient en effet au-dessus, \\
  comme les grilles en haut dans ma cellule. \\
\end{verse}

\newpage

\selectlanguage{english}

\noindent \textbf{Original Hungarian}

\selectlanguage{hungarian}

%% Fülelt a csend --- egyet ütött
%% Fölkereshetnéd ifjúságod;
%% nyirkos cementfalak között
%% képzelhetsz egy kis sabadságot ---
%% gondoltam. S hát hát amint fölállok
%% a csillagok, a Göncölök
%% úgy fénylenek fönt, mint a rácsok
%% a hallgatag cella fölött.

\settowidth{\versewidth}{gondoltam. S hát hát amint fölállok}

\begin{verse}[\versewidth]
  Fülelt\myfoot{\emph{fül} `ear' + vb.-forming \emph{-el}: `to
  listen (intently)' + 3rd pers. ind. past indef. \emph{-t}} a
  csend\myfoot{`silence'. alt. \emph{csönd}}
  --- egyet\myfoot{\emph{egy} + acc. \emph{-t}}
  ütött\myfoot{vb. \emph{üt} `to strike' + 3rd
  pers. sg. ind. past indef. \emph{-(ö)tt}} \\
  Fölkereshetnéd\myfoot{\emph{föl-}
  `upwards' + \emph{keres} `to seek': `to visit' +
  pot. `may/might' \emph{-het} + 2nd
  pers. sg. ind. pres. cond. \emph{-ned}, alt. \emph{felkeres}}
  ifjúságod;\myfoot{adj. \emph{ifjú} `young' + noun-forming
  \emph{-ság} (state of being): `youth' + 2nd pers. informal sg.
  poss. single poss. \emph{-od}: `your childhood' (informal)} \\
  nyirkos\myfoot{adj. `humid/damp'}
  cementfalak\myfoot{\emph{cement} `cement'+ \emph{fal} `wall' +
  pl. \emph{-ak}} között\myfoot{post. `between/amid', \emph{köz}
  `interval' + loc. \emph{-ött}} \\
  képzelhetsz\myfoot{\emph{kép} `image' + vb.-forming
  \emph{-(z)el}: `to imagine' + pot. \emph{-het} + 2nd
  pers. sg. ind. pres. \emph{-sz}} egy kis
  sabadságot\myfoot{\emph{szabad} `free' + \emph{-ság} (state of
  being): `freedom' + acc. \emph{-(a)t}} --- \\
  gondoltam.\myfoot{vb. \emph{gondol} `to
  think/plan/guess/worry' + 1st pers. sg. ind. past
  \emph{-tam}} S hát\myfoot{`back body','well...',
  `then/and/but' (questioning back), `surely'}
  amint\myfoot{\emph{az} + \emph{mint} `like': `as soon as',
  literary `as'} fölállok\myfoot{\emph{föl-}
  `upwards' +  vb. \emph{áll} `to stand' + 1st
  pers. sg. ind. pres. indef. \emph{-ok}, alt. \emph{felállok}} \\
  a csillagok,\myfoot{\emph{csillag} `star' + pl. \emph{-ak}} a
  Göncölök\myfoot{\emph{Göncöl} + pl. \emph{-ök}: `the Great Bear'
  (constellation)} \\
  úgy fénylenek\myfoot{\emph{fény} `light' +
  freq. vb.-forming \emph{-lik}: `to glitter' + 3rd
  pers. pl. ind. pres. indef. \emph{-ek}}
  fönt,\myfoot{adv. `above/up/awake/upstairs', alt. \emph{fent}}
  mint a rácsok\myfoot{\emph{rács} `grid/grille' + pl. \emph{-ok}} \\
  a hallgatag\myfoot{vb. \emph{hallgat} `to remain silent/to
  listen' + adj.-forming \emph{-ag}} cella\myfoot{`cell'}
  fölött.\myfoot{\emph{föl-} `up/upwards' +  loc. \emph{-ött}:
  `above', alt. \emph{elett}} \\
\end{verse}

\selectlanguage{english}

\noindent Chiastic structure: \\
1-4: ü-ö[tt](8)/á-o(9)/ö-ö[tt](8)/á-o(9) \\
5-8: á-o[k](9)/ö-ö(8)/á-o[k](9)/ö-ö[tt](8)

\newpage

\selectlanguage{english}

\poemtitle*{IX}

\settowidth{\versewidth}{and that only a memory may be forgotten;}

\begin{verse}[\versewidth]
  I heard iron weeping, \\
  I heard rain laughing. \\
  I saw that the past was cracked \\
  and that only memories may be forgotten, \\
  and how I cannot but love, \\
  bending under my burdens --- \\
  why should I also forge a weapon \\
  out of you, golden self-awareness! \\
\end{verse}

\bigskip

\noindent \textbf{Cursory remark}

\medskip

Verses 1-2 refer to the sounds of iron being forged and quenched in
water. The blacksmith is fixing a crack, which reminds J of a painful
past event, possibly a break\hyp{}up. Verses 3-6 form a chiasm:
verse~5 echoes verse~4: to avoid his love fading away, he will keep
loving; verse~6 answers verse~3: instead of hammering the crack to
make it disappear, J bends under it. Then there is no need to forge a
sword blade to become strong. J's self\hyp{}consciousness is golden,
like the glow that illuminates all the interior of the kiln.

\bigskip

\noindent \textbf{French}

\selectlanguage{french}

\settowidth{\versewidth}{et que seuls les souvenirs peuvent s'oublier,}

\begin{verse}[\versewidth]
  J'ai entendu le fer sangloter, \\
  j'ai entendu la pluie rire. \\
  Je vis que le passé était craquelé \\
  et que seuls les souvenirs peuvent s'oublier, \\
  et comment je ne peux qu'aimer, \\
  pliant sous mes fardeaux --- \\
  pourquoi devrais-je aussi forger une arme \\
  de toi, for intérieur doré! \\
\end{verse}

\newpage

\selectlanguage{english}

\noindent \textbf{Original Hungarian}

\selectlanguage{hungarian}

%% Hallottam sírni a vasat,
%% hallottam az esőt nevetni.
%% Láttam, hogy a mult meghasadt
%% s csak képzetet lehet feledni;
%% s hogy nem tudok mást, mint szeretni,
%% görnyedve terheim alatt ---
%% minek is kell fegyvert veretni
%% belőled, arany öntudat!

\settowidth{\versewidth}{s hogy nem tudok mást, mint szeretni,}

\begin{verse}[\versewidth]
  Hallottam\myfoot{vb. \emph{hall} `to hear' + 1st
  pers. sg. ind. past def. \emph{-ottam}} sírni\myfoot{vb.
  \emph{sír} `to cry/weep' + inf. \emph{-ni}} a
  vasat,\myfoot{\emph{vas} `iron' + acc. \emph{-(a)t}} \\
  hallottam az esőt\myfoot{vb. \emph{es}, inf. \emph{esik} +
  pres. part. \emph{-ő} + acc. \emph{-t}} nevetni.\myfoot{vb.
  \emph{nevet} `to laugh', inf. \emph{nevetni}} \\
  Láttam,\myfoot{vb. \emph{lát} `to see' +
  short base \emph{-t}, 1st pers. sg. past def. \emph{-tam}}
  hogy\myfoot{adv. `how'} a mult\myfoot{\textbf{\emph{múlt}}?
  rare, past part. `past', inf. \emph{múlik}}
  meghasadt\myfoot{perfective \emph{meg-} + vb.
  \emph{hasad} `to crack/split apart' + past
  part. \emph{-t}} \\
  s csak\myfoot{adv. `only'} képzetet\myfoot{\emph{képzet}
  `idea/image/notion' + acc. \emph{-t}}
  lehet\myfoot{pot. vb. \emph{van/lesz}:
  `may be/may become', 3rd pers. sg. ind.}
  feledni;\myfoot{lit. vb. \emph{feled} `to forget', alt. \emph{felejt}} \\
  s hogy\myfoot{adv. `how'} nem tudok\myfoot{vb.
  \emph{tud} `to know'/aux. `can + inf.' + 1st
  pers. sg. ind. pres. indef. \emph{-ok}}
  mást,\myfoot{adj. \emph{más} `other/different' + acc. \emph{-t}}
  mint\myfoot{comp. different degrees `than'}
  szeretni,\myfoot{inf. vb. \emph{szeret} `to love'} \\
  görnyedve\myfoot{vb. \emph{görnyed} `to bend (under a
  burden)' + adv.-forming \emph{-ve}} terheim\myfoot{\emph{teher}
  `freight/burden' + 1st
  pers. sg. poss. mult. poss. \emph{-eim} $\rightarrow$ \emph{terheim}}
  alatt\myfoot{post. \emph{al-} `lower part'+ loc. \emph{-(a)tt}}
  --- \\
  minek\myfoot{\emph{mi} + dat. \emph{-nek}: `for what'}
  is\myfoot{doublet of \emph{és} `and', adv. `also'}
  kell\myfoot{aux. vb. `must/need to/have to + inf.'}
  fegyvert\myfoot{\emph{fegyever} `weapon' + acc. \emph{-t}}
    veretni\myfoot{inf. vb. `to hammer', alt. \emph{veretés}} \\
  belőled,\myfoot{pron. \emph{belőle} `out of him/her/it' + 2nd
  pers. sg. \emph{-d}} arany\myfoot{`gold/golden'}
  öntudat!\myfoot{`self-awareness'} \\
\end{verse}

\selectlanguage{english}

\noindent Chiastic structure: \\
1-4: a-a[t](8)/e[t]-[n]i(9)/a-a[dt](8)/e[d]-[n]i(9) \\
5-8: e[t]-[n]i(9)/a-a[tt](8)/e[t]-[n]i(9)/u-a[t](8)

\newpage

\selectlanguage{english}

\poemtitle*{X}

\settowidth{\versewidth}{he who knows that he receives so much life}

\begin{verse}[\versewidth]
  He is a grown man he who has \\
  neither mother nor father in his heart, \\
  he who knows that he receives life \\
  as an extra to death, and that he returns it \\
  any time, like a found object \\
  --- therefore he treasures it, \\
  he who is neither a god nor a priest, \\
  neither to himself nor to others. \\
\end{verse}

\bigskip

\noindent \textbf{Cursory remark}

\medskip

An integral man (the Hungarian language is actually gender neutral)
knows that his parents cannot protect him anymore from the truth of
mortality. This truth is not treasured by gods, as they are immortal,
nor Christian priests, who preach eternal life.

\bigskip

\noindent \textbf{French}

\selectlanguage{french}

\settowidth{\versewidth}{tel un supplément à la mort, et qu'il la rendra}

\begin{verse}[\versewidth]
  Il est un homme accompli celui qui n'a \\
  ni mère ni père en son cœur, \\
  celui qui sait qu'il reçoit la vie \\
  tel un supplément à la mort, et qu'il la rendra \\
  à tout moment, comme un objet trouvé \\
  --- par conséquent il la garde précieusement, \\
  celui qui n'est ni un dieu ni un prêtre, \\
  ni pour lui-même, ni pour autrui. \\
\end{verse}

\newpage

\selectlanguage{english}

\noindent \textbf{Original Hungarian}

\selectlanguage{hungarian}

%% Az meglett ember, akinek
%% szívében nincs se anyja, apja,
%% ki tudja, hogy az életet
%% halálra ráadásul kapja
%% s mint talált tárgyat visszaadja
%% bármikor --- ezért őrzi meg,
%% ki nem istene és nem papja
%% se magának, sem senkinek.

\settowidth{\versewidth}{s mint talált tárgyat visszaadja}

\begin{verse}[\versewidth]
  Az meglett\myfoot{obs. `adult'} ember\myfoot{`person'},
  akinek\myfoot{pron. \emph{aki} `who' + attr. poss. \emph{-nek}} \\
  szívében\myfoot{\emph{szív} `heart' + poss. `his' \emph{-e} +
  innes. `in' \emph{-ben}} nincs se anyja,\myfoot{\emph{anya}
  `mother' + 3rd pers. sg. single poss. \emph{-(j)a}}
  apja,\myfoot{\emph{apa} `father' + 3rd pers. sg. single
  poss. \emph{-(j)a}} \\
  ki\myfoot{pron. rel. arch. `he who', alt. \emph{aki}}
  tudja,\myfoot{vb. \emph{tud} `to know' + 3rd
  pers. sg. ind. pres. def. \emph{-ja}}
  hogy\myfoot{conj. `so that'} az életet\myfoot{\emph{élet}
  `life' + acc. \emph{-(e)t}} \\
  halálra\myfoot{vb. \emph{hal} `to die' + arch. noun-forming
  \emph{-ál} $\rightarrow$ `death' + subl. `on/for' \emph{-ra}
  $\rightarrow$ `to death'}
  ráadásul\myfoot{\emph{ráadás} `addition/extra' + essive
  `as/with the intention of' \emph{-ul}}
  kapja\myfoot{vb. \emph{kap} `to receive' + 3rd
  pers. sg. ind. pers. def. \emph{-ja}} \\
  s mint\myfoot{`like'} talált\myfoot{vb. \emph{talált} `to
  find' + past part.} tárgyat\myfoot{\emph{tárgy}
  `object/thing/subject' + acc. \emph{-(a)t}}
  visszaadja\myfoot{vb. \emph{visszaad} `to give back' + 3rd
  pers. sg. ind. pres. def. \emph{-ja}} \\
  bármikor\myfoot{\emph{bár} `any' + adv. \emph{mikor} `when'
  $\rightarrow$ `any time'} --- ezért\myfoot{conj. `therefore'}
  őrzi\myfoot{vb. \emph{őriz} `to guard' + 3rd
  pers. sg. ind. pres. def. \emph{-i}} meg,\myfoot{arch. emph.} \\
  ki\myfoot{`he who'} nem istene\myfoot{\emph{isten} `god' +
  3rd pers. sg. sing. poss. \emph{-e}} és nem
  papja\myfoot{\emph{pap} `priest' + 3rd
  pers. sg. sing. poss. \emph{-(j)a}} \\
  se\myfoot{alt. \emph{sem}, \emph{sem ..., sem ...}  `neither
  ... nor ...'} magának,\myfoot{pron. refl. 3rd pers. `himself'}
  sem senkinek.\myfoot{pron. \emph{senki} `no one/nobody'} \\
\end{verse}

\selectlanguage{english}

\noindent Chiastic structure: \\
1-4: [k]i-[n]e[k](8)/a[p]-[j]a(9)/e-e(8)/a[p]-[j]a(9)\\
5-8: a[d]-[j]a(9)/i-e(8)/a[p]-[j]a(9)/[k]i-[n]e[k](8)

\newpage

\selectlanguage{english}

\poemtitle*{XI}

\settowidth{\versewidth}{It plopped into the squishy, tepid puddle,}

\begin{verse}[\versewidth]
  I saw happiness; it was soft, bright \\
  and one and half a quintal. \\
  On the coarse grass of the farmyard \\
  its curly smile swayed. \\
  It plopped into the squishy, tepid puddle, \\
  squinted, further grunted at me once --- \\
  To this day I see how hesitantly \\
  the sunlight toyed with its bristles. \\
\end{verse}

\bigskip

\noindent \textbf{Cursory remark}

\medskip

While this seems a rather descriptive poem, the first and last two
verses are subjective. Also the last verse echoes the first, in the
fashion of a chiasm: verse~1 is about seeing happiness, and the last
features light and playfulness: the Hungarian uses \emph{fény} for
`light', but it could figuratively also mean `joy', hence my
translation: `the sunlight toyed'.

\bigskip

\noindent \textbf{French}

\selectlanguage{french}

\settowidth{\versewidth}{Il se laissa tomber dans la marre tiède et spongieuse,}

\begin{verse}[\versewidth]
  J'ai vu le bonheur; il était doux, brillant \\
  et un quintal et demi. \\
  Sur la mauvaise herbe de la cour de ferme \\
  son sourire courbé se balançait. \\
  Il se laissa tomber dans la marre tiède et spongieuse, \\
  plissa ses yeux, puis me grogna une fois --- \\
  Jusqu'à ce jour, je vois avec quelle hésitation \\
  la lumière du jour jouait avec ses poils. \\
\end{verse}

\newpage

\selectlanguage{english}

\noindent \textbf{Original Hungarian}

\selectlanguage{hungarian}

%% Láttam a boldogságot én,
%% lágy volt, szőke és másfél mázsa.
%% Az udvar szigorú gyöpén
%% imbolygott göndör mosolygása.
%% Ledőlt a puha, langy tócsába,
%% hunyorgott, röffent még felém ---
%% ma is látom, mily tétovázva
%% babrált pihéi közt a fény.

\settowidth{\versewidth}{hunyorgott, röffent még felém ---}

\begin{verse}[\versewidth]
  Láttam\myfoot{vb. \emph{lát} `to see' +
  short base \emph{-t}, 1st pers. sg. past def. \emph{-tam}} a
  boldogságot\myfoot{adj. \emph{boldog} `happy' + noun-forming
  \emph{-ság} (state of being) + acc. \emph{-(o)t}}
  én,\myfoot{pron. `I'} \\
  lágy\myfoot{adj. `soft'} volt,\myfoot{vb. \emph{van} `to
  be'+ 3rd pers. sg. ind. past indef.} szőke\myfoot{adj. \emph{sző}
  `blonde' + dim. \emph{-ke}} és másfél\myfoot{\emph{más}
  `other/different' + \emph{fél} `half' $\rightarrow$ `one and a
  half'} mázsa.\myfoot{`quintal'} \\
  Az udvar\myfoot{`yard'} szigorú\myfoot{adj. `severe/strict'}
  gyöpén\myfoot{\emph{gyep} `lawn/grass' + super. \emph{-én}}  \\
  imbolygott\myfoot{vb. \emph{imbolyog} `to sway'+ 3rd
  pers. sg. single poss. \emph{-e}} göndör\myfoot{adj. `curly'}
  mosolygása.\myfoot{\emph{mosolygása} `smile' + 3rd
  pers. sg. single poss. \emph{-e}} \\
  Ledőlt\myfoot{vb. \emph{ledoől} `to fall down' + 3rd
  pers. sg. ind. past indef. \emph{-t}} a puha,\myfoot{`soft/squishy'}
  langy\myfoot{arch. poetic \emph{langyos} `tepid'}
  tócsába,\myfoot{\emph{tócsa} `puddle' + illative \emph{-ba}
  `into'} \\
  hunyorgott,\myfoot{vb \emph{hunyorg} `to squint' + 3rd
  pers. sg. ind. past. indef. \emph{-rgott}}
  röffent\myfoot{\emph{röffen} `to grunt once (of a pig)'}
  még\myfoot{adv. `still/moreover'} felém\myfoot{post. \emph{felé}
  `to/towards' + 1st pers. sg. poss \emph{-m} $\rightarrow$
  pron. `towards me'} --- \\
  ma\myfoot{adv. `today'} is\myfoot{doublet of \emph{és} `and',
  adv. `also'} látom,\myfoot{vb. \emph{lát} `to see'+ 1st
  pers. sg. ind. pres. def.} mily\myfoot{poetic `how ...(!)'}
  tétovázva\myfoot{vb. \emph{tétova} `to hesitate'
  (inf. \emph{tétovazik}) $\rightarrow$ adv. part. `doing smth
  hesitantly'} \\
  babrált\myfoot{vb. \emph{babrál} `to fiddle'}
  pihéi\myfoot{\emph{pihe} `down, fluff' + adj.-forming \emph{-i}}
  közt\myfoot{post. `between/amid' from \emph{között}: \emph{köz}
  `interval' + loc. \emph{-ött}} a fény.\myfoot{`light/glitter',
  fig. `happiness/joy'} \\
\end{verse}

\selectlanguage{english}

\noindent Chiastic structure: \\
1-4: o-é(8)/á-a(9)/ö-é(8)/á-a(9) \\
5-8: á-a(9)/e-é(8)/á-a(9)/a-é(8)

\newpage

\selectlanguage{english}

\poemtitle*{XII}

\settowidth{\versewidth}{and I stand in the light from the compartments,}

\begin{verse}[\versewidth]
  I live by a railroad. \\
  Many a train comes and goes, \\
  and I watch from afar \\
  how the bright windows fly by \\
  in the wavering, fluffy darkness. \\
  So the lit-up days rush into the eternal night, \\
  and I stand in the light from the compartments, \\
  I rest on my elbow and remain silent. \\
\end{verse}

\bigskip

\noindent \textbf{French}

\selectlanguage{french}

\settowidth{\versewidth}{Ainsi les jours clairs se pressent dans la nuit éternelle,}

\begin{verse}[\versewidth]
  Je vis près du chemin de fer. \\
  Nombreux sont les trains \\
  qui viennent et vont, et j'observe de loin \\
  comment les fenêtres illuminées défilent \\
  dans l'obscurité fluctuante et peluchée. \\
  Ainsi les jours clairs se pressent dans la nuit éternelle, \\
  et je me tiens dans la lumière des compartiments, \\
  je m'accoude et garde le silence. \\
\end{verse}

\newpage

\selectlanguage{english}

\noindent \textbf{Original Hungarian}

\selectlanguage{hungarian}

%% Vasútnál lakom. Erre sok
%% vonat jön-megy és el-elnézem,
%% hogy' szállnak fényes ablakok
%% a lengedező szösz-sötétben.
%% Így iramlanak örök éjben
%% kivilágított nappalok
%% s én állok minden fülke-fényben,
%% én könyöklök és hallgatok.

\settowidth{\versewidth}{s én állok minden fülke-fényben,}

\begin{verse}[\versewidth]
  Vasútnál\myfoot{\emph{vas} `iron' + \emph{ut} `way/road'+
    adess. \emph{-nal} `at/by'} lakom.\myfoot{vb. \emph{lak} `to
    dwell', inf. \emph{lakik}, 1st pers. sg. ind. pres. \emph{-(o)m}}
  Erre\myfoot{pron. \emph{ez} `this/it' + subl. sg. \emph{-re}
    `onto/to'}
  sok\myfoot{`much/many'} \\
  vonat\myfoot{vb. \emph{von} `to pull/to draw' + noun-forming
    \emph{-at} $\rightarrow$ `train`} jön\myfoot{vb. `to come' + 3rd
    pers. sg. ind. pres. indef.}-megy\myfoot{vb. `to go/travel/pass by
    (of time)' + 3rd pers. sg. ind. pres. indef.} és
  el\myfoot{adv./conj. `afar/far
    away'}-elnézem,\myfoot{vb. pref. \emph{el} `continuity over a long
    time' + vb. \emph{nez} `to look at'} \\
  hogy'\myfoot{short. \textbf{\emph{hogyan?}} `how'}
  szállnak\myfoot{vb. \emph{száll} `to fly/land/to come down' + 3rd
  pers. pl. ind. pres. indef.} fényes\myfoot{\emph{fény}
    `light/glitter' + adj.-forming \emph{-es} $\rightarrow$ `bright'}
  ablakok\myfoot{\emph{ablak} `window' + pl. \emph{-(o)k}} \\
  a lengedező\myfoot{vb. \emph{leng} `to swing/sway/wave' +
    freq. vb.-forming \emph{-edezik} (unrounded front\hyp{}vowel) +
    pres. part. \emph{-ő} $\rightarrow$ `swinging/waving'}
  szösz\myfoot{`harl/fluff'}-sötétben.\myfoot{adj./subst. \emph{sötét}
    `dark[ness]' + iness. sg. \emph{-ben} `in'} \\
  Így\myfoot{`just so'} iramlanak\myfoot{\emph{ir} `movement' +
  rare instr. vb.-forming \emph{-amlik} (inf.) $\rightarrow$ `to
  rush/hurry' + 3rd pers. pl. ind. pres. indef. \emph{-(a)nak}}
  örök\myfoot{`perpetual/eternal'} éjben\myfoot{poetic \emph{éj} `night' +
  innes. `in' \emph{-ben}} \\
  kivilágított\myfoot{vb. pref. \emph{ki} `outwardly' + \emph{világ}
    `light (arch.)/world' $\rightarrow$ arch. `illumination' +
    caus. vb.-forming \emph{-ít} `to make smth ...-like' $\rightarrow$
    `to illuminate' + past part. \emph{-ott}}
  nappalok\myfoot{\emph{nap} `day' + instr. \emph{-val}
  `with' $\rightarrow$ \emph{nappal} + pl. \emph{-(o)k}} \\
  s én\myfoot{pron. `I'} állok\myfoot{vb. \emph{áll} `to stand' + 1st
    pers. sg. ind. pres. indef. \emph{-ok}}
  minden\myfoot{det. `every'} fülke\myfoot{\emph{fül}
    `semi\hyp{}circular obj./shape' + dim. \emph{-ke} $\rightarrow$
    `compartment'}-fényben,\myfoot{\emph{fény}
    `light/glitter' + innes. \emph{-ben} `in'} \\
  én könyöklök\myfoot{\emph{könyök} `elbow' + vb.-forming \emph{-öl}
    $\rightarrow$ `to rest on one's elbow' + 1st
    pers. sg. ind. pers. \emph{-ök}} és
    hallgatok.\myfoot{vb. \emph{hallgat} `to remain silent/to listen'
      + 1st pers. sg. ind. pres. indef. \emph{-ok}} \\
\end{verse}

\selectlanguage{english}

\noindent Chiastic structure: \\
1-4: o(8)/é-e(9)/a-o(8)/é-e(9) \\
5-8: é-e(9)/a-o(8)/é-e(9)/o(8)

%% =====================================================================

\chapter{\oldstylenums{1936}}

\selectlanguage{hungarian}

\poemtitle*{}

\settowidth{\versewidth}{s nem lesz emlék, melyben magadra hagyna.}

\begin{verse}[\versewidth]
  Majd megöregszel és bánni fogod, \\
  hogy bántasz, - azt, amire büszke vagy ma. \\
  A lelkiismeret majd bekopog \\
  s nem lesz emlék, melyben magadra hagyna. \\

  Lesz vén ebed s az melléd települ. \\
  Nappal pihensz majd, széken szunyókálva, \\
  mert éjjel félni fogsz majd egyedül. \\
  Árnyak ütnek a rezgő anyókára. \\

  Az öreg kutya néha majd nyafog, \\
  de a szobában csend lesz, csupa rend lesz; \\
  hanem valaki hiányozni fog \\
  a multból ahhoz a magányos csendhez. \\

  Majd tipegsz s ha eleget totyogott \\
  rossz lábod, leülsz. Fönn aranykeretben \\
  áll ifju képed. Hozzá motyogod: \\
  "Nem öleltem meg, hiszen nem szerettem." \\

  "Mit is tehettem volna?" - kérdezed, \\
  de fogatlan szád már nem válaszolhat; \\
  s ki a nap előtt lehunyod szemed, \\
  alig várod, hogy feljöjjön, a holdat. \\

  Mert ha elalszol, ugrál majd az ágy, \\
  mint a csikó, hogy a hámot levesse. \\
  S a félelem tünődik, nem a vágy, \\
  a fejedben: Szeress-e, ne szeress-e. \\

  Magadban döntöd el. Én fájlalom, \\
  hogy nem felelhetek, ha kérded: él-e. \\
  Mert elfárad bennem a fájdalom, \\
  elalszik, mint a gyermek s én is véle. \\
\end{verse}

1936. nov.

\end{document}
