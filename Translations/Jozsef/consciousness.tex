\vspace*{-15mm}
\centeredornament
\vspace*{2mm}

\selectlanguage{english}

\begin{center}
  \textbf{\large Consciousness}
\end{center}

\addcontentsline{toc}{section}{Consciousness}

\poemtitle*{I}

\addcontentsline{toc}{subsection}{I.\ \emph{Dawn unbinds sky from earth}}

\begin{verse}
  Dawn unbinds sky from earth \\
  and upon its clear, soft word \\
  beetles and children spin forth \\
  into the daylight; \\
  there is no moisture in the air, \\
  the bright levity floats! \\
  Overnight they alighted on the trees \\
  like small butterflies, the leaves.
\end{verse}

\bigskip

\poemtitle*{II}
%% \footnote{Verses 1-4 are about the night, verses 5-7 transition to the day, and
%% verse~8 brings us back to the night.}}

\addcontentsline{toc}{subsection}{II. \ \emph{I saw paintings daubed with blue}}

\selectlanguage{english}

\begin{verse}
  I saw paintings daubed with blue, \\
  red, yellow in my dreams \\
  and felt that everything was just so ---\\
  not one speck of dust flying madly about. \\
  Now blurred, my dream comes down \\
  into my limbs, and the iron world is the order.\hspace*{-1mm} \\
  During the day a moon rises inwardly and, \\
  when the night is out, a sun shines herewithin.\hspace*{-1mm}
\end{verse}

\newpage

\vspace*{-15mm}
\centeredornament
\vspace*{2mm}

\selectlanguage{hungarian}

\begin{center}
  \textbf{\large Eszmélet}
\end{center}

\poemtitle*{I}
%% \footnote{Chiastic structure: \\
%% 1-4: e-e(8)/á-ra(9)/e-e(8)/á-ra(9) \\
%% 5-8: á-ra(9)/e-e(8)/á-ra(9)/e-e(8)}}

\begin{verse}
  Földtől eloldja az eget \\
  a hajnal s tiszta, lágy szavára \\
  a bogarak, a gyerekek \\
  kipörögnek a napvilágra; \\
  a levegőben semmi pára, \\
  a csilló könnyűség lebeg! \\
  Az éjjel rászálltak a fákra, \\
  mint kis lepkék, a levelek.
\end{verse}

\bigskip

\poemtitle*{II}
%% \footnote{Chiastic structure: \\
%% 1-4: e-ent(8)/ai-an(9)/a-rend(8)/i-an(9) \\
%% 5-8: i-an(9)/a-rend(8)/ai-an(9)/e-ent(8)}}

\begin{verse}
  Kék, piros, sárga, összekent \\
  képeket láttam álmaimban \\
  és úgy éreztem, ez a rend --- \\
  egy szalló porszem el nem hibbant. \\
  Most homályként száll tagjaimban \\
  álmom s a vas világ a rend. \\
  Nappal hold kél bennem s ha kinn van \\
  az éj --- egy nap süt idebent.
\end{verse}

\newpage

\vspace*{-15mm}
\centeredornament

\selectlanguage{english}

\poemtitle*{III}
%% \footnote{József cannot pursue his art and, at the same time, keep a job that
%% would put food on his table. Cuddling a small child to his heart is
%% his metaphor for the creation of poetry. József is the cat in verses
%% 6-7.}}

\addcontentsline{toc}{subsection}{III.\ \emph{I am thin, at times I only eat bread}}

\begin{verse}
  I am thin, at times I only eat bread; \\
  amidst those idle and blithering souls, \\
  I seek in vain more certainty, like the die. \\
  No chuck roast reaches my mouth \\
  while I cuddle a small child to my heart --- \\
  however clever, the cat cannot catch at once \\
  the mouse outdoors and the mouse indoors.
\end{verse}

\bigskip

\poemtitle*{IV}
%% \footnote{`To be or not to be' is usually construed as the dichotomy between
%% life and death. József tells us that not-to-be is potential beauty,
%% whereas to-be is actual death.}}

\addcontentsline{toc}{subsection}{IV. \ \emph{Just like a pile of logs}}

\begin{verse}
  Just like a pile of logs, \\
  the world lies in a jumble; \\
  each thing presses, weighs on, \\
  holds fast to the next, \\
  and so everything is determined. \\
  Only what is not has a shrub, \\
  only what will be can flower; \\
  what is falls to pieces.
\end{verse}

\newpage

\vspace*{-15mm}
\centeredornament

\selectlanguage{hungarian}

\poemtitle*{III}
%% \footnote{Chiastic structure: \\
%% 1-4: e-ret(8)/o-cska(9)/e-ek(8)/o-cska(9) \\
%% 5-8: o-cska(8)/e-ek(8)/a-cska(8)/e-ret(8)}}

\begin{verse}
  Sovány vagyok, csak kenyeret \\
  eszem néha, e léha, locska \\
  lelkek közt ingyen keresek \\
  bizonyosabbat, mint a kocska. \\
  Nem dörgölődzik sült lopcska \\
  számhoz s szivemhez kisgyerek --- \\
  ügyeskedhet, nem fog a macska \\
  egyszerre kint s bent egeret.
\end{verse}

\bigskip

\poemtitle*{IV}
%% \footnote{Chiastic structure: \\
%% 1-4: o-a(9)/i-á(8)/o-a(9)/i-á(8) \\
%% 5-8: i-á(8)/o-a(9)/i-á(8)/o-a(9)}}

\begin{verse}
  Akár egy halom hasított fa, \\
  hever egymáson a világ, \\
  szorítjanyomja, összefogja \\
  egyík dolog a másikát \\
  s így mindenik determinált. \\
  Csak ami nincs, annak van bokra, \\
  csak ami lesz, az a virág, \\
  ami van, széthull darabokra.
\end{verse}

\newpage

\vspace*{-15mm}
\centeredornament

\poemtitle*{V\footnote{This is likely a memory from Józef's childhood, when his family was so
poor that he would occasionally steal lumps of coal from the wagons at
the freight train station near his home. See \emph{They unload the wood}, page~\pageref{they_unload_the_wood}.}}

\addcontentsline{toc}{subsection}{V. \ \emph{At the freight train station}}

\begin{verse}
  At the freight train station, \\
  I lay down behind the tree \\
  like a chunk of silence; \\
  a grey weed reached my mouth, \\
  raw, strange-sweet. \\
  Playing dead, I watched the guard \\
  --- sensing what? --- \\
  and his shadow on the quiet wagons \\
  stubbornly leaping at the bright, dewy coal.
\end{verse}

\bigskip

\poemtitle*{VI}
%% \footnote{We are not captives of the world, but of our own heart, which invited
%% the world (the landlord) in for us to contend with. The inner struggle
%% is felt as anguish and yields a feverish soul.}}

\addcontentsline{toc}{subsection}{VI. \ \emph{Behold the anguish inside}}

\begin{verse}
  Behold the anguish inside, \\
  yet the explanation lies outside. \\
  Your wound is the world \\
  --- on fire, burning up --- \\
  and you feel your soul, the fever. \\
  You are captive as long as your heart rebels \\
  --- so you will be free if it indulges \\
  not building for yourself a house \\
  where a landlord settles in.
\end{verse}

\bigskip

\newpage

\vspace*{-15mm}
\centeredornament

\poemtitle*{V}
%% \footnote{Chiastic structure: \\
%% 1-4: a-o(8)/é-e(9)/e-o(8)/é-e(9) \\
%% 5-8: é-e(9)/o-o(8)/é-e(9)/o-o(8)}}

\begin{verse}
  A teherpályaudvaron \\
  úgy lapultam a fa tövéhez, \\
  mint egy darab csönd; szürke gyom \\
  ért számhoz, nyers, különös-édes. \\
  Holtan lestem az őrt, mit érez, \\
  s a hallgatag vagónokon \\
  árnyát, mely ráugrott a fényes, \\
  harmatos szénre konokon.
\end{verse}

\bigskip

\poemtitle*{VI}
%% \footnote{Chiastic structure: \\
%% 1-4: e-ül(8)/á-at(9)/e-ül(8)/á-at(9) \\
%% 5-8: á-ad(9)/e-ül(8)/á-at(9)/e-ül(8)}}

\begin{verse}
  Im itt a szenvedés belül, \\
  ám ott kívül a magyarázat. \\
  Sebed a világ --- ég, hevül \\
  s te lelkedet érzed, a lázat. \\
  Rab vagy, amíg a szíved lázad --- \\
  úgy szabadulsz, ha kényedül \\
  nem raksz magadnak olyan házat, \\
  melybe háziúr települ.
\end{verse}

\newpage

\vspace*{-15mm}
\centeredornament

\selectlanguage{english}

\poemtitle*{VII}

\addcontentsline{toc}{subsection}{VII. \ \emph{I looked up from beneath the evening}}

\begin{verse}
  I looked up from beneath the evening \\
  to the cogwheels of the heavens: \\
  from the glittering threads of chance \\
  the loom of the past wove the law; \\
  I looked up again onto the sky \\
  from beneath the yearnings of my dreams,\footnote{The
original reads \emph{álmaim gőzei}, literally `steam of my dreams',
or, figuratively, `my turbulent dreams, the yearnings of my dreams'.} \\
  and I saw the tapestry of the law \\
  always tearing up somewhere.
\end{verse}

\bigskip

\poemtitle*{VIII\footnote{This is likely a memory of József's imprisonment in \oldstylenums{1931}.}}

\addcontentsline{toc}{subsection}{VIII. \ \emph{The silence listened intently}}

\begin{verse}
  The silence listened intently --- One struck. \\
  You could visit your childhood; \\
  between the damp cement walls \\
  you could imagine a bit of freedom \\
  --- I thought. And as soon as I stood up, \\
  the stars, the Great Bear were glittering above\\
  like the grilles up in the silent cell.
\end{verse}

\newpage

\vspace*{-15mm}
\centeredornament

\selectlanguage{hungarian}

\poemtitle*{VII}
%% \footnote{Chiastic structure: \\
%% 1-4: a-ól(8)/é-e(9)/ai-ól(9)/é-e(9) \\
%% 5-8: é-e(9)/ia-ól(8)/é-e(9)/a-ol(8)}}

\begin{verse}
  Én fölnéztem az est alól \\
  az eget fogaskerekére --- \\
  csilló véletlen szálaiból \\
  törvényt szőtt a mult szövőszéke \\
  és megint fölnéztem az égre \\
  álmaim gőzei alól \\
  s láttam, a törvény szövedéke \\
  mindig fölfeslik valahol.
\end{verse}

\bigskip

\poemtitle*{VIII}
%% \footnote{Chiastic structure: \\
%% 1-4: ü-ött(8)/á-o(9)/ö-ö(8)/á-o(9) \\
%% 5-8: á-o(9)/ö-ö(8)/á-o(9)/ö-ött(8)}}

\begin{verse}
  Fülelt a csend --- egyet ütött \\
  Fölkereshetnéd ifjúságod; \\
  nyirkos cementfalak között \\
  képzelhetsz egy kis sabadságot --- \\
  gondoltam. S hát hát amint fölállok \\
  a csillagok, a Göncölök \\
  úgy fénylenek fönt, mint a rácsok \\
  a hallgatag cella fölött.
\end{verse}

\newpage

\vspace*{-15mm}
\centeredornament

\selectlanguage{english}

\poemtitle*{IX\footnote{Verses 1-2 refer to the sounds of iron being forged and quenched in
water. The blacksmith is fixing a crack, which reminds József of a
painful past event, possibly a break\hyp{}up. Verses 3-6 form a
chiasm: verse~5 echoes verse~4: to avoid his love fading away, he will
keep loving; verse~6 answers verse~3: instead of hammering the crack
to make it disappear, József bends under it. Then there is no need to
forge a sword blade to become strong. József's self\hyp{}consciousness
is golden like the glow of the kiln's interior.}}

\addcontentsline{toc}{subsection}{IX. \ \emph{I heard iron weeping}}

\begin{verse}
  I heard iron weeping, \\
  I heard rain laughing. \\
  I saw that the past was cracked \\
  and that only memories may be forgotten, \\
  and how I cannot but love, \\
  bending under my burdens \\
  --- why should I also forge a weapon \\
  out of you, golden self-awareness!
\end{verse}

\bigskip

\poemtitle*{X\footnote{The integral man (or woman)
knows that his parents cannot protect him anymore from the truth of
mortality. This truth is not treasured by gods, as they are immortal,
nor by Christian priests, who preach eternal life. See also page~\pageref{for_flora}.}}
\label{consciousness_X}
\addcontentsline{toc}{subsection}{X. \ \emph{He is a grown man}}

\begin{verse}
  He is a grown man he who has \\
  neither mother nor father in his heart, \\
  he who knows that he receives life \\
  as a supplement to death, and will return it \\
  anytime, like a found object \\
  --- therefore he treasures it, \\
  he who is neither a god nor a priest, \\
  neither to himself nor to others.
\end{verse}

\newpage

\vspace*{-15mm}
\centeredornament

\selectlanguage{hungarian}

\poemtitle*{IX}
%% \footnote{Chiastic structure: \\
%% 1-4: a-at(8)/et-ni(9)/a-adt(8)/ed-ni(9) \\
%% 5-8: et-ni(9)/a-att(8)/et-ni(9)/u-at(8)}}

\begin{verse}
  Hallottam sírni a vasat, \\
  hallottam az esőt nevetni. \\
  Láttam, hogy a mult meghasadt \\
  s csak képzetet lehet feledni; \\
  s hogy nem tudok mást, mint szeretni, \\
  görnyedve terheim alatt --- \\
  minek is kell fegyvert veretni \\
  belőled, arany öntudat!
\end{verse}

\bigskip

\poemtitle*{X}
%% \footnote{Chiastic structure: \\
%% 1-4: ki-nek(8)/ap-ja(9)/e-e(8)/ap-ja(9)\\
%% 5-8: ad-ja(9)/i-e(8)/ap-ja(9)/ki-nek(8)}}

\begin{verse}
  Az meglett ember, akinek \\
  szívében nincs se anyja, apja, \\
  ki tudja, hogy az életet \\
  halálra ráadásul kapja \\
  s mint talált tárgyat visszaadja \\
  bármikor --- ezért őrzi meg, \\
  ki nem istene és nem papja \\
  se magának, sem senkinek.
\end{verse}

\newpage

\vspace*{-15mm}
\centeredornament

\selectlanguage{english}

\poemtitle*{XI\footnote{The last verse echoes the first: the first is about seeing happiness, and the last
features light and playfulness: the Hungarian uses \emph{fény} for
`light', but it could figuratively also mean `joy', hence my
translation: `the sunlight toyed'.}}

\addcontentsline{toc}{subsection}{XI. \ \emph{I saw happiness}}

\begin{verse}
  I saw happiness; it was soft, bright \\
  and one and half a quintal. \\
  On the coarse grass of the farmyard \\
  its curly smile swayed. \\
  It plopped into the squishy, tepid puddle, \\
  squinted, further grunted at me once --- \\
  To this day I see how hesitantly \\
  the sunlight toyed with its bristles.
\end{verse}

\bigskip

\poemtitle*{XII}

\addcontentsline{toc}{subsection}{XII. \emph{I live by the railroad}}

\begin{verse}
  I live by the railroad. \\
  Many a train comes and goes, \\
  and I watch from afar \\
  how the bright windows fly by \\
  in the wavering, fluffy darkness. \\
  So the lit-up days rush into the eternal night, \\
  and I stand in the compartments' light, \\
  I rest on my elbow and remain silent.
\end{verse}

\newpage

\vspace*{-15mm}
\centeredornament

\selectlanguage{hungarian}

\poemtitle*{XI}
%% \footnote{Chiastic structure: \\
%% 1-4: o-é(8)/á-a(9)/ö-é(8)/á-a(9) \\
%% 5-8: á-a(9)/e-é(8)/á-a(9)/a-é(8)}}

\begin{verse}
  Láttam a boldogságot én, \\
  lágy volt, szőke és másfél mázsa. \\
  Az udvar szigorú gyöpén \\
  imbolygott göndör mosolygása. \\
  Ledőlt a puha, langy tócsába, \\
  hunyorgott, röffent még felém --- \\
  ma is látom, mily tétovázva \\
  babrált pihéi közt a fény.
\end{verse}

\bigskip

\poemtitle*{XII}
%% \footnote{Chiastic structure: \\
%% 1-4: e-o(8)/é-e(9)/a-o(8)/é-e(9) \\
%% 5-8: é-e(9)/a-o(8)/é-e(9)/a-o(8)}}

\begin{verse}
  Vasútnál lakom. Erre sok \\
  vonat jön-megy és el-elnézem, \\
  hogy' szállnak fényes ablakok \\
  a lengedező szösz-sötétben. \\
  Így iramlanak örök éjben \\
  kivilágított nappalok \\
  s én állok minden fülke-fényben, \\
  én könyöklök és hallgatok.
\end{verse}
