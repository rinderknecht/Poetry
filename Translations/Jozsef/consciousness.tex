\vspace*{-15mm}
\centeredornament
\vspace*{5mm}

\selectlanguage{english}

\begin{center}
  \textbf{\large Consciousness (\oldstylenums{1934})}
\end{center}

\addcontentsline{toc}{section}{\emph{Consciousness}}

\poemtitle*{I}

\addcontentsline{toc}{subsection}{I.\ \emph{Dawn unbinds sky from earth}}

\begin{verse}
  Dawn unbinds sky from earth \\
  and upon its clear, soft word \\
  beetles and children spin forth \\
  into the daylight; \\
  there is no moisture in the air, \\
  the bright levity floats! \\
  Overnight they alighted on the trees \\
  like small butterflies, the leaves.
\end{verse}

\bigskip

\poemtitle*{II}

\addcontentsline{toc}{subsection}{II. \ \emph{I saw paintings daubed with blue}}

\selectlanguage{english}

\begin{verse}
  I saw paintings daubed with blue, \\
  red, yellow in my dreams \\
  and felt that everything was just so ---\\
  not one speck of dust flying madly about. \\
  Now blurred, my dream comes down into my limbs, \\
  and the iron world is the order. \\
  During the day a moon rises inwardly and, \\
  when the night is out, a sun shines herewithin.
\end{verse}

\newpage

\vspace*{-15mm}
\centeredornament
\vspace*{5mm}

\selectlanguage{hungarian}

\begin{center}
  \textbf{\large Eszmélet (\oldstylenums{1934})}
\end{center}

\poemtitle*{I}

\begin{verse}
  Földtől eloldja az eget \\
  a hajnal s tiszta, lágy szavára \\
  a bogarak, a gyerekek \\
  kipörögnek a napvilágra; \\
  a levegőben semmi pára, \\
  a csilló könnyűség lebeg! \\
  Az éjjel rászálltak a fákra, \\
  mint kis lepkék, a levelek.
\end{verse}

\bigskip

\poemtitle*{II}

\begin{verse}
  Kék, piros, sárga, összekent \\
  képeket láttam álmaimban \\
  és úgy éreztem, ez a rend --- \\
  egy szalló porszem el nem hibbant. \\
  Most homályként száll tagjaimban \\
  álmom s a vas világ a rend. \\
  Nappal hold kél bennem s ha kinn van \\
  az éj --- egy nap süt idebent.
\end{verse}

\newpage

\vspace*{-15mm}
\centeredornament

\selectlanguage{english}

\poemtitle*{III}

\addcontentsline{toc}{subsection}{III.\ \emph{I am thin, at times I only eat bread}}

\begin{verse}
  I am thin, at times I only eat bread; \\
  amidst those idle and blithering souls, \\
  I seek in vain more certainty, like the die. \\
  No chuck roast reaches my mouth \\
  while I cuddle a small child to my heart --- \\
  however clever, the cat cannot catch at once \\
  the mouse outdoors and the mouse indoors.
\end{verse}

\bigskip

\poemtitle*{IV}

\addcontentsline{toc}{subsection}{IV. \ \emph{Just like a pile of logs}}

\begin{verse}
  Just like a pile of logs, \\
  the world lies in a jumble; \\
  each thing presses, weighs on, \\
  holds fast to the next, \\
  and so everything is determined. \\
  Only what is not has a shrub, \\
  only what will be can flower; \\
  what is falls to pieces.
\end{verse}

\bigskip

\poemtitle*{V}

\addcontentsline{toc}{subsection}{V. \ \emph{At the freight train station}}

\begin{verse}
  At the freight train station, \\
  I lay down behind the tree \\
  like a chunk of silence; \\
  a grey weed reached my mouth, \\
  raw, strange-sweet. \\
  Playing dead, I watched the guard \\
  --- sensing what? --- \\
  and his shadow on the quiet wagons \\
  stubbornly leaping at the bright, dewy coal.
\end{verse}

\newpage

\vspace*{-15mm}
\centeredornament

\selectlanguage{hungarian}

\poemtitle*{III}

\begin{verse}
  Sovány vagyok, csak kenyeret \\
  eszem néha, e léha, locska \\
  lelkek közt ingyen keresek \\
  bizonyosabbat, mint a kocska. \\
  Nem dörgölődzik sült lopcska \\
  számhoz s szivemhez kisgyerek --- \\
  ügyeskedhet, nem fog a macska \\
  egyszerre kint s bent egeret.
\end{verse}

\bigskip

\poemtitle*{IV}

\begin{verse}
  Akár egy halom hasított fa, \\
  hever egymáson a világ, \\
  szorítjanyomja, összefogja \\
  egyík dolog a másikát \\
  s így mindenik determinált. \\
  Csak ami nincs, annak van bokra, \\
  csak ami lesz, az a virág, \\
  ami van, széthull darabokra.
\end{verse}

\bigskip

\poemtitle*{V}

\begin{verse}
  A teherpályaudvaron \\
  úgy lapultam a fa tövéhez, \\
  mint egy darab csönd; szürke gyom \\
  ért számhoz, nyers, különös-édes. \\
  Holtan lestem az őrt, mit érez, \\
  s a hallgatag vagónokon \\
  árnyát, mely ráugrott a fényes, \\
  harmatos szénre konokon.
\end{verse}

\newpage

\vspace*{-15mm}
\centeredornament

\selectlanguage{english}

\poemtitle*{VI}

\addcontentsline{toc}{subsection}{VI. \ \emph{Behold the anguish inside}}

\begin{verse}
  Behold the anguish inside, \\
  yet the explanation lies outside. \\
  Your wound is the world \\
  --- on fire, burning up --- \\
  and you feel your soul, the fever. \\
  You are captive as long as your heart rebels \\
  --- so you will be free if it indulges \\
  not building for yourself a house \\
  where a landlord settles in.
\end{verse}

\bigskip

\poemtitle*{VII}

\addcontentsline{toc}{subsection}{VII. \ \emph{I looked up from beneath the evening}}

\begin{verse}
  I looked up from beneath the evening \\
  to the cogwheels of the heavens: \\
  from the glittering threads of chance \\
  the loom of the past wove the law; \\
  and I looked up again onto the sky \\
  from beneath the steam of my dreams, \\
  and I saw the fabric of the law \\
  always tearing up somewhere.
\end{verse}

\bigskip

\poemtitle*{VIII}

\addcontentsline{toc}{subsection}{VIII. \ \emph{The silence listened intently}}

\begin{verse}
  The silence listened intently --- One struck. \\
  You could visit your childhood; \\
  between the damp cement walls \\
  you could imagine a bit of freedom \\
  --- I thought. And as soon as I stood up, \\
  the stars, the Great Bear were glittering above\\
  like the grilles up in the silent cell.
\end{verse}

\newpage

\vspace*{-15mm}
\centeredornament

\selectlanguage{hungarian}

\poemtitle*{VI}

\begin{verse}
  Im itt a szenvedés belül, \\
  ám ott kívül a magyarázat. \\
  Sebed a világ --- ég, hevül \\
  s te lelkedet érzed, a lázat. \\
  Rab vagy, amíg a szíved lázad --- \\
  úgy szabadulsz, ha kényedül \\
  nem raksz magadnak olyan házat, \\
  melybe háziúr települ.
\end{verse}

\bigskip

\poemtitle*{VII}

\begin{verse}
  Én fölnéztem az est alól \\
  az eget fogaskerekére --- \\
  csilló véletlen szálaiból \\
  törvényt szőtt a mult szövőszéke \\
  és megint fölnéztem az égre \\
  álmaim gőzei alól \\
  s láttam, a törvény szövedéke \\
  mindig fölfeslik valahol.
\end{verse}

\bigskip

\poemtitle*{VIII}

\begin{verse}
  Fülelt a csend --- egyet ütött \\
  Fölkereshetnéd ifjúságod; \\
  nyirkos cementfalak között \\
  képzelhetsz egy kis sabadságot --- \\
  gondoltam. S hát hát amint fölállok \\
  a csillagok, a Göncölök \\
  úgy fénylenek fönt, mint a rácsok \\
  a hallgatag cella fölött.
\end{verse}

\newpage

\vspace*{-15mm}
\centeredornament

\selectlanguage{english}

\poemtitle*{IX}

\addcontentsline{toc}{subsection}{IX. \ \emph{I heard iron weeping}}

\begin{verse}
  I heard iron weeping, \\
  I heard rain laughing. \\
  I saw that the past was cracked \\
  and that only memories may be forgotten, \\
  and how I cannot but love, \\
  bending under my burdens \\
  --- why should I also forge a weapon \\
  out of you, golden self-awareness!
\end{verse}

\bigskip

\poemtitle*{X}

\addcontentsline{toc}{subsection}{X. \ \emph{He is a grown man}}

\begin{verse}
  He is a grown man he who has \\
  neither mother nor father in his heart, \\
  he who knows that he receives life \\
  as a supplement to death, and will return it \\
  anytime, like a found object \\
  --- therefore he treasures it, \\
  he who is neither a god nor a priest, \\
  neither to himself nor to others.
\end{verse}

\newpage

\vspace*{-15mm}
\centeredornament

\selectlanguage{hungarian}

\poemtitle*{IX}

\begin{verse}
  Hallottam sírni a vasat, \\
  hallottam az esőt nevetni. \\
  Láttam, hogy a mult meghasadt \\
  s csak képzetet lehet feledni; \\
  s hogy nem tudok mást, mint szeretni, \\
  görnyedve terheim alatt --- \\
  minek is kell fegyvert veretni \\
  belőled, arany öntudat!
\end{verse}

\bigskip

\poemtitle*{X}

\begin{verse}
  Az meglett ember, akinek \\
  szívében nincs se anyja, apja, \\
  ki tudja, hogy az életet \\
  halálra ráadásul kapja \\
  s mint talált tárgyat visszaadja \\
  bármikor --- ezért őrzi meg, \\
  ki nem istene és nem papja \\
  se magának, sem senkinek.
\end{verse}

\newpage

\vspace*{-15mm}
\centeredornament

\selectlanguage{english}

\poemtitle*{XI}

\addcontentsline{toc}{subsection}{XI. \ \emph{I saw happiness}}

\begin{verse}
  I saw happiness; it was soft, bright \\
  and one and half a quintal. \\
  On the coarse grass of the farmyard \\
  its curly smile swayed. \\
  It plopped into the squishy, tepid puddle, \\
  squinted, further grunted at me once --- \\
  To this day I see how hesitantly \\
  the sunlight toyed with its bristles.
\end{verse}

\bigskip

\poemtitle*{XII}

\addcontentsline{toc}{subsection}{XII. \emph{I live by a railroad}}

\begin{verse}
  I live by a railroad. \\
  Many a train comes and goes, \\
  and I watch from afar \\
  how the bright windows fly by \\
  in the wavering, fluffy darkness. \\
  So the lit-up days rush into the eternal night, \\
  and I stand in the compartments' light, \\
  I rest on my elbow and remain silent.
\end{verse}

\newpage

\vspace*{-15mm}
\centeredornament

\selectlanguage{hungarian}

\poemtitle*{XI}

\begin{verse}
  Láttam a boldogságot én, \\
  lágy volt, szőke és másfél mázsa. \\
  Az udvar szigorú gyöpén \\
  imbolygott göndör mosolygása. \\
  Ledőlt a puha, langy tócsába, \\
  hunyorgott, röffent még felém --- \\
  ma is látom, mily tétovázva \\
  babrált pihéi közt a fény.
\end{verse}

\bigskip

\poemtitle*{XII}

\begin{verse}
  Vasútnál lakom. Erre sok \\
  vonat jön-megy és el-elnézem, \\
  hogy' szállnak fényes ablakok \\
  a lengedező szösz-sötétben. \\
  Így iramlanak örök éjben \\
  kivilágított nappalok \\
  s én állok minden fülke-fényben, \\
  én könyöklök és hallgatok.
\end{verse}
