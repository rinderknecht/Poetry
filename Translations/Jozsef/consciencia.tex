\vspace*{-15mm}
\centeredornament
\vspace*{2mm}

\selectlanguage{french}

\begin{center}
  \textbf{\large Consciencia}
\end{center}

\addcontentsline{toc}{section}{Consciencia}

\poemtitle*{I}

\selectlanguage{spanish}

\addcontentsline{toc}{subsection}{I.\ \emph{El amanecer desata el cielo de la tierra}}

\begin{verse}
  El amanecer desata el cielo de la tierra \\
  y, al son de su voz clara y suave, \\
  escarabajos y niños piruetean \\
  al entrar en la luz del día; \\
  el aire no es húmedo, \\
  ¡la brillante levedad flota! \\
  Con la noche se asentaron en los árboles \\
  como pequeñas mariposas, las ojas. \\
\end{verse}

\bigskip

\poemtitle*{II}

\addcontentsline{toc}{subsection}{II.\ \emph{Vi cuadros embadurnados con azul}}

\begin{verse}
  Vi cuadros embadurnados con azul, \\
  rojo y amarillo en mis sueños \\
  y sentí que todo estaba en orden \\
  --- ni una sola mota de polvo volteando. \\
  Ya esfumado, \\
  mi sueño se pone en mis miembros, \\
  y el mundo de hierro es el orden. \\
  Con el día, una luna sale en mí y, \\
  al llegar la noche, un sol brilla aquí dentro. \\
\end{verse}

\newpage

\vspace*{-15mm}
\centeredornament
\vspace*{2mm}

\selectlanguage{hungarian}

\begin{center}
  \textbf{\large Eszmélet}
\end{center}

\poemtitle*{I}

\begin{verse}
  Földtől eloldja az eget \\
  a hajnal s tiszta, lágy szavára \\
  a bogarak, a gyerekek \\
  kipörögnek a napvilágra; \\
  a levegőben semmi pára, \\
  a csilló könnyűség lebeg! \\
  Az éjjel rászálltak a fákra, \\
  mint kis lepkék, a levelek.
\end{verse}

\bigskip

\poemtitle*{II}

\begin{verse}
  Kék, piros, sárga, összekent \\
  képeket láttam álmaimban \\
  és úgy éreztem, ez a rend --- \\
  egy szalló porszem el nem hibbant. \\
  Most homályként száll tagjaimban \\
  álmom s a vas világ a rend. \\
  Nappal hold kél bennem s ha kinn van \\
  az éj --- egy nap süt idebent.
\end{verse}

\newpage

\vspace*{-15mm}
\centeredornament

\selectlanguage{spanish}

\poemtitle*{III}

\addcontentsline{toc}{subsection}{III.\ \emph{Estoy flaco, a veces sólo como pan}}

\begin{verse}
  Estoy flaco, a veces sólo como pan; \\
  rodeado por esas almas ociosas y facundas, \\
  busco en vano más certidumbre, \\
  como el dado. \\
  Ninguna paleta asada halla mi boca \\
  cuando ciño a un niño sobre mi corazón \\
  --- por muy astuto que sea, \\
  el gato no puede pillar a la vez \\
  el ratón de fuera y el ratón de dentro.
\end{verse}

\bigskip

\poemtitle*{IV}

\addcontentsline{toc}{subsection}{IV. \ \emph{Así como un montón de leña}}

\begin{verse}
  Así como un montón de leña, \\
  el mundo yace a granel; \\
  cada cosa aprieta, pesa, \\
  se arrima a la próxima, \\
  y así todo está determinado. \\
  Sólo lo que no es tiene una mata, \\
  sólo lo que será puede florecer; \\
  lo que es caera a pedazos.
\end{verse}

\newpage

\vspace*{-15mm}
\centeredornament

\selectlanguage{hungarian}

\poemtitle*{III}

\begin{verse}
  Sovány vagyok, csak kenyeret \\
  eszem néha, e léha, locska \\
  lelkek közt ingyen keresek \\
  bizonyosabbat, mint a kocska. \\
  Nem dörgölődzik sült lopcska \\
  számhoz s szivemhez kisgyerek --- \\
  ügyeskedhet, nem fog a macska \\
  egyszerre kint s bent egeret.
\end{verse}

\bigskip

\poemtitle*{IV}

\begin{verse}
  Akár egy halom hasított fa, \\
  hever egymáson a világ, \\
  szorítjanyomja, összefogja \\
  egyík dolog a másikát \\
  s így mindenik determinált. \\
  Csak ami nincs, annak van bokra, \\
  csak ami lesz, az a virág, \\
  ami van, széthull darabokra.
\end{verse}

\newpage

\vspace*{-15mm}
\centeredornament

\selectlanguage{spanish}

\poemtitle*{V}

\addcontentsline{toc}{subsection}{V. \ \emph{A la estación de tren de carga}}

\begin{verse}%[\versewidth]
  A la estación de tren de carga, \\
  me tumbé detrás del pie del árbol, \\
  como una masa de silencio; \\
  una hierba gris alcanzó mi boca, \\
  cruda, extraña-dulce. \\
  Haciendome el muerto, miraba al guardia \\
  --- ¿que sentía qué? --- \\
  y, sobre los quietos vagones, a su sombra \\
  que se empeñaba en saltar \\
  sobre los relucientes, rociados carbones. \\
\end{verse}

\bigskip

\poemtitle*{VI}

\addcontentsline{toc}{subsection}{VI. \ \emph{He aquí el tormento interior}}

\begin{verse}
  He aquí el tormento interior, \\
  aunque la explicación yace fuera. \\
  Tu herida es el mundo \\
  --- en llamas, caldeando --- \\
  y sientes tú alma, la fiebre. \\
  Quedas cautivo mientras tu corazón se rebela \\
  --- así serás libre si él se complace \\
  en no edificarte una casa \\
  donde un dueño se aposenta. \\
\end{verse}

\newpage

\vspace*{-15mm}
\centeredornament

\selectlanguage{hungarian}

\poemtitle*{V}

\begin{verse}
  A teherpályaudvaron \\
  úgy lapultam a fa tövéhez, \\
  mint egy darab csönd; szürke gyom \\
  ért számhoz, nyers, különös-édes. \\
  Holtan lestem az őrt, mit érez, \\
  s a hallgatag vagónokon \\
  árnyát, mely ráugrott a fényes, \\
  harmatos szénre konokon.
\end{verse}

\bigskip

\poemtitle*{VI}

\begin{verse}
  Im itt a szenvedés belül, \\
  ám ott kívül a magyarázat. \\
  Sebed a világ --- ég, hevül \\
  s te lelkedet érzed, a lázat. \\
  Rab vagy, amíg a szíved lázad --- \\
  úgy szabadulsz, ha kényedül \\
  nem raksz magadnak olyan házat, \\
  melybe háziúr települ.
\end{verse}

\newpage

\vspace*{-15mm}
\centeredornament

\selectlanguage{spanish}

\poemtitle*{VII}

\addcontentsline{toc}{subsection}{VII. \ \emph{Por debajo del atardecer, levanté los ojos}}

\begin{verse}
  Por debajo del atardecer, levanté los ojos \\
  a las ruedas dentadas de los cielos: \\
  de los hilos brillantes de la suerte \\
  el telar del pasado había tejado la ley; \\
  por debajo del vapor de mis sueños, \\
  miré de nuevo en los cielos \\
  y vi la tela de la ley \\
  siempre destramarse en alguna parte. \\
\end{verse}

\bigskip

\poemtitle*{VIII}

\addcontentsline{toc}{subsection}{VIII. \ \emph{El silencio escuchaba atentamente}}

\begin{verse}
  El silencio escuchaba atentamente \\
  --- La una sonó. \\
  Podrías revisitar tu infancia; \\
  entre las paredes de cemento húmedo \\
  podrías imaginar un poco de libertad \\
  --- pensé. Y al ponerme en pie, \\
  las estrellas, la Osa Mayor \\
  destellaban por encima, \\
  como las rejas arriba en mi celda. \\
\end{verse}

\newpage

\vspace*{-15mm}
\centeredornament

\selectlanguage{hungarian}

\poemtitle*{VII}

\begin{verse}
  Én fölnéztem az est alól \\
  az eget fogaskerekére --- \\
  csilló véletlen szálaiból \\
  törvényt szőtt a mult szövőszéke \\
  és megint fölnéztem az égre \\
  álmaim gőzei alól \\
  s láttam, a törvény szövedéke \\
  mindig fölfeslik valahol.
\end{verse}

\bigskip

\poemtitle*{VIII}

\begin{verse}
  Fülelt a csend --- egyet ütött \\
  Fölkereshetnéd ifjúságod; \\
  nyirkos cementfalak között \\
  képzelhetsz egy kis sabadságot --- \\
  gondoltam. S hát hát amint fölállok \\
  a csillagok, a Göncölök \\
  úgy fénylenek fönt, mint a rácsok \\
  a hallgatag cella fölött.
\end{verse}

\newpage

\vspace*{-15mm}
\centeredornament

\selectlanguage{spanish}

\poemtitle*{IX}

\addcontentsline{toc}{subsection}{IX. \ \emph{Oí el hierro sollozar}}

\begin{verse}
  Oí el hierro sollozar, \\
  oí la lluvia reírse. \\
  Vi que el pasado estaba agrietado \\
  y que solos los recuerdos pueden olvidarse, \\
  y cómo sólo puedo amar, \\
  doblándome debajo de mis cargas --- \\
  ¿por qué debería también fraguar un arma \\
  de ti, fuero interior dorado! \\
\end{verse}

\bigskip

\poemtitle*{X}

\addcontentsline{toc}{subsection}{X. \ \emph{Es un hombre mayor}}

\begin{verse}
  Es un hombre mayor el que no tiene \\
  ni madre ni padre en su corazón, \\
  el que sabe que acoge la vida \\
  como un suplemento a la muerte, \\
  y la devolverá en cualquier momento, \\
  como un objeto encontrado \\
  --- por lo cual la guardará \\
  el que no es ni un diós ni un sacerdote, \\
  ni para él mismo ni para los demás. \\
\end{verse}

\newpage

\vspace*{-15mm}
\centeredornament

\selectlanguage{hungarian}

\poemtitle*{IX}

\begin{verse}
  Hallottam sírni a vasat, \\
  hallottam az esőt nevetni. \\
  Láttam, hogy a mult meghasadt \\
  s csak képzetet lehet feledni; \\
  s hogy nem tudok mást, mint szeretni, \\
  görnyedve terheim alatt --- \\
  minek is kell fegyvert veretni \\
  belőled, arany öntudat!
\end{verse}

\bigskip

\poemtitle*{X}

\begin{verse}
  Az meglett ember, akinek \\
  szívében nincs se anyja, apja, \\
  ki tudja, hogy az életet \\
  halálra ráadásul kapja \\
  s mint talált tárgyat visszaadja \\
  bármikor --- ezért őrzi meg, \\
  ki nem istene és nem papja \\
  se magának, sem senkinek.
\end{verse}

\newpage

\vspace*{-15mm}
\centeredornament

\selectlanguage{spanish}

\poemtitle*{XI}

\addcontentsline{toc}{subsection}{XI. \ \emph{Vi la felicidad}}

\begin{verse}
  Vi la felicidad; era suave, brillante \\
  y un quintal y medio. \\
  En la maleza del patio de la granja \\
  su sonriza curva se balanceaba. \\
  Se abatió en el charco blando y tibio, \\
  entrecerró los ojos, pues me gruñó una vez \\
  --- hasta hoy veo con que vacilación \\
  la luz del día jugueteaba con sus cerdas.
\end{verse}

\bigskip

\poemtitle*{XII}

\addcontentsline{toc}{subsection}{XII. \emph{Vivo cerca del ferrocarril}}

\begin{verse}
  Vivo cerca del ferrocarril. \\
  Son muchos los trenes que vienen y van, \\
  y miro desde lejos \\
  cómo las brillantes ventanas vuelan de paso \\
  en la oscuridad vacilante y vellosa. \\
  Así los días lucientes se apuran \\
  en la noche eterna, \\
  y me quedo en la luz de los compartimentos, \\
  me acodo y guardo silencio.
\end{verse}

\newpage

\vspace*{-15mm}
\centeredornament

\selectlanguage{hungarian}

\poemtitle*{XI}

\begin{verse}
  Láttam a boldogságot én, \\
  lágy volt, szőke és másfél mázsa. \\
  Az udvar szigorú gyöpén \\
  imbolygott göndör mosolygása. \\
  Ledőlt a puha, langy tócsába, \\
  hunyorgott, röffent még felém --- \\
  ma is látom, mily tétovázva \\
  babrált pihéi közt a fény.
\end{verse}

\bigskip

\poemtitle*{XII}

\begin{verse}
  Vasútnál lakom. Erre sok \\
  vonat jön-megy és el-elnézem, \\
  hogy' szállnak fényes ablakok \\
  a lengedező szösz-sötétben. \\
  Így iramlanak örök éjben \\
  kivilágított nappalok \\
  s én állok minden fülke-fényben, \\
  én könyöklök és hallgatok.
\end{verse}
