\documentclass[11pt]{book}

% Geometry
%
\usepackage[papersize={12.85cm,19.84cm},%
            includehead,%
            includefoot,%
            bindingoffset=1cm,%
            showframe,%
            showcrop]{geometry}%, showframe, showcrop

%\setlength{\evensidemargin}{-50pt}
%\setlength{\oddsidemargin}{-50pt}
\setlength{\topmargin}{-50pt}
\setlength{\textheight}{500pt}
%\setlength{\textwidth}{530pt}

\usepackage[T1]{fontenc}
\usepackage[main=francais,hungarian]{babel}
\AddToHook{begindocument}[test]{\catcode\string``=12 }
\DeclareHookRule{begindocument}{test}{before}{babel}
\usepackage{ebgaramond}

\usepackage{microtype}

\usepackage{verse} % \poemlines{1} % Number each verse
\usepackage[final]{pdfpages}
\usepackage{hyphenat}
\usepackage{graphicx}
\usepackage{pgfornament}

\frenchspacing  % Follow French conventions after a period
\hfuzz 2pt      % Do not report overhang less than 2pt

%% TEMPORARY

%% TEMPORARY ($\Box$)

\usepackage{amsmath}
\usepackage{amssymb}

\newcommand{\clearemptydoublepage}{\newpage{\pagestyle{empty}\cleardoublepage}}
\newcommand*{\centeredornament}{\centerline{\pgfornament[width=6cm]{88}}}

% Title and author
%
\title{Le tigre et le chevreuil\\ {\large Une anthologie bilingue}}
\author{{\Large Attila József}\\\\\\
Traduction de Christian Rinderknecht}
\date{\oldstylenums{2025}}

\begin{document}

\includepdf{front_cover}

\thispagestyle{empty}

\mbox{}
\thispagestyle{empty}

\newpage\leavevmode\thispagestyle{empty}

\mbox{}
\thispagestyle{empty}

\maketitle

\thispagestyle{empty}

\vspace*{\fill}

\begin{flushright}
  \textcopyright{} Traduction de Christian Rinderknecht,
  \oldstylenums{2025}.\\
  \textcopyright{} Couverture conçue par Max Wang ($\Box\Box\Box$), \oldstylenums{2025}.
\end{flushright}

\newpage

\vspace*{4cm}

\begin{flushright}
  Cette traduction est dédiée à Réka Dudás.
\end{flushright}

\thispagestyle{empty}

\newpage\leavevmode\thispagestyle{empty}\newpage

\pagenumbering{arabic}

\newpage\leavevmode\thispagestyle{empty}\newpage

%----------------------------------------------------------------------

\vspace*{-15mm}
\centeredornament
\vspace*{7mm}

\selectlanguage{french}

%\settowidth{\versewidth}{}

\begin{verse}%[\versewidth]
  Seul toi devrait lire mon poème, \\
  toi qui me connaît et m'aime, \\
  puisque tu navigues le néant \\
  et sais ce qui adviendra, comme le devin, \\
  \ \\
  car le silence apparut dans tes rêves \\
  sous forme humaine, \\
  et dans ton cœur parfois s'attarde \\
  le tigre et le chevreuil docile. \\
  \ \\
  Début juin \oldstylenums{1936}
\end{verse}

\newpage

\vspace*{-15mm}
\centeredornament
\vspace*{7mm}

\selectlanguage{hungarian}

%\poemtitle*{}

%\settowidth{\versewidth}{}

\begin{verse}%[\versewidth]
  Csak az olvassa versemet, \\
  ki ismer engem és szeret, \\
  mivel a semmiben hajóz \\
  s hogy mi lesz, tudja, mint a jós, \\
  \ \\
  mert álmaiban megjelent \\
  emberi formában a csend \\
  s szivében néha elidőz \\
  a tigris meg a szelid őz. \\
  \ \\
  \oldstylenums{1936} június eleje
\end{verse}

\newpage

\vspace*{-15mm}
\centeredornament
\vspace*{5mm}

\selectlanguage{french}

\poemtitle*{Un homme fatigué}

%\settowidth{\versewidth}{}

\begin{verse}%[\versewidth]
  Dans les champs, des paysans solennels \\
  commencent à rentrer en silence chez eux. \\
  Nous sommes allongés côte à côte, \\
  la rivière et moi; \\
  des herbes tendres dorment sous mon cœur. \\
  \ \\
  Une vaste sérénité coule avec la calme rivière, \\
  soucis et fardeaux me quittent \\
  pour devenir rosée; \\
  ni homme ni enfant, ni Hongrois ni frère, \\
  seulement un homme fatigué, allongé ici. \\
  \ \\
  Le soir prodigue la paix, \\
  je suis une tranche de son pain chaud; \\
  le ciel aussi se repose \\
  sur la tranquille rivière Maros, \\
  et elles s'assoient dehors sur mon front, \\
  les étoiles. \\
  \ \\
  Août \oldstylenums{1923}
\end{verse}

\newpage

\vspace*{-15mm}
\centeredornament
\vspace*{5mm}

\selectlanguage{hungarian}

\poemtitle*{Megfáradt ember}

%\settowidth{\versewidth}{}

\begin{verse}%[\versewidth]
  A földeken néhány komoly paraszt \\
  hazafelé indul hallgatag. \\
  Egymás mellett fekszünk: a folyó meg én, \\
  gyenge füvek alusznak a szívem alatt. \\
  \ \\
  A folyó csöndes, nagy nyugalmat görget, \\
  harmattá vált bennem a gond és teher; \\
  se férfi, se gyerek, se magyar, se testvér, \\
  csak megfáradt ember, aki itt hever. \\
  \ \\
  A békességet szétosztja az este, \\
  meleg kenyeréből egy karaj vagyok, \\
  pihen most az ég is, a nyugodt Marosra \\
  s homlokomra kiülnek a csillagok. \\
  \ \\
  \oldstylenums{1923} august
\end{verse}

\newpage

\vspace*{-15mm}
\centeredornament
\vspace*{5mm}

\selectlanguage{french}

%\settowidth{\versewidth}{et maintenant la lumière suave de mon pain est plus pure.}

\begin{verse}%[\versewidth]
  Ne sois pas si bête. \\
  Tu cours comme le vent matinal, \\
  un jour tu seras renversée par une auto. \\
  D'ailleurs, j'ai récuré ma petite table, \\
  et maintenant la lumière suave de mon pain est plus pure. \\
  Hé bien, reviens; si tu veux \\
  j'achèterai une couverture pour mon lit de fer. \\
  Une couverture ordinaire, grise. \\
  Elle conviendra à ma Pauvreté, qui t'aime, \\
  et le Seigneur t'aime aussi beaucoup \\
  et Il m'aime aussi. \\
  Le Seigneur ne vient jamais dans toute sa splendeur, \\
  Il ne veut pas abîmer mes yeux, \\
  qui ont hâte de te voir \\
  et qui te regarderont avec beauté. \\
  Quand tu reviendras, \\
  je t'embrasserai doucement, \\
  je n'arracherai pas ton manteau. \\
  Je te raconterai toutes les nouvelles blagues, \\
  parce que j'en ai inventé beaucoup depuis, \\
  et que tu seras gaie et rougieras, \\
  et tu baisseras les yeux vers le sol, \\
  et nous nous esclafferons, \\
  et nos voisins nous entendrons \\
  et jusqu'aux journaliers taciturnes et austères \\
  qui, parmi leurs rêves fatigués et brisés, \\
  esquisseront un sourire aussi. \\
  \ \\
  Automne \oldstylenums{1925}
\end{verse}

\newpage

\vspace*{-15mm}
\centeredornament
\vspace*{5mm}

\selectlanguage{hungarian}

%\settowidth{\versewidth}{a szótlan, komoly napszámosokhoz is behallik}

\begin{verse}%[\versewidth]
  Olyan bolond vagy \\
  szaladsz \\
  akár a reggeli szél. \\
  Még elüt valamelyik autó. \\
  Pedig lesikáltam kis asztalomat \\
  és most \\
  tisztábban világít kenyerem enyhe fénye. \\
  No gyere vissza, ha akarod \\
  veszek takarót vaságyamra. \\
  Egyszerű, szürke takarót. \\
  Illik az \\
  szegénységemhez, aki szeret téged \\
  és az Úr is szereti nagyon \\
  és engem is szeret az Úr \\
  nem jön soha nagy fényességgel \\
  Nem akarja, hogy elromoljanak \\
  szemeim, akik \\
  nagyon kívánnak látni téged. \\
  És nagyon szépen néznek majd terád \\
  ha visszajössz \\
  vigyázva foglak megcsókolni, \\
  nem tépem le rólad a kabátot \\
  és elmondom mind a sok tréfát, \\
  mert sokat kieszeltem azóta, \\
  hogy te is örülj, \\
  majd elpirulsz, \\
  lenézel a földre és kacagunk \\
  hangosan, hogy behallatszik szomszédunkba \\
  a szótlan, komoly napszámosokhoz is behallik \\
  és fáradt, összetört \\
  álmukban majd elmosolyodnak ők is. \\
  \ \\
  1925 Ősze
\end{verse}

\newpage

\vspace*{-15mm}
\centeredornament
\vspace*{5mm}

\selectlanguage{french}

%% % Note: The edition by \emph{Szépirodalmi könyvkiadó}, page~169, misses
%% % two verses and a conjunction.

%\settowidth{\versewidth}{What shard of the vast nighttime}

\begin{verse}%[\versewidth]
  Trains de fret aiguillés; \\
  le cliquetis onirique \\
  passe de légères menottes \\
  au paysage muet. \\
  \ \\
  La lune jaillit sans effort, \\
  comme un prisonnier libéré. \\
  \ \\
  Les pierres concassées reposent \\
  dans leur ombre propre, \\
  elles scintillent pour elles-mêmes, \\
  elles sont à leur place \\
  comme jamais avant. \\
  \ \\
  Quel éclat de la vaste nuit \\
  est cette lourde nuitée, \\
  qu'elle tombe sur nous \\
  comme un fer sur la poussière? \\
  \ \\
  Désir né du soleil! \\
  Quand l'ombre couvre le lit, \\
  pourrais-tu aussi veiller toute la nuit? \\
  \ \\
  Devant l'entrepôt \\
  une lampe poussiéreuse brûle. \\
  Elle est seulement visible, pas lumineuse; \\
  ainsi est le vœu pieux: il cligne vivement, \\
  mais le ciel est une grande lumière morte. \\
  \ \\
  \oldstylenums{1933}
\end{verse}

\newpage

\vspace*{-15mm}
\centeredornament
\vspace*{5mm}

\selectlanguage{hungarian}

%\settowidth{\versewidth}{Ha majd árnyat fogad az ágy,}

\begin{verse}%[\versewidth]
  Tehervonatok tolatnak, \\
  a méla csörömpölés \\
  könnyű bilincseket rak \\
  a néma tájra. \\
  \ \\
  Oly könnyen száll a hold, \\
  mint a fölszabadult. \\
  \ \\
  A megtört kövek \\
  önnön árnyukon fekszenek, \\
  csillognak \\
  maguknak, \\
  úgy a helyükön vannak, \\
  mint még soha. \\
  \ \\
  Milyen óriás éjszaka \\
  szilánkja ez a sulyos éj, \\
  mely úgy hull le ránk, \\
  mint a porra a vasszilánk? \\
  \ \\
  Napszülte vágy! \\
  Ha majd árnyat fogad az ágy, \\
  abban az egész éjben \\
  is ébren \\
  maradnál? \\
  \ \\
  A raktár \\
  előtt poros lámpa ég. \\
  Csak látszik, nem világit, \\
  ilyen az ész, ha áhit. \\
  Pislog élénken, holott \\
  nagy halott \\
  fény az ég. \\
  \ \\
  \oldstylenums{1933}
\end{verse}

\newpage

\vspace*{-15mm}
\centeredornament
\vspace*{5mm}

\selectlanguage{french}

\begin{center}
  \textbf{Conscience (\oldstylenums{1934})}
\end{center}

\poemtitle*{I}

%\settowidth{\versewidth}{Avec la nuit, elles se posèrent sur les arbres}

\begin{verse}%[\versewidth]
  L'aube détache le ciel de la terre \\
  et, au son de sa voix claire et douce, \\
  scarabées et enfants pirouettent \\
  en entrant dans la lumière du jour; \\
  l'air n'est pas humide, la brillante légèreté flotte!\hspace*{-4mm} \\
  Avec la nuit, elles se posèrent sur les arbres \\
  comme de petits papillons, les feuilles. \\
\end{verse}

\bigskip

\poemtitle*{II}

%\settowidth{\versewidth}{Maintenant mon rêve descend, estompé, dans mes membres}

\begin{verse}%[\versewidth]
  J'ai vu des tableaux barbouillés de bleu, \\
  rouge et jaune dans mes rêves \\
  et je sentis que tout était en ordre --- \\
  pas un seul grain de poussière qui virevolte. \\
  Maintenant mon rêve descend, estompé, dans mes membres, \\
  et le monde de fer est l'ordre. \\
  Avec le jour, une lune point en moi et, \\
  à la nuit tombée, un soleil brille ci-dedans. \\
\end{verse}

\newpage

\vspace*{-15mm}
\centeredornament
\vspace*{5mm}

\selectlanguage{hungarian}

\begin{center}
  \textbf{Eszmélet (\oldstylenums{1934})}
\end{center}

\poemtitle*{I}

\settowidth{\versewidth}{a hajnal s tiszta, lágy szavára}

\begin{verse}%[\versewidth]
  Földtől eloldja az eget \\
  a hajnal s tiszta, lágy szavára \\
  a bogarak, a gyerekek \\
  kipörögnek a napvilágra; \\
  a levegőben semmi pára, \\
  a csilló könnyűség lebeg! \\
  Az éjjel rászálltak a fákra, \\
  mint kis lepkék, a levelek.
\end{verse}

\bigskip

\poemtitle*{II}

\settowidth{\versewidth}{Nappal hold kél bennem s ha kinn van}

\begin{verse}%[\versewidth]
  Kék, piros, sárga, összekent \\
  képeket láttam álmaimban \\
  és úgy éreztem, ez a rend --- \\
  egy szalló porszem el nem hibbant. \\
  Most homályként száll tagjaimban \\
  álmom s a vas világ a rend. \\
  Nappal hold kél bennem s ha kinn van \\
  az éj --- egy nap süt idebent.
\end{verse}

\newpage

\vspace*{-15mm}
\centeredornament
%\vspace*{5mm}

\selectlanguage{french}

\poemtitle*{III}

%\settowidth{\versewidth}{aussi fûté soit-il, le chat ne peut attrapper d'un coup}

\begin{verse}%[\versewidth]
  Je suis maigre, parfois je ne mange que du pain; \\
  entouré par ces âmes oisives et bavardes, \\
  je cherche en vain plus de certitude, \\
  comme le dé. \\
  Aucun rôti de palette ne trouve ma bouche \\
  quand j'étreins un enfant sur mon cœur --- \\
  aussi fûté soit-il, le chat ne peut attrapper d'un coup \\
  la souris dehors et la souris dedans. \\
\end{verse}

\bigskip

\poemtitle*{IV}

%\settowidth{\versewidth}{Seul ce qui n'est pas a un arbrisseau,}

\begin{verse}%[\versewidth]
  Tout comme un tas de bûches, \\
  le monde gît en vrac; \\
  chaque chose presse, pèse, \\
  s'arrime à l'autre, \\
  et ainsi tout est déterminé. \\
  Seul ce qui n'est pas a un arbrisseau, \\
  seul ce qui sera peut fleurir; \\
  ce qui est tombera en pièces. \\
\end{verse}

\newpage

\vspace*{-15mm}
\centeredornament
%\vspace*{-7mm}

\selectlanguage{hungarian}

\poemtitle*{III}

\settowidth{\versewidth}{számhoz s szivemhez kisgyerek ---}

\begin{verse}%[\versewidth]
  Sovány vagyok, csak kenyeret \\
  eszem néha, e léha, locska \\
  lelkek közt ingyen keresek \\
  bizonyosabbat, mint a kocska. \\
  Nem dörgölődzik sült lopcska \\
  számhoz s szivemhez kisgyerek --- \\
  ügyeskedhet, nem fog a macska \\
  egyszerre kint s bent egeret.
\end{verse}

\bigskip

\poemtitle*{IV}

\settowidth{\versewidth}{Csak ami nincs, annak van bokra,}

\begin{verse}%[\versewidth]
  Akár egy halom hasított fa, \\
  hever egymáson a világ, \\
  szorítjanyomja, összefogja \\
  egyík dolog a másikát \\
  s így mindenik determinált. \\
  Csak ami nincs, annak van bokra, \\
  csak ami lesz, az a virág, \\
  ami van, széthull darabokra.
\end{verse}

\newpage

\vspace*{-15mm}
\centeredornament
%\vspace*{5mm}

\selectlanguage{french}

\poemtitle*{V}

%\settowidth{\versewidth}{s'entêtait à bondir sur les charbons reluisants, couverts de rosée.}

\begin{verse}%[\versewidth]
  À la gare de fret, \\
  je m'étalai derrière le pied de l'arbre, \\
  comme une masse de silence; \\
  une herbe grise atteignit ma bouche, \\
  crue, étrange-sucrée. \\
  Faisant le mort, je regardais le garde \\
  --- ressentant quoi? --- \\
  et son ombre qui, sur les wagons silencieux, \\
  s'entêtait à bondir sur les charbons reluisants,\hspace*{-4mm} \\
  couverts de rosée. \\
\end{verse}

\bigskip

\poemtitle*{VI}

%\settowidth{\versewidth}{Ta blessure est le monde --- en feu, s'échauffant ---}

\begin{verse}%[\versewidth]
  Voici le tourment intérieur, \\
  pourtant l'explication gît à l'extérieur. \\
  Ta blessure est le monde \\
  --- en feu, s'échauffant --- \\
  et tu sens ton âme, la fièvre. \\
  Tu es captif tant que ton cœur se révolte \\
  --- ainsi tu seras libre s'il se complait \\
  à ne pas bâtir pour toi une maison \\
  où un propriétaire vient demeurer. \\
\end{verse}

\newpage

\vspace*{-15mm}
\centeredornament
%\vspace*{-7mm}

\selectlanguage{hungarian}

\poemtitle*{V}

%\settowidth{\versewidth}{mint egy darab csönd; szürke gyom}

\begin{verse}%[\versewidth]
  A teherpályaudvaron \\
  úgy lapultam a fa tövéhez, \\
  mint egy darab csönd; szürke gyom \\
  ért számhoz, nyers, különös-édes. \\
  Holtan lestem az őrt, mit érez, \\
  s a hallgatag vagónokon \\
  árnyát, mely ráugrott a fényes, \\
  harmatos szénre konokon.
\end{verse}

\bigskip

\poemtitle*{VI}

%\settowidth{\versewidth}{nem raksz magadnak olyan házat,}

\begin{verse}%[\versewidth]
  Im itt a szenvedés belül, \\
  ám ott kívül a magyarázat. \\
  Sebed a világ --- ég, hevül \\
  s te lelkedet érzed, a lázat. \\
  Rab vagy, amíg a szíved lázad --- \\
  úgy szabadulsz, ha kényedül \\
  nem raksz magadnak olyan házat, \\
  melybe háziúr települ.
\end{verse}

\newpage

\vspace*{-15mm}
\centeredornament
%\vspace*{5mm}

\selectlanguage{french}

\poemtitle*{VII}

%\settowidth{\versewidth}{j'ai levé les yeux aux rouages des cieux: }

\begin{verse}%[\versewidth]
  Par dessous le soir, \\
  j'ai levé les yeux aux rouages des cieux: \\
  des fils scintillants de la chance \\
  le métier du passé avait tissé la loi; \\
  par dessous la vapeur de mes rêves, \\
  j'ai regardé à nouveau dans les cieux \\
  et j'ai vu le tissu de la loi \\
  toujours se déchirer quelque part. \\
\end{verse}

\bigskip

\poemtitle*{VIII}

%\settowidth{\versewidth}{Le silence écoutait attentivement --- Une heure sonna.}

\begin{verse}%[\versewidth]
  Le silence écoutait attentivement \\
  --- Une heure sonna. \\
  Tu pourrais visiter ton enfance; \\
  entre les murs de ciment humide \\
  tu pourrais imaginer un peu de liberté \\
  --- me dis-je. Et dès que je fus sur pied, \\
  les étoiles, la Grande Ourse \\
  scintillaient au-dessus, \\
  comme les grilles en haut dans ma cellule. \\
\end{verse}

\bigskip

\poemtitle*{IX}

%\settowidth{\versewidth}{et que seuls les souvenirs peuvent s'oublier,}

\begin{verse}%[\versewidth]
  J'ai entendu le fer sangloter, \\
  j'ai entendu la pluie rire. \\
  Je vis que le passé était craquelé \\
  et que seuls les souvenirs peuvent s'oublier, \\
  et comment je ne peux qu'aimer, \\
  pliant sous mes fardeaux --- \\
  pourquoi devrais-je aussi forger une arme \\
  de toi, for intérieur doré! \\
\end{verse}

\newpage

\vspace*{-15mm}
\centeredornament
%\vspace*{5mm}

\selectlanguage{hungarian}

\poemtitle*{VII}

%\settowidth{\versewidth}{törvényt szőtt a mult szövőszéke}

\begin{verse}%[\versewidth]
  Én fölnéztem az est alól \\
  az eget fogaskerekére --- \\
  csilló véletlen szálaiból \\
  törvényt szőtt a mult szövőszéke \\
  és megint fölnéztem az égre \\
  álmaim gőzei alól \\
  s láttam, a törvény szövedéke \\
  mindig fölfeslik valahol.
\end{verse}

\bigskip

\poemtitle*{VIII}

%\settowidth{\versewidth}{gondoltam. S hát hát amint fölállok}

\begin{verse}%[\versewidth]
  Fülelt a csend --- egyet ütött \\
  Fölkereshetnéd ifjúságod; \\
  nyirkos cementfalak között \\
  képzelhetsz egy kis sabadságot --- \\
  gondoltam. S hát hát amint fölállok \\
  a csillagok, a Göncölök \\
  úgy fénylenek fönt, mint a rácsok \\
  a hallgatag cella fölött.
\end{verse}

\bigskip

\poemtitle*{IX}

\settowidth{\versewidth}{s hogy nem tudok mást, mint szeretni,}

\begin{verse}%[\versewidth]
  Hallottam sírni a vasat, \\
  hallottam az esőt nevetni. \\
  Láttam, hogy a mult meghasadt \\
  s csak képzetet lehet feledni; \\
  s hogy nem tudok mást, mint szeretni, \\
  görnyedve terheim alatt --- \\
  minek is kell fegyvert veretni \\
  belőled, arany öntudat!
\end{verse}

\newpage

\vspace*{-15mm}
\centeredornament
%\vspace*{5mm}

\selectlanguage{french}

\poemtitle*{X}

\settowidth{\versewidth}{tel un supplément à la mort, et qu'il la rendra}

\begin{verse}%[\versewidth]
  Il est un homme accompli celui qui n'a \\
  ni mère ni père en son cœur, \\
  celui qui sait qu'il reçoit la vie \\
  tel un supplément à la mort, et la rendra \\
  à tout moment comme un objet trouvé \\
  --- par conséquent il la garde, \\
  celui qui n'est ni un dieu ni un prêtre, \\
  ni pour lui-même ni pour autrui. \\
\end{verse}

\bigskip

\poemtitle*{XI}

%\settowidth{\versewidth}{Il s'affala dans la marre tende et tiède,}

\begin{verse}%[\versewidth]
  J'ai vu le bonheur; il était doux, brillant \\
  et un quintal et demi. \\
  Sur la mauvaise herbe de la cour de ferme \\
  son sourire courbé se balançait. \\
  Il s'affala dans la marre tendre et tiède, \\
  plissa les yeux, puis me grogna une fois --- \\
  Jusqu'à ce jour, je vois avec quelle hésitation \\
  la lumière du jour s'amusait avec ses soies. \\
\end{verse}

\bigskip

\poemtitle*{XII}

%\settowidth{\versewidth}{Ainsi les jours luisants se pressent dans la nuit éternelle,}

\begin{verse}%[\versewidth]
  Je vis près du chemin de fer. \\
  Nombreux sont les trains \\
  qui viennent et vont, et j'observe de loin \\
  comment les fenêtres illuminées défilent \\
  dans l'obscurité vacillante et peluchée. \\
  Ainsi les jours luisants se pressent dans la nuit éternelle, \\
  et je me tiens dans la lueur des compartiments,\hspace*{-4mm} \\
  je m'accoude et garde le silence. \\
\end{verse}

\newpage

\vspace*{-15mm}
\centeredornament
%\vspace*{5mm}

\selectlanguage{hungarian}

\poemtitle*{X}

\settowidth{\versewidth}{s mint talált tárgyat visszaadja}

\begin{verse}%[\versewidth]
  Az meglett ember, akinek \\
  szívében nincs se anyja, apja, \\
  ki tudja, hogy az életet \\
  halálra ráadásul kapja \\
  s mint talált tárgyat visszaadja \\
  bármikor --- ezért őrzi meg, \\
  ki nem istene és nem papja \\
  se magának, sem senkinek.
\end{verse}

\bigskip

\poemtitle*{XI}

%\settowidth{\versewidth}{hunyorgott, röffent még felém ---}

\begin{verse}%[\versewidth]
  Láttam a boldogságot én, \\
  lágy volt, szőke és másfél mázsa. \\
  Az udvar szigorú gyöpén \\
  imbolygott göndör mosolygása. \\
  Ledőlt a puha, langy tócsába, \\
  hunyorgott, röffent még felém --- \\
  ma is látom, mily tétovázva \\
  babrált pihéi közt a fény.
\end{verse}

\bigskip

\poemtitle*{XII}

%\settowidth{\versewidth}{s én állok minden fülke-fényben,}

\begin{verse}%[\versewidth]
  Vasútnál lakom. Erre sok \\
  vonat jön-megy és el-elnézem, \\
  hogy' szállnak fényes ablakok \\
  a lengedező szösz-sötétben. \\
  Így iramlanak örök éjben \\
  kivilágított nappalok \\
  s én állok minden fülke-fényben, \\
  én könyöklök és hallgatok.
\end{verse}

\newpage

\vspace*{-15mm}
\centeredornament
\vspace*{5mm}

\selectlanguage{french}

%\settowidth{\versewidth}{}

\begin{verse}%%[\versewidth]
  Tu vieilliras et regretteras combien tu m'as blessé \\
  --- ce dont tu t'enorgueillis aujourd'hui. \\
  La conscience viendra heurter à la porte \\
  et ne te laissera seule dans aucun souvenir. \\
  \ \\
  Tu auras un vieux chien qui s'installera à tes côtés. \\
  Tu te reposeras durant le jour, t'assoupissant sur une chaise \\
  parce que tu as peur seule la nuit. \\
  Les ombres couvriront la vieille dame tremblotante. \\
  \ \\
  Ton vieux chien gémira parfois, \\
  mais la pièce redevient silencieuse, \\
  tout est en ordre, \\
  et pourtant un être passé \\
  te manque dans le silence solitaire. \\
  \ \\
  Tu chancelleras et quand ta mauvaise jambe aura assez titubé, \\
  tu t'asseyeras. Ton portrait de jeunesse trône \\
  dans un cadre doré. Tu lui marmonneras: \\
  «~Je ne l'ai pas serré dans mes bras, c'est que je ne l'aimais pas.~» \\
  \ \\
  «~Qu'aurais-je pu faire?~» --- tu demandes,\hspace*{-4mm} \\
  mais ta bouche édentée ne peut plus répondre;\hspace*{-4mm} \\
  face au soleil dehors, tu fermes les yeux, \\
  tu peux à peine attendre que la lune se lève. \\
  \ \\
  \hfill \textbf{\ldots}
\end{verse}

\newpage

\vspace*{-15mm}
\centeredornament
\vspace*{5mm}

\selectlanguage{hungarian}

%\poemtitle*{}

%\settowidth{\versewidth}{s nem lesz emlék, melyben magadra hagyna.}

\begin{verse}%[\versewidth]
  Majd megöregszel és bánni fogod, \\
  hogy bántasz --- azt, amire büszke vagy ma. \\
  A lelkiismeret majd bekopog \\
  s nem lesz emlék, melyben magadra hagyna. \\
  \ \\
  Lesz vén ebed s az melléd települ. \\
  Nappal pihensz majd, széken szunyókálva, \\
  mert éjjel félni fogsz majd egyedül. \\
  Árnyak ütnek a rezgő anyókára. \\
  \ \\
  Az öreg kutya néha majd nyafog, \\
  de a szobában csend lesz, csupa rend lesz; \\
  hanem valaki hiányozni fog \\
  a multból ahhoz a magányos csendhez. \\
  \ \\
  Majd tipegsz s ha eleget totyogott \\
  rossz lábod, leülsz. Fönn aranykeretben \\
  áll ifju képed. Hozzá motyogod: \\
  \textqq{Nem öleltem meg, hiszen nem szerettem.} \\
  \ \\
  \textqq{Mit is tehettem volna?} --- kérdezed, \\
  de fogatlan szád már nem válaszolhat; \\
  s ki a nap előtt lehunyod szemed, \\
  alig várod, hogy feljöjjön, a holdat. \\
  \ \\
  \hfill \textbf{\ldots}
\end{verse}

\newpage

\selectlanguage{french}

\begin{verse}%%[\versewidth]
  Car quand tu t'endors, le lit bondit \\
  tel un poulain qui tente de se défaire de son harnais. \\
  Et la peur, non le désir, occupera ta tête: \\
  devrais-tu l'aimer, devrais-tu ne pas l'aimer? \\
  \ \\
  Décide toi-même. Je suis au regret \\
  de ne pouvoir répondre si tu demandes: \\
  est-il vivant? \\
  Parce que la peine dedans est fatiguée, \\
  elle s'endort comme un enfant, et moi avec. \\
  \ \\
  Novembre \oldstylenums{1936}
\end{verse}

\newpage

\selectlanguage{hungarian}

\begin{verse}%[\versewidth]
  Mert ha elalszol, ugrál majd az ágy, \\
  mint a csikó, hogy a hámot levesse. \\
  S a félelem tünődik, nem a vágy, \\
  a fejedben: Szeress-e, ne szeress-e. \\
  \ \\
  Magadban döntöd el. Én fájlalom, \\
  hogy nem felelhetek, ha kérded: él-e. \\
  Mert elfárad bennem a fájdalom, \\
  elalszik, mint a gyermek s én is véle. \\
  \ \\
  \oldstylenums{1936} november
\end{verse}


%----------------------------------------------------------------------

\newpage

\vspace*{-15mm}
\centeredornament
%\vspace*{5mm}

\section*{Glossaire en anglais}
\selectlanguage{english}

\noindent \textbf{\large A} \\
\textbf{abban} \emph{az} `that' + innes. \emph{-ban} `in' \\
\textbf{ablakok} \emph{ablak} `window' + pl. \emph{-ok} \\
\textbf{ahhoz} \emph{az} `that' + allat. \emph{-hoz} `towards' \\
\textbf{ahol} art. \emph{az} + interr. adv. \emph{hol} `where?' \\
\textbf{akár} lit. `just like' \\
\textbf{akarja} vb. `to want' + 3rd
  pers. sg. ind. pres. indef. \emph{-ja} \\
\textbf{akarod} vb. \emph{akar} `to want' + 2nd
pers. sg. ind. pres. def. \emph{-od} \\
\textbf{aki} conj. `who' \\
\textbf{akik} pron. `who' + pl. \emph{-k} \\
\textbf{akinek} pron. \emph{aki} `who' + attr. poss. \emph{-nek} \\
\textbf{alatt} post. \emph{al-} `lower part, below' +
locat. \emph{-att} \\
\textbf{alig} adv. `barely' \\
\textbf{alól} \emph{al} `lower part' + abl. \emph{-l} `from' \\
\textbf{alusznak} vb. \emph{alszik} `to sleep' $\rightarrow$ `to fall
asleep' + 3rd pers. pl. ind. pres. indef. \emph{-nak} \\
\textbf{ami} rel. pron. `which, that' \\
\textbf{amíg} adv. `as long as'; `until' if followed by negated
verb \\
\textbf{amint} \emph{az} + \emph{mint} `like': `as soon as', lit. `as' \\
\textbf{amire} \emph{ami} `that, which' + subl. \emph{-re} `onto' \\
\textbf{annak} art. \emph{az} `that' + attr. poss. \emph{-nak}
  $\rightarrow$ \emph{annak} \\
\textbf{anyja} \emph{anya} `mother' + 3rd
pers. sg. single\hyp{}poss. poss. \emph{-ja} \\
\textbf{anyókára} \emph{anyó} `elderly woman' + dim. \emph{-ka} (term
of endearment; rare. term of address to the wife by her husband in old
age) + subl. \emph{-ra} `on[to]' \\
\textbf{apja} \emph{apa} `father' + 3rd
pers. sg. single\hyp{}poss. poss. \emph{-ja} \\
\textbf{arany} `gold, golden' \\
\textbf{aranykeretben} \emph{arany} `gold' + \emph{keret} `frame' +
innes. \emph{-ben} `in' \\
\textbf{asztalomat} \emph{asztal} `table' + 1st
pers. sg. single\hyp{}poss. poss. \emph{-om} + acc. \emph{-at} \\
\textbf{autó} `car' \\
\textbf{azért} art. \emph{az} `that' + causal \emph{-ért}
$\rightarrow$ adv. `for that reason, that is why' \\
\textbf{azóta} art. \emph{az} `that' + \emph{óta} `since'
$\rightarrow$ `since then' \\
\textbf{azt} \emph{az} `that' + acc. \emph{-t} \\
%%
\ \\
\noindent \textbf{\large Á} \\
\textbf{ágy} `bed' \\
\textbf{áhit} vb. `to wish (for smth)' \\
\textbf{áll} vb. `to stand' \\
\textbf{állok} vb. \emph{áll} `to stand' + 1st
pers. sg. ind. pres. indef. \emph{-ok} \\
\textbf{álmukban} \emph{álom} `dream'+ 3rd
pers. pl. single\hyp{}poss. poss. \emph{-uk} + innes. \emph{-ban} `in' \\
\textbf{álmaiban} \emph{álom} `dream' + 2nd
pers. sg. inform. poss. \emph{-a} + poss. pl. \emph{-i} +
innes. \emph{-ban} `in' $\rightarrow$ \emph{álmaiban} \\
\textbf{álmaim} \emph{álom} `dream' + 1st pers. sg. poss. \emph{-a} +
poss. pl. \emph{-i} + 1st pers. sg. pers. \emph{-m} \\
\textbf{álmaimban} \emph{álom} `dream' + 1st pers. sg. poss. \emph{-a}
+ poss. pl. \emph{-i} + 1st pers. sg. pers. \emph{-m} +
  innes. \emph{-ban} `in' $\rightarrow$ \emph{álmaimban} \\
\textbf{ám} poetic, interj. contrad. `but' \\
\textbf{árnyak} \emph{árny} `shadow' + nom. pl. \emph{-ak} \\
\textbf{árnyat} \emph{árny} `shadow' + acc. \emph{-t} \\
\textbf{árnyát} \emph{árny} `shadow' + 3rd
  pers. sg. single\hyp{}poss. poss. \emph{-a} + acc. \emph{-t} $\rightarrow$ \emph{-át} \\
\textbf{árnyukon} \emph{árny} `shadow' + 3rd
pers. pl. single\hyp{}poss. poss. \emph{-uk} + super. \emph{-on} `on' \\
%%
\ \\
\noindent \textbf{\large B} \\
\textbf{babrált} vb. \emph{babrál} `to fiddle' \\
\textbf{barátom} \emph{barát} `friend' + 1st
pers. sg. single\hyp{}poss. poss. \emph{-om} \\
\textbf{bánni} vb. \emph{megbán} `to regret':
perfect. vb. pref. \emph{meg-} + vb. \emph{bán} `to mind (negatively),
to be bothered' \\
\textbf{bántasz} vb. \emph{bán} `to be bothered' + caus. \emph{-t}
$\rightarrow$ `to hurt' + 2nd pers. sg. ind. pres. indef. \\
\textbf{bántottalak} vb. \emph{bán} `to mind (negatively),
to be bothered' + caus. \emph{-t} $\rightarrow$ \emph{bánt} `to hurt',
inf. \emph{bántani} + 1st pers. sg. ind. pres. 2nd obj. \\
\textbf{bármikor} \emph{bár} `any' + adv. \emph{mikor} `when'
  $\rightarrow$ `any time' \\
\textbf{behallatszik} vb. pref. \emph{be-} `inwardly' +
vb. \emph{hall} `to hear' + caus. \emph{-at} +
vb.-forming \emph{-szik} $\rightarrow$ `to cause to sound in' \\
\textbf{behallik} syn. \emph{behallatszik} \\
\textbf{bekopog} vb. pref. \emph{be-} `inwardly' + vb. \emph{kopog}
`to knock' \\
\textbf{belénk} pron. \emph{belé} (folksy, syn. \emph{bele}) + 1st
pers. pl $\rightarrow$ `with us, into us' \\
\textbf{belőled} pron. \emph{belőle} `out of him' + 2nd
pers. sg. \emph{-d} \\
\textbf{belül} `inside' \\
\textbf{bennem} pron. \emph{benne} `inside smth, smb' + 1st
pers. sg. pers. \emph{-m} \\
\textbf{bent} adv. `inside' \\
\textbf{békés} \emph{béke} `peace' + adj.\hyp{}forming \emph{-s}
$\rightarrow$ `peaceful' \\
\textbf{békességet} \emph{béke} `peace' + abst. noun\hyp{}forming
  \emph{-ség} + acc. \emph{-et} \\
\textbf{bilincseket} \emph{bilincs} `handcuff, shackle' +
pl. \emph{-ek} + acc. \emph{-et} \\
\textbf{bizonyosabbat} adj. \emph{bizonyos} `certain' +
comp. \emph{-abb} `more' + acc. \emph{-t} \\
\textbf{bocsánatot} \emph{bocsánat} `pardon' + acc. \emph{-ot} \\
\textbf{bogarak} \emph{bogár} `beetle' + pl. \emph{-ak}
  $\rightarrow$ \emph{bogarak} \\
\textbf{bogáncsokat} \emph{bogáncs} `thistle' + pl. \emph{-ok} +
acc. \emph{-at} \\
\textbf{bokra} \emph{bokor} `shrub, bush' + 3rd
pers. sg. single\hyp{}poss. poss. \emph{-a} \\
\textbf{boldogságot} adj. \emph{boldog} `happy' + noun\hyp{}forming
  \emph{-ság} (state of being) + acc. \emph{-ot} \\
\textbf{bolond} adj. `silly, foolish' \\
\textbf{büszke} adj. `proud' \\
%%
\ \\
\noindent \textbf{\large C} \\
\textbf{cella} `cell' \\
\textbf{cementfalak} \emph{cement} `cement'+ \emph{fal} `wall' +
pl. \emph{-ak} \\
\textbf{csak} adv. `only' \\
\textbf{csend} `silence'. alt. \emph{csönd} \\
\textbf{csendhez} `silence' + allat. \emph{-hez} `to' \\
\textbf{csikó} `foal' \\
\textbf{csillagok} \emph{csillag} `star' + pl. \emph{-ok} \\
\textbf{csillognak} vb. \emph{csillog} `to glitter' + 3rd
pers. pl. ind. pres. indef. \\
\textbf{csilló} vb. \emph{csillog} `to glitter' +
  adj.-forming \emph{-o} $\rightarrow$ \emph{csilló};
  usually \emph{csillogó} \\
\textbf{csönd} `silence', alt. \emph{csend} \\
\textbf{csöndes} adj. sg. `silent', alt. \emph{csendes} \\
\textbf{csörömpölés} vb. \emph{csörömpöl} `to clank' +
  vb. noun\hyp{}forming \emph{-és} $\rightarrow$ `clanking' \\
\textbf{csupa} `all, mere, pure' \\
%%
\ \\
\noindent \textbf{\large D} \\
\textbf{darab} `piece (self\hyp{}contained, not as a fraction of
  smth)' \\
\textbf{darabokra} adv. \emph{darab} `piece, chunk (self\hyp{}contained)' +
  pl. \emph{-ok} + subl. \emph{-ra} `on[to]' \\
\textbf{de} adv. `how!' \textbar\textbar{} conj. `but, surely'
\textbf{determinált} Latin \emph{determinare} + vb.-forming \emph{-ál}
`to determine' + past part. \emph{-t} \\
\textbf{dolog} `thing' \\
\textbf{döntöd} perfect. vb. pref. \emph{el-} + vb. \emph{dönt} `to
decide' + 2nd pers. sg. ind. pres. def. \\
\textbf{dörgölődzik} vb. `to press against', 3rd
pers. sg. ind. pres. indef. \\
%%
\ \\
\noindent \textbf{\large E} \\
\textbf{e} poetic abbrv. \emph{eme, ez} `this' \\
\textbf{ebed} noun. \emph{eb} `dog' + 2nd
pers. sg. single\hyp{}poss. poss. $\rightarrow$ `your dog' \\
\textbf{egeret} \emph{egér} `mouse' + acc. \emph{-et}
$\rightarrow$ \emph{egeret} \\
\textbf{eget} \emph{ég} `sky' + acc. \emph{-et} $\rightarrow$
  \emph{eget} \\
\textbf{egész} adj. ``whole' \textbar\textbar{} adv. `wholly' \\
\textbf{egyedül} \emph{egy} `one' + noun\hyp{}forming \emph{-ed} +
adv.-forming \emph{-ül} $\rightarrow$ `alone' \\
\textbf{egyet} \emph{egy} + acc. \emph{-t} \\
\textbf{egymás} pron. `each other, each one another' \\
\textbf{egymáson} pron. \emph{egymás} `each other, each one
another' + super. \emph{-on} `on' \\
\textbf{egyik} \emph{egy} `one' + 3rd pers. pl. mult. poss. \emph{-ik}
`of' \\
\textbf{egyszerre} adv. \emph{egyszer} `at once' +
  abl. \emph{-re} `onto' \\
\textbf{egyszerű} \emph{egy} `one' + \emph{-szerű} `-like'
$\rightarrow$ `plain, simple, unadorned, humble (< 1945)' \\
\textbf{együtt} \emph{egy} `one' + locat. \emph{-ütt} $\rightarrow$ `in the company of, together, with' \\
\textbf{el} adv., conj. `afar, far away' \textbar\textbar{} \emph{el
nem} + past. part. \textbar\textbar{} perf. vb. pref. \\
\textbf{elalszik} perfect. vb. pref. \emph{el-} + vb. \emph{alszik}
`to sleep' $\rightarrow$ `to fall asleep' + 3rd
pers. sg. ind. pres. indef. \\
\textbf{elalszol} vb. pref. \emph{el-} `away' + vb. \emph{alszik} `to
sleep' $\rightarrow$ `to fall asleep' + 2nd
pers. sg. ind. pres. indef. \\
\textbf{eleget} \emph{elég} `enough' + acc. \emph{-et} $\rightarrow$
adv. \\
\textbf{elfárad} perfect. vb. pref. \emph{el-} + vb. \emph{fárad} `to
tire, to be tired' + 3rd pers. sg. ind. pres. indef. \\
\textbf{elidőz} vb. perf. pref. `el-' + \emph{idő} `time' +
vb.-forming \emph{-z} $\rightarrow$ `to tarry, linger' \\
\textbf{elmondom} vb. \emph{elmond} `to tell' + 1st
pers. sg. ind. pres. def. \emph{-om} \\
\textbf{elmosolyodnak} vb. pref. \emph{el-}
+ \emph{mosoly} `smile' $\rightarrow$ `the process of smiling' +
vb.-forming \emph{-odik} `to become smth' (here `smile') + 3rd
pers. ind. pres. indef. $\rightarrow$ `they start to smile' \\
\textbf{elnézem} vb. pref. \emph{el-} `continuity over a long time' +
vb. \emph{nez} `to look at' \\
\textbf{eloldja} vb. pref. \emph{el-} `away' + vb. \emph{old} `to
  unbind' + 3rd pers. sg. ind. pres. def. \emph{-ja} \\
\textbf{előtt} \emph{elő} `the front part of smth' + locat. \emph{-tt}
  `in front of, ahead of, before, prior to' \\
\textbf{elpirulsz} perfect. vb. pref. \emph{el-} + vb. \emph{pirul}
`to blush' + 2nd pers. sg. ind. pres. indef. \emph{-sz} \\
\textbf{elromoljanak} perfect. vb. pref. \emph{el-} +
vb. \emph{romlik} `to break down, to deteriorate' + 3rd
pers. pl. subj. indef. \emph{-janak} \\
\textbf{elüt} vb. pref. \emph{el-} `away' + vb. \emph{üt} `to hit' +
3rd pers. sg. ind. pres. indef. \\
\textbf{ember} `person' \\
\textbf{engem} pron. \emph{én} `I' + acc. $\rightarrow$ `me' \\
\textbf{emlék} `memory' \\
\textbf{enyhe} \emph{enyh} `light, mild, tender, faint' + 3rd
pers. sg. single\hyp{}poss. poss. \emph{-e} `of' \\
\textbf{erdőn} \emph{erdőn} `forest' + iness. \emph{-n} `in' \\
\textbf{erejébe} \emph{erő} `force, power, strength' + 3rd
pers. sg. single\hyp{}poss. poss. \emph{-je} + illative \emph{-be}
`into (the inside of)' \\
\textbf{erre} pron. \emph{ez} `this, it' + subl. sg. \emph{-re} `onto,
to' \\
\textbf{esőt} vb. \emph{es}, inf. \emph{esik} + pres. part. \emph{-ő} + acc. \emph{-t} \\
\textbf{est} arch. `(in the) evening' \\
\textbf{este} arch.  `(in the) evening'
(syn. \emph{est}) \textbar\textbar{} noun `evening' \\
\textbf{eszem} vb. \emph{esz} `to eat', rare without verbal pref.,
inf. \emph{enni} + 1st pers. sg. ind. pres. def. \emph{-em}
  $\rightarrow$ \emph{eszem} \\
\textbf{ezért} conj. `therefore' \\
%%
\ \\
\noindent \textbf{\large É} \\
\textbf{ébren} adv. `awake' \\
\textbf{édes} adj. `sweet' \\
\textbf{ég} `sky' \textbar\textbar{} vb. `to burn, to be lit' + 3rd
pers. sg. ind. pres. indef. \\
\textbf{égre} \emph{ég} `sky' + subl. sg. \emph{-re} `onto' \\
\textbf{éj} `night' \\
\textbf{éjben} poetic \emph{éj} `night' + innes. \emph{-ben} `in' \\
\textbf{éjjel} poetic \emph{éj} `night' + instr. \emph{-vel}
  `with' $\rightarrow$ \emph{éjjel} \\
\textbf{éjszaka} \emph{éj} `night' + \emph{szaka} `period of'
    $\rightarrow$ `at night, nighttime, night' \\
\textbf{él-e} vb. \emph{él} `to live' + 3rd
pers. sg. ind. pres. indef. + polar interr. \emph{-e} \\
\textbf{életet} \emph{élet} `life' + acc. \emph{-et} \\
\textbf{élénken} vb. \emph{él} `to live, exist' + rare
adj.-forming \emph{-énk} + super. \emph{-en} `on' $\rightarrow$
`lively' \\
\textbf{én} pron. `I' \\
\textbf{éppolyan} pref. \emph{épp-} `exactly' + arch. pron. \emph{oly}
`such, so' + deadj. adv.-forming \emph{-an} $\rightarrow$ adv. `the
same kind, just as, just like' \\
\textbf{ért} vb. \emph{ért} `to reach', arch. `to touch', fig. `to
understand', 3rd pers. sg. past indef. \\
\textbf{érez} vb. `to feel, smell, taste' + 3rd
pers. sg. ind. pres. indef. \\
\textbf{éreztem} vb. \emph{érez} `to feel', 1st pers. sg. past def. \\
\textbf{érzed} vb. \emph{érez} `to feel', 2nd
pers. sg. ind. pres. def. \\
%%
\ \\
\noindent \textbf{\large F} \\
\textbf{fa} `tree' \\
\textbf{fájdalom} vb. \emph{fáj} `to ache, to hurt' +
noun\hyp{}forming \emph{-dalom} $\rightarrow$ `pain, grief, sorrow' \\
\textbf{fájlalom} vb. \emph{fáj} `to hurt' + vb.-forming \emph{-lal}
$\rightarrow$ lit., arch. `to regret' + 1st
pers. sg. ind. pres. def. \\
\textbf{favágókat} \emph{fa} `tree' + \emph{vágó} `cutter' +
pl. \emph{-k} + acc. \emph{-at} \\
\textbf{fákra} \emph{fa} `tree' + pl. \emph{-k} + subl. \emph{-ra}
`onto' $\rightarrow$ \emph{fákra} \\
\textbf{fáradt} vb. \emph{fárad} `to tire' + past. part. \emph{-t} \\
\textbf{fáradtan} vb. \emph{fárad} `to tire' + past. part. \emph{-t} +
adv.\hyp{}forming \emph{-an} $\rightarrow$ `tiredly, wearily' \\
\textbf{fegyvert} \emph{fegyever} `weapon' + acc. \emph{-t} \\
\textbf{fejedben} \emph{fej} `head' + 2nd
pers. sg. single\hyp{}poss. poss. \emph{-ed} + innes. \emph{-ben}
`in' \\
\textbf{fejünk} \emph{fej} `head' + 1st
pers. pl. single\hyp{}poss. poss. \\
\textbf{fekszenek} vb. inf. \emph{fekszik} `to lie (horizontally)' +
3rd pers. pl. ind. pres. indef. \emph{-enek} \\
\textbf{fekszünk} vb. inf. \emph{fekszik} `to lie (horizontally)' +
1st pers. pl. ind. pres. indef. \emph{-ünk} \\
\textbf{feledni} lit. vb. \emph{feled} `to forget',
alt. \emph{felejt} \\
\textbf{felelhetek} \emph{fél} `human being' + vb.-forming \emph{-el}
$\rightarrow$ vb. \emph{felel} `to answer, reply' + poten. 1st
pers. sg. ind. pres. indef. \\
\textbf{felém} post. \emph{felé} `to[wards]' + 1st
pers. sg. poss \emph{-m} $\rightarrow$ pron. `towards me' \\
\textbf{felhők} \emph{felhő} `cloud, danger, trouble' +
pl. \emph{-k} \\
\textbf{feljöjjön} vb. pref. \emph{fel-} `upwards' (see
also \emph{föl-}) + vb. \emph{jön} `to come' + 3rd
pers. sg. subj. pres. indef. \\
\textbf{félelem} vb. \emph{fél} `to be afraid' +
noun\hyp{}forming \emph{-elem} $\rightarrow$ fear' \\
\textbf{félni} vb. inf. \emph{fél} `to fear, to be afraid of' \\
\textbf{fény} `light,  glitter', fig. `happiness, joy' \\
\textbf{fénye} \emph{fény} `light' + 3rd
pers. sg. single\hyp{}poss. poss. \emph{-e} \\
\textbf{fényes} \emph{fény} `light, glitter' + adj.-forming \emph{-es}
$\rightarrow$ `bright' \\
\textbf{fényességgel} \emph{fény} `light' + adj.-forming \emph{-es}
    `with, having' $\rightarrow$ `bright' + abst. noun\hyp{}forming
  \emph{-ség} $\rightarrow$ `brightness' +
  instr. \emph{-vel} `with' (\emph{-gel} after consonant
  \emph{g}) $\rightarrow$ `with brightness' \\
\textbf{fényben} \emph{fény} `light, glitter' + innes. \emph{-ben}
`in' \\
\textbf{fénylenek} \emph{fény} `light' +
freq. vb.-forming \emph{-lik}: `to glitter' + 3rd
pers. pl. ind. pres. indef. \emph{-ek} \\
\textbf{fog} vb. `to hold' + 3rd
pers. sg. ind. pres. indef. \textbar\textbar{} aux. vb. `to will' \\
\textbf{fogad} vb. `to receive' + 3rd pers. sg. ind. pres. indef. \\
\textbf{fogaskerekére} \emph{fogas} `tooth' + \emph{kerék} `wheel':
`cogwheel' + 3rd pers. sg. single\hyp{}poss. poss. \emph{-e} + subl. \emph{-re}
`towards' \\
\textbf{fogatlan} \emph{fog} `tooth' +
priv. adj.-forming \emph{-atlan} `-less' \\
\textbf{foglak} vb. aux. \emph{fog} `to will' + 1st
pers. sg. ind. pres. 2nd obj. \emph{-lak} `you' \\
\textbf{fogod} aux. vb. \emph{fog} `to will' + 2nd
pers. sg. ind. pres. def. (e.g. \emph{Meg fog bánni.}) \\
\textbf{fogsz} aux. vb. `to will' + 2nd
pers. sg. inform. ind. pres. def. \\
\textbf{folyó} vb. inf. \emph{folyik} `to flow, go on' +
pres. part. \emph{-ó} $\rightarrow$ `river' \textbar\textbar{}
adj. `flowing, leaky, current' \\
\textbf{földek} \emph{föld} `earth, ground' + pl. \emph{-ek} \\
\textbf{földre} \emph{föld} `earth, ground' + subl. \emph{-re}
`onto' \\
\textbf{fölkereshetnéd} \emph{föl-} `upwards' + \emph{keres} `to
seek': `to visit' + pot. `may, might' \emph{-het} + 2nd
pers. sg. ind. pres. cond. \emph{-ned}, alt. \emph{felkeres} \\
\textbf{fölállok} \emph{föl-} `upwards' + vb. \emph{áll} `to stand' + 1st
  pers. sg. ind. pres. indef. \emph{-ok}, alt. \emph{felállok} \\
\textbf{földeken} \emph{föld} `earth, field' + pl. \emph{-ek} +
super. \emph{-en} `on'\\
\textbf{földtől} \emph{föld} `earth' + abl. \emph{-től} `without,
from' \\
\textbf{fölfeslik} vb. pref. \emph{föl-} `upwards'+ arch., poetic,
vb. \emph{feslik} `to come unstitched', 3rd
pers. sg. ind. pres. indef. \\
\textbf{fölnéztem} vb. pref. \emph{föl-} `upwards' + vb.
  \emph{néz} `to look' + 1st pers. sg. past. def. \emph{-tem} \\
\textbf{fölött} \emph{föl-} `up[wards]' +  locat. \emph{-ött}:
`above', alt. \emph{elett} \\
1st pers. sg. ind. pres. indef. \emph{-ok}, alt. \emph{felállok} \\
\textbf{fölszabadult} vb. pref. \emph{föl-} completion (see
also \emph{fel-}) + \emph{szabad} `free' + vb.-forming \emph{-ul}
    $\rightarrow$ `to set free' + past part. \emph{-t} \\
\textbf{fönn} adv. `above', alt. `fent, fenn' \\
\textbf{fönt} adv. `above, up, awake, upstairs', alt. \emph{fent} \\
\textbf{fülelt} \emph{fül} `ear' + vb.-forming \emph{-el}: `to listen
(intently)' + 3rd pers. ind. past indef. \emph{-t} \\
\textbf{fülke} \emph{fül} `semi\hyp{}circular obj., shape' +
dim. \emph{-ke} $\rightarrow$ `compartment' \\
\textbf{füvek} \emph{fű} `grass' + pl. \emph{-ek} \\
%%
\ \\
\noindent \textbf{\large G} \\
\textbf{gond} arch. sg. `care' \\
\textbf{Göncölök} \emph{Göncöl} + pl. \emph{-ök} $\rightarrow$ `the Great Bear'
  (constellation) \\
\textbf{göndör} adj. `curly' \\
\textbf{görnyedve} vb. \emph{görnyed} `to bend (under a burden)' +
adv.-forming \emph{-ve} \\
\textbf{görög} vb. `to roll, scroll' + 3rd
pers. sg. ind. pres. indef. \\
\textbf{gőzei} \emph{gőz} `steam, vapour' + 3rd
pers. sg. poss. \emph{-e} + pl. poss. \emph{-i} \\
\textbf{gyenge} adj. `weak', alt. \emph{gyönge} \\
\textbf{gyere} vb. \emph{jön} `to come' + 2nd
pers. sg. ind. pres. indef. \\
\textbf{gyermek} arch. `child', alt. \emph{gyerek} \\
\textbf{gyom} `weed' \\
\textbf{gyöpén} \emph{gyep} `lawn, grass' + super. \emph{-en} `on' \\
%%
\ \\
\noindent \textbf{\large H} \\
\textbf{ha} conj. `if', `when, once' \\
\textbf{hajnal} `dawn' \\
\textbf{hajóz} vb. `to navigate, sail' \\
\textbf{hagyna} vb. \emph{hagy} `to allow, leave' + 3rd
pers. sg. cond. pres. indef. \emph{-na} \\
\textbf{halálra} vb. \emph{hal} `to die' + arch. noun\hyp{}forming
  \emph{-ál} $\rightarrow$ `death' + subl. `on[to], for' \emph{-ra}
  $\rightarrow$ `to death' \\
\textbf{hallgatag} vb. \emph{hallgat} `to remain silent, to listen' +
adj.-forming \emph{-ag} \\
\textbf{hallgatok} vb. \emph{hallgat} `to remain silent, to listen' +
1st pers. sg. ind. pres. indef. \emph{-ok} \\
\textbf{hallottam} vb. \emph{hall} `to hear' + 1st pers. sg. ind. past
def. \emph{-ottam} \\
\textbf{halom} `pile, heap' \\
\textbf{halott} noun./adj. `dead' \\
\textbf{hanem} \emph{ha} conj. `if', `when, once' + \emph{nem} `no':
conj. `but (following a negative sentence, making it positive)' \\
\textbf{hangosan} adv. `loudly' \\
\textbf{haragszol} \emph{harag} `anger' + vb.-forming \emph{-szol}
$\rightarrow$ vb. inf. \emph{haragszik} `to be angry` + 2nd
pers. inform. sg. ind. pres. indef. \\
\textbf{harangszó} \emph{harang} `(church) bell'+ \emph{szó} `word,
voice' $\rightarrow$ `bell chime' \\
\textbf{harmattá} \emph{harmat} `dew' + transl. \emph{-vá} `(change,
turn) into' \\
\textbf{harmatos} adj. `dewy' \\
\textbf{hasított} vb. \emph{hasít} `to cleave, split' +
past. part. \emph{-ott} \\
\textbf{hazaballagunk} \emph{haza} `home' (\emph{haz} arch. `house' (modern
\emph{ház}) + vb. \emph{ballag} `to walk slowly, to stroll' + 1st
pers. pl. ind. pres. indef. \\
\textbf{hazafelé} \emph{haza} `home' (\emph{haz} arch. `house' (modern
\emph{ház}) + short. arch. lative (non\hyp{}extant case in modern
Hungarian) \emph{-á} $\rightarrow$ `(to) home') + \emph{felé}
`towards' $\rightarrow$ adv. `homewards' \\
\textbf{hámot} \emph{hám} `harness' + acc. \emph{-ot} \\
\textbf{háromszoros} \emph{három} `three' + \emph{-szoros} `-fold' \\
\textbf{hát} `back (body)','well...', `then, and, but' (questioning
back), `surely' \\
\textbf{házat} \emph{ház} `house' + acc. \emph{-at} \\
\textbf{háziúr} \emph{ház} `house' + adj.-forming `of the' \emph{-i}
+ \emph{úr} `master' \\
\textbf{helyükön} \emph{hely} `place/space' + 3rd
pers. pl. single\hyp{}poss. poss. \emph{-ük} `their' +
super. \emph{-ön} `on' \\
\textbf{hever} vb. `to lie (down)' + 3rd pers. sg. ind. pres. indef. \\
\textbf{hevül} vb. `to grow hot', 3rd pers. sg. ind. pres. indef. \\
\textbf{hiányozni} \emph{hiány} `lack, absence, break' +
vb.-forming \emph{-zik} $\rightarrow$ `(smb) to be missing, missed',
inf. \emph{hiányozni} \\
\textbf{hibbant} perfect. vb. pref. \emph{el-} + vb. \emph{hibban}
  `to become imbalanced', arch. + past part. \emph{-t} \\
\textbf{hiszen} conj. `as, because (smth obvious)' \\
\textbf{hogy} adv. interr. `how', conj. `that', `if'. \\
\textbf{hogy'} short. \emph{hogyan} `how' \\
\textbf{hogyha} coloq. conj. syn. \emph{ha} `if', `when, once, supposing' \\
\textbf{hold} `moon' \\
\textbf{holdat} \emph{hold} `moon' + acc. \emph{-at} \\
\textbf{holott} adv. arch. `where' \textbar\textbar{}
conj. lit. `though, but, for, since' \\
\textbf{holtan} \emph{holt} `dead' + adv.-forming \emph{-an} \\
\textbf{homályként} \emph{homály} `blur' + essive formal \emph{-ként}
`as' \\
\textbf{homlokomra} \emph{homlok} `forehead' + 1st
pers. sg. single\hyp{}poss. poss. \emph{-om} + subl. \emph{-ra} `on[to]' \\
\textbf{hozzá} pron. `to it, him, her' \\
\textbf{hull} vb. `to fall', alt. \emph{hullik} + 3rd
pers. sg. ind. pres. indef. \\
\textbf{hullámzanak} vb. \emph{hullámzik} `to wave, billow, undulate' +
3rd pers. pl. ind. pres. indef. \\
\textbf{hunyorgott} vb. \emph{hunyorg} `to squint' + 3rd
pers. sg. ind. past. indef. \emph{-rgott} \\
\textbf{hűs} adj. `cool (water, weather)' \\
%%
\ \\
\noindent \textbf{\large I} \\
\textbf{idebent} adv. \emph{ide} `here' + \emph{bent} `within' \\
\textbf{ifju} \emph{ifjú} adj., noun `young' \\
\textbf{ifjúságod} adj. \emph{ifjú} `young' + noun\hyp{}forming \emph{-ság}
(state of being): `youth' + 2nd pers. informal
sg. single\hyp{}poss. poss. \emph{-od}: `your childhood' (informal) \\
\textbf{illik} vb. `to suit, match, fit' + 3rd
pers. sg. ind. pres. indef. \\
\textbf{ilyen} det. `such' \\
\textbf{im} lit., abbrv. \emph{íme} `here is, behold' \\
\textbf{imbolygott} vb. \emph{imbolyog} `to sway'+ 3rd
pers. sg. single\hyp{}poss. poss. \emph{-e} \\
\textbf{indul} vb. `to set off, to head, to start' + 3rd
pers. sg. ind. pres. indef. \\
\textbf{ingyen} fig. arch. `in vain'; usually `free of charge' \\
\textbf{iramlanak} \emph{ir} `movement' + rare
instr. vb.-forming \emph{-amlik} (inf.) $\rightarrow$ `to rush, hurry'
+ 3rd pers. pl. ind. pres. indef. \emph{-anak} \\
\textbf{is} doublet of \emph{és} `and', adv. `also' \\
\textbf{ismer} vb. `to know'  + 2nd pers. sg. inform. (or 3rd
pers. sg.) ind. pres. indef.\\
\textbf{istene} \emph{isten} `god' + 3rd
pers. sg. sing. poss. \emph{-e} \\
%%
\ \\
\noindent \textbf{\large Í} \\
\textbf{így} `just so' \\
%%
\ \\
\noindent \textbf{\large J} \\
\textbf{jön} vb. `to come' + 3rd pers. sg. ind. pres. indef. \\
\textbf{jönnél} vb. \emph{jön} `to come' + 2nd pers. sg.
cond. pres. indef. \\
\textbf{jós} `fortunetelle, soothsayer' \\
%%
\ \\
\noindent \textbf{\large K} \\
\textbf{kabátot} \emph{kabát} `coat' + acc. \emph{-ot} \\
\textbf{kacagunk} vb. \emph{kacag} `to laugh (a lot)' + 1st
pers. pl. ind. pres. indef. \emph{-unk} \\
\textbf{kapás} Farm hand using a hoe, from \emph{kapa} `hoe' \\
\textbf{kapja} vb. \emph{kap} `to receive' + 3rd
pers. sg. ind. pres. def. \emph{-ja} \\
\textbf{karaj} folk. `slice' (syn. \emph{karéj}) \\
\textbf{karol} vb. `to put one's arm around
someone' \textbar\textbar{} lit. `to embrace' \\
\textbf{kell} aux. vb. `must, need to, have to + inf.' \\
\textbf{kenyerem} \emph{kenyér} `bread' + 1st
pers. sg. single\hyp{}poss. poss. \emph{-em} \\
\textbf{kenyeret} \emph{kenyér} `bread' + acc. \emph{-et}
$\rightarrow$ \emph{kenyeret} \\
\textbf{kenyeréből} \emph{kenyér} `bread' + 3rd
pers. single\hyp{}poss. poss. \emph{-é} + elative \emph{-ből} `from, out of' \\
\textbf{keresek} vb. \emph{keres} `to seek' + 1st
pers. sg. ind. pres. indef. \\
\textbf{kettő} `two' \\
\textbf{kék} `blue' \\
\textbf{kél} vb. `to rise', 3rd pers. sg. ind. pres. def.,
  arch. alt. \emph{kel} \\
\textbf{képed} \emph{kép} `picture' + 2nd
pers. sg. single\hyp{}poss. poss. \emph{-ed} \\
\textbf{képeket} \emph{kép} `picture' + pl. \emph{-ek} + acc. \emph{-et} \\
\textbf{képzelhetsz} \emph{kép} `image' + vb.-forming \emph{-zel}:
`to imagine' + pot. \emph{-het} + 2nd
pers. sg. ind. pres. \emph{-sz} \\
\textbf{képzetet} \emph{képzet} `idea, image, notion' +
acc. \emph{-t} \\
\textbf{kérded} arch. vb. \emph{kérd} `to ask, to question' + 2nd
pers. sg. ind. pres. def. \\
\textbf{kérdezed} vb. \emph{kérdez} `to ask' + 2nd
pers. sg. ind. pres. \\
\textbf{kérek} vb. \emph{kér} `to ask for' + 1st
pers. sg. ind. pres. indef. \emph{-ek} \\
\textbf{kényedül} lit. vb. `to indulge', 3rd
pers. sg. ind. pres. indef. \\
\textbf{ki} pron. rel. arch. `he who',
alt. \emph{aki} \textbar\textbar{} adv. `out[wardly]' \\
\textbf{kieszeltem} vb. \emph{kieszel} `to devise' + 1st
pers. sg. past. \emph{-tem} \\
\textbf{kinn} `outside' \\
\textbf{kint} adv. `outside' \\
\textbf{kipörögnek} vb. \emph{kipörög} `to spin' + 3rd
  pers. pl. ind. pres. indef. \emph{-nek} \\
\textbf{kis} adj. `small' \\
\textbf{kisgyerek} \emph{kis} `small' + \emph{gyerek} `child' \\
\textbf{kiülnek} vb. pref. \emph{ki} `outwardly' + vb. \emph{ül}
`to sit' + 3rd pers. pl. ind. pres. indef. \\
\textbf{kivilágított} vb. pref. \emph{ki} `outwardly' + \emph{világ}
`light (arch.), world' $\rightarrow$ arch. `illumination' +
    caus. vb.-forming \emph{-ít} `to make smth ...-like' $\rightarrow$
`to illuminate' + past part. \emph{-ott} \\
\textbf{kívánnak} vb. \emph{kíván} `to wish' + 3rd
pers. pl. ind. pres. indef. \emph{-nak} \\
\textbf{kívül} adv. `outdoor, outside' \\
\textbf{kocska} `a die' \\
\textbf{komoly} adj. `severe, austere' \\
\textbf{konokon} adj. \emph{konok} `stubborn' + super. \emph{-on}
`on (that manner)' \\
\textbf{könnyen} \emph{könnyű} `easy, light' + super. \emph{-en} `on'
$\rightarrow$ `effortlessly, easily' \\
\textbf{könnyű} \emph{könnyű} `simple, easy, light' \\
\textbf{könnyűség} \emph{könnyű} `simple, easy, light' +
  abst. noun\hyp{}forming \emph{-ség} \\
\textbf{könyöklök} \emph{könyök} `elbow' + vb.-forming \emph{-öl}
$\rightarrow$ `to rest on one's elbow' + 1st
pers. sg. ind. pers. \emph{-ök} \\
\textbf{kövek} \emph{kő} `stone' + pl. \emph{-ek} \\
\textbf{közt} post. `between, amid' from \emph{között}: \emph{köz}
  `interval' + locat. \emph{-ött} \\
\textbf{között} post. `between, amid', \emph{köz} `interval' +
locat. \emph{-ött} \\
\textbf{közt} post. `between, amid' from \emph{között}: \emph{köz}
  `interval' + locat. \emph{-ött} \\
\textbf{kutya} `dog' \\
\textbf{különös} `strange' \\
%%
\ \\
\noindent \textbf{\large L} \\
\textbf{lakom} vb. \emph{lak} `to dwell', inf. \emph{lakik}, 1st
pers. sg. ind. pres. \emph{-om} \\
\textbf{langy} arch. poetic \emph{langyos} `tepid' \\
\textbf{lapultam} vb. \emph{lapul} `to lie low' + 1st pers. sg. past
indef. \emph{-tam} \\
\textbf{lábod} \emph{láb} `leg (human, animal), foot' + 2nd
pers. inform. single\hyp{}poss. poss. \emph{-od} \\
\textbf{lágy} adj. `soft' \\
\textbf{lámpa} `lamp' \\
\textbf{látom} vb. \emph{lát} `to see'+ 1st
pers. sg. ind. pres. def. \\
\textbf{látni} vb. \emph{lát} `to see' + inf. \emph{-ni} \\
\textbf{látszik} vb. \emph{lát} `to see' + vb.-forming \emph{-szik}
$\rightarrow$ `to be visible, to appear' \\
\textbf{láttam} vb. \emph{lát} `to see' + short base \emph{-t}, 1st
pers. sg. past def. \emph{-tam} \\
\textbf{lázad} vb. \emph{to rebel}, 3rd pers. sg. ind. pres. indef. \\
\textbf{lázat} \emph{láz} `fever' + acc. \emph{-at} \\
\textbf{le} comp. `down' \\
\textbf{lebeg} vb. \emph{lebeg} `to float (in the air)', 3rd
  pers. sg. ind. pres. indef. \\
\textbf{ledőlt} vb. \emph{ledoől} `to fall down' + 3rd
pers. sg. ind. past indef. \emph{-t} \\
\textbf{lehet} pot. vb. \emph{van, lesz} `may be, may become', 3rd
pers. sg. ind. \\
\textbf{leheverünk} vb. pref. \emph{le-} completion, downward +
vb. \emph{hever} vb. `to lie (down)' + 1st
pers. pl. ind. pres. indef. \emph{-ünk} \\
\textbf{lehunyod} vb. pref. \emph{le-} completion, downward +
vb. \emph{huny} `to close (one's eyes)' + 2nd
pers. sg. ind. pres. def. \\
\textbf{lelkedet} \emph{lélek} `soul' + 2nd
  pers. sg. single\hyp{}poss. poss. \emph{-ed} $\rightarrow$ \emph{lelked} + acc. \emph{-et} \\
\textbf{lelkek} \emph{lélek} `soul' + pl. \emph{-ek} $\rightarrow$ \emph{lelkek} \\
poss. \emph{-ed} $\rightarrow$ \emph{lelked} + acc. \emph{-et} \\
\textbf{lelkiismeret} arch. dial. \emph{lelki} `his, her soul'
(alt. \emph{lelke}) + arch. dial. \emph{ismereti} `his, her, its
knowledge' (alt. \emph{ismerete}) $\rightarrow$ `(moral)
conscience' \\
\textbf{lengedező} vb. \emph{leng} `to swing, sway' +
freq. vb.-forming \emph{-edezik} (unrounded front\hyp{}vowel) +
pres. part. \emph{-ő} $\rightarrow$ `swinging, swaying' \\
\textbf{lenézel} vb. `to look down'+ 2nd
pers. sg. ind. pres. indef. \emph{-el} \\
\textbf{lennél} vb. \emph{van} `to be' \textbar\textbar{}
vb. \emph{lesz} `to become' + 2nd pers. sg. cond. pres. \\
\textbf{lepkék} \emph{lepke} `butterfly' + pl. \emph{-k}
$\rightarrow$ \emph{lepkék} \\
\textbf{lesikáltam} vb. pref. \emph{le-} completion, downward +
vb. \emph{sikál} `to scrub, scour (out)' + 1st pers. sg. ind. past.def.
\emph{-am} \\
\textbf{lestem} vb. \emph{les} `to spy' + 1st pers. sg. past
indef. \emph{-tem} \\
\textbf{lesz} vb. \emph{van} `to be', 3rd
pers. sg. ind. fut. indef. \\
\textbf{leszedjük} vb. pref. \emph{le-} completion + vb. \emph{szed}
`to take from somewhere, to collect, to take medicine regularly'
$\rightarrow$ `to pick smth off smth, smb' + 1st
pers. pl. ind. pres. def. \textbar\textbar{} 1st
pers. pl. subj. pres. def. \emph{-jük} \\
\textbf{leülsz} vb. pref. \emph{le-} completion, downward + vb. \emph{ül}
`to sit' $\rightarrow$ `to sit down' \\
\textbf{levegőben} \emph{levegő} `air' + innes. \emph{-ben} `in' \\
\textbf{levelek} \emph{level} `leaf' + pl. \emph{-ek} \\
\textbf{levesse} vb. \emph{levet} `to buck, throw off' + 3rd
pers. sg. subj. pres. def. \\
\textbf{léha} `improvident, idler' \\
\textbf{locska} adj. `blethering' \\
\textbf{lombok} \emph{lomb} `foliage, leaves' + pl. \emph{-ok} \\
\textbf{lombosodnak} \emph{lomb} `foliage, leaves' + \emph{-os}
adj.\hyp{}forming + vb.-forming \emph{-odik} `to become smth' (here, `foliage') + 3rd
pers. ind. pres. indef. $\rightarrow$ `they start to become leaves' \\
\textbf{lopcska} `chuck' (shoulder meat cut, probably pork) \\
%%
\ \\
\noindent \textbf{\large M} \\
\textbf{ma} `today' \\
\textbf{macska} `cat' \\
\textbf{magadban} \emph{maga} `oneself' + 2nd pers. sg. single
  poss. \emph{-d} + iness. \emph{-ban} `in' \\
\textbf{magadnak} \emph{maga} `oneself' + 2nd pers. sg. single
  poss. \emph{-d} + dat. \emph{-nak} \\
\textbf{magadra} \emph{maga} `oneself, itself' + subl. \emph{-ra}
`onto' \\
\textbf{magányos} adv., adj. \emph{magán} `separate[ly]' +
adj.-forming \emph{-os} $\rightarrow$ `lonely (psych.)' \\
\textbf{magának} pron. refl. 3rd pers. `himself' \\
\textbf{magokat} \emph{mag} `seed' + pl. \emph{-ok} +
acc. \emph{-at} \\
\textbf{maguknak} \emph{mag(a)} `oneself' + 3rd pers. pl. \emph{-uk} +
dat. \emph{-nak} \\
\textbf{magyarázat} vb. \emph{magyaráz} `to explain' +
noun\hyp{}forming \emph{-at} \\
\textbf{majd} adv. `sometime, later' \\
\textbf{maradnál} vb. \emph{marad} `to stay, remain (also state of
being)' + 2nd pers. sg. cond. pres. indef. \\
\textbf{Marosra} prop. noun. \emph{Maros}, a Hungarian and Romanian
river + subl. \emph{-ra} `onto' \\
\textbf{már} adv. `already, yet, any more' \\
\textbf{másfél} \emph{más} `other, different' + \emph{fél} `half'
$\rightarrow$ `one and a half' \\
\textbf{egymásról} pron. \emph{egymá} `each other, each one another' +
delat. \emph{-ról} \\
\textbf{másikát} adj. \emph{más} `other' +
noun\hyp{}forming \emph{-ik} $\rightarrow$ `another' + 3rd
  pers. sg. single\hyp{}poss. \emph{-a} + acc. \emph{-t}
  $\rightarrow$ \emph{-át} \\
\textbf{mást} adj. \emph{más} `other, different' + acc. \emph{-t} \\
\textbf{mázsa} `quintal' \\
\textbf{meg} arch. emph. \textbar\textbar{} conj. `and, plus' \\
\textbf{megcsókolni} perfect. vb. pref. \emph{meg-} +
vb. \emph{csókol} `to kiss' + inf. \emph{-ni} \\
\textbf{elfárad} emph. vb. pref. \emph{meg-} + vb. \emph{fárad} `to
tire, to be tired' + 3rd pers. sg. ind. pres. indef. \\
\textbf{meghasadt} perfect. vb. pref. \emph{meg-} + vb. \emph{hasad}
`to crack, split apart' + past part. \emph{-t} \\
\textbf{megint} \emph{meg} `again' (short action) +
adv.-forming \emph{-int} \\
\textbf{megjelent} vb. inf. \emph{megjelenik} `to appear' + 2nd
pers. sg. inform. (or 3rd pers. sg.) ind. past. indef. \\
\textbf{meglett} obs. `adult' \\
\textbf{megnézzük} vb \emph{megnéz} `to see, take a look' + 1st
pers. pl. ind. pres. indef. \\
\textbf{megöregszel} perfect. vb. pref. \emph{meg-} +
\emph{öreg} `old' + vb.-forming \emph{-szik}
$\rightarrow$ `to grow, become old' + 2nd
pers. sg. inform. pres. indef. \\
\textbf{megtört} perfect. vb. pref. \emph{meg-} + vb. \emph{tör} `to
    break, crush' + past. part. \emph{-t} \\
\textbf{megy} vb. `to go, travel, pass by (of time)' + 3rd
pers. sg. ind. pres. indef. \\
\textbf{meleg} adj. `warm, hot, heartfelt, cordial' \\
\textbf{mellett} \emph{mell} `breast' + locat. \emph{-ett}
$\rightarrow$ `next to' \\
\textbf{melléd} post. \emph{mellé} `next to' + 2nd pers. sg. single
  poss. \emph{-d} $\rightarrow$ pron. `next yo you' \\
\textbf{mely} lit. `which, that' \\
\textbf{melybe} lit. arch. \emph{mely} `which' + illative \emph{-be}
`into (the inside of)' \\
\textbf{melyben} lit. \emph{mely} `which, that' + innes. \emph{-ben} `in' \\
\textbf{mert} conj. `because' \\
\textbf{még} adv. `still, moreover, later, sometime' \\
\textbf{méla} adj. `dreamy, pensive' \\
\textbf{miért} \emph{mi} `what' + causal\hyp{}final \emph{ért} `for'
$\rightarrow$ `why' \\
\textbf{miként} \emph{mi} `what' + \emph{-kent} $\rightarrow$
essive\hyp{}formal sg. `as a, like a' (task, role,
manner) \textbar\textbar{} syn. \emph{hogyan} form. adv. `how?, in
what way?' \textbar\textbar{} syn. \emph{ahogyan} form. adv. `as,
like' \\
\textbf{míg} lit. adv. `as long as, while (of
time)' \textbar\textbar{} `until' \\
\textbf{mily} poetic `how ...(!)' \\
\textbf{milyen} pron. `what' \\
\textbf{mind} pron. sg. `all, each of' \\
\textbf{minden} det. `every' \\
\textbf{mindenik} folk. `each, every one'; alt. \emph{mindegyik}
  $leftarrow$ \emph{mind} `all, every' + \emph{egyik} `one [of]'. \\
\textbf{mindig} \emph{mind} `always' (repetition) + term. `until'
  \emph{-ig} \\
\textbf{minek} \emph{mi} `what' + dat. \emph{-nek} $\rightarrow$ `for what' \\
\textbf{mint} conj. `like' \textbar\textbar{} comp. different degrees
`than' \\
\textbf{mit} interj. \emph{mi} `what' + acc. \emph{-t} \\
\textbf{mivel} \emph{mi} `what' + instr. \emph{-vel}
`with' \textbar\textbar{} conj. `as (because)' \\
\textbf{mondottál} arch. \emph{mondtál}, vb. \emph{mond} `to say,
tell' + 2nd pers. sg. ind. past indef. \\
\textbf{most} adv. `now' \\
\textbf{mosolygása} \emph{mosolygás} `smile' + 3rd pers. sg. single
poss. \emph{-a} \\
\textbf{motyogod} vb. \emph{motyog} `to mumble, mutter' + 2nd
pers. sg. ind. pres. def. \emph{-od} \\
\textbf{mult} \emph{múlt} rare, past part. `past',
inf. \emph{múlik} \\
\textbf{multból} \emph{múlt} rare, past part. `past' +
elative \emph{-ból} `from, out of' \\
\textbf{munkások} \emph{munka} `work' + noun\hyp{} and
adj.\hyp{}forming \emph{-s} $\rightarrow$ \emph{munkás} `labourer,
worker' + pl. \emph{-ok} \\
%%
\ \\
\noindent \textbf{\large N} \\
\textbf{nagy} adj. `great' \\
\textbf{nagyon} adj. \emph{nagy} `great' + adv.-forming \emph{-on}
    $\rightarrow$ `very (to a high degree)' \\
\textbf{nap} \emph{nap} `sun, day' \\
\textbf{nappal} \emph{nap} `day' + instr. \emph{-val} `with'
$\rightarrow$ \emph{nappal} \\
\textbf{nappalok} \emph{nap} `day' + instr. \emph{-val}
  `with' $\rightarrow$ \emph{nappal} + pl. \emph{-ok} \\
\textbf{napszámosokhoz} \emph{napszámos} `day labourer' +
pl. \emph{-ok} + allat. `\emph{-hoz} `towards' \\
\textbf{napszülte} \emph{nap} `sun/day' + vb. \emph{szül} `to give
birth, bear' + 3rd pers. sg. ind. past def. \emph{-te} \\
\textbf{napvilágra} \emph{nap} `sun/day' + \emph{világ}
  `world' + subl. \emph{-ra} `onto' \\
\textbf{ne} \emph{nem} becomes \emph{ne} in imperative, subjunctive \\
\textbf{nevetni} vb. \emph{nevet} `to laugh', inf. \emph{nevetni} \\
\textbf{néha} adv. `sometimes' \\
\textbf{néhány} pron. `some (of us, you, them)' \\
\textbf{nékünk} lit. \emph{néki} (alt. \emph{neki}) `to/for him' +
1st pers. pl. $\rightarrow$ `to us, for us' \\
\textbf{néma} `mute' \\
\textbf{néznek} vb. \emph{néznek} `to look at smth' + 3rd
  pers. pl. ind. pres. indef. \emph{-nek} \\
\textbf{nézzük} vb. \emph{nez} `to look at' + 1st
pers. pl. ind. pres. \emph{-jük} \\
\textbf{no} interj. `well, now' \\
%%
\ \\
\noindent \textbf{\large Ny} \\
\textbf{nyafog} vb. `to whine' + 3rd pers. sg. ind. pres. indef. \\
\textbf{nyers} adj. `raw' \\
\textbf{nyirkos} adj. `humid, damp' \\
\textbf{nyomja} vb. \emph{nyom} `to weigh' + 3rd
pers. sg. ind. pres. def. \emph{-ja} \\
\textbf{nyugalmat} \emph{nyugalom} `serinity, peacefulness' +
acc. \emph{-at} \\
\textbf{nyugalom} `serinity, peacefulness' \\
\textbf{nyugodt} adj. `tranquil, calm'\\
%%
\ \\
\noindent \textbf{\large O} \\
\textbf{odahallik} vb. pref. (adv.) \emph{oda-} `there' +
arch. vb. \emph{hallik} syn. \emph{hallatszik} `to be able to be
heard, to sound, to be audible' \\
\textbf{oly} arch. pron. \emph{oly} `so' \\
\textbf{olyan} arch. pron. \emph{oly} `such, so' + deadj. adv.-forming
\emph{-an} \\
\textbf{olyanigen} adj. \emph{olyan} `such, so' + \emph{igen} `yes'
$\rightarrow$ emphatic affirmation, `yes indeed, exactly so' \\
\textbf{olvassa} vb. \emph{olvas} `to read' + 2nd
pers. sg. inform. (or 3rd pers. sg.) subj. pres. indef. \\
\textbf{ondoltam} vb. \emph{gondol} `to think, plan, guess, worry' +
1st pers. sg. ind. past \emph{-tam} \\
\textbf{ott} demons. `there, over there' \\
%%
\ \\
\noindent \textbf{\large Ó} \\
\textbf{óriás} adj. `giant, huge, vast' \\
%%
\ \\
\noindent \textbf{\large Ö} \\
\textbf{öleltem} perfect. vb. pref. \emph{meg-} + vb. \emph{ölel} `to
embrace' + 1st pers. ind. past \\
\textbf{önnön} arch. emph. poss. adj. by duplication of
pers. pron. \emph{ön} (3rd pers. sg.) $\rightarrow$ `one's own' \\
\textbf{öntudat} `self-awareness' \\
\textbf{öreg} adj. `old' \\
\textbf{örök} `perpetual, eternal' \\
\textbf{örülj} vb. \emph{örül} `to rejoice' + 2nd
pers. sg. subj. indef. \emph{-j} \\
\textbf{összefogja} \emph{össze} `together' + vb. \emph{fog}
  `to hold' + 3rd pers. sg. ind. pres. def. \emph{-ja} \\
\textbf{összekent} \emph{össze} `together' + \emph{ken} `smear'
$\rightarrow$ \emph{összeken} + 3rd pers. sg. past indef. \emph{-t} \\
\textbf{összetört} \emph{össze} `together' + vb. \emph{tör} `to break'
+ past. part. \emph{-t} \\
%%
\ \\
\noindent \textbf{\large Ő} \\
\textbf{ők} pron. `they' \\
\textbf{őrt} \emph{őr} `guard' + acc. \emph{-t} \\
\textbf{őrzi} vb. \emph{őriz} `to guard' + 3rd
pers. sg. ind. pres. def. \emph{-i} \\
\textbf{őz} roe deer \\
%%
\ \\
\noindent \textbf{\large P} \\
\textbf{papja} \emph{pap} `priest' + 3rd
pers. sg. sing. poss. \emph{-ja} \\
\textbf{paraszt} sg. `peasant'\\
\textbf{pára} `moisture, mist' \\
\textbf{pedig} conj. expressing contrast `whereas, but, although' or
addition `and (by contrast)' \\
\textbf{pihen} vb. `to rest' + 3rd pers. sg. ind. pres. indef. \\
\textbf{pihensz} vb. `to rest' + 2nd
pers. sg. inform. ind. pres. indef. \\
\textbf{pihéi} \emph{pihe} `down, fluff' + adj.-forming \emph{-i} \\
\textbf{piros} `red' \\
\textbf{pislog} vb. `to blink' \\
\textbf{poros} \emph{por} `dust' + adj.-forming \emph{-os}
$\rightarrow$ `dusty' \\
\textbf{porra} \emph{por} `dust' + subl. \emph{-ra} `onto' \\
\textbf{porszem} \emph{por} `dust' + \emph{szem} `eye, speck' \\
\textbf{puha} adj. `soft, squishy' \\
%%
\ \\
\noindent \textbf{\large R} \\
\textbf{rab} `captive' \\
\textbf{rak} vb. `to put, set' + 3rd pers. sg. ind. pres. indef.\\
\textbf{raksz} vb. \emph{rak} `to set up, to build', 2rd
pers. sg. ind. pres. indef. \emph{-sz} \\
\textbf{raktár} `warehouse, storage room' \\
\textbf{ráadásul} \emph{ráadás} `addition, extra' + essive
  `as, with the intention of' \emph{-ul} \\
\textbf{rácsok} \emph{rács} `grid, grille' + pl. \emph{-ok} \\
\textbf{ránk} pron. \emph{rá} `upon, on, unto him' + 1st
pers. pl. poss. single\hyp{}poss. \emph{-nk} \\
\textbf{rászálltak} pointed action \emph{rá-} `on' + vb. \emph{száll}
`to fly, land, to come down' + 3rd pers. sg. past indef. \emph{-tak} \\
\textbf{ráugrott} pointed action \emph{ra-} `at' + vb. \emph{ugrik}
`to leap/jump' + 3rd pers. sg. ind. past. indef. \emph{-ott}
$\rightarrow$ \emph{ugrott} \\
\textbf{reggeli} \emph{reggel} `morning' + adj.-forming \emph{-i} \\
\textbf{rend} `order' \\
\textbf{rezgő} poetic, arch. vb. \emph{rezg} `to tremble, quiver'
(alt. \emph{rezeg}) + pres. part. \emph{-ő} \\
\textbf{rossz} adj. `bad' \\
\textbf{rólad} pron. \emph{róla} `about, off her' + 2nd pers. sg. \\
\textbf{röffent} \emph{röffen} `to grunt once (of a pig)' \\
%%
\ \\
\noindent \textbf{\large S} \\
\textbf{s} poetic abbrv. \emph{es} `and' (does not count as a
syllable) \\
\textbf{sabadságot} \emph{szabad} `free' + \emph{-ság} (state of
being): `freedom' + acc. \emph{-at} \\
\textbf{sárga} `yellow' \\
\textbf{se} alt. \emph{sem}, \emph{sem ..., sem ...}  `neither
  ... nor ...' \\
\textbf{sebed} \emph{seb} `wound' + 2nd pers. sg. single
poss. \emph{-ed} \\
\textbf{senkinek} pron. \emph{senki} `no one, nobody' \\
\textbf{semmi} `nothing' \\
\textbf{semmiben} \emph{semmi} `nothing' + innes. \emph{-ben} `in' \\
\textbf{sírni} vb. \emph{sír} `to cry, weep' + inf. \emph{-ni} \\
\textbf{soha} adv. `never' \\
\textbf{sok} adj. `much, many' \\
\textbf{sokat} \emph{sok} `many' + acc. \emph{-at} \\
\textbf{sovány} `slim, thin', fig. `poor' \\
\textbf{sötétben} adj., subst. \emph{sötét} `dark[ness]' +
iness. sg. \emph{-ben} `in' \\
\textbf{súlyos} \emph{súly} `weight' + adj.-forming \emph{-os}
$\rightarrow$ adj. `severe, austere' \\
\textbf{sült} vb. \emph{sül} `to roast' + past part. \emph{-t} \\
\textbf{süt} vb. `to roast, shine', 3rd pers. sg. ind. pres. indef. \\
%%
\ \\
\noindent \textbf{\large Sz} \\
\textbf{szabadulsz} \emph{szabad} `free' + vb.-forming \emph{-ul} +
2nd pers. sg. ind. pres. indef. \emph{-sz} \\
\textbf{szaladsz} vb. \emph{szalad} `to run' + 2nd
pers. sg. ind. pres. indef. \emph{-sz} \\
\textbf{szalló} vb. \emph{száll} `to fly' + pres. part. \emph{-ó} \\
\textbf{szavára} \emph{szó} `voice', `word' + 3rd pers. sg. single
poss. \emph{-a} $\rightarrow$ \emph{szava} + subl. \emph{-ra} `by, on[to]'
$\rightarrow$ \emph{szavára} \\
\textbf{szád} \emph{száj} `mouth' + 2nd
pers. sg. single\hyp{}poss. poss. \\
\textbf{szálaiból} \emph{szál} `thread' + 3rd
pers. sg. poss. \emph{-a} + poss. pl. \emph{-i} + elative \emph{-ból}
`from, out of' \\
\textbf{száll} vb. `to fly, to land, to come down', 3rd
pers. sg. ind. pres. \\
\textbf{szállnak} vb. \emph{száll} `to fly, land, come down' + 3rd
pers. pl. ind. pres. indef. \\
\textbf{számhoz} \emph{száj} `mouth' + 1st pers. sg. single
poss. \emph{-m} $\rightarrow$ \emph{szám} +
allat. `towards' \emph{-hoz} $\rightarrow$ \emph{számhoz} \\
\textbf{százannyit} \emph{száz} `hundred' + \emph{annyit} `that much,
as much' $\rightarrow$ `hundredfold' \\
\textbf{szegénységemhez} \emph{szegény} `poor' +
abst. noun\hyp{}forming \emph{-ség} + 1st pers. sg. single\hyp{}poss. \emph{-em} +
allat. \emph{-hez} `to' \\
\textbf{szelid} adj. `tame (of an animal), gentle' \\
\textbf{szemed} \emph{szem} `eye' + 2nd
pers. sg. single\hyp{}poss. poss. \emph{-ed} (but plural implied
here!) \\
\textbf{szemeim} \emph{szem} `eye' + 1st
pers. sg. mult. poss. \emph{-eim} \\
\textbf{szenvedés} vb. \emph{szenved} `to suffer' +
noun\hyp{}forming \emph{-és} \\
\textbf{szeress-e} vb. \emph{szeret} `to love' + 2nd pers. sg. subj. +
polar interr. \emph{-e} \\
\textbf{szeret} vb. `to love, like' + 2nd
pers. sg. inform. (or 3rd pers. sg.) ind. pres. indef. \\
\textbf{szereti} vb. \emph{szeret} `to love' + 3rd
pers. sg. ind. pres. indef. \emph{-i} \\
\textbf{szeretném} vb. \emph{szeret} `to love' + 1st
pers. sg. cond. pres. def. \\
\textbf{szeretni} inf. vb. \emph{szeret} `to love' \\
\textbf{szerettem} vb. \emph{szeret} `to love' + 1st
pers. sg. ind. past \\
\textbf{széken} \emph{szék} `chair' +  super. \emph{-en} `on' \\
\textbf{szél} `wind' \\
\textbf{szénre} \emph{szén} `coal' + subl. \emph{-re} `onto' \\
\textbf{szépen} adj. \emph{szép} `beautiful' \\
\textbf{széthull} vb. `to fall apart', 3rd
pers. sg. ind. pres. indef. \\
\textbf{szétosztja} vb. pref. \emph{szét-} `apart, around'
+ \emph{oszt} `dispense, divide, deal' + 3rd
pers. sg. ind. pres. def. \emph{-ja} \\
$\rightarrow$ `to distrivute, give out' \\
\textbf{szigorú} adj. `severe, strict' \\
\textbf{szilánkja} \emph{szilánk} `chip, shard, splinter' + 3rd
    pers. sg. single\hyp{}poss. \emph{-ja} \\
\textbf{szívem} \emph{szív} `heart' + 1st pers. sg. single
poss. \emph{-em} `my' \\
\textbf{szivemhez} \emph{szívemhez}: \emph{szívem} `heart' +
allat. \emph{-hez} `towards' \
\textbf{szíved} \emph{szív} `heart' + 2nd pers. sg. single
poss. \emph{-ed} \\
\textbf{szívesen} \emph{szív} `heart' + adj.\hyp{}forming \emph{-es} +
adv.\hyp{}forming \emph{-en} $\rightarrow$ `heartily, gladly' \\
\textbf{szívében} \emph{szív} `heart' + poss. \emph{-e} `his' + innes. `in' \emph{-ben} \\
\textbf{szomszédunkba} \emph{szomszéd} `neighbour' + 1st
pers. pl. single\hyp{}poss. \emph{-unk} + illative \emph{-ba} `into' \\
\textbf{szorítja} vb. \emph{szorít} `to press, constrict',
inf. \emph{szorítani} + 3rd pers. sg. ind. pres. def. \emph{-ja} \\
\textbf{szobában} \emph{szoba} `room' + innes. \emph{-ban} `in' \\
\textbf{szótlan} \emph{szó} `word, voice' + privative
adj.-forming \emph{-tlan} `-less' \\
\textbf{szösz} `harl, fluff' \\
\textbf{szövedéke} \emph{szövedék} `fabric, cloth, web' + 3rd
pers. sg. single\hyp{}poss. \emph{-e} \\
\textbf{szövőszéke} \emph{szövő} `weaver' + \emph{szék} `chair':
`loom' + 3rd pers. sg. single\hyp{}poss. \emph{-e} \\
\textbf{szőke} adj. \emph{sző} `blonde' + dim. \emph{-ke} \\
\textbf{szőtt} vb. \emph{sző} `to weave', past part. \\
\textbf{szunyókálva} vb. pref. \emph{el-} `away' +
vb. \emph{szunyókál} `to take a nap, to doze off' +
adv.-part. \emph{-va} \\
\textbf{szürke} adj. `grey' \\
%%
\ \\
\noindent \textbf{\large T} \\
\textbf{tagjaimban} \emph{tag} `member' + 1st
pers. sg. poss. \emph{-ja} + poss. pl. \emph{-i} + 1st
  pers. sg. pers. \emph{-m} + innes. \emph{-ban} `in'
  $\rightarrow$ \emph{tagjaimban} \\
\textbf{takarót} vb. \emph{takar} `to cover (protect)' +
pres. part. \emph{-ó} $\rightarrow$ `cover, blanket' + acc. \emph{-t} \\
\textbf{talált} vb. \emph{talált} `to find' + past part. \\
\textbf{tájra} \emph{táj} `landscape, region' + subl. \emph{-ra}
`on[to]' \\
\textbf{tárgyat} \emph{tárgy} `object, thing, subject' +
acc. \emph{-at} \\
\textbf{te} inform. sg. `you' \\
\textbf{tehettem} vb. \emph{tesz} `to do' + poten. \emph{-het} + 1st
pers. sg. ind. past \\
\textbf{teher} `burden, freight, load'\\
\textbf{teherpályaudvaron} \emph{teher} `freight' + \emph{pálya}
  `track' + \emph{udvar} `courtyard' + super. \emph{-on} `on' \\
\textbf{tehervonatok} \emph{teher} `freight' + vb. \emph{von} `to
pull, draw' + noun\hyp{}forming \emph{-at} $\rightarrow$ `freight train' +
pl. \emph{-ok} \\
\textbf{települ} vb. `to settle (somewhere)' + 3rd
pers. sg. pres. ind. indef. \\
\textbf{tengerfenékről} \emph{tenger} `sea' + \emph{fenék} `bottom' +
delat. \emph{-ől} `from, about, off' \\
\textbf{terád} pron. sg. \emph{te} `you' (here emph.) +
pron. \emph{rá} + 2nd pers. sg. `on, of you' \\
\textbf{terheim} \emph{teher} `freight, burden' + 1st
pers. sg. poss. mult. poss. \emph{-eim}
$\rightarrow$ \emph{terheim} \\
\textbf{termő} adj. `productive (plants, fruits), yielding, fertile,
fruitful' \\
\textbf{testvér} `sibling, sister, brother' \\
\textbf{téged} pron. \emph{te} `you (sg.)' + acc. \\
\textbf{tépem} vb. \emph{tép} `to tear, rend' + 1st
pers. sg. ind. pres. def. \emph{-em} \\
\textbf{tétovázva} vb. \emph{tétova} `to hesitate'
(inf. \emph{tétovazik}) $\rightarrow$ adv. part. `doing smth
hesitantly' \\
\textbf{tipegsz} vb. \emph{tipeg} `to toddle' + 2nd
pers. sg. inform. ind. pres. \\
\textbf{tiszta} adj. `bright' \\
\textbf{tisztábban} adj. \emph{tiszta} `clean, clear' +
comp. \emph{-bb} +   deadj. adv.-forming \emph{-an} $\rightarrow$
`more clearly, cleanly' \\
\textbf{tolatnak} vb. \emph{tol} `to reverse, back up' +
caus. \emph{-at} `to make smth, smb do smth' + 3rd
pers. pl. ind. pres. indef. \\
\textbf{totyogott} vb. \emph{totyog} `to toddle' (syn. \emph{tipeg}) +
2nd pers. form. ind. pres. indef. \\
\textbf{tócsába} \emph{tócsa} `puddle' + illative \emph{-ba} `into' \\
\textbf{törvényt} \emph{törveny} `law, principle' + acc. \emph{-t} \\
\textbf{tövéhez} \emph{tö} `base of the stem of a plant' + 3rd
pers. sg. single\hyp{}poss. \emph{-e} `of' $\rightarrow$ \emph{töve} +
  allat. \emph{-hez} `towards' $\rightarrow$ \emph{tövéhez} \\
\textbf{tréfát} \emph{tréfá} `joke' + acc. \emph{-at} \\
\textbf{tudja} vb. \emph{tud} `to know' + 3rd
pers. sg. ind. pres. def. \emph{-ja} \\
\textbf{tudok} vb. \emph{tud} `to know' \textbar\textbar{} aux. `can +
inf.' + 1st pers. sg. ind. pres. indef. \emph{-ok} \\
\textbf{tünődik} \emph{tűnődik} vb. `to ponder, wonder' \\
%%
\ \\
\noindent \textbf{\large U} \\
\textbf{udvar} `yard' \\
\textbf{ugrál} vb. \emph{ugrik} `to jump' + freq. \emph{-ál}
    $\rightarrow$ `to bounce, leap (repeatedly)' \\
%%
\ \\
\noindent \textbf{\large Ú} \\
\textbf{úr} `Lord' (the Christian god), `master' \\
\textbf{úgy} adv. `so (in this/that manner)'; \emph{úgy ... mint ...}
`like'; \emph{úgy ... ha ...} `then ... if ...' \\
%%
\ \\
\noindent \textbf{\large Ü} \\
\textbf{ügyeskedhet} vb. \emph{ügyesked} `to be clever' +
  3rd pers. sg. ind. pres. potential \emph{-het} \\
\textbf{ütött} vb. \emph{üt} `to strike' + 3rd pers. sg. ind. past
indef. \emph{-ött} \\
\textbf{ütnek} vb. \emph{üt} `to hit' + 3rd
pers. pl. ind. pres. indef. \emph{-nek} \\
%%
\ \\
\noindent \textbf{\large V} \\
\textbf{vagónokon} \emph{vagón} `wagon' + pl. \emph{-ok} +
super. \emph{-on} `on' \\
\textbf{vagy} vb. \emph{van} `to be' + 2nd pers. sg. ind. pres. indef. \\
\textbf{valahol} pron. pref. \emph{vala} `some-' + interr. adv. \emph{hol} `where?' $\rightarrow$ `somewhere' \\
\textbf{valaki} pron. pref. \emph{vala} `some-' +
pron. rel. arch. \emph{ki} `who (the person(s) that)' $\rightarrow$
`somebody, someone' \\
\textbf{valamelyik} pron. pref. \emph{vala} `some-' +
    interj. \emph{myelik} `which, what' $\rightarrow$ `one, any of' \\
\textbf{valamiért} pron. pref. \emph{vala} `some-' + \emph{mi} `what' $\rightarrow$ `something' +
causal\hyp{}final \emph{ért} `for' $\rightarrow$ `for some reason' \\
\textbf{vannak} vb. \emph{van} `to be'+ 3rd
pers. pl. ind. pres. indef. \\
\textbf{vas} `iron' \\
\textbf{vasat} \emph{vas} `iron' + acc. \emph{-at} \\
\textbf{vaságyamra} \emph{vas} `iron' + \emph{ágy} `bed' + 1st
pers. sg. single\hyp{}poss. \emph{-am} + subl. \emph{-ra} `on[to]' \\
\textbf{vasútnál} \emph{vas} `iron' + \emph{ut} `way, road'+
adess. \emph{-nál} `at, by' \\
\textbf{vasszilánk} \emph{vas} `iron' + \emph{szilánk} `shard' \\
\textbf{vágy} `desire, longing' \\
\textbf{válaszolhat} \emph{válasz} `reply, answer' +
vb.-forming \emph{-ol} + poten. \emph{-hat} + 3rd
pers. sg. ind. pres. indef. \\
\textbf{vált} vb. inf. \emph{válik} `to separate, grow apart' +
caus. \emph{-t} \\
\textbf{várod} vb. \emph{vár} `to wait' + 2nd
pers. sg. ind. pres. def. \\
\textbf{velem} pron. \emph{vele} `with him' + 1st
pers. sg. $\rightarrow$ `with me' \\
\textbf{veretni} inf. vb. `to hammer', alt. \emph{veretés} \\
\textbf{versemet} \emph{vers} `verse, poem' + 1st pers. sg. single
poss. \emph{-em} + acc. \emph{-et} \\
\textbf{veszek} vb. \emph{vesz} `to take, grab, buy, get, consider, put on'
  + 1st pers. sg. ind. pres. indef. \emph{-ek} \\
\textbf{vetettek} vb. \emph{vet} `to sow', inf. \emph{vetni} + 3rd
pers. pl. ind. past. indef. \emph{-tek} \\
\textbf{véle} poetic, arch. form of \emph{vele} `with him'
\textbf{véletlen} `chance' \\
\textbf{vén} adj. `old' \\
\textbf{vigyázva} vb. \emph{vigyáz} `to protect, pay attention' +
  adv. part. \emph{-va} result of action or simultaneous actions \\
\textbf{világ} `light (arch.), world' \\
\textbf{világít} \emph{világ} `light (arch.), world' +
caus. vb.-forming \emph{-ít} `to make smth ...-like' + 1st
pers. sg. ind. pres. indef. `to make shine' \\
\textbf{vissza} adv. `back[wards]' \\
\textbf{visszaadja} vb. \emph{visszaad} `to give back' + 3rd
  pers. sg. ind. pres. def. \emph{-ja} \\
\textbf{visszajössz} vb. pref. \emph{vissza-} `back-' + vb. \emph{jön}
  `to come'+ 2nd pers. sg. ind. pres. indef. \emph{-sz} \\
\textbf{virág} `flower' \\
\textbf{volna} vb. \emph{van} `to be' + 3rd
pers. sg. cond. pres. `would be' \\
\textbf{volt} vb. \emph{van} `to be'+ 3rd pers. sg. ind. past
indef. \\
\textbf{vonat} vb. \emph{von} `to pull, draw' +
noun\hyp{}forming \emph{-at} $\rightarrow$ `train' \\
%%
\ \\
\noindent \textbf{\large Zs} \\
\textbf{zsenge} `first born, young, immature' \\


%----------------------------------------------------------------------

\thispagestyle{empty} \ \clearemptydoublepage
\thispagestyle{empty} \ \clearemptydoublepage

\end{document}
