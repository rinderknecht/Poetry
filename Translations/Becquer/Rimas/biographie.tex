%-*-latex-*-

\documentclass[a4paper,fontsize=13pt,twoside,final]{scrbook}

% Page geometry for L'Harmattan
%
\usepackage[bindingoffset=0cm,a4paper,left=55mm,top=6cm,textheight=185mm,textwidth=10cm,includefoot,includehead,dvips]{geometry}% showframe, showcrop

\usepackage[T1]{fontenc}
\usepackage[utf8]{inputenc}
\usepackage[french]{babel}
\usepackage{ebgaramond}

\hyphenation{Gon-zá-lez thè-mes stro-phes con-trainte}

\begin{document}

\pagestyle{empty}

Gustavo Adolfo Bécquer naît à Séville en \oldstylenums{1836}, dans une
famille nombreuse. Son père, qui s'était distingué comme peintre,
meurt en \oldstylenums{1841}. Gustavo va à l'école en
\oldstylenums{1846}, où il reçoit une éducation en lettres
classiques. Un an après le décès de sa mère en \oldstylenums{1847}, il
rejoint une école des beaux arts, mais abandonne ses études en
\oldstylenums{1850}. Il reprend un cursus normal et publie prose et
poésie dans des revues sévillanes en \oldstylenums{1853} et
\oldstylenums{1854} ---~année où il se rend à Madrid et y publie des
critiques musicales et théâtrales.

En \oldstylenums{1856}, il s'emploie à dessiner et décrire une
histoire des lieux de cultes espagnols. En \oldstylenums{1858} sont
publiés certains de ses récits en proses, nommés \emph{Leyendas}
(Légendes), où un Moyen-Âge romantisé évoque une gloire passée.

Julia Espín devient sa muse de \oldstylenums{1858}
à \oldstylenums{1860}. En plus de ses poèmes, qu'il appelle
ses \emph{rimas} (rimes), il écrit aussi durant cette période des
opérettes espagnoles (\emph{zarzuelas}) et collabore avec son frère
Valeriano, peintre.

Il épouse Casta Esteban Navarro en
\oldstylenums{1861}. Jusqu'en \oldstylenums{1864}, il continue à
publier ses \emph{Leyendas} et le ministre Luis González Bravo le
nomme censeur des romans, ce qui le met enfin à l'abri du besoin.

En \oldstylenums{1868}, une révolution antimonarchique détrône la
rei-ne Isabel~II, et fait choir son gouvernement, dont Luis González
Bravo faisait partie. Celui-ci avait reçu de Bécquer un manuscrit
contenant, entre autres, les \emph{rimas}, et le saccage par la foule
du domicile du ministre déchu vit la disparition de ce premier
manuscrit. Bécquer perd son poste et retourne à la vie précaire de
journaliste. Il réécrit ce texte, qu'il intitule \emph{El libro de los
gorriones} (Le livre des moineaux). Le projet prévoyait une première
partie en prose (restée inachevée), et la seconde en vers
---~complète, semble-t-il. Il se sépare de son épouse, mais parvient à
rester en bons termes avec elle.

Il meurt en \oldstylenums{1870}, peu après son frère Valeriano. Un an
après paraît son œuvre en deux volumes, intitulée \emph{Obras}
(Œuvres), où presque tous les poèmes du \emph{livre des moineaux} sont
repris, corrigés parfois, et ordonnés par thèmes. Il retint du
romantisme le lyrisme, mais le dépassa par ses thèmes tantôt
symboliques, tantôt réalistes, ainsi que par une recherche esthétique
pour elle-même et un style direct.

\end{document}
