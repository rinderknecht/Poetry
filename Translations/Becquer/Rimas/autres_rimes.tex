\begin{center}
  \textbf{80}
  \addcontentsline{toc}{section}{\emph{80.\ La vie est un songe}}
\end{center}

\begin{verse}
  La vie est un songe, \\
  mais un songe fébrile qui dure un point; \\
  quand on s'en éveille \\
  on voit que tout est vanité et fumée...

  Si seulement elle était un songe \\
  très long et très profond, \\
  un songe durant jusqu'à la mort... \\
  Je rêverais de mon amour et du tien.
\end{verse}

\begin{center}
  \textbf{81}
\end{center}

\vspace*{-20pt}
\poemtitle{Amour éternel}

\begin{verse}
  Le soleil peut bien s'ennuager éternellement; \\
  la mer s'assécher en un instant; \\
  l'axe de la Terre se rompre \\
  comme un cristal fragile.

  Advienne que pourra! La mort peut bien \\
  me recouvrir de sa crêpe funèbre, \\
  mais jamais ne s'éteindra en moi \\
  la flamme de ton amour.
\end{verse}

\begin{center}
  \textbf{82}
\end{center}

\vspace*{-20pt}
\poemtitle{Pour Casta}

\begin{verse}
  Ton\footnote{Casta Esteban Navarro, qui épousa
  l'auteur en 1861.} haleine est l'haleine des fleurs, \\
  ta voix est l'harmonie des cygnes, \\
  ton regard est la splendeur du jour, \\
  et la couleur des roses est ta couleur.

  Tu prêtes vie neuve et espoir \\
  à un cœur pour l'amour déjà mort; \\
  tu croîs de ma vie dans le désert \\
  comme la fleur dans les plateaux.
\end{verse}

\bigskip

\begin{center}
  \textbf{83}
\end{center}

\vspace*{-20pt}
\poemtitle{La goutte de rosée}

\begin{verse}
  La goutte de rosée qui dort \\
  dans le calice du lys blanc \\
  est le palais de cristal où \\
  vit le génie heureux de la pureté.\footnote{Voir les rimes~52 et~85.}

  Il lui donne son mystère et sa poésie, \\
  il lui prête son arôme balsamique. \\
  Ah! Que de la lumière au baiser \\
  ne s'évapore cette perle de la fleur!
\end{verse}

\bigskip

\begin{center}
  \textbf{84}
  \addcontentsline{toc}{section}{\emph{84.\ Loin, parmi les arbres de la jungle intriquée}}
\end{center}

\begin{verse}
  Loin, parmi les arbres \\
  de la jungle intriquée, \\
  ne vois-tu quelque chose qui brille \\
  et pleure? C'est une étoile.

  On la voit déjà plus proche \\
  briller au portique d'une ermitane,\footnote{Femme ermite.} \\
  comme au travers d'un tulle. \\
  C'est un réverbère.

  La course rapide s'achève ici. \\
  Désillusion. La lumière que nous avons suivie \\
  n'est ni réverbère ni étoile: \\
  c'est une lampe à huile.
\end{verse}

\bigskip\bigskip

\begin{center}
  \textbf{85}
\end{center}

\vspace*{-20pt}
\poemtitle{À tous les saints (Premier novembre)}

\begin{verse}
  Patriarches, qui furent la semence \\
  de l'arbre de la foi des siècles lointains, \\
  priez pour nous \\
  le divin vainqueur de la mort.

  Prophètes inspirés, qui déchirèrent \\
  le voile mystérieux de l'avenir, \\
  priez pour nous \\
  celui qui sépara la lumière des ténèbres.

  Âmes candides, Saints Innocents, \\
  qui accrurent le chœur des anges, \\
  priez pour nous \\
  celui qui appela les enfants à son côté.

  Apôtres, qui établirent les fondations \\
  de l'Église dans le monde, \\
  priez pour nous \\
  le dépositaire de la vérité.

  Martyres qui remportèrent leur palme \\
  rouge de sang dans l'arène des cirques, \\
  priez pour nous \\
  celui qui vous donna fortitude dans les luttes.

  Vierges semblables au lys, \\
  que l'été vêtit de neige de d'or,\footnote{Voir rimes~52 et~83.} \\
  priez pour nous \\
  celui qui est source et perfection.

  Moines, qui dans le combat de la vie \\
  demandèrent paix au cloître silencieux, \\
  priez pour nous \\
  celui qui est arc-en-ciel de calme dans les tempêtes.

  Docteurs, dont les plumes nous léguèrent \\
  des trésors de vertu et de savoir, \\
  priez pour nous \\
  celui qui est torrent de science intarissable.

  Soldats de l'armée du Christ, \\
  tous Saintes et Saints, \\
  priez celui qui vit et règne parmi nous \\
  pour que nos fautes nous soient pardonnées.
\end{verse}

\bigskip

\begin{center}
  \textbf{86}
\end{center}

\vspace*{-20pt}
\poemtitle{Dans l'album de Madame}

\begin{verse}
  Ce cimetière \\
  est solitaire, triste et muet; \\
  ses habitants ne pleurent pas... \\
  Qu'ils sont heureux, les morts!
\end{verse}
