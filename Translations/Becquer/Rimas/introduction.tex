\selectlanguage{francais}

Notre traduction reprend l'ordre du \emph{Livre des moineaux}, où
chaque poème est numéroté avec des chiffres arabes, mais nous
adjoignons aussi la numérotation des \emph{Œuvres}, en chiffres
romains. Nous avons repris les corrections posthumes et parfois pris
la peine de citer en notes les variantes de certains vers ou strophes,
non pour constituer un semblant d'apparat critique, mais pour faire
sentir au lecteur le processus créateur. Nous avons consulté une
source de référence, celle du \emph{Centro Virtual Cervantes} en
Espagne, mais aussi une édition de \emph{Rimas y leyendas}
de \oldstylenums{1984} aux éditions Orbis. Nous avons ajouté un
chapitre intitulé \emph{Autre rimes} regroupant des poèmes attribués à
l'auteur par la critique, mais qui ne faisaient ni partie
du \emph{Livre des moineaux} ni des
\emph{Œuvres}.

Les poèmes de Bécquer sont connus aujourd'hui sous le vocable de
\emph{Rimas} (Rimes), car l'auteur les appelaient ainsi auprès de ses
amis. Malgré l'apparence de vers libres, beaucoup de ses poèmes riment
au sens où ils contiennent des correspondances internes, souvent des
allitérations, des répétitions de mots, des structures parallèles, des
progressions etc. Bécquer joue beaucoup avec la syntaxe espagnole pour
réaliser ces rimes. Plutôt que faire systématiquement de même en
français, où l'ordre des propositions et des adjectifs est plus
contraint, nous avons opté pour une traduction plus fluide, surtout
dans les longs poèmes, pour ne pas égarer le lecteur. Nous avons
néanmoins tâché de recréer certaines allitérations, mais surtout les
structures entre strophes et vers, comme par exemple la mise en
exergue de certains mots, au début ou à la fin de certains vers.

L'espagnol du milieu du XIX\ieme{} siècle a changé: nous avons
consulté des sources philologiques pour traduire correctement certains
mots. Par ailleurs, certains termes religieux sont devenus obscurs:
nous avons fourni des notes pour les expliquer brièvement. Bécquer
avait une ponctuation idiosyncratique ---~quand elle n'était pas
absente ou surnuméraire: nous avons pris la liberté d'user d'une
ponctuation moderne qui sert la compréhension plutôt que le style,
surtout dans les poèmes les plus longs.

Gustavo Adolfo Bécquer est bien connu des espagnols pour sa poésie,
bien que sa prose soit plus volumineuse, parce que certains de ses
poèmes sont étudiés et appris par cœur dans les écoles, mais surtout
parce que leur lyrisme original sait toucher les jeunes cœurs. Voici
quelques thèmes qui traversent son œuvre: l'existence, avec sa cohorte
habituelle: destin, incertitude, mort, aspiration au repos
existentiel; les galanteries amoureuses; l'amour perdu avec son
aréopage de regrets, insomnies, mais aussi dépit et rancune; la
musique; la nature; enfin, la métapoésie, c'est-à-dire des poèmes sur
l'écriture poétique elle-même, sur le sujet poétique (en particulier,
l'idéal féminin), avec parfois des éléments platoniciens qui touchent
au symbolisme.

\bigskip
\bigskip
\bigskip
\bigskip
\hfill Christian Rinderknecht
