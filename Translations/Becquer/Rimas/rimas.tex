%-*-latex-*-

\documentclass[a4paper,12pt]{book}

\usepackage[T1]{fontenc}
\usepackage[utf8]{inputenc}
\usepackage[french]{babel}
\usepackage[charter]{mathdesign}
\usepackage{verse}
\usepackage{url}

\begin{document}
\thispagestyle{empty}
\vspace*{70mm}
\begin{center}
{\Huge\textbf{RIMES}} \\
\vspace*{10mm}
{\Large Gustavo Adolfo Bécquer} \\
\vspace*{10mm}
Traduction de Christian Rinderknecht\\
\url{rinderknecht@free.fr}
\end{center}

\cleardoublepage

%\selectlanguage{francais}
\frenchspacing  % Follow French conventions after a period

%% Parmi les recoins ténébreux de mon cerveau dorment, blottis et nus,
%% les enfants extravagants de ma fantaisie, attendant en silence que
%% l'Art les vêtent de mots pour se présenter décemment sur la scène du
%% monde.


\newpage

\begin{center}
  \textbf{1}
  \addcontentsline{toc}{subsection}{\emph{1.\ Je sais un hymne géant et étrange...}}
\end{center}

\settowidth{\versewidth}{qui puisse l'enfermer, et c'est à peine, ô ma belle!,}

\begin{verse}[\versewidth]
  Je sais un hymne géant et étrange \\
  qui annonce dans la nuit de l'âme une aurore, \\
  et ces pages sont de cet hymne \\
  des cadences que l'air dilate dans l'ombre.

  Je voudrais l'écrire, domptant \\
  de l'homme la rebelle langue mesquine, \\
  avec des mots qui soient à la fois \\
  soupirs et rires, couleurs et notes.

  Mais vaine est la lutte: il n'est aucune mesure \\
  qui puisse l'enfermer, et c'est à peine, ô ma belle!, \\
  si, en tenant dans mes mains les tiennes, \\
  je peux te le conter seul à seul à l'oreille.
\end{verse}

\bigskip

\begin{center}
  \textbf{2}
  \addcontentsline{toc}{subsection}{\emph{2.\ \emph{Saeta} qui
    traverse en volant...}}
\end{center}

\settowidth{\versewidth}{où, tremblante, elle se plantera;}

\begin{verse}[\versewidth]
  \emph{Saeta}\footnote{Courte prière chantée
  depuis les balcons au passage des trônes portant des scènes de la
  Passion du Christ, pendant la Semaine Sainte, principalement en
  Andalousie. L'étymologie est le latin \emph{sagitta},
  signifiant \emph{flèche}, d'où la métaphore qui suit.} qui traverse en volant, \\
  lancée au hasard \\
  sans qu'on ne sache \\
  où, tremblante, elle se plantera;

  feuille sèche de l'arbre \\
  emportée par la bourrasque,\footnote{Il pourrait s'agir aussi, au
  sens propre, du \emph{vendaval}, un vent du sud soufflant sur la
  vallée du Guadalquivir, qui traverse Séville.} \\
  et on ne devine le sillon \\
  où elle retombera;

  vague géante que le vent \\
  enfle et pousse dans la mer, \\
  et roule et passe, et ne sait \\
  quelle rivage elle va cherchant;

  lueur qui, prête à s'éteindre, \\
  brille en ronds tremblants, \\
  et on ne sait d'eux \\
  lequel sera le dernier:

  c'est moi qui, au hasard, \\
  traverse le monde sans penser \\
  d'où je viens, ni où \\
  mes pas me mèneront.
\end{verse}

\bigskip

\begin{center}
  \textbf{3}
  \addcontentsline{toc}{subsection}{\emph{3.\ Secousse étrange qui
    agite les idées...}}
\end{center}

\settowidth{\versewidth}{comme au travers d'un tulle,}

\begin{verse}[\versewidth]
  Secousse étrange \\
  qui agite les idées, \\
  comme ouragan qui pousse \\
  les vagues au galop;

  murmure qui dans l'âme \\
  s'élève et va croissant, \\
  comme volcan qui, sourd, \\
  annonce qu'il va s'embraser;

  silhouettes difformes \\
  d'êtres impossibles; \\
  paysages qui apparaissent \\
  comme au travers d'un tulle;

  couleurs qui, en se fondant, \\
  imitent dans l'air \\
  les atomes de l'iris, \\
  qui nagent dans la lumière;

  idées sans paroles, \\
  paroles insensées; \\
  cadences qui n'ont \\
  ni rythme ni mesure;

  souvenirs et désirs \\
  de ce qui n'existe pas; \\
  transports de joie, \\
  envies de pleurer;

  activité nerveuse \\
  qui erre sans emploi, \\
  sans rênes qui guident \\
  ce cheval ailé;

  folie que l'âme \\
  exalte et enflamme, \\
  ivresse divine \\
  du génie créateur...

  Telle est l'inspiration!

  Voix géante qui ordonne \\
  le chaos dans le cerveau, \\
  et, parmi les ombres, fait \\
  apparaître la lumière;

  brillante rêne d'or \\
  qui, puissante, freine \\
  de l'esprit exalté \\
  le coursier volant;

  fil de lumière qui en gerbes \\
  noue les pensées, \\
  soleil qui rompt les nuées \\
  et atteint le zénith;

  main intelligente \\
  qui, en un collier de perles, \\
  parvient à réunir \\
  les mots indociles;

  rythme harmonieux \\
  qui, avec cadence et nombre, \\
  enserre dans la mesure \\
  les notes fugitives;

  ciseau qui mord dans le bloc, \\
  modelant la statue, \\
  et la beauté plastique \\
  ajoute à l'idéale;

  atmosphère où tournent \\
  les idées en ordre, \\
  telles des atomes que réunit \\
  une attraction secrète;

  torrent où la fièvre \\
  éteint sa soif; \\
  oasis qui à l'esprit \\
  rend sa vigueur...

  Telle est notre raison!

  Avec ces deux\footnote{Inspiration et raison.} toujours en lutte \\
  et des deux vainqueur, \\
  tant il n'est donné qu'au génie \\
  de les mettre sous le même joug.
\end{verse}

\bigskip

\begin{center}
  \textbf{3}
  \addcontentsline{toc}{subsection}{\emph{3.\ Ne dites pas que, épuisé
    son trésor...}}
\end{center}

\settowidth{\versewidth}{tant qu'il existera une femme splendide,}

\begin{verse}[\versewidth]
  Ne dites pas que, épuisé son trésor, \\
  faute de sujet, la lyre s'est tue: \\
  il pourrait ne pas y avoir de poètes, \\
  mais toujours il y aura la poésie.

  Tant que les ondes embrasées \\
  de la lumière palpiteront aux baisers, \\
  tant que le soleil vêtira \\
  les nuées déchirées de feu et d'or; \\
  tant que l'air en son giron portera \\
  parfums et harmonies; \\
  tant qu'il aura un printemps au monde, \\
  il y aura la poésie!

  Tant que la science échouera à découvrir \\
  la source de la vie, \\
  et qu'en mer ou au ciel il y aura un abîme \\
  qui résiste au calcul; \\
  tant que l'humanité, toujours progressant, \\
  ne saura où elle va; \\
  tant qu'il aura un mystère pour l'homme, \\
  il y aura la poésie!

  Tant que l'on sentira l'âme se réjouir \\
  sans que les lèvres rient; \\
  tant que l'on pleurera sans que le sanglot \\
  ne vienne troubler la pupille; \\
  tant que le c{\oe}ur et la tête \\
  continueront à batailler; \\
  tant qu'il y aura espoirs et souvenirs, \\
  il y aura la poésie!

  Tant qu'il y aura des yeux qui reflètent \\
  les yeux qui les regardent, \\
  tant que répondra la lèvre soupirant \\
  à la lèvre qui soupire; \\
  tant que deux âmes en un baiser \\
  confondues pourront se toucher; \\
  tant qu'il existera une femme splendide, \\
  il y aura la poésie!
\end{verse}

\bigskip

\begin{center}
  \textbf{5}
  \addcontentsline{toc}{subsection}{\emph{5.\ Esprit sans nom,
    indéfinissable essence...}}
\end{center}

\settowidth{\versewidth}{que tapissent de blanches perles,}

\begin{verse}[\versewidth]
  Esprit sans nom, \\
  indéfinissable essence, \\
  je vis avec la vie \\
  sans formes de l'idée.

  Je nage dans le vide, \\
  tremble dans le brasier solaire, \\
  je palpite parmi les ombres \\
  et flotte avec les brumes.

  Je suis la frange d'or \\
  de la lointaine étoile, \\
  je suis de la haute lune \\
  la lumière tiède et sereine.

  Je suis l'ardent nuage \\
  qui ondoie dans le couchant, \\
  je suis de l'astre errant \\
  le sillage lumineux.

  Je suis neige sur les cimes, \\
  je suis feu sur les sables, \\
  onde bleue sur les mers \\
  et écume sur les rivages.

  Dans le luth je suis note, \\
  parfum dans la violette, \\
  flamme fugace dans les tombes \\
  et lierre dans les ruines.

  Je chante avec l'alouette \\
  et bourdonne avec l'abeille; \\
  j'imite les bruits \\
  qui résonnent en pleine nuit.\footnote{NDT. Ce quatrain ne
figure pas dans le manuscrit original, mais dans la publication dans
le journal \emph{El Museo Universal}, page~31, le~28 janvier 1866 (voir \url{prensahistorica.mcu.es}).}

  Je tonne dans le torrent \\
  et siffle dans la foudre, \\
  et aveugle dans l'éclair \\
  et rugis dans la tempête.

  Je ris sur les collines, \\
  susurre dans les herbes hautes, \\
  soupire dans l'onde pure \\
  et pleure sur les feuilles sèches.

  J'ondule avec les atomes \\
  de la fumée qui s'élève \\
  et monte lentement au ciel \\
  en spirales immenses.

  Parmi les fils dorés \\
  que les insectes suspendent, \\
  je me mêle aux arbres \\
  dans l'ardente sieste.

  Je cours après les nymphes \\
  qui, dans le courant frais\footnote{La publication dans
le journal \emph{El Museo Universal}, page~31, le~28 janvier 1866
(voir \url{prensahistorica.mcu.es}) recense: «~le courant inquiet~».} \\
  de la rivière cristalline, \\
  s'ébattent nues.

  Dans des bois de coraux \\
  qui tapissent de blanches perles, \\
  je poursuis dans l'Océan \\
  les naïades légères.

  Dans les cavernes concaves \\
  où le soleil ne pénètre jamais, \\
  me mêlant aux gnomes, \\
  je contemple leurs richesses.

  Je cherche des siècles \\
  les traces effacées, \\
  et je sais de ces empires \\
  dont il ne reste même pas le nom.\footnote{Variante dans le journal
  \emph{El Museo Universal}, page~31, le~28 janvier 1866 (voir
  \url{prensahistorica.mcu.es}): «~Je rencontre les traces effacées~/~de ces siècles,~/~dont il ne reste aucun souvenir~/~sur la face du globe.~»}

  Je poursuis en un brusque vertige \\
  les mondes qui voltigent, \\
  et ma pupille embrasse \\
  la création entière.\footnote{Variante dans le journal
  \emph{El Museo Universal}, page~31, le~28 janvier 1866 (voir
  \url{prensahistorica.mcu.es}): «~J'embrasse du regard~/~la création
  entière,~/~et poursuis en un brusque vertige~/~les astres qui voltigent.~»}

  Je sais de ces régions \\
  qu'une rumeur n'atteint pas, \\
  et où d'informes astres \\
  attendent un souffle de vie.

  Je suis sur l'abîme \\
  le pont qui traverse, \\
  et l'échelle inconnue \\
  qui unit le ciel à la terre.\footnote{Variante dans le journal
  \emph{El Museo Universal}, page~31, le~28 janvier 1866 (voir
  \url{prensahistorica.mcu.es}): «~Je suis l'échelle inconnue~/~qui unit
  le ciel à la terre,~/~et ouvre à la pensée~/~un chemin vers d'autres
  sphères.~»}

  Je suis l'anneau invisible \\
  qui fixe \\
  le monde de la forme \\
  au monde de l'idée.

  Enfin, je suis cet esprit, \\
  essence inconnue,\footnote{Variante dans le journal
  \emph{El Museo Universal}, page~31, le~28 janvier 1866 (voir
  \url{prensahistorica.mcu.es}): «~l'essence du sentiment,~»} \\
  parfum mystérieux \\
  dont le vase est le poète.
\end{verse}

\bigskip

\begin{center}
  \textbf{6}
  \addcontentsline{toc}{subsection}{\emph{6.\ Comme la brise qui
    rafraîchit le sang...}}
\end{center}

\settowidth{\versewidth}{chante et cueille des fleurs en passant.}

\begin{verse}[\versewidth]
  Comme la brise qui rafraîchit le sang \\
  sur le champ sombre des batailles, \\
  chargée de parfums et d'harmonies \\
  dans le silence de la nuit, elle erre;

  symbole de la douleur et de la tendresse, \\
  dans l'horrible drame du barde anglais, \\
  la douce Ophélie,\footnote{Personnage de la pièce de Shakespeare \emph{Hamlet}.} la raison égarée, \\
  chante et cueille des fleurs en passant.
\end{verse}

\bigskip

\begin{center}
  \textbf{7}
  \addcontentsline{toc}{subsection}{\emph{7.\ Dans l'angle obscur du
    salon...}}
\end{center}

\settowidth{\versewidth}{comme dorment les oiseaux sur les branches,}

\begin{verse}[\versewidth]
  Dans l'angle obscur du salon, \\
  de son maître peut-être oubliée, \\
  silencieuse et couverte de poussière, \\
  trônait la harpe.

  Que de notes dormaient sur ses cordes, \\
  comme dorment les oiseaux sur les branches, \\
  attendant la main de neige \\
  qui les fait s'envoler!

  Hélas! pensai-je. Que de fois le génie \\
  ainsi dort au fond de l'âme, \\
  et une voix attend, comme Lazare, \\
  qui lui dise: \emph{Lève-toi et marche!}
\end{verse}

\bigskip

\begin{center}
  \textbf{8}
  \addcontentsline{toc}{subsection}{\emph{8.\ Quand je regarde
    l'horizon bleu...}}
\end{center}

\settowidth{\versewidth}{et m'inonder de leur lumière, et avec elles}

\begin{verse}[\versewidth]
  Quand je regarde l'horizon bleu \\
  se perdre au lointain, \\
  au travers d'une gaze de poussière \\
  dorée et inquiète,

  je crois possible de m'arracher \\
  du sol misérable \\
  et flotter avec la brume dorée \\
  en atomes légers, \\
  défait comme elle.

  Quand je vois de nuit, dans le fond \\
  obscur du ciel, \\
  trembler les étoiles comme d'ardentes \\
  pupilles de feu,

  je crois possible de m'envoler \\
  là où elles brillent, \\
  et m'inonder de leur lumière, et avec elles, \\
  en un feu qui a pris, \\
  me fondre en un baiser.

  Sur la mer de doute où je vogue \\
  je ne sais même pas ce que je crois; \\
  pourtant ces désirs me disent \\
  que je porte quelque chose \\
  de divin, ici en moi.
\end{verse}

\bigskip

\begin{center}
  \textbf{9}
  \addcontentsline{toc}{subsection}{\emph{9.\ Le zéphir qui gémit
    faiblement...}}
\end{center}

\settowidth{\versewidth}{jusqu'à ce que de pourpre et d'or il la nuance;}

\begin{verse}[\versewidth]
  Le zéphyr qui gémit faiblement \\
  baise les ondes légères qu'il plisse en jouant; \\
  le soleil baise la nuée à l'Occident \\
  jusqu'à ce que de pourpre et d'or il la nuance; \\
  la flamme à l'entour du tronc ardent \\
  s'étale en baisant une autre flamme, \\
  et jusqu'au saule pesant, qui se penche \\
  vers la rivière qui le baise, renvoie un baiser.
\end{verse}

\bigskip

\begin{center}
  \textbf{10}
  \addcontentsline{toc}{subsection}{\emph{10.\ Les invisibles atomes de
    l'air alentour palpitent et s'enflamment...}}
\end{center}

\settowidth{\versewidth}{Dis-moi...? Silence! C'est l'amour qui passe!}

\begin{verse}[\versewidth]
  Les invisibles atomes de l'air \\
  alentour palpitent et s'enflamment, \\
  le ciel se défait en rayons d'or, \\
  la terre frémit de joie; \\
  j'entends, flottant sur des ondes d'harmonie, \\
  rumeur de baisers et battements d'ailes, \\
  mes paupières se closent... Qu'arrive-t-il? \\
  ---~C'est l'amour qui passe!
\end{verse}

\bigskip

\begin{center}
  \textbf{11}
  \addcontentsline{toc}{subsection}{\emph{11.\ Je suis ardente, je
    suis brune...}}
\end{center}

\settowidth{\versewidth}{de désirs de jouissance mon âme est pleine.}

\begin{verse}[\versewidth]
  ---~Je suis ardente, je suis brune, \\
  je suis le symbole de la passion; \\
  de désirs de jouissance mon âme est pleine. \\
  Est-ce moi que tu cherches?

  \hfill ---~Ce n'est pas toi, non.

  ---~Mon front est pâle, mes tresses d'or; \\
  je peux t'offrir des bonheurs sans fin; \\
  je garde un trésor de tendresse. \\
  Est-ce moi que tu appelles?

  \hfill ---~Ce n'est pas toi, non.

  ---~Je suis un songe, fantôme \\
  impossible et vain de brume et lumière; \\
  je suis incorporelle, je suis intangible, \\
  je ne puis t'aimer.

  \hfill ---~Oh viens, toi, viens!
\end{verse}

\bigskip

\begin{center}
  \textbf{12}
  \addcontentsline{toc}{subsection}{\emph{12.\ Petite, parce que tes
    yeux sont verts...}}
\end{center}

\settowidth{\versewidth}{Petite, parce que tes yeux}

\begin{verse}[\versewidth]
  Petite, parce que tes yeux \\
  sont verts comme la mer, tu te plains; \\
  verts sont ceux des naïades, \\
  verts les eut Minerve, \\
  et vertes sont les pupilles \\
  des houris\footnote{NDT. Beauté céleste que le Coran promet au musulman dans le paradis d'Allah.} du Prophète.

  Le vert est gala et ornement \\
  de la forêt au printemps; \\
  parmi ses sept couleurs, \\
  l'iris brillant l'affiche; \\
  les émeraudes sont vertes, \\
  verte la couleur de qui espère, \\
  et les ondes de l'Océan \\
  et le laurier des poètes.

  $$\star \ \ \ \star \ \ \ \star$$

  Ta joue est une rose matinale \\
  couverte de rosée congelée, \\
  où le carmin des pétales \\
  se voit à travers des perles.

  Et pourtant, \\
  je sais que tu te plains \\
  car tu crois que tes yeux \\
  l'enlaidissent: \\
  eh bien ne le crois pas,

  car tes pupilles humides, \\
  vertes et inquiètes, \\
  semblent de jeunes feuilles d'amandier, \\
  qui tremblent dans la brise.

  Ta bouche pourpre-rubis \\
  est grenade entrouverte \\
  qui dans l'été invite \\
  à éteindre la soif en elle.

  Et pourtant, \\
  je sais que tu te plains \\
  car tu crois que tes yeux
  l'enlaidissent: \\
  eh bien ne le crois pas,

  car, si fâchée, \\
  tes pupilles scintillent, \\
  tes yeux ressemblent \\
  aux vagues qui se brisent \\
  sur les rochers cantabriques.

  $$\star \ \ \ \star \ \ \ \star$$

  Ton front, couronné \\
  de l'or crépu d'une large tresse, \\
  est une cime enneigée où le jour \\
  reflète sa première lueur.

  Et pourtant, \\
  je sais que tu te plains \\
  car tu crois que tes yeux \\
  l'enlaidissent: \\
  eh bien ne le crois pas,

  car parmi les cils blonds, \\
  proche des tempes, ils semblent \\
  des broches d'émeraude et or \\
  haussant une blanche hermine.

  Petite, parce que tes yeux \\
  sont verts comme la mer, tu te plains; \\
  peut-être, si noirs ou bleus \\
  ils devenaient, tu le regretterais.
\end{verse}

\bigskip

\begin{center}
  \textbf{13}
  \addcontentsline{toc}{subsection}{\emph{13.\ Ta pupille est bleue...}}
\end{center}

\settowidth{\versewidth}{comme un point de lumière irradie une idée,}

\begin{verse}[\versewidth]
  Ta pupille est bleue et quand tu ris \\
  sa clarté suave me rappelle \\
  l'éclat tremblant du matin \\
  qui se reflète dans la mer.

  Ta pupille est bleue et quand tu pleures \\
  les larmes transparentes en elle \\
  me semblent gouttes de rosée \\
  sur une violette.

  Ta pupille est bleue et si au fond \\
  comme un point de lumière irradie une idée, \\
  elle paraît dans le ciel du soir \\
  une étoile perdue.
\end{verse}

\bigskip

\begin{center}
  \textbf{14}
  \addcontentsline{toc}{subsection}{\emph{14.\ Je t'entrevis et l'image de tes yeux resta...}}
\end{center}

\settowidth{\versewidth}{Je sais qu'il est des feux follets la nuit}

\begin{verse}[\versewidth]
  Je t'entrevis et l'image de tes yeux resta, \\
  flottant devant mes yeux \\
  comme la tâche sombre bordée de feu \\
  qui flotte et aveugle si l'on fixe le soleil.

  Et où que je pose le regard \\
  je revois tes pupilles flamboyer \\
  mais tu n'es pas là; c'est ton regard, \\
  des yeux, les tiens; rien de plus.

  Dans l'angle de mon alcôve je les regarde \\
  luire, détachés, fantastiques; \\
  quand je dors je les sens m'examiner, \\
  grand ouverts sur moi.

  Je sais qu'il est des feux follets la nuit \\
  qui mènent le voyageur à sa perte; \\
  moi je me sens entraîné par tes yeux, \\
  mais où ils m'entraînent, je ne le sais.
\end{verse}

\bigskip

\begin{center}
  \textbf{15}
  \addcontentsline{toc}{subsection}{\emph{15.\ Voile flottant de brume légère...}}
\end{center}

\settowidth{\versewidth}{Moi, qui dans mon agonie, vers tes yeux}

\begin{verse}[\versewidth]
  Voile flottant de brume légère, \\
  ruban plissé de blanche écume, \\
  rumeur sonore \\
  d'une harpe d'or, \\
  baiser du zéphir, onde de lumière, \\
  tu es cela.

  Toi, ombre aérienne, qui t'évanouis \\
  quand je crois enfin te saisir. \\
  Comme la flamme, comme le son, \\
  comme la brume, comme le gémissement \\
  du lac bleu!

  En mer, onde sonnante sans rivages; \\
  dans le vide, comète errante, \\
  longue complainte \\
  du vent rauque, \\
  soif perpétuelle de mieux, \\
  je suis cela.

  Moi, qui dans mon agonie, vers tes yeux \\
  retourne mes yeux jour et nuit; \\
  moi, qui infatigable et dément, \\
  cours après une ombre, la fille ardente \\
  d'une vision!
\end{verse}

\bigskip

\begin{center}
  \textbf{16}
  \addcontentsline{toc}{subsection}{\emph{16.\ Si, quand les clochettes bleues de ton balcon...}}
\end{center}

\settowidth{\versewidth}{Si, quand les clochettes bleues de ton balcon}

\begin{verse}[\versewidth]
  Si, quand les clochettes bleues de ton balcon \\
  se bercent, \\
  tu crois qu'en soupirant passe le vent \\
  qui murmure, \\
  sache que, caché parmi les feuilles vertes, \\
  moi je soupire.

  Si, quand résonne, confuse derrière toi, \\
  une vague rumeur, \\
  tu crois que par ton nom t'a appelé \\
  une voix lointaine, \\
  sache que, parmi les ombres qui t'entourent, \\
  moi je t'appelle.

  Si, quand se trouble ton cœur craintif \\
  en pleine nuit, \\
  si tu sens sur tes lèvres une haleine \\
  qui embrase, \\
  sache que, bien qu'invisible à tes côtés, \\
  moi je respire.
\end{verse}

\bigskip

\begin{center}
  \textbf{17}
  \addcontentsline{toc}{subsection}{\emph{17.\ Aujourd'hui la terre et les cieux me sourient...}}
\end{center}

\settowidth{\versewidth}{aujourd'hui je l'ai vue..., je l'ai vue et elle m'a regardé...}

\begin{verse}[\versewidth]
  Aujourd'hui la terre et les cieux me sourient, \\
  aujourd'hui le soleil atteint le fond de mon âme, \\
  aujourd'hui je l'ai vue..., je l'ai vue et elle m'a regardé... \\
  Aujourd'hui je crois en Dieu!
\end{verse}

\bigskip

\begin{center}
  \textbf{18}
  \addcontentsline{toc}{subsection}{\emph{18.\ Fatiguée par la danse...}}
\end{center}

\settowidth{\versewidth}{que pousse la mer et caresse le zéphir,}

\begin{verse}[\versewidth]
  Fatiguée par la danse, \\
  ardente la couleur, brève l'haleine, \\
  appuyée à mon bras, \\
  elle s'arrêta à un bout du salon.

  Parmi la gaze légère \\
  que soulevait le sein palpitant, \\
  une fleur était bercée \\
  d'un mouvement doux et mesuré.

  Comme dans un berceau de nacre \\
  que pousse la mer et caresse le zéphir, \\
  peut-être dormait-elle là-bas du souffle \\
  de ses lèvres entrouvertes.

  Oh! Qui, pensai-je, pourrait ainsi \\
  laisser filer le temps! \\
  Oh! Si les fleurs dorment, \\
  quel sommeil\footnote{NDT. On peut lire aussi «songe» (\emph{sueño})} si doux!
\end{verse}

\bigskip

\begin{center}
  \textbf{19}
  \addcontentsline{toc}{subsection}{\emph{19.\ Quand sur ta poitrine
    tu penches un front mélancolique...}}
\end{center}

\settowidth{\versewidth}{Quand sur ta poitrine tu penches}

\begin{verse}[\versewidth]
  Quand sur ta poitrine tu penches \\
  un front mélancolique, \\
  tu me sembles \\
  un lys brisé,

  car, en te donnant la pureté \\
  qui est symbole céleste, \\
  comme lui te fit Dieu \\
  d'or et de neige.
\end{verse}

\bigskip

\begin{center}
  \textbf{20}
  \addcontentsline{toc}{subsection}{\emph{20.\ Elle sait, si parfois ses lèvres rouges...}}
\end{center}

\settowidth{\versewidth}{sont brûlées par une invisible atmosphère,}

\begin{verse}[\versewidth]
  Elle sait, si parfois ses lèvres rouges \\
  sont brûlées par une invisible atmosphère, \\
  que l'âme qui peut parler avec les yeux \\
  aussi peut embrasser avec le regard.
\end{verse}

\bigskip

\begin{center}
  \textbf{21}
  \addcontentsline{toc}{subsection}{\emph{21.\ Qu'est la poésie?...}}
\end{center}

\settowidth{\versewidth}{Qu'est la poésie! Et toi tu me le demandes?}

\begin{verse}[\versewidth]
  Qu'est la poésie? dis-tu en plantant \\
  dans ma pupille ta pupille bleue. \\
  Qu'est la poésie! Et toi tu me le demandes? \\
  La poésie... c'est toi.
\end{verse}

\bigskip

\begin{center}
  \textbf{22}
  \addcontentsline{toc}{subsection}{\emph{22.\ Comment vit cette rose que tu as prise...}}
\end{center}

\settowidth{\versewidth}{Comment vit cette rose que tu as prise}

\begin{verse}[\versewidth]
  Comment vit cette rose que tu as prise \\
  contre ton cœur? \\
  Sur un volcan, avant de la trouver, \\
  jamais je n'avais vu de fleur.
\end{verse}

\bigskip

\begin{center}
  \textbf{23}
  \addcontentsline{toc}{subsection}{\emph{23.\ Pour un regard, un monde;...}}
\end{center}

\settowidth{\versewidth}{que t'offrir pour un baiser!}

\begin{verse}[\versewidth]
  Pour un regard, un monde; \\
  pour un sourire, un ciel; \\
  pour un baiser... j'ignore \\
  que t'offrir pour un baiser!
\end{verse}

\bigskip

\begin{center}
  \textbf{24}
  \addcontentsline{toc}{subsection}{\emph{24.\ Deux rouges langues de feu...}}
\end{center}

\settowidth{\versewidth}{deux baisers qui à l'unisson éclatent,}

\begin{verse}[\versewidth]
  Deux rouges langues de feu \\
  qui, enlacées au même tronc, \\
  s'approchent et, en se baisant, \\
  forment une seule flamme;

  deux notes que la main fait jaillir \\
  du luth en même temps, \\
  et qui dans l'espace se réunissent \\
  et s'embrassent en harmonie;

  deux vagues qui viennent ensemble \\
  mourir sur une plage \\
  et, en se brisant, se couronnent \\
  d'un panache d'argent;

  deux lambeaux de vapeur \\
  qui s'élèvent du lac, \\
  et, en se joignant dans le ciel, \\
  forment un nuage blanc;

  deux idées qui surgissent de pair, \\
  deux baisers qui éclatent de concert, \\
  deux échos qui se confondent... \\
  c'est cela nos deux âmes.
\end{verse}

\bigskip

\begin{center}
  \textbf{25}
  \addcontentsline{toc}{subsection}{\emph{25.\ Quand t'enveloppent dans la nuit...}}
\end{center}

\settowidth{\versewidth}{Quand t'enveloppent dans la nuit}

\begin{verse}[\versewidth]
  Quand t'enveloppent dans la nuit \\
  les ailes de tulle du sommeil, \\
  et tes cils tendus \\
  imitent des arcs d'ébène,

  pour écouter les battements \\
  de ton cœur inquiet \\
  et sentir ta tête endormie \\
  pencher sur ma poitrine,

  je donnerais, mon amour, \\
  tout ce que je possède: \\
  la lumière, l'air \\
  et la pensée!

  Quand se fixent tes yeux \\
  sur un objet invisible \\
  et le reflet illumine \\
  tes lèvres d'un sourire,

  pour lire sur ton front \\
  la pensée secrète \\
  qui passe comme un nuage marin \\
  sur le large miroir,

  je donnerais, mon amour, \\
  tout ce que je désire: \\
  la renommée, l'or, \\
  la gloire, le génie!

  Quand ta langue devient muette, \\
  et ton haleine se presse, \\
  et tes joues s'allument, \\
  et tu entrouvres tes yeux noirs,

  pour voir entre tes cils \\
  briller d'un feu humide \\
  l'étincelle ardente qui jaillit \\
  du volcan des désirs,

  je donnerais, mon amour, \\
  tout ce que en quoi j'espère: \\
  la foi, l'âme, \\
  la terre, le ciel!
\end{verse}

\bigskip

\begin{center}
  \textbf{26}
  \addcontentsline{toc}{subsection}{\emph{26.\ Je vais contre mes intérêts en le confessant...}}
\end{center}

\settowidth{\versewidth}{je pense comme toi qu'une ode est seule bonne}

\begin{verse}[\versewidth]
  Je vais contre mes intérêts en le confessant. \\
  Néanmoins, mon aimée, \\
  je pense comme toi qu'une ode est seule bonne \\
  écrite au dos d'un billet de banque\footnote{NDT. Il s'agit des ordres de paiement, dont les versos étaient vierges.}. \\
  Il ne manquera pas quelque sot qui en l'entendant \\
  ne se signe et dise: \\
  \emph{Femme, à la fin du dix-neuvième siècle, \\
    matérielle et prosaïque...} Sottises! \\
  Des voix qui font courir quatre poètes \\
  qui se drapent en hiver avec une lyre! \\
  Aboiements des chiens à la lune! \\
  Tu sais et je sais qu'en cette vie, \\
  celui qui \emph{l'écrit} avec génie est très rare, \\
  et, avec de l'or, quiconque \emph{fait} de la poésie.
\end{verse}

\bigskip

\begin{center}
  \textbf{27}
  \addcontentsline{toc}{subsection}{\emph{27.\ Éveillée, je tremble à ta vue...}}
\end{center}

\settowidth{\versewidth}{Éveillée, tu regardes et, en regardant, tes yeux}

\begin{verse}[\versewidth]
  Éveillée, je tremble à ta vue; \\
  assoupie, j'ose te regarder; \\
  c'est pour cela, âme de mon âme, \\
  que je veille pendant que tu dors.

  Éveillée, tu ris et, en riant, tes lèvres \\
  inquiètes me semblent \\
  des éclairs carmins qui serpentent \\
  sur un ciel enneigé.

  Assoupie, un léger sourire plisse \\
  les bords de ta bouche, \\
  suave comme le sillage brillant \\
  que laisse un soleil mourrant...

  Dors!

  Éveillée, tu regardes et, en regardant, tes yeux \\
  humides resplendissent \\
  comme la vague bleue dont la crête \\
  est illuminée par un soleil étincelant.

  Au travers de tes paupières, assoupie, \\
  ils déversent un éclat calme, \\
  comme la lueur tiède que répand \\
  une lampe transparente...

  Dors!

  Éveillée, tu parles et, en parlant, \\
  tes paroles vibrantes semblent \\
  une pluie de perles se déversant à torrents
  en une coupe dorée.

  Assoupie, dans le murmure de ton haleine \\
  rythmée et ténue, \\
  j'entends un poème que mon âme \\
  amoureuse comprend...

  Dors!

  J'ai posé une main sur mon cœur \\
  pour que son battement \\
  ne sonne et ne trouble \\
  le calme solennel de la nuit.

  J'ai fermé enfin les persiennes \\
  de ton balcon \\
  pour que le flamboiement fâcheux \\
  de l'aurore n'entre et ne t'éveille...

  Dors!
\end{verse}

\bigskip

\begin{center}
  \textbf{28}
  \addcontentsline{toc}{subsection}{\emph{28.\ Quand, parmi l'ombre obscure...}}
\end{center}

\settowidth{\versewidth}{dis-moi: est-ce que je touche et respire}

\begin{verse}[\versewidth]
  Quand, parmi l'ombre obscure, \\
  une voix perdue murmure, \\
  troublant sa triste paix; \\
  si, au fond de mon âme, \\
  je l'entends résonner doucement,

  dis-moi: est-ce le vent virevoltant \\
  qui se plaint, ou bien tes soupirs \\
  me parlent-ils d'amour en passant?

  Quand le soleil à ma fenêtre \\
  brille rouge au matin, \\
  et mon amour évoque ton ombre; \\
  si sur ma bouche je crois sentir \\
  l'impression d'une autre bouche,

  dis-moi: est-ce que je délire aveuglément, \\
  ou bien un baiser m'envoie ton cœur \\
  dans un soupir?

  Et, dans le jour lumineux \\
  et la pleine nuit noire, \\
  si dans tout ce qui entoure \\
  mon âme qui te désire \\
  je crois te sentir et voir,

  dis-moi: est-ce que je touche et respire \\
  en rêve, ou que, dans un soupir, \\
  tu me donnes ton haleine à boire?
\end{verse}

\bigskip

\begin{center}
  \textbf{29}
  \addcontentsline{toc}{subsection}{\emph{29.\ Sur sa jupe elle tenait
  le livre ouvert...}}
\end{center}

\settowidth{\versewidth}{Je sais seulement que nous nous tournâmes}

\begin{flushright}
  \emph{La bocca mi baciò tutto tremante.}\\ \textsc{Dante}
\end{flushright}

\begin{verse}[\versewidth]
  Sur sa jupe elle tenait \\
  le livre ouvert, \\
  ses boucles noires \\
  touchaient ma joue:\\
  nous ne voyions pas les lettres, \\
  aucun des deux, je crois, \\
  mais nous gardions \\
  un profond silence. \\
  Combien cela dura? Ni alors \\
  je ne pus le savoir. \\
  Je sais seulement qu'on n'entendait \\
  rien d'autre que l'haleine \\
  pressée qui s'échappait \\
  des lèvres sèches, \\
  je sais seulement que nous nous tournâmes \\
  les deux en même temps, \\
  et nos yeux se trouvèrent, \\
  et sonna un baiser!

\ldots\ldots\ldots\ldots\ldots\ldots\ldots\ldots\ldots\ldots\ldots\ldots\ldots\ldots\ldots\ldots

  Le livre était la création de Dante, \\
  son \emph{Enfer}. \\
  Quand nous y baissâmes les yeux, \\
  je dis, tremblant: \\
  --- Comprends-tu maintenant qu'un poème \\
    tient dans un vers? \\
  Et elle répondit, enflammée: \\
  --- Je le comprends maintenant!
\end{verse}

\bigskip

\begin{center}
  \textbf{30}
  \addcontentsline{toc}{subsection}{\emph{30.\ Une larme pointait à
    ses yeux...}}
\end{center}

\settowidth{\versewidth}{je dis encore: \emph{Pourquoi ce jour-là n'avoir rien dit?}}

\begin{verse}[\versewidth]
  Une larme pointait à ses yeux \\
  et à ma lèvre une phrase de pardon; \\
  l'orgueil parla et son pleur s'assècha, \\
  et la phrase sur mes lèvres expira.

  Je vais mon chemin; elle, un autre; \\
  mais en repensant à notre amour mutuel, \\
  je dis encore: \emph{Pourquoi n'ai-je rien dit ce jour-là?} \\
  et elle doit se dire: \emph{Pourquoi n'ai-je pas pleuré?}
\end{verse}

\bigskip

\begin{center}
  \textbf{31}
  \addcontentsline{toc}{subsection}{31.\ Notre passion fut une tragique saynète...}
\end{center}

\settowidth{\versewidth}{Notre passion fut une tragique saynète}

\begin{verse}[\versewidth]
  Notre passion fut une tragique saynète \\
  dont l'absurde fable \\
  produit rires et pleurs,
  le comique et le grave confondus.

  Mais le pire de cette histoire fut \\
  qu'à la fin de l'acte \\
  à elle échurent larmes et rires, \\
  et à moi seulement les larmes.
\end{verse}

\bigskip

\begin{center}
  \textbf{32}
  \addcontentsline{toc}{subsection}{\emph{32.\ Elle passait, irrésistible dans sa splendeur...}}
\end{center}

\settowidth{\versewidth}{je poursuivis sans me retourner, et pourtant}

\begin{verse}[\versewidth]
  Elle passait, irrésistible dans sa splendeur, \\
  et je lui cédai le pas; \\
  je poursuivis sans me retourner, et pourtant \\
  quelque chose à mon oreille murmura \emph{«~C'est elle.~»}

  Qui unit le soir au matin? \\
  Je l'ignore: je sais seulement \\
  que lors d'une brève nuit d'été \\
  s'unirent les crépuscules et... \emph{ainsi fut-il}.
\end{verse}

\bigskip

\begin{center}
  \textbf{33}
  \addcontentsline{toc}{subsection}{\emph{33.\ C'est une question de mots, et pourtant...}}
\end{center}

\settowidth{\versewidth}{C'est une question de mots, et pourtant}

\begin{verse}[\versewidth]
  C'est une question de mots, et pourtant \\
  ni toi ni moi, jamais, \\
  après ce qui advint, ne conviendra \\
  à qui la faute incombe.

  Quel dommage que l'Amour n'ait \\
  de dictionnaire où chercher \\
  quand l'orgueil est simplement orgueil \\
  et quand il est dignité!
\end{verse}

\bigskip

\begin{center}
  \textbf{34}
  \addcontentsline{toc}{subsection}{\emph{34.\ Muette, elle traverse
      et ses mouvements...}}
\end{center}

\settowidth{\versewidth}{flamboient d'un nouvel éclat dans ses pupilles.}

\begin{verse}[\versewidth]
  Muette, elle traverse et ses mouvements \\
  sont harmonie silencieuse; \\
  ses pas sonnent et, en sonnant, ils rappellent \\
  la cadence rythmée de l'hymne ailé.

  Elle entrouvre les yeux, ces yeux \\
  aussi clairs que le jour; \\
  et la terre et le ciel, ce qu'ils embrassent, \\
  flamboient d'un nouvel éclat dans ses pupilles.

  Elle rie, et ses éclats de rire ont des notes \\
  de l'eau fugitive; \\
  elle pleure, et chaque larme est un poème \\
  de tendresse infinie.

  Elle a la lumière, elle a le parfum, \\
  la couleur et la ligne, \\
  la forme qui engendre les désirs, \\
  l'expression qui est la source éternelle de poésie.

  Qu'elle est stupide? Bah! Tant que se taisant \\
  elle garde l'énigme secrète, \\
  toujours vaudra ce que je crois qu'elle tait \\
  plus que ce qu'aucune autre me dirait.
\end{verse}

\bigskip

\begin{center}
  \textbf{35}
  \addcontentsline{toc}{subsection}{\emph{35.\ Ton oubli ne m'admira pas!...}}
\end{center}

\settowidth{\versewidth}{Ton oubli ne m'admira pas! Bien que d'un jour}

\begin{verse}[\versewidth]
  Ton oubli ne m'admira pas! Bien que d'un jour \\
  ta tendresse m'admira bien plus; \\
  car ce qui en moi a de la valeur, \\
  cela... tu ne le soupçonnas même pas.
\end{verse}

\bigskip

\begin{center}
  \textbf{36}
  \addcontentsline{toc}{subsection}{\emph{36.\ Si l'on écrivait dans un livre...}}
\end{center}

\settowidth{\versewidth}{et si s'effaçait de nos âmes autant}

\begin{verse}[\versewidth]
  Si l'on écrivait dans un livre \\
  l'histoire de nos préjudices, \\
  et si s'effaçait de nos âmes autant \\
  que s'effacerait de ses pages... \\
  Je t'aime tant encore: ton amour laissa \\
  sur ma poitrine des traces si profondes \\
  que si tu n'en effaçais qu'une, \\
  je les effacerais toutes!
\end{verse}

\bigskip

\begin{center}
  \textbf{37}
  \addcontentsline{toc}{subsection}{\emph{37.\ Avant toi je mourrai...}}
\end{center}

\settowidth{\versewidth}{je porte le fer avec lequel ta main ouvrit}

\begin{verse}[\versewidth]
  Avant toi je mourrai: caché \\
  dans les entrailles déjà \\
  je porte le fer avec lequel ta main ouvrit \\
  la large blessure mortelle.

  Avant toi je mourrai; et mon âme, \\
  dans son entêtement tenace, \\
  s'assiéra aux portes de la mort, \\
  t'attendant là-bas.

  Avec les heures les jours, avec les jours \\
  les années s'envoleront, \\
  et tu frapperas à cette porte à la fin... \\
  Qui renonce à frapper?

  Puis la terre gardera \\
  tes fautes et ta dépouille, \\
  tu te laveras dans les ondes de la mort \\
  comme dans un autre Jourdain;

  là-bas, où le murmure de la vie \\
  va mourir en tremblant, \\
  comme la vague qui va en silence \\
  expirer sur le rivage;

  là-bas, où le sépulcre qui se ferme \\
  ouvre une éternité, \\
  tout ce que nous deux avons tu, \\
  là-bas nous devrons en parler.
\end{verse}

\bigskip

\begin{center}
  \textbf{38}
  \addcontentsline{toc}{subsection}{\emph{38.\ Les soupirs sont air, et
    à l'air ils vont!...}}
\end{center}

\settowidth{\versewidth}{je porte le fer avec lequel ta main ouvrit}

\begin{verse}[\versewidth]
  Les soupirs sont air, et à l'air ils vont! \\
  Les larmes sont eau, et à la mer elles vont! \\
  Dis-moi, femme: quand l'amour s'oublie, \\
  sais-tu où il va?
\end{verse}

\bigskip

\begin{center}
  \textbf{39}
  \addcontentsline{toc}{subsection}{\emph{39.\ Pourquoi me le dire?...}}
\end{center}

\settowidth{\versewidth}{Pourquoi me le dire? Je sais: elle est changeante,}

\begin{verse}[\versewidth]
  Pourquoi me le dire? Je sais: elle est changeante, \\
  altière et vaine et capricieuse; \\
  l'eau jaillirait d'une roche stérile \\
  avant que les sentiments ne jaillissent de son âme.

  Je sais qu'en son cœur, nid de serpents, \\
  il n'y a fibre qui réponde à l'amour; \\
  qu'elle est une statue inanimée... mais... \\
  elle est si belle!
\end{verse}

\bigskip

\begin{center}
  \textbf{40}
  \addcontentsline{toc}{subsection}{\emph{40.\ Sa main dans mes mains...}}
\end{center}

\settowidth{\versewidth}{Sa main dans mes mains,}

\begin{verse}[\versewidth]
  Sa main dans mes mains, \\
  ses yeux dans mes yeux, \\
  la tête amoureuse \\
  appuyée sur mon épaule, \\
  Dieu sait combien de fois, \\
  d'un pas paresseux, \\
  nous avons erré ensemble \\
  sous les grands ormes \\
  qui prêtent mystère et ombre \\
  au porche de sa maison. \\
  Et hier..., un an à peine \\
  passé en coup de vent, \\
  avec quelle exquise grâce, \\
  avec quel admirable aplomb, \\
  elle me dit, me présentant  \\
  quelque ami officieux: \\
  \emph{«Je crois qu'en quelque endroit \\
    je vous ai vu.»} Ah! Sots \\
  qui êtes des salons \\
  commères de bon ton \\
  et marchiez là en chasse \\
  de galants imbroglios: \\
  quelle histoire vous avez manquée! \\
  Quelle ambroisie \\
  pour être dévorée \\
  \emph{sotto voce} en un cercle, \\
  derrière l'éventail \\
  de plumes et d'or!

\ldots\ldots\ldots\ldots\ldots\ldots\ldots\ldots\ldots\ldots\ldots\ldots

  Lune discrète et chaste, \\
  ormes touffus et grands, \\
  murs de sa demeure, \\
  seuils de son porche, \\
  taisez-vous, et que le secret \\
  ne sorte pas de vous! \\
  Taisez-vous, pour ma part \\
  j'ai tout oublié; \\
  et elle..., elle, il n'y a de masque \\
  semblable à son visage!
\end{verse}

\bigskip

\begin{center}
  \textbf{41}
  \addcontentsline{toc}{subsection}{\emph{41.\ Tu étais l'ouragan et
    moi la haute tour...}}
\end{center}

\settowidth{\versewidth}{l'un à emporter, l'autre à ne pas céder;}

\begin{verse}[\versewidth]
  Tu étais l'ouragan et moi la haute \\
  tour qui défie son pouvoir: \\
  tu devais te fracasser ou m'abattre!... \\
  Impossible!

  Tu étais l'océan et moi la roche \\
  dressée qui attend son va-et-vient: \\
  tu devais te briser ou m'arracher!... \\
  Impossible!

  Belle, toi; moi, altier; habitués \\
  l'un à emporter, l'autre à ne pas céder: \\
  étroite, la sente; inévitable, le choc... \\
  Impossible!
\end{verse}

\bigskip

\begin{center}
  \textbf{42}
  \addcontentsline{toc}{subsection}{\emph{42.\ Quand on me le conta, je
    sentis le froid...}}
\end{center}

\settowidth{\versewidth}{Il m'avait rendu service!... Je le remerciai.}

\begin{verse}[\versewidth]
  Quand on me le conta, je sentis le froid \\
  d'une lame d'acier dans les entrailles; \\
  je m'appuyai contre le mur, et un instant \\
  je perdis la conscience du lieu où j'étais.

  La nuit s'abattit sur mon être; \\
  d'ire et de pitié s'inonda mon âme \\
  et je compris pourquoi on pleure, \\
  et je compris pourquoi on tue!

  Le nuage de douleur passa..., avec peine \\
  je parvins à balbutier quelques mots... \\
  Et qui me donna la nouvelle?... Un ami fidèle. \\
  Il m'avait rendu un grand service!... Je le remerciai.
\end{verse}

\bigskip

\begin{center}
  \textbf{43}
  \addcontentsline{toc}{subsection}{\emph{43.\ J'écartai la lumière...}} \end{center}

\settowidth{\versewidth}{je me souviens seulement avoir pleuré et maudit,}

\begin{verse}[\versewidth]
  J'écartai la lumière, et au bord \\
  du lit défait je m'assis, \\
  muet, sombre, les pupilles immobiles \\
  plantées dans le mur.

  Combien de temps restai-je ainsi? Je ne sais; \\
  quand me quitta l'horrible ivresse de douleur, \\
  la lumière expirait et sur mes balcons \\
  riait le soleil.

  Je ne sais non plus, en de si terribles heures, \\
  à quoi je pensai ou ce qui me traversa; \\
  je me souviens seulement avoir pleuré et maudit, \\
  et avoir en cette nuit-là vieilli.
\end{verse}

\bigskip

\begin{center}
  \textbf{44}
  \addcontentsline{toc}{subsection}{\emph{44.\ Comme d'un livre
    ouvert...}}
\end{center}

\settowidth{\versewidth}{Vois: je suis un homme... et je pleure aussi!}

\begin{verse}[\versewidth]
  Comme d'un livre ouvert \\
  je lis dans le fond de tes pupilles; \\
  À quoi bon feignent les lèvres \\
  des rires que démentent les yeux?

  Pleure! N'ai honte \\
  de confesser que tu m'aimas un peu. \\
  Pleure! Personne ne nous voit. \\
  Vois: je suis un homme... et je pleure aussi.
\end{verse}

\bigskip

\begin{center}
  \textbf{45}
  \addcontentsline{toc}{subsection}{\emph{45.\ À la clef d'un arc mal assuré...}}
\end{center}

\settowidth{\versewidth}{Panache de son heaume de granit,}

\begin{verse}[\versewidth]
  À la clef d'un arc mal assuré, \\
  aux pierres rougies par le temps, \\
  campait le blason gothique, \\
  œuvre d'un rude ciseau.

  Panache de son heaume de granit, \\
  le lierre qui pendait autour \\
  ombrait l'écu où une main \\
  tenait un cœur.

  Pour le contempler en ce lieu désert, \\
  nous nous arrêtâmes tous deux: \\
  et cela, me dit-elle, est le parfait emblème \\
  de mon amour constant.

  Hélas! Ce qu'elle me dit alors était vrai: \\
  vrai que le cœur, \\
  elle le porterait sur la main... partout..., \\
  mais dans la poitrine, non.
\end{verse}

\bigskip

\begin{center}
  \textbf{46}
  \addcontentsline{toc}{subsection}{\emph{46.\ Elle m'a blessé en
    se cachant dans l'ombre...}}
\end{center}

\settowidth{\versewidth}{Elle m'a blessé en se retirant dans l'ombre,}

\begin{verse}[\versewidth]
  Elle m'a blessé en se retirant dans l'ombre, \\
  scellant d'un baiser sa trahison. \\
  Elle se pendit à mon cou, et, dans le dos, \\
  elle me brisa le cœur de sang froid.

  Et elle poursuit, joyeuse, son chemin, \\
  heureuse, gaie, impavide; et pourquoi? \\
  Parce que la blessure ne saigne pas, \\
  Parce que le mort est debout.
\end{verse}

\bigskip

\begin{center}
  \textbf{47}
  \addcontentsline{toc}{subsection}{\emph{47.\ Je me suis penché sur
    les gouffres béants...}}
\end{center}

\settowidth{\versewidth}{Je me suis penché sur les gouffres béants}

\begin{verse}[\versewidth]
  Je me suis penché sur les gouffres béants \\
  de la terre et du ciel, \\
  et j'en ai vu la fin, avec les yeux \\
  ou avec la pensée.

  Mais, hélas!, d'un cœur je vins à l'abîme \\
  et je m'inclinai un moment; \\
  et mon âme et mes yeux se troublèrent: \\
  il était si profond et si noir!
\end{verse}

\bigskip

\begin{center}
  \textbf{48}
  \addcontentsline{toc}{subsection}{\emph{48.\ Comme s'arrache le fer d'une plaie...}}
\end{center}

\settowidth{\versewidth}{et la lumière de la foi, qui en elle brûlait}

\begin{verse}[\versewidth]
  Comme s'arrache le fer d'une plaie, \\
  j'arrachai son amour de mes entrailles, \\
  bien que je sentis ce faisant \\
  que je m'arrachais la vie avec lui!

  De l'autel que je lui dressai dans mon âme, \\
  la volonté abattit son image, \\
  et la lumière de la foi, qui en elle brûlait \\
  devant l'autel désert, s'éteignit.

  Sa vision tenace vient encore à mon esprit \\
  pour combattre ma determination... \\
  Quand pourrai-je dormir de ce sommeil \\
  où s'achève le rêve!
\end{verse}

\bigskip

\begin{center}
  \textbf{49}
  \addcontentsline{toc}{subsection}{\emph{49.\ Parfois je la rencontre
    de par le monde...}}
\end{center}

\settowidth{\versewidth}{Puis point à ma lèvre un autre sourire,}

\begin{verse}[\versewidth]
  Parfois je la rencontre de par le monde \\
  et elle passe près de moi; \\
  et elle passe en souriant, et je dis: \\
  \emph{Comment peut-elle \emph{rire}?}

  Puis point à ma lèvre un autre sourire, \\
  masque de la douleur, \\
  et je pense alors: \emph{Peut-être rit-elle \\
  comme je ris moi-même.}
\end{verse}

\bigskip

\begin{center}
  \textbf{50}
  \addcontentsline{toc}{subsection}{\emph{50.\ Comme le sauvage aux mains malhabiles...}}
\end{center}

\settowidth{\versewidth}{et, l'idole une fois là, nous sacrifiâmes}

\begin{verse}[\versewidth]
  Comme le sauvage aux mains malhabiles \\
  fait à discrétion un dieu d'un tronc, \\
  et ensuite devant son œuvre s'agenouille, \\
  cela nous le fîmes toi et moi.

  Nous donnâmes forme réelle à un fantôme, \\
  invention ridicule de l'esprit, \\
  et, l'idole une fois là, nous sacrifiâmes \\
  notre amour sur son autel.
\end{verse}

%\newpage

\begin{center}
  \textbf{51}
  \addcontentsline{toc}{subsection}{\emph{51.\ Du peu de vie qu'il me
    reste...}}
\end{center}

\settowidth{\versewidth}{je donnerais volontiers les meilleures années,}

\begin{verse}[\versewidth]
  Du peu de vie qu'il me reste \\
  je donnerais volontiers les meilleures années, \\
  pour savoir ce que tu as conté \\
  de moi à d'autres.

  Et cette vie mortelle et de l'éternelle \\
  ce qu'il me reviendra, s'il m'en revient, \\
  pour savoir ce que, seule, \\
  de moi tu as pensé.
\end{verse}

\bigskip

\begin{center}
  \textbf{52}
  \addcontentsline{toc}{subsection}{\emph{52.\ Lames géantes qui vous
    brisez en mugissant...}}
\end{center}

\settowidth{\versewidth}{Lames géantes qui vous brisez en mugissant}

\begin{verse}[\versewidth]
  Lames géantes qui vous brisez en mugissant \\
  sur les rivages déserts et lointains: \\
  enveloppé dans le drap d'écumes, \\
  emportez-moi avec vous!

  Rafales d'ouragans qui arrachent \\
  de la grande forêt les feuilles mortes: \\
  entraîné dans l'aveugle toubillon, \\
  emportez-moi avec vous!

  Nuées de tempête que rompt l'éclair \\
  et qui ornez les orles défaits en feu: \\
  enlevé dans la brume obscure, \\
  emportez-moi avec vous!

  Emportez-moi, par pitié, là où le vertige \\
  m'arracherait la mémoire et la raison. \\
  Par pitié! J'ai peur de rester \\
  seul à seul avec ma douleur!
\end{verse}

\bigskip

\begin{center}
  \textbf{53}
  \addcontentsline{toc}{subsection}{\emph{53.\ Elles reviendront, les
    noires hirondelles...}}
\end{center}

\settowidth{\versewidth}{et, à nouveau, leurs fleurs s'ouvriront le soir,}

\begin{verse}[\versewidth]
  Elles reviendront, les noires hirondelles, \\
  pendre leurs nids à ton balcon, \\
  et, à nouveau, avec leurs ailes \\
  elles toqueront aux carreaux en jouant.

  Mais celles qui réfrènaient leur vols, \\
  en contemplant ta beauté et mon bonheur, \\
  celles qui apprirent nos noms... \\
  celles-ci ne reviendront pas!

  Ils reviendront, les épais chèvrefeuilles, \\
  escalader les murs de ton jardin, \\
  et, à nouveau, leurs fleurs s'ouvriront le soir, \\
  encore plus belles.

  Mais celles figées par la rosée, \\
  dont nous regardions les gouttes trembler \\
  et tomber comme larmes du jour... \\
  celles-ci ne reviendront pas!

  Ils reviendront, les mots ardents d'amour, \\
  sonner à ton oreille, \\
  ton cœur peut-être se réveillera \\
  de son profond sommeil.

  Mais, muet et absorbé et à genoux, \\
  comme on adore Dieu devant son autel, \\
  comme moi je t'ai aimée..., détrompe-toi, \\
  ainsi personne ne t'aimera plus.
\end{verse}

\bigskip

\begin{center}
  \textbf{54}
  \addcontentsline{toc}{subsection}{\emph{54.\ Quand à nouveau les
    fugaces heures du passé nous évoquons...}}
\end{center}

\settowidth{\versewidth}{et, à nouveau, leurs fleurs s'ouvriront le soir,}

\begin{verse}[\versewidth]
  Quand, à nouveau, nous évoquons \\
  les heures fugaces du passé, \\
  une larme tremblante brille, \\
  prompte à glisser sur ses cils noirs.

  Et, enfin, elle glisse et tombe comme goutte \\
  de rosée à la pensée que, \\
  tel ce jour pour hier, pour ce jour demain, \\
  tous deux nous soupirerons à nouveau.
\end{verse}

\bigskip

\begin{center}
  \textbf{55}
  \addcontentsline{toc}{subsection}{\emph{55.\ Dans le tumulte
    discordant de l'orgie...}}
\end{center}

\settowidth{\versewidth}{---~À rien... --~À rien, et tu pleures? --~J'ai la tristesse}

\begin{verse}[\versewidth]
  Dans le tumulte discordant de l'orgie, \\
  l'écho d'un soupir \\
  caressa mon oreille, \\
  comme une note de musique lointaine.

  L'écho d'un soupir que je connais, \\
  formé d'une haleine que j'ai bue, \\
  parfum d'une fleur qui croît cachée \\
  dans un cloître sombre.

  Mon adorée d'un jour, ma tendre, me dit: \\
  ---~À quoi penses-tu? \\
  ---~À rien... --~À rien, et tu pleures? --~J'ai la tristesse \\
    gaie et le vin triste.
\end{verse}

\bigskip

\begin{center}
  \textbf{56}
  \addcontentsline{toc}{subsection}{\emph{56.\ Aujourd'hui comme hier, demain comme aujourd'hui...}}
\end{center}

\settowidth{\versewidth}{Aujourd'hui comme hier, demain comme aujourd'hui,}

\begin{verse}[\versewidth]
  Aujourd'hui comme hier, demain comme aujourd'hui, \\
  et toujours pareil! \\
  Un ciel gris, un horizon éternel, \\
  et marcher... marcher.

  Le cœur battant la mesure \\
  comme un machine stupide; \\
  l'intelligence obtuse du cerveau \\
  endormie dans un recoin.

  L'âme, dans son ambition du Paradis, \\
  le recherche sans foi. \\
  Fatigue sans objet, vague qui roule \\
  sans savoir pourquoi.

  La voix, d'un ton égal, \\
  chante incessamment le même chant. \\
  La goutte d'eau monotone qui tombe, \\
  et tombe, sans cesse.

  Ainsi vont les jours, glissant \\
  les uns après les autres, \\
  aujourd'hui comme hier... et tous \\
  sans plaisir ni douleur.

  Hélas! Parfois je me souviens en un soupir \\
  d'une affliction ancienne. \\
  Amère est la douleur, mais au moins \\
  souffrir est vivre!
\end{verse}

\bigskip

\begin{center}
  \textbf{57}
  \addcontentsline{toc}{subsection}{\emph{57.\ Cette carcasse d'os et de peau...}}
\end{center}

\settowidth{\versewidth}{horribles gravent sur le cœur, si ce n'est au front.}

\begin{verse}[\versewidth]
  Cette carcasse d'os et de peau, \\
  à tant promener une tête folle, \\
  se fatigue à la fin et je ne le regrette pas; \\
  car, bien qu'il soit vrai que je ne sois pas vieux,

  de la part de vie qu'il me revient \\
  de la vie du monde, \\
  j'ai fait un tel usage à mes dépens que je jurerais \\
  avoir condensé un siècle en chaque jour.

  Ainsi, si je mourais à l'instant, \\
  je ne pourrais dire que je n'ai vécu; \\
  si la casaque paraît neuve dehors \\
  je sais qu'elle a vieilli dedans.

  Elle a vieilli, oui; malgré mon étoile! \\
  suffisamment le dit mon ardeur dolente; \\
  c'est qu'il est des douleurs qui leurs empreintes \\
  horribles gravent sur le cœur, si ce n'est au front.
\end{verse}

\bigskip

\begin{center}
  \textbf{58}
  \addcontentsline{toc}{subsection}{\emph{58.\ Veux-tu de ce
    nectar délicieux éviter l'amertume la lie?...}}
\end{center}

\settowidth{\versewidth}{Alors sens-le, approche-le de tes lèvres}

\begin{verse}[\versewidth]
  Veux-tu de ce nectar délicieux \\
  éviter l'amertume de la lie? \\
  Alors sens-le, approche-le de tes lèvres \\
  et laisse-le après.

  Veux-tu que nous gardions un doux \\
  souvenir de cet amour? \\
  Alors aimons-nous aujourd'hui, et demain \\
  disons-nous adieu!
\end{verse}

\bigskip

\begin{center}
  \textbf{59}
  \addcontentsline{toc}{subsection}{\emph{59.\ Moi, je sais quel est
    l'objet de tes soupirs...}}
\end{center}

\settowidth{\versewidth}{pendant que tu éprouves tant et ne sais rien,}

\begin{verse}[\versewidth]
  Moi, je sais quel est l'objet \\
  de tes soupirs; \\
  Moi, je sais la cause de ta douce \\
  et secrète langueur.

  Tu ris?... Un jour \\
  tu sauras, petite, pourquoi. \\
  Toi, tu le soupçonnes, \\
  et moi je le sais.

  Moi, je sais quand tu rêves \\
  et ce qu'en songe tu vois. \\
  Comme d'un livre je peux lire \\
  sur ton front ce que tu tais.

  Tu ris? Un jour \\
  tu sauras, petite, pourquoi. \\
  Toi, tu le soupçonnes, \\
  et moi je le sais.

  Moi, je sais pourquoi tu souris \\
  et pleures à la fois; \\
  moi, je pénètre les recoins mystérieux \\
  de ton âme de femme.

  Tu ris?... Un jour \\
  tu sauras, petite, pourquoi. \\
  Pendant que tu éprouves tant et ne sais rien, \\
  moi, qui ne ressens plus rien, je sais tout.
\end{verse}

\bigskip

\begin{center}
  \textbf{60}
  \addcontentsline{toc}{subsection}{\emph{60.\ Ma vie est une friche...}}
\end{center}

\settowidth{\versewidth}{fleur que je touche s'effeuille.}

\begin{verse}[\versewidth]
  Ma vie est une friche; \\
  fleur que je touche s'effeuille. \\
  Sur mon chemin fatal \\
  on va semant le mal \\
  pour que moi je le recueille.
\end{verse}

\bigskip

\begin{center}
  \textbf{61}
  \addcontentsline{toc}{subsection}{\emph{61.\ En voyant mes heures de
    fièvre...}}
\end{center}

\settowidth{\versewidth}{quand le soleil brillera à nouveau:}

\begin{verse}[\versewidth]
  En voyant mes heures de fièvre \\
  et d'insomnie, lentes, passer: \\
  au bord de ma couche, \\
  qui s'assiéra?

  Quand ma main tremblante \\
  se tendra, prête à expirer: \\
  cherchant une main amie, \\
  qui la serrera?

  Quand la mort dépolira \\
  de mes yeux le cristal: \\
  mes paupières encore ouvertes, \\
  qui les clora?

  Quand la cloche sonnera \\
  (si elle sonne à mon enterrement): \\
  une prière en l'entendant, \\
  qui la murmurera?

  Quand mes pâles restes \\
  opprimeront la terre enfin: \\
  sur la fosse oubliée, \\
  qui viendra pleurer?

  Enfin, le jour suivant, \\
  quand le soleil brillera à nouveau: \\
  de mon passage de par le monde, \\
  qui se souviendra?
\end{verse}

\bigskip

\begin{center}
  \textbf{62}
  \addcontentsline{toc}{subsection}{\emph{62.\ D'abord une aube
    tremblante...}}
\end{center}

\settowidth{\versewidth}{puis elle étincelle et croît et se dilate}

\begin{verse}[\versewidth]
  D'abord une aube tremblante et vague, \\
  rai de lumière inquiète qui coupe la mer; \\
  puis elle étincelle et croît et se dilate \\
  en une ardente explosion de clarté.

  Le foyer brillant est la joie, \\
  l'ombre craintive est la peine; \\
  Hélas! Dans la nuit obscure de mon âme, \\
  quand poindra le jour?
\end{verse}

\bigskip

\begin{center}
  \textbf{63}
  \addcontentsline{toc}{subsection}{\emph{63.\ Comme des essaims
    d'abeilles irritées...}}
\end{center}

\settowidth{\versewidth}{et, l'un après l'autre, ils viennent planter}

\begin{verse}[\versewidth]
  Comme un essaim d'abeilles irritées, \\
  d'un recoin sombre de la mémoire \\
  sortent, pour me poursuivre, les souvenirs \\
  des heures passées.

  Je veux les chasser. Effort inutile! \\
  Ils m'encerclent, me harcèlent, \\
  et, l'un après l'autre, ils viennent planter \\
  le fin aiguillon qui envenime l'âme.
\end{verse}

\bigskip

\begin{center}
  \textbf{64}
  \addcontentsline{toc}{subsection}{\emph{64.\ Comme l'avare garde son
    trésor, je gardais ma douleur...}}
\end{center}

\settowidth{\versewidth}{Mais aujourd'hui en vain je l'appelle et le Temps,}

\begin{verse}[\versewidth]
  Comme l'avare garde son trésor, \\
  je gardais ma douleur; \\
  je voulais prouver que l'éternel existe \\
  à celle qui me jura un amour éternel.

  Mais aujourd'hui en vain je l'appelle et le Temps, \\
  qui l'épuisa, me dit: \\
  \emph{Ah, boue misérable! Éternellement \\
    tu ne saurais même souffrir!}
\end{verse}

\bigskip

\begin{center}
  \textbf{65}
  \addcontentsline{toc}{subsection}{\emph{65.\ Vint la nuit et point
    d'asile...}}
\end{center}

\settowidth{\versewidth}{parvenait le rauque bouillonnement de la multitude,}

\begin{verse}[\versewidth]
  Vint la nuit et point d'asile; \\
  et j'eus soif!... Je bus mes larmes. \\
  Et j'eus faim!... J'ai clos mes yeux enflés \\
  pour mourir!

  Étais-je dans un désert? Bien qu'à mon oreille \\
  parvenait le rauque bouillonnement de la multitude, \\
  j'étais orphelin et pauvre... Le monde était \\
  un désert... pour moi!
\end{verse}

\bigskip

\begin{center}
  \textbf{66}
  \addcontentsline{toc}{subsection}{\emph{66.\ D'où je viens? Cherche
    le plus horrible et âpre des sentiers...}}
\end{center}

\settowidth{\versewidth}{des empreintes de pieds ensanglantés}

\begin{verse}[\versewidth]
  D'où je viens? Cherche le plus \\
  horrible et âpre des sentiers; \\
  des empreintes de pieds ensanglantés \\
  sur la roche dure; \\
  les restes d'une âme en lambeaux \\
  dans les ronces acérées: \\
  ils te diront le chemin \\
  qui conduit à mon berceau.

  Où vais-je? Traverse la plus \\
  sombre et triste des déserts froids; \\
  vallée de neiges éternelles \\
  et de brumes mélancoliques. \\
  Où se trouve une pierre solitaire \\
  sans aucune inscription, \\
  où habite l'oubli: \\
  là se trouvera ma tombe.
\end{verse}

\bigskip

\begin{center}
  \textbf{67}
  \addcontentsline{toc}{subsection}{\emph{67.\ Quelle merveille que de
    voir le jour...}}
\end{center}

\settowidth{\versewidth}{que de bien dormir... et ronfler comme un sous-chantre...}

\begin{verse}[\versewidth]
  Quelle merveille que de voir le jour \\
  se lever, couronné de feu, \\
  et, à son baiser enflammé, \\
  voir briller les vagues et s'incendier l'air!

  Quelle merveille, après la pluie, \\
  dans le soir bleuté de l'automne triste, \\
  que de respirer le parfum \\
  des fleurs humides jusqu'à satiété!

  Quelle merveille, quand la blanche neige \\
  tombe silencieusement en flocons, \\
  que de voir s'agiter les langues rougeâtres \\
  des flammes inquiètes!

  Quelle merveille, après la fatigue, \\
  que de bien dormir... et ronfler comme un sous-chantre... \\
  et manger... et grossir... Et quel malheur \\
  que cela seulement ne suffise pas!
\end{verse}

\bigskip

\begin{center}
  \textbf{68}
  \addcontentsline{toc}{subsection}{\emph{68.\ Je ne sais ce que j'ai
    rêvé la nuit dernière...}}
\end{center}

\settowidth{\versewidth}{et, pour la première fois, je sentis en le notant}

\begin{verse}[\versewidth]
  Je ne sais ce que j'ai rêvé \\
  la nuit dernière. \\
  Triste, très triste dû être le rêve, \\
  car, éveillé, l'angoisse perdurait.

  Je notai, en reprenant corps, \\
  l'humidité de l'oreiller, \\
  et, pour la première fois, je sentis en le notant \\
  mon âme s'emplir d'un plaisir amer.

  Triste affaire qu'un rêve \\
  qui nous arrache des pleurs; \\
  mais j'ai une joie dans ma tristesse: \\
  je sais qu'il me reste encore des larmes.
\end{verse}

\bigskip

\begin{center}
  \textbf{69}
  \addcontentsline{toc}{subsection}{\emph{69.\ Nous naissons de l'éclair lorsqu'il brille,...}}
\end{center}

\settowidth{\versewidth}{et son éclat perdure encore quand nous mourons:}

\begin{verse}[\versewidth]
  Nous naissons de l'éclair lorsqu'il brille, \\
  et son éclat perdure encore quand nous mourons: \\
  si courte est la vie!

  Nous courons après gloire et amour, \\
  ombres d'un rêve que nous poursuivons: \\
  s'éveiller est mourir!
\end{verse}

\bigskip

\begin{center}
  \textbf{70}
  \addcontentsline{toc}{subsection}{\emph{70.\ Combien de fois, au
    pied des murs moussus qui la gardent...}}
\end{center}

\settowidth{\versewidth}{Bien que le vent sifflât dans les angles obscurs}

\begin{verse}[\versewidth]
  Combien de fois, au pied des murs \\
  moussus qui la gardent, \\
  n'ai-je entendu la clochette à minuit \\
  sonner aux matines!

  Combien de fois la lune argentée traça \\
  ma silhouette, \\
  contre celle du cyprès qui, de son verger, \\
  point sur les murailles!

  Quand l'église se drapait d'ombres, \\
  de son ogive en coiffe enfoncée, \\
  combien de fois sur les vitraux \\
  n'ai-je vu trembler l'éclat de la lampe!

  Bien que le vent sifflât dans les angles obscurs \\
  de la tour, \\
  parmi les voix du chœur je percevais \\
  sa voix vibrante et claire.

  Dans les nuits d'hiver, si un poltron \\
  osait traverser la place déserte, \\
  il hâtait son pas \\
  quand il m'apercevait. \\

  Et il ne manqua pas une vieille qui ne racontât \\
  au matin suivant \\
  que de quelque sacristain mort en pécheur \\
  j'étais l'âme.

  À tâtons, je connaissais les recoins \\
  de l'atrium et de la façade; \\
  de mes pieds les orties qui là-bas poussent \\
  peut-être gardent les empreintes.

  Les hiboux effrayés, qui me suivaient \\
  de leurs yeux de flammes, \\
  finirent par me considérer, avec le temps, \\
  comme un bon camarade.

  À mon côté, les reptiles sans peur \\
  avançaient en se traînant: \\
  je crois que même les saints de granit muets \\
  me saluaient!
\end{verse}

\bigskip

\begin{center}
  \textbf{71}
  \addcontentsline{toc}{subsection}{\emph{71.\ Je ne dormais pas;
    errant dans la limbe...}}
\end{center}

\settowidth{\versewidth}{Je dormis et au réveil je m'exclamai: \emph{«~Quelqu'un}}

\begin{verse}[\versewidth]
  Je ne dormais pas; errant dans la limbe \\
  où changent de forme les objets, \\
  mystérieux espaces qui séparent \\
  la veille du sommeil.

  Les idées, qui en rondes silencieuses \\
  tournaient dans mon cerveau, \\
  bougeaient peu à peu en leur danse \\
  d'un rythme plus lent.

  De la lumière qui parvient à l'âme par les yeux, \\
  les paupièrent en voilaient le reflet; \\
  mais le monde de visions \\
  allumait à l'intérieur une autre lumière.

  À ce moment résonna dans mon oreille \\
  une rumeur comme celle qui, au temple, \\
  erre confusément quand les fidèles terminent \\
  leurs prières par un \emph{Amen}.

  Et j'entendis comme une voix fine et triste \\
  qui m'appela de loin par mon nom, \\
  et je sentis une odeur de cierges éteints, \\
  d'humidité et d'encens.

\ldots\ldots\ldots\ldots\ldots\ldots\ldots\ldots\ldots\ldots\ldots\ldots\ldots\ldots\ldots\ldots\ldots\ldots\ldots\ldots

  La nuit entra et, dans les bras de l'oubli, \\
  je tombai comme pierre en son sein profond. \\
  Je dormis et au réveil je m'exclamai: \emph{«~Quelqu'un \\
que j'aimais est mort!~»}.
\end{verse}

\bigskip

\begin{center}
  \textbf{72}
  \addcontentsline{toc}{subsection}{\emph{72.\ Les ondes ont une vague
    harmonie...}}
\end{center}

\settowidth{\versewidth}{\emph{T'embarques-tu?}, me criaient-ils. Et moi, souriant,}

\begin{center} \emph{Première voix} \end{center}

\begin{verse}[\versewidth]
  Les ondes ont une vague harmonie; \\
  les violettes, une suave odeur; \\
  les brumes d'argent, la froide nuit; \\
  la lumière et l'or, le jour; \\
  moi, quelque chose de meilleur: \\
  moi, j'ai l'\emph{Amour}!
\end{verse}

\smallskip

\begin{center} \emph{Deuxième voix} \end{center}

\begin{verse}[\versewidth]
  Aura de liesse, nuée radieuse, \\
  vague d'envie qui baise le pied, \\
  île de songes où repose \\
  l'âme inassouvie. \\
  Douce ivresse, \\
  c'est la \emph{Gloire}.
\end{verse}

\begin{center} \emph{Troisième voix} \end{center}

\begin{verse}[\versewidth]
  Braise allumée est le trésor, \\
  ombre qui fuit la vanité. \\
  Mensonges que la gloire et l'or, \\
  seul ce que moi j'adore \\
  est vrai: \\
  la \emph{Liberté}!
\end{verse}

\begin{verse}
  Ainsi passaient les bateliers en chantant \\
  l'éternelle chanson, \\
  et l'écume sautait du coup de rame \\
  et le soleil la blessait.

  \emph{T'embarques-tu?}, me criaient-ils. Et moi, souriant, \\
  je leur dis au passage: \\
  «~\emph{J'ai déjà embarqué.}~», et je leur pointai \\
  mes habits étendus qui sèchaient encore sur la plage.
\end{verse}

\bigskip

\begin{center}
  \textbf{73}
  \addcontentsline{toc}{subsection}{\emph{73.\ On clôt ses yeux
    qu'elle avait encore ouverts...}}
\end{center}

\settowidth{\versewidth}{XXXXXXXXXXXXXXXXXXXXXXX}

\begin{verse}[\versewidth]
  On clôt ses yeux \\
  qu'elle avait encore ouverts, \\
  on couvrit son visage \\
  d'une étoffe blanche, \\
  et d'aucuns sanglotant, \\
  et d'autres silencieux, \\
  tous sortirent \\
  de la triste alcôve.

  La lumière, qui flamboyait \\
  dans un vase sur le sol, \\
  projetait sur le mur \\
  l'ombre de la couche, \\
  et parmi cette ombre \\
  on voyait, par intervalles, \\
  se dessiner, rigide, \\
  la forme du corps.

  Le jour s'éveillait, \\
  et à la première lueur, \\
  il réveillait le village \\
  de ses mille bruits. \\
  Devant ce contraste \\
  de vie et mystère, \\
  de lumière et ténèbres, \\
  je pensai un moment:

  \emph{Mon Dieu, oh combien \\
  seuls restent les morts!}

  Sur les épaules on la porta \\
  de la maison au temple, \\
  et on laissa le cercueil \\
  dans une chapelle. \\
  Là-bas on entoura \\
  sa pâle dépouille \\
  de cierges jaunes \\
  et d'étoffes noires.

  En sonnant des Âmes\footnote{NDT. Service nocturne pendant lequel les
  fidèles prient pour les âmes des défunts.} \\
  la dernière cloche, \\
  une vieille acheva \\
  ses ultimes prières; \\
  elle traversa la large nef, \\
  les portes gémirent, \\
  et le saint lieu \\
  resta désert.

  D'une horloge on entendait \\
  le balancier mesuré \\
  le crépitement \\
  de quelques cierges. \\
  Tout était \\
  si craintif et triste, \\
  si obscur et transi, \\
  que je pensai un moment:

  \emph{Mon Dieu, oh combien \\
  seuls restent les morts!}

  La langue de fer \\
  de la haute cloche \\
  lui dédia une volée \\
  d'«~adieu!~» plaintif. \\
  Le deuil aux habits, \\
  amis et proches \\
  passèrent en file, \\
  formant cortège.

  De l'ultime asile, \\
  obscur et étroit, \\
  le pic ouvrit la niche \\
  à une extrémité. \\
  Là on la coucha, \\
  et puis la mura , \\
  et avec un salut \\
  se retira le cortège.

  Le pic sur l'épaule, \\
  le fossoyeur \\
  chantonnant dans sa barbe \\
  se perdit au loin. \\
  La nuit s'avançait, \\
  le soleil s'était couché; \\
  perdu dans les ombres, \\
  je pensai un moment:

  \emph{Mon Dieu, oh combien \\
  seuls restent les morts!}

  Dans les longues nuits \\
  de l'hiver gelé \\
  quand le vent \\
  fait craquer les bois \\
  et la forte averse \\
  fouette les carreaux, \\
  de la pauvre enfant \\
  parfois je me souviens.

  Là-bas tombe la pluie \\
  d'un bruit éternel; \\
  là-bas la combat \\
  le souffle de la bise. \\
  Étendue dans le creux \\
  de l'humide mur, \\
  peut-être de froid \\
  se gèlent ses os!...

\ldots\ldots\ldots\ldots\ldots\ldots\ldots\ldots\ldots\ldots\ldots\ldots\ldots\ldots

  La poussière retourne-t-elle à la poussière? \\
  L'âme s'envole-t-elle au ciel? \\
  Tout est-il, sans âme, \\
  pauvreté et bourbe? \\
  Je ne sais; mais il y a \\
  quelque chose que je n'explique pas, \\
  quelque chose qui répugne, \\
  bien qu'il soit courageux le faire, \\
  à laisser si tristes, \\
  si seuls, les morts!
\end{verse}

%%%%%%%%%%%

\bigskip

\begin{center} {\bf 23 [LXXV]}\addcontentsline{toc}{subsection}{\em 23.\ Serait-il vrai que quand le sommeil touche...} \end{center}

\begin{verse}
Serait-il vrai que quand le sommeil touche \\
de ses doigts de rose nos yeux, \\
de la prison qu'elle habite l'âme \\
s'enfuit en vol pressé? \\ \

Serait-il vrai qu'hôte des brumes, \\
au souffle ténu de la brise nocturne, \\
ailée elle monte à la région vide \\
pour en rencontrer d'autres? \\ \

%\newpage

Et là dévêtue de l'humaine forme, \\
là les liens terrestres rompus, \\
de brèves heures elle habite \\
le monde silencieux de l'idée? \\ \

Et qu'elle rit et pleure, et exècre et aime \\
et garde un visage de douleur et de joie, \\
pareil à celui qu'elle laisse quand traverse \\
le ciel un météore? \\ \

Moi je ne sais si ce monde de visions \\
vit hors ou dans nous; \\
ce que je sais c'est que je connais beaucoup de gens \\
que je ne connais pas. \\
\end{verse}

\bigskip

\begin{center} {\bf 24 [LXXIV]}\addcontentsline{toc}{subsection}{\em 24.\ Les habits défaits, les épées nues...} \end{center}

\begin{verse}
Les habits défaits, \\
les épées nues, \\
sur le linteau d'or de la porte \\
deux anges veillaient. \\ \

Je m'approchai des fers forgés \\
qui défendent l'entrée, \\
et des doubles grilles au fond \\
je la vis confuse et blanche. \\ \

Je la vis comme l'image \\
qui dans une rêverie passe, \\
comme un rai de lumière ténu et diffus \\
qui entre des ténèbres nage. \\ \

Je sentis mon âme pleine \\
d'un ardent désir; \\
comme attire un abîme, ce mystère \\
vers lui m'entraînait, \\ \

mais, hélas!, des anges \\
paraissaient me dire les regards: \\
{\em Le seuil de cette porte \\
seul Dieu le passe!} \\
\end{verse}

\bigskip

\begin{center} {\bf 44 [LXXVII]}\addcontentsline{toc}{subsection}{\em 44.\ Tu dis que tu as un c{\oe}ur...} \end{center}

\begin{verse}
Tu dis que tu as un\footnote{NDT. On peut lire aussi ``du''.} c{\oe}ur, et tu le dis \\
seulement parce que tu sens ses battements. \\
Cela n'est pas un c{\oe}ur..., c'est une machine \\
qui en suivant sa mesure fait du bruit. \\
\end{verse}

\bigskip

\begin{center} {\bf 48 [LXXVIII]}\addcontentsline{toc}{subsection}{\em 48.\ Feignant des réalités avec ombre vaine...} \end{center}

\begin{verse}
Feignant des réalités \\
avec ombre vaine \\
devant le Désir \\
va l'Espérance; \\
et ses mensonges, \\
comme le Phénix, renaissent \\
de ses cendres. \\
\end{verse}

\bigskip

\begin{center} {\bf 55 [LXXIX]}\addcontentsline{toc}{subsection}{\em 55.\ Une femme m'a empoisonné l'âme...} \end{center}

\begin{verse}
Une femme m'a empoisonné l'âme, \\
une autre m'a empoisonné le corps; \\
aucune des deux ne vint me chercher, \\
moi, d'aucune des deux je ne me plains. \\ \

Comme le monde est rond, le monde tourne. \\
Si demain, tournant, ce poison \\
empoisonne à son tour, pourquoi m'accuser? \\
Puis-je donner plus que ce que l'on me donna?\footnote{NDT. Cette stance 55 apparait barrée dans le manuscrit original.} \\
\end{verse}

\bigskip

\begin{center} {\bf 74 [LXXVI]}\addcontentsline{toc}{subsection}{\em 74.\ Dans l'imposante nef du temple byzantin...} \end{center}

\begin{verse}
Dans l'imposante nef \\
du temple byzantin, \\
je vis la tombe gothique à l'indécise \\
lueur qui tremblait sur les vitraux. \\ \

Les mains sur la poitrine, \\
et dans les mains un livre, \\
une belle femme reposait \\
sur l'urne, prodige du ciseau. \\ \

Au doux poids enfoncé \\
du corps abandonné, \\
comme de tendre plume et lisse, \\
se pliait sa couche de granit. \\ \

Le divin éclat \\
de l'ultime sourire \\
le visage gardait, comme le ciel garde \\
du soleil qui meurt le rai fugitif. \\ \

Assis sur le bord \\
de l'oreiller de pierre, \\
deux anges, le doigt sur la lèvre, \\
imposaient silence à l'enceinte. \\ \

Elle ne semblait pas morte; \\
on l'aurait dit dormant \\
dans la pénombre des arcs massifs \\
et en songe voyant le paradis. \\ \

Je m'approchai \\
de l'angle sombre de la nef, \\
avec le pas retenu de qui vient \\
au berceau d'un enfant assoupi. \\ \

Je la contemplai un moment. \\
Et cet éclat tiède, \\
ce lit de pierre qui offrait, \\
proche du mur, un autre lieu vide, \\ \

dans l'âme avivèrent \\
la soif de l'infini, \\
le désir de cette vie de la mort, \\
pour laquelle un instant sont les siècles... \\ \

\ldots\ldots\ldots\ldots\ldots\ldots\ldots\ldots\ldots\ldots\ldots\ldots\ldots\ldots\ldots\ldots\ldots\ldots\ldots\ldots \\ \

Fatigué du combat \\
dans lequel je lutte, \\
parfois je me souviens avec envie \\
de ce recoin obscur et caché. \\ \

De cette silencieuse et pâle \\
femme je me souviens et dis: \\
{\em « Oh, quel amour si muet, celui de la mort! \\
Quel sommeil, celui du sépulcre si calme!} \\
\end{verse}

\bigskip

\tableofcontents

\end{document}
