%-*-latex-*-

\documentclass[a4paper,fontsize=13pt,twoside,final]{scrbook}

% Page geometry for L'Harmattan
%
\usepackage[bindingoffset=0cm,
            a4paper,
            left=55mm,
            top=6cm,
            textheight=185mm,
            textwidth=10cm,
            footskip=13mm,
            includefoot,
            includehead,
            dvips]{geometry}% showframe, showcrop

\usepackage[T1]{fontenc}
\usepackage[utf8]{inputenc}
\usepackage[french]{babel}
\usepackage{ebgaramond}

\usepackage{verse}
\usepackage{url}
\usepackage{hyphenat}

%\selectlanguage{francais}

\hyphenation{Gon-zá-lez thè-mes stro-phes con-trainte}

\frenchspacing  % Follow French conventions after a period

% Page style
%
%% \usepackage{fancyhdr}
%% \fancyhf{}
%% \cfoot{\thepage}
%% \pagestyle{plain}

\begin{document}

\pagestyle{empty}

%% \vspace*{70mm}

%% \begin{center}
%% {\Huge\textbf{RIMES}} \\
%% \bigskip
%% {\Large Gustavo Adolfo \textsc{Bécquer}} \\
%% \bigskip
%% Traduction de Christian Rinderknecht
%% \end{center}

%% \bigskip\bigskip\bigskip

%% \vfill
%% \begin{center}
%% L'Harmattan
%% \end{center}
%% %\centerline\today

\cleardoublepage
\
\cleardoublepage
\
\cleardoublepage
\

\begin{center}
  {\Large Gustavo Adolfo \textsc{Bécquer}}
\end{center}

\bigskip\bigskip\bigskip\bigskip\bigskip\bigskip

\begin{center}
  {\Huge\textbf{RIMES}}
\end{center}

\bigskip\bigskip\bigskip

\vfill
\begin{center}
L'Harmattan
\end{center}
%\centerline\today

\clearpage

\pagestyle{plain}

%\setcounter{page}{7}

\part{Biographie de l'auteur}
Gustavo Adolfo Bécquer naît à Séville en \oldstylenums{1836}, dans une
famille nombreuse. Son père, qui s'était distingué comme peintre,
meurt en \oldstylenums{1841}. Gustavo va à l'école en
\oldstylenums{1846}, où il reçoit une éducation en lettres
classiques. Un an après le décès de sa mère en \oldstylenums{1847}, il
rejoint une école des beaux arts, mais abandonne ses études en
\oldstylenums{1850}. Il reprend un cursus normal et publie prose et
poésie dans des revues sévillanes en \oldstylenums{1853} et
\oldstylenums{1854} ---~année où il se rend à Madrid et y publie des
critiques musicales et théâtrales.

En \oldstylenums{1856}, il s'emploie à dessiner et décrire une
histoire des lieux de cultes espagnols. En \oldstylenums{1858} sont
publiés certains de ses récits en proses, nommés \emph{Leyendas}
(Légendes), où un Moyen-Âge romantisé évoque une gloire passée.

Julia Espín devient sa muse de \oldstylenums{1858}
à \oldstylenums{1860}. En plus de ses poèmes, qu'il appelle
ses \emph{rimas} (rimes), il écrit aussi durant cette période des
opérettes espagnoles (\emph{zarzuelas}) et collabore avec son frère
Valeriano, peintre.

Il épouse Casta Esteban Navarro en
\oldstylenums{1861}. Jusqu'en \oldstylenums{1864}, il continue à
publier ses \emph{Leyendas} et le ministre Luis González Bravo le
nomme censeur des romans, ce qui le met enfin à l'abri du besoin.

En \oldstylenums{1868}, une révolution antimonarchique détrône la
rei-ne Isabel~II, et fait choir son gouvernement, dont Luis González
Bravo faisait partie. Celui-ci avait reçu de Bécquer un manuscrit
contenant, entre autres, les \emph{rimas}, et le saccage par la foule
du domicile du ministre déchu vit la disparition de ce premier
manuscrit. Bécquer perd son poste et retourne à la vie précaire de
journaliste. Il réécrit ce texte, qu'il intitule \emph{El libro de los
gorriones} (Le livre des moineaux). Le projet prévoyait une première
partie en prose (restée inachevée), et la seconde en vers
---~complète, semble-t-il. Il se sépare de son épouse, mais parvient à
rester en bons termes avec elle.

Il meurt en \oldstylenums{1870}, peu après son frère Valeriano. Un an
après paraît son œuvre en deux volumes, intitulée \emph{Obras}
(Œuvres), où presque tous les poèmes du \emph{livre des moineaux} sont
repris, corrigés parfois, et ordonnés par thèmes. Il retint du
romantisme le lyrisme, mais le dépassa par ses thèmes tantôt
symboliques, tantôt réalistes, ainsi que par une recherche esthétique
pour elle-même et un style direct.


\part{Introduction du traducteur}
\selectlanguage{francais}

Notre traduction reprend l'ordre du \emph{Livre des moineaux}, où
chaque poème est numéroté avec des chiffres arabes, mais nous
adjoignons aussi la numérotation des \emph{Œuvres}, en chiffres
romains. Nous avons repris les corrections posthumes et parfois pris
la peine de citer en notes les variantes de certains vers ou strophes,
non pour constituer un semblant d'apparat critique, mais pour faire
sentir au lecteur le processus créateur. Nous avons consulté une
source de référence, celle du \emph{Centro Virtual Cervantes} en
Espagne, mais aussi une édition de \emph{Rimas y leyendas}
de \oldstylenums{1984} aux éditions Orbis. Nous avons ajouté un
chapitre intitulé \emph{Autre rimes} regroupant des poèmes attribués à
l'auteur par la critique, mais qui ne faisaient ni partie
du \emph{Livre des moineaux} ni des
\emph{Œuvres}.

Les poèmes de Bécquer sont connus aujourd'hui sous le vocable de
\emph{Rimas} (Rimes), car l'auteur les appelaient ainsi auprès de ses
amis. Malgré l'apparence de vers libres, beaucoup de ses poèmes riment
au sens où ils contiennent des correspondances internes, souvent des
allitérations, des répétitions de mots, des structures parallèles, des
progressions etc. Bécquer joue beaucoup avec la syntaxe espagnole pour
réaliser ces rimes. Plutôt que faire systématiquement de même en
français, où l'ordre des propositions et des adjectifs est plus
contraint, nous avons opté pour une traduction plus fluide, surtout
dans les longs poèmes, pour ne pas égarer le lecteur. Nous avons
néanmoins tâché de recréer certaines allitérations, mais surtout les
structures entre strophes et vers, comme par exemple la mise en
exergue de certains mots, au début ou à la fin de certains vers.

L'espagnol du milieu du XIX\ieme{} siècle a changé: nous avons
consulté des sources philologiques pour traduire correctement certains
mots. Par ailleurs, certains termes religieux sont devenus obscurs:
nous avons fourni des notes pour les expliquer brièvement. Bécquer
avait une ponctuation idiosyncratique ---~quand elle n'était pas
absente ou surnuméraire: nous avons pris la liberté d'user d'une
ponctuation moderne qui sert la compréhension plutôt que le style,
surtout dans les poèmes les plus longs.

Gustavo Adolfo Bécquer est bien connu des espagnols pour sa poésie,
bien que sa prose soit plus volumineuse, parce que certains de ses
poèmes sont étudiés et appris par cœur dans les écoles, mais surtout
parce que leur lyrisme original sait toucher les jeunes cœurs. Voici
quelques thèmes qui traversent son œuvre: l'existence, avec sa cohorte
habituelle: destin, incertitude, mort, aspiration au repos
existentiel; les galanteries amoureuses; l'amour perdu avec son
aréopage de regrets, insomnies, mais aussi dépit et rancune; la
musique; la nature; enfin, la métapoésie, c'est-à-dire des poèmes sur
l'écriture poétique elle-même, sur le sujet poétique (en particulier,
l'idéal féminin), avec parfois des éléments platoniciens qui touchent
au symbolisme.

\bigskip
\bigskip
\bigskip
\bigskip
\hfill Christian Rinderknecht



\part{Rimes}
\input{rimes}

\part{Autres rimes}
\begin{center}
  \textbf{80}
  \addcontentsline{toc}{section}{\emph{80.\ La vie est un songe}}
\end{center}

\settowidth{\versewidth}{un songe qui durerait jusqu'à la mort...!}

\begin{verse}[\versewidth]
  La vie est un songe, \\
  mais un songe fébrile qui dure un point; \\
  quand on s'en éveille \\
  on voit que tout est vanité et fumée...

  Si seulement elle était un songe \\
  très long et très profond, \\
  un songe durant jusqu'à la mort... \\
  Je rêverais de mon amour et du tien.
\end{verse}

\bigskip

\begin{center}
  \textbf{81}
%  \addcontentsline{toc}{section}{\emph{81.\ Le soleil peut bien s'ennuager éternellement}}
\end{center}

\settowidth{\versewidth}{Le soleil peut bien s'ennuager éternellement;}

\poemtitle{Amour éternel}

\begin{verse}[\versewidth]
  Le soleil peut bien s'ennuager éternellement; \\
  la mer s'assécher en un instant; \\
  l'axe de la Terre se rompre \\
  comme un cristal fragile.

  Advienne que pourra! La mort peut bien \\
  me recouvrir de sa crêpe funèbre, \\
  mais jamais ne s'éteindra en moi \\
  la flamme de ton amour.
\end{verse}

\bigskip

\begin{center}
  \textbf{82}
  \addcontentsline{toc}{section}{\emph{82.\ Ton haleine est l'haleine des fleurs}}
\end{center}

\settowidth{\versewidth}{et la couleur des roses est ta couleur.}

\poemtitle{Pour Casta}

\begin{verse}[\versewidth]
  Ton\footnote{Casta Esteban Navarro, qui épousa
  l'auteur en 1861.} haleine est l'haleine des fleurs, \\
  ta voix est l'harmonie des cygnes, \\
  ton regard est la splendeur du jour, \\
  et la couleur des roses est ta couleur.

  Tu prêtes vie neuve et espoir \\
  à un cœur pour l'amour déjà mort; \\
  tu croîs de ma vie dans le désert \\
  comme la fleur dans les plateaux.
\end{verse}

\bigskip

\begin{center}
  \textbf{83}
%  \addcontentsline{toc}{section}{\emph{83.\ La goutte de rosée qui dort}}
\end{center}

\settowidth{\versewidth}{Il lui donne son mystère et sa poésie,}

\poemtitle{La goutte de rosée}

\begin{verse}[\versewidth]
  La goutte de rosée qui dort \\
  dans le calice du lys blanc \\
  est le palais de cristal où \\
  vit le génie heureux de la pureté.\footnote{Voir les rimes~52 et~85.}

  Il lui donne son mystère et sa poésie, \\
  il lui prête son arôme balsamique. \\
  Ah! Que de la lumière au baiser \\
  ne s'évapore cette perle de la fleur!
\end{verse}

\bigskip

\begin{center}
  \textbf{84}
%  \addcontentsline{toc}{section}{\emph{84.\ Loin, parmi les arbres de la jungle intriquée}}
\end{center}

\settowidth{\versewidth}{Désillusion. La lumière que nous avons suivie}

\begin{verse}[\versewidth]
  Loin, parmi les arbres \\
  de la jungle intriquée, \\
  ne vois-tu quelque chose qui brille \\
  et pleure? C'est une étoile.

  On la voit déjà plus proche \\
  briller au portique d'une ermitane,\footnote{Femme ermite.} \\
  comme au travers d'un tulle. \\
  C'est un réverbère.

  La course rapide s'achève ici. \\
  Désillusion. La lumière que nous avons suivie \\
  n'est ni réverbère ni étoile: \\
  c'est une lampe à huile.
\end{verse}

\bigskip

\begin{center}
  \textbf{85}
%  \addcontentsline{toc}{section}{\emph{85.\ Patriarches, qui furent la semence de l'arbre de la foi}}
\end{center}

\settowidth{\versewidth}{celui qui est arc-en-ciel de calme dans les tempêtes.}

\poemtitle{À tous les saints (Premier novembre)}

\begin{verse}[\versewidth]
  Patriarches, qui furent la semence \\
  de l'arbre de la foi des siècles lointains, \\
  priez pour nous \\
  le divin vainqueur de la mort.

  Prophètes inspirés, qui déchirèrent \\
  le voile mystérieux de l'avenir, \\
  priez pour nous \\
  celui qui sépara la lumière des ténèbres.

  Âmes candides, Saints Innocents, \\
  qui accrurent le chœur des anges, \\
  priez pour nous \\
  celui qui appela les enfants à son côté.

  Apôtres, qui établirent les fondations \\
  de l'Église dans le monde, \\
  priez pour nous \\
  le dépositaire de la vérité.

  Martyres qui remportèrent leur palme \\
  rouge de sang dans l'arène des cirques, \\
  priez pour nous \\
  celui qui vous donna fortitude dans les combats.

  Vierges semblables au lys, \\
  que l'été vêtit de neige de d'or,\footnote{Voir rimes~52 et~83.} \\
  priez pour nous \\
  celui qui est source et perfection.

  Moines, qui dans le combat de la vie \\
  demandèrent paix au cloître silencieux, \\
  priez pour nous \\
  celui qui est arc-en-ciel de calme dans les tempêtes.

  Docteurs, dont les plumes nous léguèrent \\
  des trésors de vertu et de savoir, \\
  priez pour nous \\
  celui qui est torrent de science intarissable.

  Soldats de l'armée du Christ, \\
  tous Saintes et Saints, \\
  priez celui qui vit et règne parmi nous \\
  pour que nos fautes nous soient pardonnées.
\end{verse}

\bigskip

\begin{center}
  \textbf{86}
%  \addcontentsline{toc}{section}{\emph{86.\ Ce cimetière est solitaire, triste et muet}}
\end{center}

\settowidth{\versewidth}{ses habitants ne pleurent pas}

\poemtitle{Dans l'album de Madame}

\begin{verse}[\versewidth]
  Ce cimetière \\
  est solitaire, triste et muet; \\
  ses habitants ne pleurent pas... \\
  Qu'ils sont heureux, les morts!
\end{verse}


\bigskip

\tableofcontents

\end{document}

Gustavo Adolfo Bécquer (\oldstylenums{1836}-\oldstylenums{1870}) était
un poète, écrivain, critique musical et théâtral, né à Séville
(Espagne). Il retint du romantisme le lyrisme, mais le dépassa par ses
thèmes tantôt symboliques, tantôt réalistes, ainsi que par une
recherche esthétique pour elle-même et un style direct. Il publia une
quinzaine de poèmes dans des revues littéraires et artistiques. Il
remit un recueil manuscrit de ses poèmes à son protecteur, Luis
González Bravo, alors premier ministre espagnol de la reine Isabel~II,
pour qu'il le préface. Malheureusement, la révolution antimonarchique
de \oldstylenums{1868} vit le saccage du domicile de l'homme d'État et
la disparition du manuscrit. Par la suite, Bécquer reconstitua (de
mémoire, prétend-il) le livre perdu, qu'il intitula \emph{El libro de
los gorriones} (Le livre des moineaux). Le projet prévoyait une
première partie en prose (restée inachevée), et la seconde en vers
---~complète, semble-t-il. Cette œuvre poétique fit partie de la
publication de deux volumes posthumes intitulés \emph{Obras} (Œuvres),
en \oldstylenums{1871}. Presque tous les poèmes furent retenus, mais
dans un ordre thématique.
