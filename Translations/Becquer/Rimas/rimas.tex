%-*-latex-*-

\documentclass[a4paper,12pt]{book}

\usepackage[T1]{fontenc}
\usepackage[utf8]{inputenc}
\usepackage[french]{babel}
\usepackage[charter]{mathdesign}
\usepackage{verse}
\usepackage{url}

\begin{document}
\thispagestyle{empty}
\vspace*{70mm}
\begin{center}
{\Huge\textbf{RIMES}} \\
\vspace*{10mm}
{\Large Gustavo Adolfo Bécquer} \\
\vspace*{10mm}
Traduction de Christian Rinderknecht\\
\url{rinderknecht@free.fr}
\end{center}

\cleardoublepage

%\selectlanguage{francais}
\frenchspacing  % Follow French conventions after a period

\begin{center}
  \textbf{1}
  \addcontentsline{toc}{subsection}{\emph{1.\ Je sais un hymne géant et étrange...}}
\end{center}

\settowidth{\versewidth}{qui puisse l'enfermer, et c'est à peine, ô ma belle!,}

\begin{verse}[\versewidth]
  Je sais un hymne géant et étrange \\
  qui annonce dans la nuit de l'âme une aurore, \\
  et ces pages sont de cet hymne \\
  des cadences que l'air dilate dans l'ombre.

  Je voudrais l'écrire, domptant \\
  de l'homme la rebelle langue mesquine, \\
  avec des mots qui soient à la fois \\
  soupirs et rires, couleurs et notes.

  Mais vaine est la lutte: il n'est aucune mesure \\
  qui puisse l'enfermer, et c'est à peine, ô ma belle!, \\
  si, en tenant dans mes mains les tiennes, \\
  je peux te le conter seul à seul à l'oreille.
\end{verse}

\bigskip

\begin{center}
  \textbf{2}
  \addcontentsline{toc}{subsection}{\emph{2.\ \emph{Saeta} qui s'envole...}}
\end{center}

\settowidth{\versewidth}{où, tremblante, elle se plantera;}

\begin{verse}[\versewidth]
  \emph{Saeta}\footnote{Courte prière chantée
  depuis les balcons au passage des trônes portant des scènes de la
  Passion du Christ, pendant la Semaine Sainte, principalement en
  Andalousie. L'étymologie est le latin \emph{sagitta},
  signifiant \emph{flèche}, d'où la métaphore qui suit.} qui traverse en volant, \\
  lancée au hasard \\
  sans qu'on ne sache \\
  où, tremblante, elle se plantera;

  feuille sèche de l'arbre \\
  emportée par la bourrasque,\footnote{Il pourrait s'agir aussi, au
  sens propre, du \emph{vendaval}, un vent du sud soufflant sur la
  vallée du Guadalquivir, qui traverse Séville.} \\
  et on ne devine le sillon \\
  où elle retombera;

  vague géante que le vent \\
  enfle et pousse dans la mer, \\
  et roule et passe, et ne sait \\
  quelle rivage elle va cherchant;

  lueur qui, prête à s'éteindre, \\
  brille en ronds tremblants, \\
  et on ne sait d'eux \\
  lequel sera le dernier:

  c'est moi qui, au hasard, \\
  traverse le monde sans penser \\
  d'où je viens, ni où \\
  mes pas me mèneront.
\end{verse}

\bigskip

\begin{center}
  \textbf{3}
  \addcontentsline{toc}{subsection}{\emph{3.\ Secousse étrange qui
    agite les idées...}}
\end{center}

\settowidth{\versewidth}{comme au travers d'un tulle,}

\begin{verse}[\versewidth]
  Secousse étrange \\
  qui agite les idées, \\
  comme ouragan qui pousse \\
  les vagues au galop;

  murmure qui dans l'âme \\
  s'élève et va croissant, \\
  comme volcan qui, sourd, \\
  annonce qu'il va s'embraser;

  silhouettes difformes \\
  d'êtres impossibles; \\
  paysages qui apparaissent \\
  comme au travers d'un tulle;

  couleurs qui, en se fondant, \\
  imitent dans l'air \\
  les atomes de l'iris, \\
  qui nagent dans la lumière;

  idées sans paroles, \\
  paroles insensées; \\
  cadences qui n'ont \\
  ni rythme ni mesure;

  souvenirs et désirs \\
  de ce qui n'existe pas; \\
  transports de joie, \\
  envies de pleurer;

  activité nerveuse \\
  qui erre sans emploi, \\
  sans rênes qui guident \\
  ce cheval ailé;

  folie que l'âme \\
  exalte et enflamme, \\
  ivresse divine \\
  du génie créateur...

  Telle est l'inspiration!

  Voix géante qui ordonne \\
  le chaos dans le cerveau, \\
  et, parmi les ombres, fait \\
  apparaître la lumière;

  brillante rêne d'or \\
  qui, puissante, freine \\
  de l'esprit exalté \\
  le coursier volant;

  fil de lumière qui en gerbes \\
  noue les pensées, \\
  soleil qui rompt les nuées \\
  et atteint le zénith;

  main intelligente \\
  qui, en un collier de perles, \\
  parvient à réunir \\
  les mots indociles;

  rythme harmonieux \\
  qui, avec cadence et nombre, \\
  enserre dans la mesure \\
  les notes fugitives;

  ciseau qui mord dans le bloc, \\
  modelant la statue, \\
  et la beauté plastique \\
  ajoute à l'idéale;

  atmosphère où tournent \\
  les idées en ordre, \\
  telles des atomes que réunit \\
  une attraction secrète;

  torrent où la fièvre \\
  éteint sa soif; \\
  oasis qui à l'esprit \\
  rend sa vigueur...

  Telle est notre raison!

  Avec ces deux\footnote{Inspiration et raison.} toujours en lutte \\
  et des deux vainqueur, \\
  tant il n'est donné qu'au génie \\
  de les mettre sous le même joug.
\end{verse}

\bigskip

\begin{center}
  \textbf{3}
  \addcontentsline{toc}{subsection}{\emph{3.\ Ne dites pas qu'épuisé
    son trésor...}}
\end{center}

\settowidth{\versewidth}{tant qu'il existera une femme splendide,}

\begin{verse}[\versewidth]
  Ne dites pas qu'épuisé son trésor, \\
  faute de sujet, la lyre s'est tue: \\
  il pourrait ne pas y avoir de poètes, \\
  mais toujours il y aura la poésie.

  Tant que les ondes embrasées \\
  de la lumière palpiteront aux baisers, \\
  tant que le soleil vêtira \\
  les nuées déchirées de feu et d'or; \\
  tant que l'air en son giron portera \\
  parfums et harmonies; \\
  tant qu'il aura un printemps au monde, \\
  il y aura la poésie!

  Tant que la science échouera à découvrir \\
  la source de la vie, \\
  et qu'en mer ou au ciel il y aura un abîme \\
  qui résiste au calcul; \\
  tant que l'humanité, toujours progressant, \\
  ne saura où elle va; \\
  tant qu'il aura un mystère pour l'homme, \\
  il y aura la poésie!

  Tant que l'on sentira l'âme se réjouir \\
  sans que les lèvres rient; \\
  tant que l'on pleurera sans que le sanglot \\
  ne vienne troubler la pupille; \\
  tant que le c{\oe}ur et la tête \\
  continueront à batailler; \\
  tant qu'il y aura espoirs et souvenirs, \\
  il y aura la poésie!

  Tant qu'il y aura des yeux qui reflètent \\
  les yeux qui les regardent, \\
  tant que répondra la lèvre soupirant \\
  à la lèvre qui soupire; \\
  tant que deux âmes en un baiser \\
  confondues pourront se toucher; \\
  tant qu'il existera une femme splendide, \\
  il y aura la poésie!
\end{verse}

\bigskip

\begin{center}
  \textbf{5}
  \addcontentsline{toc}{subsection}{\emph{5.\ Esprit sans nom,
    indéfinissable essence...}}
\end{center}

\settowidth{\versewidth}{que tapissent de blanches perles,}

\begin{verse}[\versewidth]
  Esprit sans nom, \\
  indéfinissable essence, \\
  je vis avec la vie \\
  sans formes de l'idée.

  Je nage dans le vide, \\
  tremble dans le brasier solaire, \\
  je palpite parmi les ombres \\
  et flotte avec les brumes.

  Je suis la frange d'or \\
  de la lointaine étoile, \\
  je suis de la haute lune \\
  la lumière tiède et sereine.

  Je suis l'ardent nuage \\
  qui ondoie dans le couchant, \\
  je suis de l'astre errant \\
  le sillage lumineux.

  Je suis neige sur les cimes, \\
  je suis feu sur les sables, \\
  onde bleue sur les mers \\
  et écume sur les rivages.

  Dans le luth je suis note, \\
  parfum dans la violette, \\
  flamme fugace dans les tombes \\
  et lierre dans les ruines.

  Je chante avec l'alouette \\
  et bourdonne avec l'abeille; \\
  j'imite les bruits \\
  qui résonnent en pleine nuit.\footnote{NDT. Ce quatrain ne
figure pas dans le manuscrit original, mais dans la publication dans
le journal \emph{El Museo Universal}, page~31, le~28 janvier 1866 (voir \url{prensahistorica.mcu.es}).}

  Je tonne dans le torrent \\
  et siffle dans la foudre, \\
  et aveugle dans l'éclair \\
  et rugis dans la tempête.

  Je ris sur les collines, \\
  susurre dans les herbes hautes, \\
  soupire dans l'onde pure \\
  et pleure sur les feuilles sèches.

  J'ondule avec les atomes \\
  de la fumée qui s'élève \\
  et monte lentement au ciel \\
  en spirales immenses.

  Parmi les fils dorés \\
  que les insectes suspendent, \\
  je me mêle aux arbres \\
  dans l'ardente sieste.

  Je cours après les nymphes \\
  qui, dans le courant frais\footnote{La publication dans
le journal \emph{El Museo Universal}, page~31, le~28 janvier 1866
(voir \url{prensahistorica.mcu.es}) recense: «~le courant inquiet~».} \\
  de la rivière cristalline, \\
  s'ébattent nues.

  Dans des bois de coraux \\
  qui tapissent de blanches perles, \\
  je poursuis dans l'Océan \\
  les naïades légères.

  Dans les cavernes concaves \\
  où le soleil ne pénètre jamais, \\
  me mêlant aux gnomes, \\
  je contemple leurs richesses.

  Je cherche des siècles \\
  les traces effacées, \\
  et je sais de ces empires \\
  dont il ne reste même pas le nom.\footnote{Variante dans le journal
  \emph{El Museo Universal}, page~31, le~28 janvier 1866 (voir
  \url{prensahistorica.mcu.es}): «~Je rencontre les traces effacées~/~de ces siècles,~/~dont il ne reste aucun souvenir~/~sur la face du globe.~»}

  Je poursuis en un brusque vertige \\
  les mondes qui voltigent, \\
  et ma pupille embrasse \\
  la création entière.\footnote{Variante dans le journal
  \emph{El Museo Universal}, page~31, le~28 janvier 1866 (voir
  \url{prensahistorica.mcu.es}): «~J'embrasse du regard~/~la création
  entière,~/~et poursuis en un brusque vertige~/~les astres qui voltigent.~»}

  Je sais de ces régions \\
  qu'une rumeur n'atteint pas, \\
  et où d'informes astres \\
  attendent un souffle de vie.

  Je suis sur l'abîme \\
  le pont qui traverse, \\
  et l'échelle inconnue \\
  qui unit le ciel à la terre.\footnote{Variante dans le journal
  \emph{El Museo Universal}, page~31, le~28 janvier 1866 (voir
  \url{prensahistorica.mcu.es}): «~Je suis l'échelle inconnue~/~qui unit
  le ciel à la terre,~/~et ouvre à la pensée~/~un chemin vers d'autres
  sphères.~»}

  Je suis l'anneau invisible \\
  qui fixe \\
  le monde de la forme \\
  au monde de l'idée.

  Enfin, je suis cet esprit, \\
  essence inconnue,\footnote{Variante dans le journal
  \emph{El Museo Universal}, page~31, le~28 janvier 1866 (voir
  \url{prensahistorica.mcu.es}): «~l'essence du sentiment,~»} \\
  parfum mystérieux \\
  dont le vase est le poète.
\end{verse}

\bigskip

\begin{center}
  \textbf{6}
  \addcontentsline{toc}{subsection}{\emph{6.\ Comme la brise qui
    rafraîchit le sang...}}
\end{center}

\settowidth{\versewidth}{chante et cueille des fleurs en passant.}

\begin{verse}[\versewidth]
  Comme la brise qui rafraîchit le sang \\
  sur le champ sombre des batailles, \\
  chargée de parfums et d'harmonies \\
  dans le silence de la nuit, elle erre;

  symbole de la douleur et de la tendresse, \\
  dans l'horrible drame du barde anglais, \\
  la douce Ophélie,\footnote{Personnage de la pièce de Shakespeare \emph{Hamlet}.} la raison égarée, \\
  chante et cueille des fleurs en passant.
\end{verse}

\bigskip

\begin{center}
  \textbf{7}
  \addcontentsline{toc}{subsection}{\emph{7.\ Dans l'angle obscur du
    salon...}}
\end{center}

\settowidth{\versewidth}{comme dorment les oiseaux sur les branches,}

\begin{verse}[\versewidth]
  Dans l'angle obscur du salon, \\
  de son maître peut-être oubliée, \\
  silencieuse et couverte de poussière, \\
  trônait la harpe.

  Que de notes dormaient sur ses cordes, \\
  comme dorment les oiseaux sur les branches, \\
  attendant la main de neige \\
  qui les fait s'envoler!

  Hélas! pensai-je. Que de fois le génie \\
  ainsi dort au fond de l'âme, \\
  et une voix attend, comme Lazare, \\
  qui lui dise: \emph{Lève-toi et marche!}
\end{verse}

\bigskip

\begin{center}
  \textbf{8}
  \addcontentsline{toc}{subsection}{\emph{8.\ Quand je vois
    l'horizon bleu...}}
\end{center}

\settowidth{\versewidth}{et m'inonder de leur lumière, et avec elles}

\begin{verse}[\versewidth]
  Quand je regarde l'horizon bleu \\
  se perdre au lointain, \\
  au travers d'une gaze de poussière \\
  dorée et inquiète,

  je crois possible de m'arracher \\
  du sol misérable \\
  et flotter avec la brume dorée \\
  en atomes légers, \\
  défait comme elle.

  Quand je vois de nuit, dans le fond \\
  obscur du ciel, \\
  trembler les étoiles comme d'ardentes \\
  pupilles de feu,

  je crois possible de m'envoler \\
  là où elles brillent, \\
  et m'inonder de leur lumière, et avec elles, \\
  en un feu qui a pris, \\
  me fondre en un baiser.

  Sur la mer de doute où je vogue \\
  je ne sais même pas ce que je crois; \\
  pourtant ces désirs me disent \\
  que je porte quelque chose \\
  de divin, ici en moi.
\end{verse}

\bigskip

\begin{center}
  \textbf{9}
  \addcontentsline{toc}{subsection}{\emph{9.\ Le zéphir qui gémit
    faiblement...}}
\end{center}

\settowidth{\versewidth}{jusqu'à ce que de pourpre et d'or il la nuance;}

\begin{verse}[\versewidth]
  Le zéphyr qui gémit faiblement \\
  baise les ondes légères qu'il plisse en jouant; \\
  le soleil baise la nuée à l'Occident \\
  jusqu'à ce que de pourpre et d'or il la nuance; \\
  la flamme à l'entour du tronc ardent \\
  s'étale en baisant une autre flamme, \\
  et jusqu'au saule pesant, qui se penche \\
  vers la rivière qui le baise, renvoie un baiser.
\end{verse}

\bigskip

\begin{center}
  \textbf{10}
  \addcontentsline{toc}{subsection}{\emph{10.\ Les invisibles atomes de
    l'air alentour palpitent et s'enflamment...}}
\end{center}

\settowidth{\versewidth}{Dis-moi...? Silence! C'est l'amour qui passe!}

\begin{verse}[\versewidth]
  Les invisibles atomes de l'air \\
  alentour palpitent et s'enflamment, \\
  le ciel se défait en rayons d'or, \\
  la terre frémit de joie; \\
  j'entends, flottant sur des ondes d'harmonie, \\
  rumeur de baisers et battements d'ailes, \\
  mes paupières se closent... Qu'arrive-t-il? \\
  ---~C'est l'amour qui passe!
\end{verse}

\bigskip

\begin{center}
  \textbf{11}
  \addcontentsline{toc}{subsection}{\emph{11.\ Je suis ardente, je
    suis brune...}}
\end{center}

\settowidth{\versewidth}{de désirs de jouissance mon âme est pleine.}

\begin{verse}[\versewidth]
  ---~Je suis ardente, je suis brune, \\
  je suis le symbole de la passion; \\
  de désirs de jouissance mon âme est pleine. \\
  Est-ce moi que tu cherches?

  \hfill ---~Ce n'est pas toi, non.

  ---~Mon front est pâle, mes tresses d'or; \\
  je peux t'offrir des bonheurs sans fin; \\
  je garde un trésor de tendresse. \\
  Est-ce moi que tu appelles?

  \hfill ---~Ce n'est pas toi, non.

  ---~Je suis un songe, fantôme \\
  impossible et vain de brume et lumière; \\
  je suis incorporelle, je suis intangible, \\
  je ne puis t'aimer.

  \hfill ---~Oh viens, toi, viens!
\end{verse}

\bigskip

\begin{center}
  \textbf{12}
  \addcontentsline{toc}{subsection}{\emph{12.\ Petite, parce que tes yeux...}}
\end{center}

\settowidth{\versewidth}{Petite, parce que tes yeux}

\begin{verse}[\versewidth]
  Petite, parce que tes yeux \\
  sont verts comme la mer, tu te plains; \\
  verts sont ceux des naïades, \\
  verts les eut Minerve, \\
  et vertes sont les pupilles \\
  des houris\footnote{NDT. Beauté céleste que le Coran promet au musulman dans le paradis d'Allah.} du Prophète.

  Le vert est gala et ornement \\
  de la forêt au printemps; \\
  parmi ses sept couleurs, \\
  l'iris brillant l'affiche; \\
  les émeraudes sont vertes, \\
  verte la couleur de qui espère, \\
  et les ondes de l'Océan \\
  et le laurier des poètes.

  $$\star \ \ \ \star \ \ \ \star$$

  Ta joue est une rose matinale \\
  couverte de rosée congelée, \\
  où le carmin des pétales \\
  se voit à travers des perles.

  Et pourtant, \\
  je sais que tu te plains \\
  car tu crois que tes yeux \\
  l'enlaidissent: \\
  eh bien ne le crois pas,

  car tes pupilles humides, \\
  vertes et inquiètes, \\
  semblent de jeunes feuilles d'amandier, \\
  qui tremblent dans la brise.

  Ta bouche pourpre-rubis \\
  est grenade entrouverte \\
  qui dans l'été invite \\
  à éteindre la soif en elle.

  Et pourtant, \\
  je sais que tu te plains \\
  car tu crois que tes yeux
  l'enlaidissent: \\
  eh bien ne le crois pas,

  car, si fâchée, \\
  tes pupilles scintillent, \\
  tes yeux ressemblent \\
  aux vagues qui se brisent \\
  sur les rochers cantabriques.

  $$\star \ \ \ \star \ \ \ \star$$

  Ton front, couronné \\
  de l'or crépu d'une large tresse, \\
  est une cime enneigée où le jour \\
  reflète sa première lueur.

  Et pourtant, \\
  je sais que tu te plains \\
  car tu crois que tes yeux \\
  l'enlaidissent: \\
  eh bien ne le crois pas,

  car parmi les cils blonds, \\
  proche des tempes, ils semblent \\
  des broches d'émeraude et or \\
  haussant une blanche hermine.

  Petite, parce que tes yeux \\
  sont verts comme la mer, tu te plains; \\
  peut-être, si noirs ou bleus \\
  ils devenaient, tu le regretterais.
\end{verse}

\bigskip

\begin{center}
  \textbf{13}
  \addcontentsline{toc}{subsection}{\emph{13.\ Ta pupille est bleue...}}
\end{center}

\settowidth{\versewidth}{comme un point de lumière irradie une idée,}

\begin{verse}[\versewidth]
  Ta pupille est bleue et quand tu ris \\
  sa clarté suave me rappelle \\
  l'éclat tremblant du matin \\
  qui se reflète dans la mer.

  Ta pupille est bleue et quand tu pleures \\
  les larmes transparentes en elle \\
  me semblent gouttes de rosée \\
  sur une violette.

  Ta pupille est bleue et si au fond \\
  comme un point de lumière irradie une idée, \\
  elle paraît dans le ciel du soir \\
  une étoile perdue.
\end{verse}

\bigskip

\begin{center}
  \textbf{14}
  \addcontentsline{toc}{subsection}{\emph{14.\ Je t'entrevis, et flottant devant mes yeux...}}
\end{center}

\settowidth{\versewidth}{Je sais qu'il est des feux follets la nuit}

\begin{verse}[\versewidth]
  Je t'entrevis et l'image de tes yeux resta, \\
  flottant devant mes yeux \\
  comme la tâche sombre bordée de feu \\
  qui flotte et aveugle si l'on fixe le soleil.

  Et où que je pose le regard \\
  je revois tes pupilles flamboyer \\
  mais tu n'es pas là; c'est ton regard, \\
  des yeux, les tiens; rien de plus.

  Dans l'angle de mon alcôve je les regarde \\
  luire, détachés, fantastiques; \\
  quand je dors je les sens m'examiner, \\
  grand ouverts sur moi.

  Je sais qu'il est des feux follets la nuit \\
  qui mènent le voyageur à sa perte; \\
  moi je me sens entraîné par tes yeux, \\
  mais où ils m'entraînent, je ne le sais.
\end{verse}

\bigskip

\begin{center}
  \textbf{15}
  \addcontentsline{toc}{subsection}{\emph{15.\ Flottant voile de légère brume,...}}
\end{center}

\settowidth{\versewidth}{Moi, qui dans mon agonie, vers tes yeux}

\begin{verse}[\versewidth]
  Voile flottant de brume légère, \\
  ruban plissé de blanche écume, \\
  rumeur sonore \\
  d'une harpe d'or, \\
  baiser du zéphir, onde de lumière, \\
  tu es cela.

  Toi, ombre aérienne, qui t'évanouis \\
  quand je crois enfin te saisir. \\
  Comme la flamme, comme le son, \\
  comme la brume, comme le gémissement \\
  du lac bleu!

  En mer, onde sonnante sans rivages; \\
  dans le vide, comète errante, \\
  longue complainte \\
  du vent rauque, \\
  soif perpétuelle de mieux, \\
  je suis cela.

  Moi, qui dans mon agonie, vers tes yeux \\
  retourne mes yeux jour et nuit; \\
  moi, qui infatigable et dément, \\
  cours après une ombre, la fille ardente \\
  d'une vision!
\end{verse}

\bigskip

\begin{center}
  \textbf{16}
  \addcontentsline{toc}{subsection}{\emph{16.\ Si, quand les clochettes bleues de ton balcon...}}
\end{center}

\settowidth{\versewidth}{Si quand les clochettes bleues de ton balcon}

\begin{verse}[\versewidth]
  Si, quand les clochettes bleues de ton balcon \\
  se bercent, \\
  tu crois qu'en soupirant passe le vent \\
  qui murmure, \\
  sache que, caché parmi les feuilles vertes, \\
  moi je soupire.

  Si, quand résonne, confuse derrière toi, \\
  une vague rumeur, \\
  tu crois que par ton nom t'a appelé \\
  une voix lointaine, \\
  sache que, parmi les ombres qui t'entourent, \\
  moi je t'appelle.

  Si, quand se trouble ton cœur craintif \\
  en pleine nuit, \\
  si tu sens sur tes lèvres une haleine \\
  qui embrase, \\
  sache que, bien qu'invisible à tes côtés, \\
  moi je respire.
\end{verse}

\bigskip

\begin{center}
  \textbf{17}
  \addcontentsline{toc}{subsection}{\emph{17.\ Aujourd'hui la terre et les cieux me sourient...}}
\end{center}

\settowidth{\versewidth}{aujourd'hui je l'ai vue..., je l'ai vue et elle m'a regardé...}

\begin{verse}[\versewidth]
  Aujourd'hui la terre et les cieux me sourient, \\
  aujourd'hui le soleil atteint le fond de mon âme, \\
  aujourd'hui je l'ai vue..., je l'ai vue et elle m'a regardé... \\
  Aujourd'hui je crois en Dieu!
\end{verse}

\bigskip

\begin{center}
  \textbf{18}
  \addcontentsline{toc}{subsection}{\emph{18.\ Fatiguée par la danse...}}
\end{center}

\settowidth{\versewidth}{que pousse la mer et caresse le zéphir,}

\begin{verse}[\versewidth]
  Fatiguée par la danse, \\
  ardente la couleur, brève l'haleine, \\
  appuyée à mon bras, \\
  elle s'arrêta à un bout du salon.

  Parmi la gaze légère \\
  que soulevait le sein palpitant, \\
  une fleur était bercée \\
  d'un mouvement doux et mesuré.

  Comme dans un berceau de nacre \\
  que pousse la mer et caresse le zéphir, \\
  peut-être dormait-elle là-bas du souffle \\
  de ses lèvres entrouvertes.

  Oh! Qui, pensai-je, pourrait ainsi \\
  laisser filer le temps! \\
  Oh! Si les fleurs dorment, \\
  quel sommeil\footnote{NDT. On peut lire aussi «songe» (\emph{sueño})} si doux!
\end{verse}

\bigskip

\begin{center}
  \textbf{19}
  \addcontentsline{toc}{subsection}{\emph{19.\ Quand sur ta poitrine
    tu penches un front mélancolique...}}
\end{center}

\settowidth{\versewidth}{Quand sur ta poitrine tu penches}

\begin{verse}[\versewidth]
  Quand sur ta poitrine tu penches \\
  un front mélancolique, \\
  tu me sembles \\
  un lys brisé,

  car, en te donnant la pureté \\
  qui est symbole céleste, \\
  comme lui te fit Dieu \\
  d'or et de neige.
\end{verse}

\bigskip

\begin{center}
  \textbf{20}
  \addcontentsline{toc}{subsection}{\emph{20.\ Elle sait, si parfois ses lèvres rouges...}}
\end{center}

\settowidth{\versewidth}{sont brûlées par une invisible atmosphère,}

\begin{verse}[\versewidth]
  Elle sait, si parfois ses lèvres rouges \\
  sont brûlées par une invisible atmosphère, \\
  que l'âme qui peut parler avec les yeux \\
  aussi peut embrasser avec le regard.
\end{verse}

\bigskip

\begin{center}
  \textbf{21}
  \addcontentsline{toc}{subsection}{\emph{21.\ Qu'est la poésie? dis-tu en plantant...}}
\end{center}

\settowidth{\versewidth}{Qu'est la poésie! Et toi tu me le demandes?}

\begin{verse}[\versewidth]
  Qu'est la poésie? dis-tu en plantant \\
  dans ma pupille ta pupille bleue. \\
  Qu'est la poésie! Et toi tu me le demandes? \\
  La poésie... c'est toi.
\end{verse}

%%%%%%%%%%%

\bigskip

\begin{center}
  \textbf{XLVIII}\addcontentsline{toc}{subsection}{\emph{XLVIII.\ Comme s'arrache le fer d'une plaie...}}
\end{center}

%\settowidth{\versewidth}{}

\begin{verse}
Comme s'arrache le fer d'une plaie \\
son amour de mes entrailles je m'arrachai, \\
bien que je sentis ce faisant que la vie \\
je m'arrachais avec lui! \\ \

De l'autel que je lui dressai dans mon âme \\
la volonté abattit son image, \\
et la lumière de la foi qui en elle brûlait \\
devant l'autel désert s'éteignit. \\ \

Troublant encore dans la nuit la ferme décision \\
la vision tenace vit dans l'idée... \\
Quand pourrai-je dormir de ce sommeil \\
dans lequel finit le rêve! \\
\end{verse}

\bigskip

\begin{center} {\bf 2 [XLVII]}\addcontentsline{toc}{subsection}{\em 2.\ Je me suis penché sur les gouffres béants...} \end{center}

\begin{verse}
Je me suis penché sur les gouffres béants\\
de la terre et du ciel, \\
et j'en ai vu la fin ou avec les yeux \\
ou avec la pensée. \\ \

Mais, hélas!, d'un c{\oe}ur je vins à l'abîme \\
et je m'inclinai un moment; \\
et mon âme et mes yeux se troublèrent: \\
si profond et si noir il était! \\
\end{verse}

\bigskip

\begin{center} {\bf 3 [XLV]}\addcontentsline{toc}{subsection}{\em 3.\ À la clef d'un arc en ruine...} \end{center}

\begin{verse}
À la clef d'un arc en ruine \\
aux pierres rougies par le temps, \\
{\oe}uvre d'un rude ciseau campait \\
le gothique blason. \\ \

Panache de son heaume de granit, \\
le lierre qui pendait autour \\
ombrait l'écu, dont une main \\
tenait un c{\oe}ur. \\ \

%\newpage

Pour le contempler en la déserte place \\
nous nous arrêtâmes tous deux: \\
et cela, me dit-elle, est le parfait emblème \\
de mon constant amour. \\ \

Hélas! Ce qu'elle me dit alors était vrai: \\
vrai que le c{\oe}ur \\
elle le portât dans la main... partout..., \\
mais dans la poitrine non. \\
\end{verse}

\bigskip

\begin{center} \textbf{4 [XXXVIII]} \addcontentsline{toc}{subsection}{\em 4.\ Les soupirs sont air et à l'air ils vont!...} \end{center}

\begin{verse}
Les soupirs sont air et à l'air ils vont! \\
Les larmes sont eau et à la mer elles vont! \\
Dis-moi, femme: quand l'amour s'oublie, \\
sais-tu toi où il va? \\
\end{verse}

\bigskip

\begin{center} {\bf 5 [LXII]}\addcontentsline{toc}{subsection}{\em 5.\ Les ondes ont une vague harmonie...} \end{center}

\smallskip

\begin{center} {\em Première voix} \end{center}

\begin{verse}
Les ondes ont vague harmonie, \\
les violettes, suave odeur; \\
les brumes d'argent la froide nuit, \\
lumière et or le jour; \\
moi, chose bien meilleure: \\
moi je détiens l'{\em Amour\/}!
\end{verse}

\smallskip

\begin{center} {\em Deuxième voix} \end{center}

\begin{verse}
Nuage radieux, bravos de liesse, \\
vague d'envie qui baise le pied, \\
île de songes où repose \\
l'âme inassouvie. \\
Douce ivresse, \\
c'est le {\em Paradis}.
\end{verse}

%\newpage

\begin{center} {\em Troisième voix} \end{center}

\begin{verse}
Braise allumée est le trésor, \\
Ombre fuyante la vanité, \\
et tout est faux: la gloire, l'or. \\
Ce que moi j'adore \\
seul est vérité: \\
La {\em Liberté\/}!
\end{verse}

\bigskip

\begin{verse}
Ainsi les bateliers passaient chantant \\
l'éternelle chanson, \\
et au coup de rame sautait l'écume \\
et la frappait le soleil. \\ \

{\em T'embarques-tu?}, criaient-ils. Et moi, souriant, \\
je leur dis au passage: \\
« {\em J'ai déjà embarqué} ~», et par gestes que \\
mes habits étendus sèchent sur la plage. \\
\end{verse}

%\newpage

\begin{center} {\bf 7 [XXVI]}\addcontentsline{toc}{subsection}{\em 7.\ Je vais contre mes intérêts en le confessant...} \end{center}

\begin{verse}
Je vais contre mes intérêts en le confessant: \\
nonobstant, mon aimée, \\
je pense comme toi qu'une ode est seule bonne \\
écrite au dos d'un billet de banque\footnote{NDT. Il s'agit des ordres de paiement, dont les versos étaient vierges.}. \\
Il ne manquera pas quelque sot qui en l'entendant \\
ne se signe et dise: \\
{\em Femme, à la fin du dix-neuvième siècle, \\
matérielle et prosaïque...} Sottises! \\
Des voix qui font courir quatre poètes \\
qui dans l'hiver se drapent de la lyre! \\
Aboiements des chiens à la lune! \\
Tu sais et je sais que dans cette vie, \\
avec génie, est très rare celui qui {\em l'écrit}, \\
et avec l'or, quiconque {\em fait} de la poésie. \\
\end{verse}

\bigskip

\begin{center} {\bf 8 [LVIII]}\addcontentsline{toc}{subsection}{\em 8.\ Veux-tu que de ce nectar délicieux ne t'éc{\oe}ure pas la lie?...} \end{center}

\begin{verse}
Veux-tu que de ce nectar délicieux \\
ne t'éc{\oe}ure pas la lie? \\
Alors aspire-le, approche-le de tes lèvres \\
et laisse-le après. \\ \

Veux-tu que nous conservions un doux \\
souvenir de cet amour? \\
Alors aimons-nous aujourd'hui, et demain \\
disons-nous adieu! \\
\end{verse}

\bigskip

\begin{center} {\bf 9 [LV]}\addcontentsline{toc}{subsection}{\em 9.\ Dans le tumulte discordant de l'orgie...} \end{center}

\begin{verse}
Dans le tumulte discordant de l'orgie \\
caressa mon oreille, \\
comme une note de lointaine musique, \\
l'écho d'un soupir. \\ \

L'écho d'un soupir que je connais, \\
formé d'une haleine que j'ai bue, \\
parfum d'une fleur qui croît cachée \\
dans un sombre cloître. \\ \

Mon adorée d'un jour, tendre, \\
{\em À quoi penses-tu?\/}, me dit-elle. \\
{\em À rien... À rien, et tu pleures? C'est que \\
gaie est ma tristesse et triste est mon vin.} \\
\end{verse}

%\newpage

\begin{center} {\bf 10 [XLIV]}\addcontentsline{toc}{subsection}{\em 10.\ Comme dans un livre ouvert...}  \end{center}

\begin{verse}
Comme dans un livre ouvert \\
je lis dans le fond de tes pupilles; \\
À quoi bon feint la lèvre \\
des rires que démentent les yeux? \\ \

Pleure! N'ai pas honte \\
de confesser que tu m'aimas un peu. \\
Pleure, personne ne nous voit! \\
Vois: je suis un homme... et je pleure aussi! \\
\end{verse}

\bigskip

\begin{center} {\bf 12 [L]}\addcontentsline{toc}{subsection}{\em 12.\ Ce que fait le sauvage qui de gauche main...} \end{center}

\begin{verse}
Ce que fait le sauvage qui de gauche main \\
fait à discrétion d'un tronc un dieu, \\
et ensuite devant son {\oe}uvre s'agenouille, \\
cela nous le fîmes toi et moi. \\ \

Nous donnâmes forme réelle à un fantôme\footnote{NDT. L'acception ``fantasme'' ({\em fantasma}) est possible mais n'a été popularisée qu'au XX$^e$ siècle, par le biais de la psychanalyse.}, \\
de l'esprit ridicule invention, \\
et l'idole une fois là, nous sacrifiâmes \\
sur son autel notre amour. \\
\end{verse}

%\newpage

\begin{center} {\bf 14 [XLIX]}\addcontentsline{toc}{subsection}{\em 14.\ Parfois je la rencontre de par le monde...} \end{center}

\begin{verse}
Parfois je la rencontre de par le monde \\
et elle passe près de moi; \\
et elle passe en souriant, et je dis: \\
{\em Comment peut-elle {\em rire\/}?} \\ \

Puis point à ma lèvre un autre sourire, \\
masque de la douleur, \\
et je pense alors: {\em Peut-être rit-elle \\
comme je ris moi-même!} \\
\end{verse}

\bigskip

\begin{center} {\bf 16 [XLII]}\addcontentsline{toc}{subsection}{\em 16.\ Quand on me le conta je sentis le froid...} \end{center}

\begin{verse}
Quand on me le conta je sentis le froid \\
d'une lame d'acier dans les entrailles; \\
je m'appuyai contre le mur, et un instant \\
je perdis la conscience du lieu où j'étais. \\ \

La nuit s'abattit sur mon être; \\
d'ire et de pitié s'inonda mon âme \\
et il m'apparut pourquoi on pleure, \\
et je compris pourquoi on tue! \\ \

Le nuage de douleur passa..., avec peine \\
je parvins à balbutier quelques mots... \\
Et que devais-je faire?.. C'était un ami. \\
Il m'avait rendu service!... Je le remerciai. \\
\end{verse}

\bigskip

\begin{center} {\bf 17 [LIX]}\addcontentsline{toc}{subsection}{\em 17.\ Moi je sais quel est l'objet...} \end{center}

\begin{verse}
Moi je sais quel est l'objet \\
de tes soupirs; \\
Moi je sais la cause de ta douce \\
secrète langueur. \\ \

Tu ris?... Un jour \\
tu sauras, petite, pourquoi \\
tu le sais à peine \\
et moi je le sais. \\ \

Moi je sais quand tu rêves \\
et ce qu'en songes tu vois. \\
Comme dans un livre je peux lire \\
sur ton front ce que tu tais. \\ \

Tu ris? Un jour \\
tu sauras, petite, pourquoi \\
tu le sais à peine, \\
et moi je le sais. \\ \

%\newpage

Moi je sais pourquoi tu souris \\
et pleures à la fois; \\
moi je pénètre les mystères \\
de ton âme de femme. \\ \

Tu ris?... Un jour \\
tu sauras, petite, pourquoi \\
pendant que tu éprouves tant et ne sais rien, \\
moi, qui ne ressens plus rien, je sais tout. \\
\end{verse}

\bigskip

\begin{center} {\bf 18 [LXVII]}\addcontentsline{toc}{subsection}{\em 18.\ Quelle merveille que de voir le jour...}  \end{center}

\begin{verse}
Quelle merveille que de voir le jour \\
couronné de feu se lever, \\
et à son baiser enflammé \\
briller les vagues et s'incendier l'air! \\ \

Quelle merveille, après la pluie, \\
dans le soir bleuté du triste automne, \\
que de boire le parfum \\
des fleurs humides jusqu'à satiété! \\ \

Quelle merveille, quand en flocons \\
la blanche neige silencieuse tombe, \\
que de voir les rousses langues \\
des inquiètes flammes s'agiter! \\ \

Quelle merveille, quand il y a le sommeil, \\
que de bien dormir... et ronfler tel un sous-chantre... \\
et manger... et grossir!... Et quel malheur \\
que cela seulement ne suffise! \\
\end{verse}

\bigskip

\begin{center} {\bf 19 [XXII]}\addcontentsline{toc}{subsection}{\em 19.\ Comment vit cette rose que tu as prise...} \end{center}

\begin{verse}
Comment vit cette rose que tu as prise\footnote{NDT. Il faut également lire ``allumée'' (autre sens du verbe {\em prender})} \\
contre ton c{\oe}ur? \\
Sur un volcan avant de la trouver \\
jamais je n'ai vu de fleur. \\
\end{verse}

\bigskip

\begin{center} {\bf 20 [LVI]}\addcontentsline{toc}{subsection}{\em 20.\ Ce jour comme hier, demain comme ce jour...} \end{center}

\begin{verse}
Ce jour comme hier, demain comme ce jour: \\
et toujours ainsi! \\
Un ciel gris, un horizon éternel, \\
et marcher... Marcher! \\ \

%\newpage

Se mouvant en mesure comme une bête \\
machine, le c{\oe}ur; \\
la gauche intelligence du cerveau \\
endormie dans un coin. \\ \

L'âme, qui ambitionne un paradis \\
en le cherchant sans foi; \\
fatigue sans objet, vague qui roule \\
en ignorant pourquoi. \\ \

Voix qui sans cesse du même ton \\
chante la même chanson; \\
goutte d'eau monotone qui tombe \\
et tombe sans arrêt. \\ \

Ainsi vont glissant les jours \\
les uns après les autres, \\
ce jour comme hier, probablement \\
demain comme ce jour. \\ \

Hélas! Parfois je me souviens en soupirant \\
de l'ancienne douleur!... \\
Amère est la peine; mais au moins \\
souffrir est vivre! \\
\end{verse}

\bigskip

\begin{center} {\bf 22 [XXIII]}\addcontentsline{toc}{subsection}{\em 22.\ Pour un regard, un monde...} \end{center}

\begin{verse}
Pour un regard, un monde; \\
pour un sourire, un ciel; \\
pour un baiser... j'ignore \\
que t'offrir pour un baiser! \\
\end{verse}

\bigskip

\begin{center} {\bf 23 [LXXV]}\addcontentsline{toc}{subsection}{\em 23.\ Serait-il vrai que quand le sommeil touche...} \end{center}

\begin{verse}
Serait-il vrai que quand le sommeil touche \\
de ses doigts de rose nos yeux, \\
de la prison qu'elle habite l'âme \\
s'enfuit en vol pressé? \\ \

Serait-il vrai qu'hôte des brumes, \\
au souffle ténu de la brise nocturne, \\
ailée elle monte à la région vide \\
pour en rencontrer d'autres? \\ \

%\newpage

Et là dévêtue de l'humaine forme, \\
là les liens terrestres rompus, \\
de brèves heures elle habite \\
le monde silencieux de l'idée? \\ \

Et qu'elle rit et pleure, et exècre et aime \\
et garde un visage de douleur et de joie, \\
pareil à celui qu'elle laisse quand traverse \\
le ciel un météore? \\ \

Moi je ne sais si ce monde de visions \\
vit hors ou dans nous; \\
ce que je sais c'est que je connais beaucoup de gens \\
que je ne connais pas. \\
\end{verse}

\bigskip

\begin{center} {\bf 24 [LXXIV]}\addcontentsline{toc}{subsection}{\em 24.\ Les habits défaits, les épées nues...} \end{center}

\begin{verse}
Les habits défaits, \\
les épées nues, \\
sur le linteau d'or de la porte \\
deux anges veillaient. \\ \

Je m'approchai des fers forgés \\
qui défendent l'entrée, \\
et des doubles grilles au fond \\
je la vis confuse et blanche. \\ \

Je la vis comme l'image \\
qui dans une rêverie passe, \\
comme un rai de lumière ténu et diffus \\
qui entre des ténèbres nage. \\ \

Je sentis mon âme pleine \\
d'un ardent désir; \\
comme attire un abîme, ce mystère \\
vers lui m'entraînait, \\ \

mais, hélas!, des anges \\
paraissaient me dire les regards: \\
{\em Le seuil de cette porte \\
seul Dieu le passe!} \\
\end{verse}

\bigskip

\begin{center} {\bf 26 [XLI]}\addcontentsline{toc}{subsection}{\em 26.\ Tu étais l'ouragan et moi la haute tour...} \end{center}

\begin{verse}
Tu étais l'ouragan et moi la haute \\
tour qui défie son pouvoir. \\
Tu devais te fracasser ou m'abattre!... \\
Impossible! \\ \

Tu étais l'Océan et moi la droite \\
roche qui attend son va-et-vient: \\
Tu devais te briser ou m'arracher!... \\
Impossible! \\ \

Belle, toi; moi, altier; habitués \\
l'un à emporter, l'autre à ne pas céder; \\
La sente, étroite; inévitable, le choc... \\
Impossible! \\
\end{verse}

\bigskip

\begin{center} {\bf 28 [XXXVII]}\addcontentsline{toc}{subsection}{\em 28.\ Avant toi je mourrai...} \end{center}

\begin{verse}
Avant toi je mourrai: caché \\
dans les entrailles déjà \\
je porte le fer avec lequel ta main ouvrit \\
la large blessure mortelle. \\ \

Avant toi je mourrai; et mon âme \\
dans son entêtement tenace, \\
s'assiéra aux portes de la mort, \\
en attendant que tu frappes. \\ \

Avec les heures les jours, avec les jours \\
les années voleront, \\
et à cette porte tu frapperas à la fin... \\
Qui ne frappe jamais? \\ \

Alors ta faute et tes restes \\
la terre gardera, \\
te lavant dans les ondes de la mort \\
comme dans un autre Jourdain; \\ \

là où le murmure de la vie \\
va mourir en tremblant, \\
comme la vague qui à la plage vient \\
en silence expirer; \\ \

là où le sépulcre qui se ferme \\
ouvre une éternité... \\
Tout ce que nous deux avons tu \\
nous devrons en parler! \\
\end{verse}

\bigskip

\begin{center} {\bf 30 [XXXI]}\addcontentsline{toc}{subsection}{\em 30.\ Notre passion fut une tragique saynète...} \end{center}

\begin{verse}
Notre passion fut une tragique saynète \\
dont de l'absurde fable, \\
le comique et le grave confondus, \\
jaillissent rires et pleurs. \\ \

Mais le pire de cette histoire fut \\
qu'à la fin de l'acte \\
à elle échurent larmes et rires, \\
et à moi seulement les larmes. \\
\end{verse}

\bigskip

\begin{center} {\bf 31 [XXV]}\addcontentsline{toc}{subsection}{\em 31.\ Quand dans la nuit t'enveloppent les ailes de tulle du sommeil...} \end{center}

\begin{verse}
Quand dans la nuit t'enveloppent \\
les ailes de tulle du sommeil, \\
et tes cils tendus \\
imitent des arcs d'ébène, \\ \

pour écouter les battements \\
de ton c{\oe}ur inquiet \\
et coucher ta tête endormie \\
sur ma poitrine, \\ \

je donnerais, mon amour, \\
tout ce que j'ai: \\
la lumière, l'air \\
et la pensée! \\ \

Quand se fixent tes yeux \\
sur un invisible objet, \\
et le reflet d'un sourire \\
illumine tes lèvres, \\ \

pour lire sur ton front \\
la secrète pensée \\
qui passe comme le nuage \\
sur le large miroir de la mer, \\ \

je donnerais, mon amour, \\
tout ce que je désire: \\
la renommée, l'or, \\
la gloire, le génie! \\ \

Quand se tait ta langue, \\
et se presse ton haleine, \\
et tes joues s'allument, \\
et tu entrouvres tes yeux noirs, \\ \

%\newpage

pour voir entre tes cils \\
briller d'un humide feu \\
l'ardente étincelle qui jaillit \\
du volcan des désirs, \\ \

je donnerais, mon amour, \\
tout ce en quoi j'espère: \\
la foi, l'âme, \\
la terre, le ciel! \\
\end{verse}

\bigskip

\begin{center} {\bf 32 [LVII]}\addcontentsline{toc}{subsection}{\em 32.\ Cette carcasse d'os et de peau...} \end{center}

\begin{verse}
Cette carcasse d'os et de peau, \\
à force de promener une tête folle \\
se fatigue à la fin, et je ne la regrette pas; \\
car bien qu'il soit vrai que je ne sois pas vieux, \\ \

de la part de vie qu'il me revient \\
de la vie du monde, à mes dépens, \\
j'ai fait un tel usage que je jurerais \\
avoir condensé un siècle en chaque jour. \\ \

Ainsi, si je mourais à l'instant \\
je ne pourrais dire que je n'ai vécu; \\
si la casaque paraît neuve du dehors \\
je sais que dedans elle a vieilli. \\ \

Elle a vieilli, oui; malgré mon étoile! \\
suffisamment le dit mon ardeur dolente; \\
c'est qu'il est des douleurs qui leur horrible \\
empreinte gravent au c{\oe}ur si ce n'est au front. \\
\end{verse}

\bigskip

\begin{center} {\bf 33 [XXIV]}\addcontentsline{toc}{subsection}{\em 33.\ Deux rouges langues de feu...} \end{center}

\begin{verse}
Deux rouges langues de feu \\
qui, enlacées au même tronc, \\
s'approchent et en se baisant \\
forment une seule flamme; \\ \

deux notes que du luth \\
en même temps la main fait jaillir, \\
et dans l'espace se réunissent \\
et harmonieuses s'embrassent; \\ \

deux vagues qui viennent ensemble \\
mourir sur une plage \\
et en se brisant se couronnent \\
d'un panache d'argent; \\ \

deux lambeaux de vapeur \\
qui du lac s'élèvent, \\
et en se joignant dans le ciel \\
forment un nuage blanc; \\ \

deux idées qui de pair surgissent, \\
deux baisers qui à l'unisson éclatent, \\
deux échos qui se confondent... \\
cela est nos deux âmes. \\
\end{verse}

\bigskip

\begin{center} {\bf 34 [XLIII]}\addcontentsline{toc}{subsection}{\em 34.\ J'écartai la lumière...} \end{center}

\begin{verse}
J'écartai la lumière, et au bord \\
du lit défait je m'assis, \\
muet, sombre, la pupille immobile \\
plantée dans le mur. \\ \

Combien de temps restai-je ainsi? Je ne sais; \\
quand me laissa l'horrible ivresse de douleur \\
la lumière expirait et sur mes balcons \\
riait le soleil. \\ \

Je ne sais non plus en de si terribles heures \\
à quoi je pensai ou ce qui me traversa; \\
je me souviens seulement avoir pleuré et maudit \\
et avoir en cette nuit-là vieilli. \\
\end{verse}

\bigskip

\begin{center} {\bf 35 [LII]}\addcontentsline{toc}{subsection}{\em 35.\ Lames géantes qui vous brisez en mugissant...} \end{center}

\begin{verse}
Lames géantes qui vous brisez en mugissant \\
sur les plages désertes et lointaines, \\
enveloppé dans le drap d'écumes, \\
emportez-moi avec vous! \\ \

Rafales d'ouragans qui arrachez \\
de la grande forêt les feuilles mortes, \\
entraîné dans l'aveugle toubillon, \\
emportez-moi avec vous! \\ \

Nuées de tempête que rompt l'éclair \\
et en feu allument les sanglants orles, \\
enlevé parmi l'obscur brouillard, \\
emportez-moi avec vous! \\ \

Emportez-moi par pitié, où le vertige \\
m'arrache la raison et la mémoire... \\
Par pitié!... J'ai si peur de rester \\
seul avec ma douleur! \\
\end{verse}

\bigskip

\begin{center} {\bf 36 [LIV]}\addcontentsline{toc}{subsection}{\em 36.\ Quand à nouveau les fugaces heures du passé nous évoquons...} \end{center}

\begin{verse}
Quand à nouveau les fugaces heures \\
du passé nous évoquons, \\
tremblante brille sur tes cils noirs \\
une larme prompte à glisser. \\ \

Et à la fin elle glisse, et tombe comme goutte \\
de rosée à la pensée que, \\
tel ce jour pour hier, pour ce jour demain, \\
tous deux nous soupirerons à nouveau. \\
\end{verse}

\bigskip

\begin{center} {\bf 38 [LXXI]}\addcontentsline{toc}{subsection}{\em 38.\ Elles reviendront les noires hirondelles...} \end{center}

\begin{verse}
Elles reviendront les noires hirondelles \\
accrocher leurs nids à ton balcon, \\
et une fois encore avec l'aile aux carreaux \\
elles frapperont en jouant, \\
mais celles qui réfrènaient leur vols, \\
en contemplant ta beauté et ma chance, \\
celles qui apprirent nos noms... \\
celles-ci ne reviendront pas! \\ \

Ils reviendront les épais chèvrefeuilles\footnote{NDT. En espagnol ``chèvrefeuille'' est du genre féminin ({\em madreselva}), ce qui complète parfaitement la symétrie avec la première strophe (notamment les premiers et derniers vers).} \\
escalader les murs de ton jardin, \\
et une fois encore le soir, plus belles, \\
leurs fleurs s'ouvriront; \\
mais ceux figés de rosée, \\
dont nous regardions trembler les gouttes, \\
et tomber comme larmes du jour... \\
ceux-ci ne reviendront pas! \\ \

Elles reviendront les paroles ardentes \\
d'amour sonner dans ton oreille, \\
ton c{\oe}ur de son profond sommeil \\
peut-être se réveillera. \\
Mais muet et absorbé et à genoux, \\
comme on adore Dieu devant son autel, \\
comme moi je t'ai aimée..., détrompe-toi, \\
ainsi personne ne t'aimera plus! \\
\end{verse}

\bigskip

\begin{center} {\bf 40 [XXX]}\addcontentsline{toc}{subsection}{\em 40.\ Pointait à son {\oe}il une larme...} \end{center}

\begin{verse}
Pointait à son {\oe}il une larme \\
et à ma lèvre une phrase de pardon; \\
l'orgueil parla et son pleur s'assècha, \\
et la phrase sur mes lèvres expira. \\ \

%\newpage

Et je vais mon chemin, et elle un autre; \\
mais en repensant à notre amour mutuel, \\
je dis encore: {\em Pourquoi ce jour-là n'avoir rien dit?} \\
et elle doit se dire: {\em Pourquoi n'ai-je pas pleuré?} \\
\end{verse}

\bigskip

\begin{center} {\bf 41 [LX]}\addcontentsline{toc}{subsection}{\em 41.\ Ma vie est une friche...} \end{center}

\begin{verse}
Ma vie est une friche: \\
fleur que je touche s'effeuille; \\
sur mon chemin fatal \\
on va semant le mal \\
pour que moi je le recueille. \\
\end{verse}

\bigskip

\begin{center} {\bf 44 [LXXVII]}\addcontentsline{toc}{subsection}{\em 44.\ Tu dis que tu as un c{\oe}ur...} \end{center}

\begin{verse}
Tu dis que tu as un\footnote{NDT. On peut lire aussi ``du''.} c{\oe}ur, et tu le dis \\
seulement parce que tu sens ses battements. \\
Cela n'est pas un c{\oe}ur..., c'est une machine \\
qui en suivant sa mesure fait du bruit. \\
\end{verse}

\bigskip

\begin{center} {\bf 45 [LXI]}\addcontentsline{toc}{subsection}{\em 45.\ En voyant mes heures de fièvre...} \end{center}

\begin{verse}
En voyant mes heures de fièvre \\
et d'insomnie, lentes, passer, \\
au bord de ma couche, \\
qui s'assiéra? \\ \

Quand la main tremblante \\
se tendra, prête à expirer, \\
cherchant une main amie, \\
qui la serrera? \\ \

%\newpage

Quand la mort dépolira \\
de mes yeux le cristal, \\
mes paupières encore ouvertes, \\
qui les clora? \\ \

Quand la cloche sonnera \\
(si elle sonne à mon enterrement), \\
une prière en l'entendant, \\
qui la murmurera? \\ \

Quand mes pâles restes \\
la terre opprimera enfin, \\
sur la fosse oubliée, \\
qui viendra pleurer? \\ \

Qui, au jour suivant \\
quand le soleil brillera à nouveau, \\
de mon passage dans le monde, \\
qui se souviendra? \\
\end{verse}

\bigskip

\begin{center} {\bf 46 [X]}\addcontentsline{toc}{subsection}{\em 46.\ Les invisibles atomes de l'air alentour palpitent et s'enflamment...} \end{center}

\begin{verse}
Les invisibles atomes de l'air \\
alentour palpitent et s'enflamment, \\
le ciel se défait en rayons d'or, \\
la terre frémit, réjouie. \\
J'entends, flottant en ondes d'harmonie, \\
rumeur de baisers et battement d'ailes; \\
mes paupières se closent... Qu'arrive-t-il? \\
Dis-moi...? Silence! C'est l'amour qui passe! \\
\end{verse}

\bigskip

\begin{center} {\bf 47 [LXV]}\addcontentsline{toc}{subsection}{\em 47.\ Vint la nuit et point d'asile...} \end{center}

\begin{verse}
Vint la nuit et point d'asile; \\
et j'eus soif!... Mes larmes je bus. \\
Et j'eus faim!... Mes yeux enflés \\
je clos pour mourir! \\ \

Étais-je dans un désert? Bien qu'à mon oreille \\
parvenait le rauque bouillonnement des foules, \\
moi j'étais orphelin et pauvre... Le monde était \\
désert... pour moi! \\
\end{verse}

\bigskip

\begin{center} {\bf 48 [LXXVIII]}\addcontentsline{toc}{subsection}{\em 48.\ Feignant des réalités avec ombre vaine...} \end{center}

\begin{verse}
Feignant des réalités \\
avec ombre vaine \\
devant le Désir \\
va l'Espérance; \\
et ses mensonges, \\
comme le Phénix, renaissent \\
de ses cendres. \\
\end{verse}

\bigskip

\begin{center} {\bf 49 [LXIX]}\addcontentsline{toc}{subsection}{\em 49.\ Lorsque brille l'éclair nous naissons...} \end{center}

\begin{verse}
Lorsque brille l'éclair nous naissons, \\
et son éclat dure encore quand nous mourons. \\
Si courte est la vie! \\ \

Gloire et amour après lesquels nous courons, \\
ombres d'un rêve que tous nous poursuivons. \\
S'éveiller est mourir! \\
\end{verse}

%\newpage

\begin{center} {\bf 53 [XXIX]}\addcontentsline{toc}{subsection}{\em 53.\ Sur la jupe elle tenait le livre ouvert...} \end{center}

\begin{verse}
Sur la jupe elle tenait \\
le livre ouvert, \\
ses boucles noires \\
touchaient ma joue:\\
nous ne voyions pas les lettres, \\
aucun des deux, je crois, \\
et pourtant nous gardions \\
un profond silence. \\
Combien cela dura? Ni alors \\
je ne pus le savoir, \\
je sais seulement qu'on n'entendait \\
rien d'autre que l'haleine \\
qui pressée s'échappait \\
de la lèvre sèche. \\
Je sais seulement que nous nous tournâmes \\
les deux en même temps, \\
et nos yeux se trouvèrent, \\
et sonna un baiser. \\ \

\ldots\ldots\ldots\ldots\ldots\ldots\ldots\ldots\ldots\ldots\ldots\ldots\ldots\ldots\ldots\ldots\ldots\ldots\ldots\ldots \\ \

Création de Dante était le Livre, \\
était son {\em Enfer}. \\
Quand nous y baissâmes les yeux, \\
je dis, tremblant: \\
{\em Comprends-tu maintenant qu'un poème \\
tient dans un vers?} \\
Et elle répondit, enflammée: \\
{\em Je le comprends maintenant!} \\
\end{verse}

\bigskip

\begin{center} {\bf 54 [XXXVI]}\addcontentsline{toc}{subsection}{\em 54.\ Si de nos turpitudes on écrivait l'histoire dans un livre...} \end{center}

\begin{verse}
Si de nos turpitudes on écrivait \\
l'histoire dans un livre, \\
et si s'effa\c{c}ait de nos âmes autant \\
que s'effacerait de ses pages... \\
Je t'aime encore, ton amour laissa \\
sur ma poitrine des traces si profondes \\
que si tu en effa\c{c}ais seulement une, \\
je les effacerais toutes! \\
\end{verse}

\bigskip

\begin{center} {\bf 55 [LXXIX]}\addcontentsline{toc}{subsection}{\em 55.\ Une femme m'a empoisonné l'âme...} \end{center}

\begin{verse}
Une femme m'a empoisonné l'âme, \\
une autre m'a empoisonné le corps; \\
aucune des deux ne vint me chercher, \\
moi, d'aucune des deux je ne me plains. \\ \

Comme le monde est rond, le monde tourne. \\
Si demain, tournant, ce poison \\
empoisonne à son tour, pourquoi m'accuser? \\
Puis-je donner plus que ce que l'on me donna?\footnote{NDT. Cette stance 55 apparait barrée dans le manuscrit original.} \\
\end{verse}

\bigskip

\begin{center} {\bf 56 [LXII]}\addcontentsline{toc}{subsection}{\em 56.\ D'abord est une aube tremblante...} \end{center}

\begin{verse}
D'abord est une aube tremblante et vague, \\
rai d'inquiète lueur qui coupe la mer; \\
puis elle étincelle et croît et se diffuse \\
en une gigantesque explosion de clarté. \\ \

La brillante flamme est la joie, \\
la craintive ombre est la peine; \\
Hélas! dans l'obscure nuit de mon âme \\
quand poindra le jour? \\
\end{verse}

\bigskip

\begin{center} {\bf 58 [XXVIII]}\addcontentsline{toc}{subsection}{\em 58.\ Quand parmi l'ombre obscure, une voix perdue murmure...} \end{center}

\begin{verse}
Quand parmi l'ombre obscure \\
une voix perdue murmure, \\
troublant sa triste paix, \\
si au fond de mon âme \\
je l'entends résonner, \\ \

dis-moi: est-ce que le vent virevoltant \\
se plaint, ou que tes soupirs \\
me parlent d'amour en passant? \\ \

Quand le soleil à ma fenêtre \\
brille rouge au matin, \\
et mon amour évoque ton ombre, \\
si sur ma bouche une autre bouche \\
je crois bien sentir, \\ \

dis-moi: est-ce qu'aveugle je délire, \\
ou qu'un baiser dans un soupir \\
m'envoie ton c{\oe}ur? \\ \

Et dans le lumineux jour \\
et la pleine nuit noire, \\
si dans tout ce qui entoure \\
mon âme qui te désire \\
je crois te sentir et voir, \\ \

dis-moi: est-ce que je touche et respire \\
en rêve, ou que dans un soupir \\
tu me donnes ton haleine à boire? \\
\end{verse}

\bigskip

\begin{center} {\bf 59 [LXX]}\addcontentsline{toc}{subsection}{\em 59.\ Combien de fois, au pied des murs moussus qui la gardent...} \end{center}

\begin{verse}
Combien de fois, au pied des murs \\
moussus qui la gardent, \\
n'ai-je entendu la clochette à minuit \\
sonner aux matines! \\ \

Combien de fois tra\c{c}a la lune argentée \\
ma silhouette, \\
contre celle du cyprès qui de son verger \\
point sur les murailles! \\ \

Quand d'ombres se drapait l'église \\
à l'ogive en coiffe enfoncée, \\
combien de fois sur les vitraux \\
n'ai-je vu trembler l'éclat de la lampe! \\ \

Bien que le vent dans les angles obscurs \\
de la tour sifflât, \\
parmi les voix du ch{\oe}ur je percevais \\
sa voix vibrante et claire. \\ \

Dans les nuits d'hiver, si un poltron \\
la place déserte \\
osait traverser, quand il m'apercevait \\
il hâtait son pas. \\ \

Et il ne manqua pas une vieille qui ne racontât \\
au matin suivant \\
que de quelque sacristain mort en pécheur \\
j'étais l'âme. \\ \

À l'aveuglette je connaissais les recoins \\
du parvis et le portail; \\
de mes pieds les orties qui là-bas poussent \\
peut-être gardent les empreintes. \\ \

Les hiboux qui effrayés me suivaient \\
avec leurs yeux de flammes \\
finirent par me considérer avec le temps \\
comme un bon camarade. \\ \

À mon côté, sans peur, les reptiles \\
avan\c{c}aient en se traînant. \\
Jusqu'aux saints de granit muets \\
je crois me saluaient! \\
\end{verse}

\bigskip

\begin{center} {\bf 61 [LXVIII]}\addcontentsline{toc}{subsection}{\em 61.\ Je ne sais ce que j'ai rêvé la nuit dernière...} \end{center}

\begin{verse}
Je ne sais ce que j'ai rêvé \\
la nuit dernière. \\
Triste, très triste dû être le rêve, \\
car éveillé l'angoisse me durait. \\ \

%\newpage

Je notai en m'incorporant \\
l'humidité de l'oreiller, \\
et pour la première fois je sentis \\
d'un amer plaisir s'emplir l'âme. \\ \

Triste chose que le rêve \\
qui nous tire des larmes; \\
mais dans ma peine j'ai une joie... \\
Je sais qu'il me reste encore des pleurs! \\
\end{verse}

\bigskip

\begin{center} {\bf 63 [XXVII]}\addcontentsline{toc}{subsection}{\em 63.\ Éveillée je tremble à ta vue...} \end{center}

\begin{verse}
Éveillée je tremble à ta vue; \\
Assoupie, j'ose te regarder; \\
c'est pour cela, âme de mon âme, \\
que je veille pendant que tu dors. \\ \

Éveillée tu ris et en riant tes lèvres \\
inquiètes me semblent \\
éclairs écarlates qui serpentent \\
sur un ciel de neige. \\ \

Assoupie, un léger sourire plie \\
les bords de ta bouche, \\
suave comme la trace brillante \\
que laisse un soleil qui meurt... \\
Dors! \\ \

Éveillée tu regardes et en regardant tes yeux \\
humides resplendissent \\
comme l'onde bleue dont frappe la crête \\
le soleil étincelant. \\ \

À travers tes paupières, assoupie, \\
tu verses un calme éclat, \\
comme répand une lueur tiède \\
une lampe transparente... \\
Dors! \\ \

Éveillée tu parles, et en parlant, vibrantes, \\
tes paroles semblent \\
pluie de perles qui en coupe dorée \\
se déverse à torrents. \\ \

Assoupie, dans le murmure de ton haleine \\
rythmée et ténue \\
j'écoute un poème que mon âme \\
amoureuse comprend... \\
Dors! \\ \

Sur le c{\oe}ur la main \\
j'ai mis pour que ne sonne pas \\
son battement, et trouble \\
de la nuit le calme solennel. \\ \

De ton balcon les persiennes \\
j'ai fermé, pour que n'entre pas \\
le flamboiement fâcheux \\
de l'aurore et t'éveille... \\
Dors! \\
\end{verse}

\bigskip

\begin{center} {\bf 64 [LXIV]}\addcontentsline{toc}{subsection}{\em 64.\ Comme garde l'avare son trésor, je gardais ma douleur...} \end{center}

\begin{verse}
Comme garde l'avare son trésor, \\
je gardais ma douleur; \\
je voulais prouver que l'éternel existe \\
à celle qui me jura éternel son amour. \\ \

Mais aujourd'hui en vain je l'appelle et le Temps \\
qui l'acheva, me dit: \\
{\em Ah, boue misérable, même éternellement \\
tu ne saurais souffrir!} \\
\end{verse}

\bigskip

\begin{center} {\bf 65 [XXXIV]}\addcontentsline{toc}{subsection}{\em 65.\ Elle traverse muette, et ses mouvements...} \end{center}

\begin{verse}
Elle traverse muette, et ses mouvements \\
sont silencieuse harmonie; \\
ses pas sonnent, et en sonnant ils rappellent \\
de l'hymne ailé la cadence rythmée. \\ \

Elle entrouvre les yeux, ces yeux \\
clairs comme le jour; \\
et la terre et le ciel, ce qu'ils embrassent, \\
flamboient d'un nouvel éclat dans ses pupilles. \\ \

Elle rie, et ses éclats de rire ont des notes \\
de l'eau fugitive; \\
elle pleure, et chaque larme est un poème \\
de tendresse infinie. \\ \

Elle possède et la lumière et le parfum, \\
et la couleur et la ligne, \\
la forme, génératrice de désirs, \\
l'expression, source éternelle de poésie. \\ \

Qu'elle est stupide? Bah! Tant que se taisant \\
elle tient secrète l'énigme, \\
toujours vaudra ce que je crois qu'elle tait \\
plus que ce qu'aucune autre me dise. \\
\end{verse}

%\newpage

\begin{center} {\bf 66 [XL]}\addcontentsline{toc}{subsection}{\em 66.\ Sa main dans mes mains, ses yeux dans mes yeux...} \end{center}

\begin{verse}
Sa main dans mes mains, \\
ses yeux dans mes yeux, \\
l'amoureuse tête \\
appuyée sur mon épaule, \\
Dieu sait combien de fois, \\
d'un pas paresseux, \\
nous avons erré ensemble \\
sous les grands ormes \\
qui prêtent mystère et ombre \\
au porche de sa maison! \\
Et hier..., un an à peine \\
passé en coup de vent, \\
avec quelle exquise grâce, \\
avec quel admirable aplomb, \\
elle me dit, nous présentant  \\
un ami officieux: \\
{\em « Je crois qu'en quelqu'endroit \\
je vous ai vu. ~»} Ah! sots \\
qui êtes des salons \\
commères de bon ton \\
et marchiez là en chasse \\
de galants imbroglios: \\
quelle histoire vous avez manquée! \\
Quelle ambroisie \\
pour être dévorée \\
{\em sotto voce} en un cercle, \\
derrière l'éventail \\
de plumes et d'or! \\ \

\ldots\ldots\ldots\ldots\ldots\ldots\ldots\ldots\ldots\ldots\ldots\ldots\ldots\ldots\ldots\ldots\ldots\ldots\ldots\ldots \\ \

Discrète et chaste lune, \\
touffus et grands ormes, \\
murs de sa demeure, \\
seuils de son porche, \\
taisez-vous, et que le secret \\
ne sorte pas de vous! \\
Taisez-vous, de mon côté \\
j'ai tout oublié; \\
et elle..., elle, \\
il n'y a pas de masque \\
qui égale son visage! \\
\end{verse}

%\newpage

\begin{center} {\bf 67 [LXVI]}\addcontentsline{toc}{subsection}{\em 67.\ D'où je viens? Cherche le plus horrible et âpre des sentiers...} \end{center}

\begin{verse}
D'où je viens? Cherche le plus \\
horrible et âpre des sentiers. \\
Des empreintes de pieds ensanglantés \\
sur la roche dure, \\
les restes d'une âme en lambeaux \\
dans les ronces pointues \\
te diront le chemin \\
qui conduit à mon berceau. \\ \

Où je vais? Traverse la plus \\
sombre et triste des banquises\footnote{NDT. Exactement: {\em p\'{a}ramo}, qui désigne une vaste étendue déserte et froide.}, \\
vallée d'éternelles neiges et d'éternelles \\
mélancoliques brumes. \\
Où se trouve une pierre solitaire \\
sans inscription, \\
où habite l'oubli, \\
là sera ma tombe. \\
\end{verse}

\bigskip

\begin{center} {\bf 68 [LXIII]}\addcontentsline{toc}{subsection}{\em 68.\ Comme des essaims d'abeilles irritées...} \end{center}

\begin{verse}
Comme des essaims d'abeilles irritées, \\
d'un coin sombre de la mémoire \\
sortent pour me poursuivre les souvenirs \\
des heures passées. \\ \

Je veux les chasser. Effort inutile! \\
Ils m'encerclent, me harcèlent, \\
et l'un après l'autre ils viennent planter \\
le fin aiguillon qui envenime l'âme. \\
\end{verse}

\bigskip

\begin{center} {\bf 69 [XXXIII]}\addcontentsline{toc}{subsection}{\em 69.\ C'est une question de mots...} \end{center}

\begin{verse}
C'est une question de mots, et pourtant \\
ni toi ni moi jamais, \\
après ce qui advint ne conviendra \\
à qui la faute incombe. \\ \

Quel dommage que l'amour n'ait pas \\
de dictionnaire à trouver \\
quand l'orgueil est simplement orgueil \\
et quand il est dignité! \\
\end{verse}

%\newpage

\begin{center} {\bf 70 [LI]}\addcontentsline{toc}{subsection}{\em 70.\ Du peu de vie qu'il me reste...} \end{center}

\begin{verse}
Du peu de vie qu'il me reste \\
je donnerais volontiers les meilleures années, \\
pour savoir ce qu'à d'autres \\
de moi tu as conté. \\ \

Et cette vie mortelle et de l'éternelle \\
ce qu'il me reviendra, s'il m'en revient, \\
pour savoir ce que, seule, \\
de moi tu as pensé. \\
\end{verse}

\bigskip

\begin{center} {\bf 71 [LXXIII]}\addcontentsline{toc}{subsection}{\em 71.\ On clôt ses yeux qu'elle avait encore ouverts...} \end{center}

\begin{verse}
On clôt ses yeux \\
qu'elle avait encore ouverts, \\
on couvrit son visage \\
d'une blanche étoffe, \\
et d'aucuns sanglotant, \\
et d'autres en silence, \\
de la triste alcôve \\
tous sortirent. \\ \

La lumière, qui flamboyait \\
dans un vase sur le sol, \\
au mur projetait \\
l'ombre de la couche, \\
et parmi cette ombre \\
on voyait, par intervalles, \\
se dessiner, rigide, \\
la forme du corps. \\ \

Le jour s'éveillait, \\
et à sa première lueur, \\
avec ses mille bruits, \\
il réveillait la ville; \\
devant ce contraste \\
de vie et mystères \\
de lumière et ténèbres, \\
je pensai un moment: \\ \

{\em Mon Dieu, oh combien \\
seuls restent les morts!} \\ \

%\newpage

De la maison sur des épaules \\
on la porta au temple, \\
et dans une chapelle \\
on laissa le cercueil. \\
Là-bas on entoura \\
sa pâle dépouille \\
de jaunes cierges \\
et d'étoffes noires. \\ \

En sonnant des Âmes\footnote{NDT. Sonnerie à certaines heures de la nuit pour que les fidèles prient pour les âmes du Purgatoire.} \\
la dernière cloche, \\
une vieille acheva \\
ses ultimes prières; \\
elle traversa la large nef, \\
les portes gémirent, \\
et le saint lieu \\
resta désert. \\ \

D'une horloge on entendait, \\
mesuré, le balancier \\
et de certains cierges \\
le crépitement. \\
Si craintif et triste, \\
si obscur et transi \\
tout était... \\
que je pensai un moment: \\ \

{\em Mon Dieu, oh combien \\
seuls restent les morts!} \\ \

De la haute cloche \\
la langue de fer \\
lui dédia, à toute volée, \\
son ``adieu!'' plaintif. \\
Le deuil aux habits, \\
amis et proches \\
passèrent en file \\
formant cortège. \\ \

De l'ultime asile, \\
obscur et étroit, \\
le pic ouvrit la niche \\
à une extrémité. \\
Là on la coucha, \\
et puis la mura , \\
et avec un salut \\
se retira le cortège. \\ \

%\newpage

Le pic sur l'épaule, \\
le fossoyeur \\
chantonnant dans sa barbe \\
se perdit au loin. \\
La nuit s'avan\c{c}ait, \\
le soleil s'était couché; \\
perdu dans les ombres, \\
je pensai un moment: \\ \

{\em Mon Dieu, oh combien \\
seuls restent les morts!} \\ \

Dans les longues nuits \\
de l'hiver gelé \\
quand le vent \\
fait craquer les bois \\
et la forte averse \\
fouette les carreaux, \\
de la pauvre enfant \\
parfois je me souviens. \\ \

Là-bas tombe la pluie \\
d'un bruit éternel; \\
là-bas la combat \\
le souffle de la bise. \\
Étendue dans le creux \\
de l'humide mur, \\
peut-être de froid \\
se gèlent ses os!... \\ \

\ldots\ldots\ldots\ldots\ldots\ldots\ldots\ldots\ldots\ldots\ldots\ldots\ldots\ldots\ldots\ldots\ldots\ldots\ldots\ldots \\ \

La poussière retourne-t-elle à la poussière? \\
L'âme s'envole-t-elle au ciel? \\
Tout est-il, sans âme, \\
pauvreté et bourbe? \\
Je ne sais; mais il y a \\
quelque chose que je n'explique pas, \\
quelque chose qui répugne, \\
bien qu'il soit courageux le faire, \\
à laisser si tristes, \\
si seuls, les morts! \\
\end{verse}

\bigskip

\begin{center} {\bf 73 [XXXII]}\addcontentsline{toc}{subsection}{\em 73.\ Elle passait, irrésistible dans sa splendeur...} \end{center}

\begin{verse}
Elle passait, irrésistible dans sa splendeur, \\
et le pas je lui cédai; \\
je poursuivis sans me retourner, et pourtant \\
quelque chose à mon ouïe murmura {\em « C'est elle. ~»} \\ \

Qui unit le soir au matin? \\
Je l'ignore; je sais seulement \\
que par une brève nuit d'été \\
s'unirent les crépuscules et... {\em elle fut}. \\
\end{verse}

\bigskip

\begin{center} {\bf 74 [LXXVI]}\addcontentsline{toc}{subsection}{\em 74.\ Dans l'imposante nef du temple byzantin...} \end{center}

\begin{verse}
Dans l'imposante nef \\
du temple byzantin, \\
je vis la tombe gothique à l'indécise \\
lueur qui tremblait sur les vitraux. \\ \

Les mains sur la poitrine, \\
et dans les mains un livre, \\
une belle femme reposait \\
sur l'urne, prodige du ciseau. \\ \

Au doux poids enfoncé \\
du corps abandonné, \\
comme de tendre plume et lisse, \\
se pliait sa couche de granit. \\ \

Le divin éclat \\
de l'ultime sourire \\
le visage gardait, comme le ciel garde \\
du soleil qui meurt le rai fugitif. \\ \

Assis sur le bord \\
de l'oreiller de pierre, \\
deux anges, le doigt sur la lèvre, \\
imposaient silence à l'enceinte. \\ \

Elle ne semblait pas morte; \\
on l'aurait dit dormant \\
dans la pénombre des arcs massifs \\
et en songe voyant le paradis. \\ \

Je m'approchai \\
de l'angle sombre de la nef, \\
avec le pas retenu de qui vient \\
au berceau d'un enfant assoupi. \\ \

Je la contemplai un moment. \\
Et cet éclat tiède, \\
ce lit de pierre qui offrait, \\
proche du mur, un autre lieu vide, \\ \

dans l'âme avivèrent \\
la soif de l'infini, \\
le désir de cette vie de la mort, \\
pour laquelle un instant sont les siècles... \\ \

\ldots\ldots\ldots\ldots\ldots\ldots\ldots\ldots\ldots\ldots\ldots\ldots\ldots\ldots\ldots\ldots\ldots\ldots\ldots\ldots \\ \

Fatigué du combat \\
dans lequel je lutte, \\
parfois je me souviens avec envie \\
de ce recoin obscur et caché. \\ \

De cette silencieuse et pâle \\
femme je me souviens et dis: \\
{\em « Oh, quel amour si muet, celui de la mort! \\
Quel sommeil, celui du sépulcre si calme!} \\
\end{verse}

\bigskip

\begin{center} {\bf 75 [XXXIV]}\addcontentsline{toc}{subsection}{\em 75.\ Pourquoi me le dire? Je sais...} \end{center}

\begin{verse}
Pourquoi me le dire? Je sais; changeante, \\
altière et vaine et capricieuse elle est; \\
avant le sentiment de son âme \\
jaillira l'eau de la roche stérile. \\ \

Je sais que son c{\oe}ur, nid de serpents, \\
n'a pas une fibre qui à l'amour réponde; \\
qu'elle est une statue inanimée... mais... \\
Elle est si belle! \\
\end{verse}

\bigskip

\begin{center} {\bf 76 [LXXI]}\addcontentsline{toc}{subsection}{\em 76.\ Je ne dormais pas; errant dans la limbe...}  \end{center}

\begin{verse}
Je ne dormais pas; errant dans la limbe \\
où changent de forme les objets, \\
mystérieux espaces qui séparent \\
la veille du sommeil. \\ \

Les idées qui en ronde silencieuse \\
tournaient autour de mon cerveau \\
peu à peu en leur danse bougeaient \\
d'un rythme plus lent. \\ \

De la lumière qui atteint l'âme par les yeux \\
les paupièrent voilaient le reflet; \\
mais une autre lumière le monde de visions \\
allumait à l'intérieur. \\ \

À ce moment résonna à mon ouïe \\
une rumeur comme celle qui au temple \\
erre confuse quand terminent les fidèles \\
d'un {\em Amen\/} leurs prières. \\ \

Et j'entendis comme une voix fine et triste \\
qui par mon nom m'appelait de loin, \\
et je sentis une odeur de cierges éteints, \\
d'humidité et d'encens. \\ \

\ldots\ldots\ldots\ldots\ldots\ldots\ldots\ldots\ldots\ldots\ldots\ldots\ldots\ldots\ldots\ldots\ldots\ldots\ldots\ldots \\ \

La nuit passa, dans les bras de l'oubli \\
je tombai comme pierre en son sein profond; \\
mais, en m'éveillant, je m'exclamai: {\em « Quelqu'un \\
que j'aimais est mort! ~»}. \\
\end{verse}

\bigskip

\begin{center} {\bf 77 [XLVI]}\addcontentsline{toc}{subsection}{\em 77.\ Elle m'a blessé en fuyant dans l'ombre...} \end{center}

\begin{verse}

Elle m'a blessé en fuyant dans l'ombre, \\
scellant d'un baiser sa trahison. \\
Elle se pendit à mon cou, et dans le dos \\
me brisa de sang froid le c{\oe}ur. \\ \

Et elle poursuit joyeuse son chemin, \\
heureuse, gaie, impavide; et pourquoi? \\
Parce que ne saigne pas la blessure... \\
Parce que le mort est debout! \\
\end{verse}

\bigskip

\begin{center} {\bf 78 [XXXV]}\addcontentsline{toc}{subsection}{\em 78.\ Ton oubli ne m'admira pas!...} \end{center}

\begin{verse}
Ton oubli ne m'admira pas! Bien que d'un jour \\
ta tendresse m'admira bien plus; \\
car ce qui en moi a de la valeur, \\
cela... tu ne le soup\c{c}onnas même pas. \\
\end{verse}

%\newpage


\cleardoublepage

\tableofcontents

\end{document}
