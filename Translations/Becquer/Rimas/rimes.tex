%----------------------------------------------------------------------------

%\selectlanguage{francais}

\begin{center}
  \textbf{1 [XLVIII]}
  \addcontentsline{toc}{section}{\emph{1.\ Comme s'arrache le fer d'une plaie}}
\end{center}

\begin{verse}
  Comme s'arrache le fer d'une plaie,\footnote{Le thème de la blessure
  par arme blanche est récurrent chez Bécquer. Voir les rimes~16, 28
  et~77.} \\
  j'arrachai son amour de mes entrailles, \\
  bien que je sentis ce faisant \\
  que je m'arrachais la vie avec lui.

  De l'autel que je lui dressai dans mon âme, \\
  la volonté abattit son image, \\
  et la lumière de la foi qui en elle brûlait \\
  s'éteignit devant l'autel désert.

  Son image tenace revient encore à mon esprit \\
  pour combattre ma détermination... \\
  Quand pourrai-je dormir de ce sommeil \\
  où s'achève le rêve?
\end{verse}

%% \poemtitle*{Original}
%% \selectlanguage{spanish}

%% \begin{verse}[\versewidth]
%%   \it
%%   Como se arranca el hierro de una herida \\
%%   su amor de las entrañas me arranqué, \\
%%   aunque sentí al hacerlo que la vida \\
%%   ¡me arrancaba con él!

%%   Del altar que le alcé en el alma mía \\
%%   la volontad su imagen arrojó, \\
%%   y la luz de la fe que en ella ardía \\
%%   ante el ara desierta se apagó.

%%   Aun para combatir mi empeño \\
%%   viene a mi mente su visión tenaz... \\
%%   ¡Cuándo podré dormir cone ese sueño \\
%%   en que acaba el soñar!
%% \end{verse}

\bigskip

%----------------------------------------------------------------------------

%\selectlanguage{francais}

\begin{center}
  \textbf{2 [XLVII]}
  \addcontentsline{toc}{section}{\emph{2.\ Je me suis penché sur
    les gouffres béants}}
\end{center}

\begin{verse}
  Je me suis penché sur les gouffres béants \\
  de la terre et du ciel, \\
  et j'en ai vu la fin avec les yeux \\
  ou la pensée.

  Mais hélas! je parvins à l'abîme d'un cœur \\
  et je m'inclinai un moment, \\
  et mon âme et mes yeux se troublèrent, \\
  si profond et si noir il était!
\end{verse}

%% \poemtitle*{Original}
%% \selectlanguage{spanish}

%% \begin{verse}[\versewidth]
%%   \it
%%   Yo me he asomado a las profundas simas \\
%%   de la tierra y del cielo, \\
%%   y les he visto el fin, o con los ojos, \\
%%   o con el pensamiento.

%%   Mas, ¡ay! de un corazón llegué al abismo \\
%%   y me incliné un momento, \\
%%   y mi alma y mis ojos se turbaron: \\
%%   ¡Tan hondo era y tan negro!
%% \end{verse}

\pagebreak

%----------------------------------------------------------------------------

%\selectlanguage{francais}

\begin{center}
  \textbf{3 [XLV]}
  \addcontentsline{toc}{section}{\emph{3.\ À la clef d'un arc mal assuré}}
\end{center}

\begin{verse}
  À la clef d'un arc mal assuré, \\
  aux pierres rougies par le temps, \\
  campait le blason gothique, \\
  œuvre d'un rude ciseau.

  Panache de son heaume de granit, \\
  le lierre qui pendait autour \\
  ombrait l'écu où une main \\
  tenait un cœur.

  Pour le contempler en ce lieu désert, \\
  nous nous arrêtâmes tous deux: \\
  et cela, me dit-elle, est le parfait emblème \\
  de mon amour constant.

  Hélas! Ce qu'elle me dit alors était vrai: \\
  vrai que le cœur, \\
  elle l'aurait sur la main, partout... \\
  mais dans la poitrine, non.
\end{verse}

%% \poemtitle*{Original}
%% \selectlanguage{spanish}

%% \begin{verse}[\versewidth]
%%   \it
%%   En la clave del arco mal seguro \\
%%   cuyas piedras el tiempo enrojeció, \\
%%   obra de cincel rudo campeaba \\
%%   el gótico blasón.

%%   Penacho de su yelmo de granito, \\
%%   la yedra que colgaba en derredor \\
%%   daba sombra al escudo en que una mano \\
%%   tenía un corazón.

%%   A contemplarle en la desierta plaza \\
%%   nos paramos los dos: \\
%%   Y, ése, me dijo, es el cabal emblema \\
%%   de mi constante amor.

%%   ¡Ay! es verdad lo que me dijo entonces: \\
%%   verdad que el corazón \\
%%   lo llevará en la mano... en cualquier parte... \\
%%   pero en el pecho no.
%% \end{verse}

\bigskip

%----------------------------------------------------------------------------

%\selectlanguage{francais}

\begin{center}
  \textbf{4 [XXXVIII]}
  \addcontentsline{toc}{section}{\emph{4.\ Les soupirs sont air, et
    à l'air ils vont}}
\end{center}

\begin{verse}
Les soupirs sont air, et à l'air ils vont. \\
Les larmes sont eau, et à la mer elles vont. \\
Dis, ma demoiselle: quand l'amour s'oublie,\\
sais-tu où il va?
\end{verse}

\bigskip\bigskip\bigskip\bigskip\bigskip

%% \poemtitle*{Original}
%% \selectlanguage{spanish}

%% \begin{verse}[\versewidth]
%%   \it
%%   Los suspiros son aire y van al aire. \\
%%   Las lágrimas son agua y van al mar. \\
%%   Dime, mujer, cuando el amor se olvida \\
%%   ¿sabes tú adónde va?
%% \end{verse}

\pagebreak

%----------------------------------------------------------------------------

%\selectlanguage{francais}

\begin{center}
  \textbf{5 [LXXII]}
  \addcontentsline{toc}{section}{\emph{5.\ Les ondes ont une vague
    harmonie}}
\end{center}

\begin{center} \emph{Première voix} \end{center}

\begin{verse}
  Les ondes ont une vague harmonie; \\
  les violettes, une suave odeur; \\
  les brumes d'argent, la froide nuit; \\
  la lumière et l'or, le jour; \\
  moi, quelque chose de meilleur: \\
  moi, j'ai l'\emph{Amour}!
\end{verse}

\begin{center} \emph{Deuxième voix} \end{center}

\begin{verse}
  Brise\footnote{\emph{aura}, dans la poésie espagnole du XIX\ieme{}
  siècle, désignait un vent doux. Nous traduisons parfois par
  \emph{zéphyr}. Voir rimes~27 et~60.} de liesse, nuée radieuse, \\
  vague d'envie qui baise le pied, \\
  île de songes où repose \\
  l'âme inassouvie. \\
  Douce ivresse, \\
  c'est la \emph{Gloire}.
\end{verse}

\begin{center} \emph{Troisième voix} \end{center}

\begin{verse}
  Braise allumée est le trésor, \\
  ombre qui fuit la vanité. \\
  Tout est mensonge: la gloire, l'or; \\
  seul ce que moi j'adore \\
  est vrai: \\
  la \emph{Liberté}!
\end{verse}

\pagebreak

\begin{verse}
  Ainsi passaient les bateliers en chantant \\
  l'éternelle chanson, \\
  et l'écume sautait aux coups de rame, \\
  blessée par le soleil.

  \emph{T'embarques-tu?}, me criaient-ils. \\
  Et moi, souriant, je leur dis au passage: \\
  «\,J'ai déjà embarqué.\,», et je leur pointai \\
  mes habits étendus qui séchaient sur la plage.
\end{verse}

%% \poemtitle*{Original}
%% \selectlanguage{spanish}

%% \begin{center} \emph{Primera voz} \end{center}

%% \begin{verse}[\versewidth]
%%   \it
%%   Las ondas tienen vaga armonía, \\
%%   las violetas suave olor, \\
%%   brumas de plata la noche fría, \\
%%   luz y oro el día, \\
%%   yo algo mejor: \\
%%   ¡yo tengo \emph{Amor}!
%% \end{verse}

%% \begin{center} \emph{Segunda voz} \end{center}

%% \begin{verse}[\versewidth]
%%   \it
%%   Aura de aplausos, nube radiosa, \\
%%   ola de envidia que besa el pie, \\
%%   isla de sueños donde reposa \\
%%   el ama ansiosa, \\
%%   ¡dulce embriaguez \\
%%   la \emph{Gloria} es!
%% \end{verse}

%% \begin{center} \emph{Tercera voz} \end{center}

%% \begin{verse}[\versewidth]
%%   \it
%%   Ascua encendida es el tesoro, \\
%%   sombra que huye la vanidad. \\
%%   Todo es mentira: la gloria, el oro, \\
%%   lo que yo adoro \\
%%   sólo es verdad, \\
%%   ¡la \emph{Libertad}!
%% \end{verse}

%% \begin{verse}[\versewidth]
%%   \it
%%   Así los barqueros pasaban cantando \\
%%   la eterna canción \\
%%   y al golpe del remo saltaba la espuma \\
%%   y heríala el sol.

%%   \emph{¿Te embarcas?}, gritaban. Y yo sonriendo \\
%%   les dije al pasar: \\
%%   Yo ya me he embarcado; por señas que aún tengo \\
%%   la ropa en la playa tendida a secar.
%% \end{verse}

\bigskip

%----------------------------------------------------------------------------

%\selectlanguage{francais}

\begin{center}
  \textbf{6 [XVIII]}
  \addcontentsline{toc}{section}{\emph{6.\ Fatiguée par la danse}}
\end{center}

\begin{verse}
  Fatiguée par la danse, \\
  la couleur ardente, le souffle court, \\
  appuyée à mon bras, \\
  elle s'arrêta à un bout du salon.

  Parmi la gaze légère \\
  que soulevait son sein palpitant, \\
  une fleur était bercée \\
  d'un mouvement doux et mesuré.\footnote{Le motif de la fleur au
  décolleté se retrouve à la rime~19.}

  Comme dans un berceau de nacre \\
  que pousse la mer et caresse le zéphyr, \\
  peut-être dormait-elle là-bas du souffle \\
  de ses lèvres entrouvertes.

  Oh! Qui, pensai-je, pourrait ainsi \\
  laisser filer le temps! \\
  Oh! Si les fleurs dorment, \\
  quel sommeil si doux!
\end{verse}

%% \poemtitle*{Original}
%% \selectlanguage{spanish}

%% \begin{verse}[\versewidth]
%%   \it
%%   Fatigada del baile, \\
%%   encendido el color, breve el aliento, \\
%%   apoyada en mi brazo, \\
%%   del salón se detuvo en un extremo.

%%   Entre la leve gasa \\
%%   que levantaba el palpitante seno, \\
%%   una flor se mecía \\
%%   en compasado y dulce movimiento.

%%   Como en una cuna de nácar \\
%%   que empuja el mar y que acaricia el céfiro \\
%%   tal vez allí dormía \\
%%   al soplo de sus labios entreabiertos.

%%   ¡Oh!, ¡Quién así, pensaba, \\
%%   dejar pudiera deslizarse el tiempo! \\
%%   ¡Oh! si las flores duermen, \\
%%   ¡qué dulcísimo sueño!
%% \end{verse}

\bigskip

%----------------------------------------------------------------------------

%\selectlanguage{francais}

\begin{center}
  \textbf{7 [XXVI]}
  \addcontentsline{toc}{section}{\emph{7.\ Je vais contre mes intérêts en le confessant}}
\end{center}

\begin{verse}
  Je vais contre mes intérêts en le confessant. \\
  Néanmoins, mon aimée, \\
  je pense comme toi qu'une ode est seule bonne \\
  écrite au dos d'un chèque.

  Il ne manquera pas quelque sot qui, \\
  en l'entendant, ne se signe et dise: \\
  \emph{Femme de la fin du dix-neuvième siècle, \\
    matérielle et prosaïque.} \\
  Sottises! \\
  Des voix qui font courir quatre poètes \\
  qui se drapent en hiver avec une lyre! \\
  Aboiements des chiens à la lune!

  Tu sais et je sais qu'en cette vie, \\
  celui qui \emph{l'écrit} avec génie est très rare, \\
  et qu'avec de l'or, quiconque \emph{fait} de la poésie.
\end{verse}

\bigskip

\begin{center}
  \textbf{8 [LVIII]}
  \addcontentsline{toc}{section}{\emph{8.\ Veux-tu éviter l'amertume de la lie de ce nectar délicieux?}}
\end{center}

\begin{verse}
  Veux-tu éviter l'amertume de la lie \\
  de ce nectar délicieux? \\
  Alors hume-le, approche-le de tes lèvres \\
  et écarte-le ensuite.

  Veux-tu que nous gardions un doux \\
  souvenir de cet amour? \\
  Alors aimons-nous aujourd'hui, et demain \\
  disons-nous adieu!\footnote{Le thème de l'amour d'un soir se
  retrouve aux rimes~9 et~73, qui offrent un contraste à l'image
  d'Épinal d'un poète transi d'amour.}
\end{verse}

\pagebreak

\begin{center}
  \textbf{9 [LV]}
  \addcontentsline{toc}{section}{\emph{9.\ Dans le tumulte
    discordant de l'orgie}}
\end{center}

\begin{verse}
  Dans le tumulte discordant de l'orgie, \\
  l'écho d'un soupir caressa mon oreille \\
  comme une note de musique lointaine.

  L'écho d'un soupir que je connais, \\
  formé d'une haleine que j'ai bue, \\
  parfum d'une fleur qui croît cachée \\
  dans un cloître sombre.\footnote{Cette rime illustre le thème de la
  novice hors d'atteinte. Voir les rimes~24 et~59.}

  Mon adorée d'un jour, ma tendre, me dit: \\
  ---~À quoi penses-tu? \\
  ---~À rien... \\
  ---~À rien, et tu pleures? \\
  ---~J'ai la tristesse gaie et le vin triste.\footnote{Le thème de l'amour d'un soir se
  retrouve aux rimes~8 et~73, qui offrent un contraste à l'image
  d'Épinal d'un poète transi d'amour.}
\end{verse}

\bigskip

\begin{center}
  \textbf{10 [XLIV]}
  \addcontentsline{toc}{section}{\emph{10.\ Comme dans un livre
    ouvert}}
\end{center}

\begin{verse}
  Comme dans un livre ouvert, \\
  je lis dans le fond de tes pupilles. \\
  À quoi bon feignent les lèvres \\
  des rires que démentent les yeux?

  Pleure! N'ai honte \\
  de confesser que tu m'aimas un peu. \\
  Pleure! Personne ne nous voit. \\
  Vois: je suis un homme... et je pleure aussi.
\end{verse}

\bigskip

\begin{center}
  \textbf{11 [I]}
  \addcontentsline{toc}{section}{\emph{11.\ Je sais un hymne géant et étrange}}
\end{center}

\begin{verse}
  Je sais un hymne géant et étrange \\
  qui annonce dans la nuit de l'âme\footnote{La «\,nuit obscure de
  l'âme\,» est une expression de Jean de la Croix, qui
  désigne l'épreuve de l'absence de Dieu chez le
  mytique. Cf. rime~56.} une aurore, \\
  et ces pages sont de cet hymne \\
  des cadences que l'air dilate dans l'ombre.

  Je voudrais l'écrire, domptant \\
  de l'homme la rebelle langue mesquine, \\
  avec des mots qui soient à la fois \\
  soupirs et rires, couleurs et notes.\footnote{Le narrateur envisage
  ici la synesthésie poétique comme issue pour exprimer
  l'inexprimable «\,hymne\,» du premier quatrain. (Bécquer maitrisait le
  dessin et la musique.)}

  Mais vaine est la lutte: il n'est aucune mesure \\
  qui puisse l'enfermer, et c'est à peine, ma belle, \\
  si je puis te le conter seul à seul à l'oreille \\
  en tenant tes mains dans les miennes.
\end{verse}

\bigskip

\begin{center}
  \textbf{12 [L]}
  \addcontentsline{toc}{section}{\emph{12.\ Comme le sauvage aux mains malhabiles}}
\end{center}

\begin{verse}
  Comme le sauvage aux mains malhabiles \\
  fait à discrétion un dieu d'un tronc, \\
  et ensuite devant son œuvre s'agenouille, \\
  cela nous le fîmes toi et moi.

  Nous donnâmes forme réelle à un fantôme, \\
  invention ridicule de l'esprit, \\
  et, l'idole une fois là, nous sacrifiâmes \\
  notre amour sur son autel.
\end{verse}

\pagebreak

\begin{center}
  \textbf{13 [VII]}
  \addcontentsline{toc}{section}{\emph{13.\ Dans l'angle obscur du
    salon}}
\end{center}

\begin{verse}
  Dans l'angle obscur du salon, \\
  de son maître peut-être oubliée, \\
  silencieuse et couverte de poussière, \\
  trônait la harpe.

  Que de notes dormaient sur ses cordes, \\
  comme dorment les oiseaux sur les branches, \\
  attendant la main de neige \\
  qui les fait s'envoler!

  Hélas! pensai-je. Que de fois le génie \\
  ainsi dort-il au fond de l'âme, \\
  et attend une voix, comme Lazare, \\
  qui lui dise: \emph{Lève-toi et marche!}
\end{verse}

\bigskip

\begin{center}
  \textbf{14 [XLIX]}
  \addcontentsline{toc}{section}{\emph{14.\ Parfois je la rencontre
    de par le monde}}
\end{center}

\begin{verse}
  Parfois je la rencontre de par le monde \\
  et elle passe près de moi; \\
  et elle passe en souriant, et je dis: \\
  \emph{Comment peut-elle \emph{rire}?}

  Puis point à ma lèvre un autre sourire, \\
  masque de la douleur, \\
  et je pense alors: \emph{Peut-être rit-elle \\
  comme je ris moi-même.}
\end{verse}

\bigskip\bigskip\bigskip\bigskip

\pagebreak

\begin{center}
  \textbf{15 [II]}
  \addcontentsline{toc}{section}{\emph{15.\ \emph{Saeta} qui
    traverse en volant}}
\end{center}

\begin{verse}
  \emph{Saeta}\footnote{Courte prière chantée
  depuis les balcons au passage des trônes portant des scènes de la
  Passion du Christ, pendant la Semaine Sainte, principalement en
  Andalousie. L'étymologie est le latin \emph{sagitta},
  signifiant \emph{flèche}, d'où la métaphore qui suit.} qui traverse
  en volant, \\
  lancée au hasard \\
  sans qu'on ne sache \\
  où, tremblante, elle se plantera;

  feuille sèche de l'arbre \\
  emportée par la bourrasque,\footnote{Il pourrait s'agir aussi, au
  sens propre, du \emph{vendaval}, un vent du sud soufflant sur la
  vallée du Guadalquivir, qui traverse Séville, la ville de Bécquer,
  mais la version publiée dans \emph{El Museo Universal} indique
  \emph{huracán} (ouragan), d'où notre choix.} \\
  et on ne devine le sillon \\
  où elle retombera;

  vague géante que le vent \\
  enfle et pousse dans la mer, \\
  et roule et passe, et ne sait \\
  quel rivage elle va cherchant;

  lueur qui, prête à s'éteindre, \\
  brille en ronds tremblants, \\
  et l'on ne sait d'entre-eux \\
  lequel sera le dernier:

  c'est moi\footnote{L'accumulation d'images de la nature sans
  référent se résout ici.} qui, au hasard, \\
  traverse le monde sans penser \\
  d'où je viens, ni où \\
  mes pas me mèneront.\footnote{Le thème du destin incertain se
  retrouve dans les rimes~60 et~67.}
\end{verse}

\pagebreak

\begin{center}
  \textbf{16 [XLII]}
  \addcontentsline{toc}{section}{\emph{16.\ Quand on me le conta, je
    sentis le froid}}
\end{center}

\begin{verse}
  Quand on me le conta, je sentis le froid \\
  d'une lame d'acier dans les entrailles;\footnote{Le thème de la
  blessure par arme blanche est récurrent chez Bécquer. Voir les
  rimes~1, 28 et~77.} \\
  je m'appuyai contre le mur, et un instant \\
  je perdis la conscience du lieu où j'étais.

  La nuit s'abattit sur mon être; \\
  d'ire et de pitié s'inonda mon âme \\
  et je compris pourquoi on pleure, \\
  et je compris pourquoi on tue!

  Le nuage de douleur passa... Avec peine \\
  je parvins à balbutier de brèves bagatelles.\footnote{L'allitération
  suggère le balbutiement qui s'entend dans le texte
  original: \emph{logré balbucear breves palabras}.} \\
  Et qui me donna la nouvelle? Un ami fidèle. \\
  Pour ce grand service je le remerciai.
\end{verse}

\bigskip

\begin{center}
  \textbf{17 [LIX]}
  \addcontentsline{toc}{section}{\emph{17.\ Moi, je sais quel est
    l'objet de tes soupirs}}
\end{center}

\begin{verse}
  Moi, je sais quel est l'objet
  de tes soupirs; \\
  Moi, je sais la cause de ta douce \\
  et secrète langueur.

  Tu ris? Un jour
  tu sauras, petite, pourquoi. \\
  Toi, tu le soupçonnes
  et moi je le sais.

  Moi, je sais quand tu rêves \\
  et ce qu'en songe tu vois. \\
  Comme dans un livre je peux lire \\
  sur ton front ce que tu tais.

  Tu ris? Un jour
  tu sauras, petite, pourquoi. \\
  Toi, tu le soupçonnes
  et moi je le sais.

  Moi, je sais pourquoi tu souris \\
  et pleures à la fois; \\
  moi, je pénètre les recoins mystérieux \\
  de ton âme de femme.

  Tu ris? Un jour
  tu sauras, petite, pourquoi. \\
  Pendant que tu éprouves tant et ne sais rien, \\
  moi, qui ne ressens plus rien, je sais tout.
\end{verse}

%\bigskip

\begin{center}
  \textbf{18 [LXVII]}
  \addcontentsline{toc}{section}{\emph{18.\ Quelle merveille que de
    voir le jour se lever}}
\end{center}

\begin{verse}
  Quelle merveille que de voir le jour \\
  se lever, couronné de feu, \\
  et, à son baiser enflammé, \\
  voir briller les vagues et s'incendier l'air!

  Quelle merveille, après la pluie, \\
  dans le soir bleuté de l'automne triste, \\
  que de respirer le parfum \\
  des fleurs humides jusqu'à satiété!

  Quelle merveille, quand la blanche neige \\
  tombe silencieusement en flocons, \\
  que de voir s'agiter les langues rougeâtres \\
  des flammes inquiètes!

  Quelle merveille, après la fatigue, \\
  que bien dormir, ronfler tel un
  sous\hyp{}chantre,\footnote{Officier du chœur subordonné au chantre.} \\
  et manger, et grossir... Et quel malheur\footnote{Une fameuse
  correction indique l'opposé: «\,bonheur\,».} \\
  que cela seulement ne suffise pas!
\end{verse}

\pagebreak

\begin{center}
  \textbf{19 [XXII]}
  \addcontentsline{toc}{section}{\emph{19.\ Comment vit encore cette rose que tu as prise}}
\end{center}

\begin{verse}
  Comment vit encore cette rose \\
  que tu as prise contre ton cœur?\footnote{Le motif de la fleur au
  décolleté se retrouve à la rime~6.} \\
  Avant de la contempler, \\
  jamais je n'avais vu de fleur sur un volcan.
\end{verse}

%\bigskip

\begin{center}
  \textbf{20 [LVI]}
  \addcontentsline{toc}{section}{\emph{20.\ Aujourd'hui comme hier, demain comme aujourd'hui}}
\end{center}

\begin{verse}
  Ce jour comme hier, demain comme ce jour, \\
  et toujours pareil! \\
  Un ciel gris, un horizon éternel, \\
  et marcher... marcher.

  Le cœur battant la mesure \\
  comme une machine stupide;\footnote{Le thème du cœur-machine se
  retrouve à la rime~44.} \\
  l'intelligence obtuse du cerveau \\
  endormie dans un recoin.

  L'âme, dans son ambition du Paradis, \\
  le cherche sans foi. \\
  Fatigue sans objet, vague qui roule \\
  sans savoir pourquoi.

  La voix, d'un ton égal, \\
  chante incessamment le même chant. \\
  La goutte d'eau monotone qui tombe, \\
  et tombe, sans cesse.

  Ainsi vont les jours, filant \\
  les uns après les autres, \\
  aujourd'hui comme hier... et tous \\
  sans plaisir ni douleur.

  Hélas! Parfois je me souviens en un soupir \\
  d'une affliction ancienne. \\
  Amère est la douleur, mais au moins \\
  souffrir est vivre!
\end{verse}

\smallskip

\begin{center}
  \textbf{21 [XXI]}
  \addcontentsline{toc}{section}{\emph{21.\ Qu'est la poésie?}}
\end{center}

\begin{verse}
  \emph{Qu'est la poésie?} dis-tu en plantant \\
  dans mes yeux tes yeux bleus. \\
  Qu'est la poésie! Et toi, tu me le demandes? \\
  La poésie... c'est toi.\footnote{La rime la plus célèbre du
  recueil. On peut la rapprocher de la rime~39 dans sa recherche d'une
  définition de la poésie, fondée sur l'expression du sentiment
  incarné par les femmes.}
\end{verse}

\smallskip

\begin{center}
  \textbf{22 [XXIII]}
  \addcontentsline{toc}{section}{\emph{22.\ Pour un regard, un monde;}}
\end{center}

\begin{verse}
  Pour un regard, un monde; \\
  pour un sourire, un ciel; \\
  pour un baiser... j'ignore \\
  que t'offrir pour un baiser!
\end{verse}

\bigskip

\begin{center}
  \textbf{23 [LXXV]}
  \addcontentsline{toc}{section}{\emph{23.\ Est-il vrai que,
    quand le sommeil touche nos yeux}}
\end{center}

\begin{verse}
  Est-il vrai que, quand le sommeil touche \\
  nos yeux de ses doigts de rose, \\
  l'âme s'enfuit en vol pressé \\
  de la prison qu'elle habite?

  Est-il vrai que, hôte des brumes, \\
  au souffle ténu de la brise nocturne, \\
  elle monte, ailée, à la région vide \\
  pour en rencontrer d'autres?

  Et là, dévêtue de la forme humaine, \\
  là, les liens terrestres rompus, \\
  elle habite de brèves heures \\
  le monde silencieux de l'idée?

  Et qu'elle rit et pleure, et exècre et aime, \\
  et conserve un visage de douleur et joie, \\
  pareil à celui que laisse un météore \\
  quand il traverse le ciel?

  Moi, je ne sais si ce monde de visions \\
  vit hors de nous ou en nous: \\
  ce que je sais, c'est que je connais maintes gens \\
  que je ne connais pas.
\end{verse}

\smallskip

\begin{center}
  \textbf{24 [LXXIV]}
  \addcontentsline{toc}{section}{\emph{24.\ Les habits défaits, les
    épaules nues}}
\end{center}

\begin{verse}
  Les habits défaits,
  les épaules nues, \\
  deux anges veillaient \\
  sur le linteau doré de la porte.

  Je m'approchai des fers forgés \\
  qui défendent l'entrée \\
  et, des doubles grilles, \\
  je la vis au fond, confuse et blanche.

  Je la vis comme l'image \\
  qui passe en rêverie, \\
  comme un rai de lumière ténu et diffus \\
  qui passe parmi les ténèbres.

  Je sentis mon âme pleine \\
  d'un désir ardent; \\
  comme attire un abîme, ce mystère \\
  vers lui m'entraînait.

  Mais hélas! le regard des anges \\
  semblait me dire: \\
  \emph{Le seuil de cette porte,
  seul Dieu le franchit!}\footnote{Cette rime illustre le thème de la
  novice hors d'atteinte. Voir les rimes~9 et~59.}
\end{verse}

\smallskip

\begin{center}
  \textbf{25 [VIII]}
  \addcontentsline{toc}{section}{\emph{25.\ Quand je regarde
    l'horizon bleu}}
\end{center}

\begin{verse}
  Quand je regarde l'horizon bleu \\
  se perdre au lointain, \\
  au travers d'une gaze de poussière \\
  dorée et inquiète,

  je crois possible de m'arracher \\
  du sol misérable \\
  et flotter avec la brume dorée \\
  en atomes légers, \\
  défait comme elle.

  Quand je vois de nuit, dans le fond \\
  obscur du ciel, \\
  trembler les étoiles comme d'ardents \\
  iris de feu,

  je crois possible de m'envoler \\
  là où elles brillent, \\
  et m'inonder de leur lumière \\
  et, en un feu qui a pris, \\
  me fondre avec elles en un baiser.

  Sur la mer de doute où je vogue, \\
  je ne sais même pas ce que je crois; \\
  pourtant ces désirs me disent \\
  que je porte quelque chose \\
  de divin, ici, en moi.
\end{verse}

\bigskip

\begin{center}
  \textbf{26 [XLI]}
  \addcontentsline{toc}{section}{\emph{26.\ Tu étais l'ouragan et
    moi la haute tour}}
\end{center}

\begin{verse}
  Tu étais l'ouragan et moi la haute \\
  tour qui défie son pouvoir: \\
  tu devais te fracasser ou m'abattre... \\
  Impossible!

  Tu étais l'océan et moi la roche \\
  dressée qui attend son va-et-vient: \\
  tu devais te briser ou m'arracher... \\
  Impossible!

  Belle, toi; moi, altier; habitués \\
  l'un à l'emporter, l'autre à ne pas céder: \\
  étroite, la sente; inévitable, le choc... \\
  Impossible!\footnote{Cette rime s'inscrit dans le thème dialogique
  «\,toi et moi\,» que l'on retrouve dans la rime~33, mais ici avec
  discordance et opposition.}
\end{verse}

\bigskip

\begin{center}
  \textbf{27 [IX]}
  \addcontentsline{toc}{section}{\emph{27.\ Le zéphyr qui gémit
    faiblement}}
\end{center}

\begin{verse}
  Le zéphyr\footnote{\emph{aura}, dans la poésie espagnole du XIX\ieme{}
  siècle, désignait un vent doux. Nous traduisons parfois par
  \emph{brise}. Voir rimes~5 et~60.} qui gémit faiblement \\
  baise les ondes légères qu'il plisse en jouant; \\
  le soleil baise la nuée à l'occident \\
  jusqu'à ce que, de pourpre et d'or, il la nuance; \\
  la flamme à l'entour du tronc ardent \\
  s'étale en baisant une autre flamme, \\
  et jusqu'au saule pesant, qui se penche \\
  vers la rivière qui le baise, renvoie un baiser.
\end{verse}

\pagebreak

\begin{center}
  \textbf{28 [XXXVII]}
  \addcontentsline{toc}{section}{\emph{28.\ Je mourrai avant toi}}
\end{center}

\begin{verse}
  Je mourrai avant toi: caché \\
  dans les entrailles déjà \\
  je porte le fer avec lequel ta main ouvrit \\
  la large blessure mortelle.\footnote{Le thème de la blessure
  par arme blanche est récurrent chez Bécquer. Voir les rimes~1, 16 et~77.}

  Je mourrai avant toi et mon âme, \\
  dans son entêtement tenace, \\
  s'assiéra aux portes de la mort, \\
  t'attendant là-bas.

  Avec les heures, les jours; avec les jours, \\
  les années s'envoleront; \\
  et tu frapperas à cette porte enfin... \\
  Qui peut ne pas frapper?

  Puis la terre gardera \\
  tes fautes et ta dépouille, \\
  tu te laveras dans les ondes de la mort \\
  comme dans un autre Jourdain;\footnote{Référence au baptême de Jésus
  par Jean le Baptiste, sauf que ce sont ici les eaux de la mort.}

  là-bas, où le murmure de la vie \\
  va mourir en tremblant, \\
  comme la vague va en silence \\
  expirer sur le rivage;

  là-bas, où le sépulcre qui se ferme \\
  ouvre une éternité, \\
  tout ce que nous deux avons tu \\
  nous devrons en parler, là-bas.
\end{verse}

\bigskip

\begin{center}
  \textbf{29 [XIII]}
  \addcontentsline{toc}{section}{\emph{29.\ Tes yeux sont bleus}}
\end{center}

\begin{verse}
  Tes yeux sont bleus et, quand tu ris, \\
  leur clarté suave me rappelle \\
  l'éclat tremblant du matin \\
  qui se reflète dans la mer.

  Tes yeux sont bleus et, quand tu pleures, \\
  les larmes transparentes en eux \\
  me semblent gouttes de rosée \\
  sur une violette.

  Tes yeux sont bleus et, si irradie une idée \\
  comme un point de lumière au fond, \\
  elle paraît une étoile perdue \\
  dans le ciel de l'après-midi.\footnote{Il s'agit de la première rime
  publiée par l'auteur, le \oldstylenums{17}~décembre
  \oldstylenums{1859}, sous le titre: «\,Imitación de Byron\,», en
  référence à des vers de Lord Byron dans \emph{Hebrew Melodies}
  (\oldstylenums{1815}).}
\end{verse}

\bigskip

\begin{center}
  \textbf{30 [XXXI]}
  \addcontentsline{toc}{section}{\emph{30.\ Notre passion fut une tragique saynète}}
\end{center}

\begin{verse}
  Notre passion fut une tragique saynète \\
  dont l'absurde fable \\
  fait jaillir rires et pleurs, \\
  le comique et le grave confondus.

  Mais le pire de cette histoire fut \\
  qu'à la fin de l'acte, \\
  à elle échurent larmes et rires, \\
  et à moi seulement les larmes.
\end{verse}

\pagebreak

\begin{center}
  \textbf{31 [XXV]}
  \addcontentsline{toc}{section}{\emph{31.\ Quand t'enveloppent dans la nuit}}
\end{center}

\begin{verse}
  Quand t'enveloppent dans la nuit\footnote{Le premier don est la
  femme endormie, dont on retrouve la figure à la rime~63.} \\
  les ailes de tulle du sommeil \\
  et que tes cils tendus
  imitent des arcs d'ébène,

  pour écouter les battements \\
  de ton cœur inquiet \\
  et sentir ta tête endormie \\
  reposer sur ma poitrine,

  je donnerais, mon amour, \\
  tout ce que je possède: \\
  la lumière, l'air
  et la pensée!

  Quand tes yeux se fixent\footnote{Le deuxième don est la femme
  contemplative, dont on retrouve la figure à la rime~17.} \\
  sur un objet invisible \\
  et le reflet d'un sourire
  illumine tes lèvres,

  pour lire sur ton front
  la pensée secrète \\
  qui passe comme un nuage marin \\
  sur le large miroir,

  je donnerais, mon amour, \\
  tout ce que je désire: \\
  la renommée, l'or,
  la gloire, le génie!

  Quand ta langue devient muette,\footnote{Le troisième et dernier don
  est la femme qui offre son désir à son amant, le narrateur.} \\
  et ton haleine se presse,
  et tes joues s'allument, \\
  et tu entrouvres tes yeux noirs,

  pour voir entre tes cils \\
  briller d'un feu humide \\
  l'étincelle ardente qui jaillit \\
  du volcan des désirs,

  je donnerais, mon amour, \\
  tout ce que en quoi j'espère: \\
  la foi, l'âme,
  la terre, le ciel!\footnote{À chaque don de la femme, chacun plus
  précieux, le narrateur échange des offrandes de plus en plus précieuses.}
\end{verse}

\bigskip

\begin{center}
  \textbf{32 [LVII]}
  \addcontentsline{toc}{section}{\emph{32.\ Cette carcasse d'os et de peau}}
\end{center}

\begin{verse}
  Cette carcasse d'os et de peau \\
  se fatigue enfin de tant promener une tête folle, \\
  et je ne le regrette pas, \\
  car, bien qu'il soit vrai que je ne sois pas vieux,

  de la part de vie qu'il me revient \\
  de la vie du monde, \\
  j'ai fait un tel usage à mes dépens que je jurerais \\
  avoir condensé un siècle en chaque jour.

  Ainsi, si je mourais à l'instant, \\
  je ne pourrais dire que je n'ai vécu; \\
  si le vêtement paraît neuf par dehors \\
  je sais qu'il a vieilli par dedans.

  Il a vieilli, oui; malgré mon étoile! \\
  mon ardeur dolente le dit suffisamment; \\
  c'est qu'il est des douleurs qui gravent au cœur \\
  leurs empreintes horribles, au lieu du front.
\end{verse}

\pagebreak

\begin{center}
  \textbf{33 [XXIV]}
  \addcontentsline{toc}{section}{\emph{33.\ Deux rouges langues de feu}}
\end{center}

\begin{verse}
  Deux rouges langues de feu \\
  qui, enlacées au même tronc, \\
  s'approchent et, en se baisant, \\
  forment une seule flamme;

  deux notes que la main fait jaillir \\
  du luth en même temps, \\
  et qui, dans l'espace, se réunissent \\
  et s'embrassent en harmonie;

  deux vagues qui viennent ensemble \\
  mourir sur une plage \\
  et, en se brisant, se couronnent \\
  d'un panache d'argent;

  deux lambeaux de vapeur \\
  qui s'élèvent du lac, \\
  et, en se joignant dans le ciel, \\
  forment un nuage blanc;

  deux idées qui surgissent de pair, \\
  deux baisers qui éclatent de concert, \\
  deux échos qui se confondent... \\
  ainsi sont nos deux âmes.\footnote{Cette rime s'inscrit dans le
  thème dialogique «\,toi et moi\,» que l'on retrouve dans la rime~26,
  mais avec harmonie ici.}
\end{verse}

%\bigskip

\begin{center}
  \textbf{34 [XLIII]}
  \addcontentsline{toc}{section}{\emph{34.\ J'écartai la lampe et
    au bord du lit défait je m'assis}}
\end{center}

\begin{verse}
  J'écartai la lampe et au bord \\
  du lit défait je m'assis, \\
  muet, sombre, les pupilles immobiles \\
  plantées dans le mur.

  Combien de temps restai-je ainsi? Je ne sais; \\
  quand me quitta l'horrible ivresse de douleur, \\
  la lueur expirait et sur mes balcons \\
  le soleil riait.

  Je ne sais non plus, en de si terribles heures, \\
  à quoi je pensais ou ce qui me traversa; \\
  je me souviens seulement avoir pleuré et maudit, \\
  et avoir vieilli cette nuit-là.\footnote{À rapprocher de la rime~16.}
\end{verse}

\bigskip

\begin{center}
  \textbf{35 [LII]}
  \addcontentsline{toc}{section}{\emph{35.\ Lames géantes qui vous
    brisez en mugissant}}
\end{center}

\begin{verse}
  Lames géantes qui vous brisez en mugissant \\
  sur les rivages déserts et lointains: \\
  enveloppé dans le drap d'écumes, \\
  emportez-moi avec vous!

  Rafales d'ouragans qui arrachent \\
  de la grande forêt les feuilles mortes: \\
  entraîné dans l'aveugle tourbillon, \\
  emportez-moi avec vous!

  Nuées de tempête que rompt l'éclair \\
  et qui ornez les orles défaits en feu: \\
  enlevé parmi la brume obscure, \\
  emportez-moi avec vous!

  Emportez-moi, par pitié, là où le vertige \\
  m'arracherait la mémoire et la raison. \\
  Par pitié! J'ai peur de rester \\
  seul à seul avec ma douleur!
\end{verse}

\pagebreak

\begin{center}
  \textbf{36 [LIV]}
  \addcontentsline{toc}{section}{\emph{36.\ Quand nous évoquons à nouveau les heures fugaces du passé}}
\end{center}

\vspace*{-15pt}

\begin{verse}
  Quand nous évoquons à nouveau \\
  les heures fugaces du passé, \\
  une larme tremblante brille, \\
  prompte à glisser sur ses cils noirs.

  Et, enfin, elle glisse et tombe comme goutte \\
  de rosée à la pensée que, \\
  tel ce jour pour hier, pour ce jour demain, \\
  tous deux nous soupirerons à nouveau.
\end{verse}

%\bigskip

\begin{center}
  \textbf{37 [XX]}
  \addcontentsline{toc}{section}{\emph{37.\ Elle sait, si parfois
    ses lèvres rouges sont brûlées}}
\end{center}

\vspace*{-15pt}

\begin{verse}
  Elle sait, si parfois ses lèvres rouges \\
  sont brûlées par une atmosphère invisible, \\
  que l'âme qui peut parler avec les yeux \\
  peut aussi embrasser avec le regard.\footnote{À rapprocher de la rime~43.}
\end{verse}

%\bigskip

\begin{center}
  \textbf{38 [LIII]}
  \addcontentsline{toc}{section}{\emph{38.\ Elles reviendront, les
    obscures hirondelles}}
\end{center}

\vspace*{-15pt}

\begin{verse}
  Elles reviendront, les obscures hirondelles, \\
  pendre leurs nids à ton balcon \\
  et, à nouveau, avec leurs ailes \\
  elles toqueront aux carreaux en jouant.

  Mais celles qui réfrénaient leur vols \\
  en contemplant ta beauté et mon bonheur, \\
  celles qui apprirent nos noms... \\
  celles-ci ne reviendront pas!

  Ils reviendront, les épais chèvrefeuilles, \\
  escalader les murs de ton jardin, \\
  et, à nouveau, leurs fleurs s'ouvriront le soir, \\
  encore plus belles.

  Mais celles figées par la rosée, \\
  dont nous regardions les gouttes trembler \\
  et tomber comme larmes du jour... \\
  celles-ci ne reviendront pas!

  Ils reviendront, les mots ardents de l'amour \\
  sonner à ton oreille, \\
  ton cœur se réveillera peut-être \\
  de son profond sommeil.

  Mais, muet et absorbé et à genoux, \\
  comme on adore Dieu devant son autel, \\
  comme moi je t'ai aimée..., détrompe-toi, \\
  ainsi personne ne t'aimera plus.\footnote{La plus célèbre des rimes,
  avec la rime~21.}
\end{verse}

\bigskip\bigskip

\begin{center}
  \textbf{39 [IV]}
  \addcontentsline{toc}{section}{\emph{39.\ Ne dites pas que, épuisé
    son trésor}}
\end{center}

\begin{verse}
  Ne dites pas que, épuisé son trésor, \\
  faute de sujet, la lyre s'est tue: \\
  il pourrait ne pas y avoir de poètes, \\
  mais toujours il y aura la poésie.

  Tant que les ondes embrasées \\
  de la lumière palpiteront aux baisers, \\
  tant que le soleil vêtira \\
  les nuées déchirées de feu et d'or; \\
  tant que l'air portera en son giron \\
  parfums et harmonies;\footnote{Voir rime~57.} \\
  tant qu'il aura un printemps au monde, \\
  il y aura la poésie!

  Tant que la science échouera à découvrir \\
  la source de la vie, \\
  et qu'en mer ou au ciel il y aura un abîme \\
  qui résiste au calcul; \\
  tant que l'humanité, toujours progressant, \\
  ne saura où elle va; \\
  tant qu'il aura un mystère pour l'homme, \\
  il y aura la poésie!

  Tant que l'on sentira l'âme se réjouir \\
  sans que les lèvres ne rient; \\
  tant que l'on pleurera sans que le sanglot \\
  ne vienne troubler la pupille; \\
  tant que le cœur et la tête \\
  continueront à batailler; \\
  tant qu'il y aura espoirs et souvenirs, \\
  il y aura la poésie!

  Tant qu'il y aura des yeux qui reflètent \\
  les yeux qui les regardent, \\
  tant que répondra la lèvre soupirant \\
  à la lèvre qui soupire; \\
  tant que deux âmes en un baiser \\
  confondues pourront se toucher; \\
  tant qu'il existera une femme splendide, \\
  il y aura la poésie!
\end{verse}

\bigskip

\begin{center}
  \textbf{40 [XXX]}
  \addcontentsline{toc}{section}{\emph{40.\ Une larme pointait à
    ses yeux}}
\end{center}

\begin{verse}
  Une larme poignait à ses yeux \\
  et une phrase de pardon à mes lèvres; \\
  l'orgueil parla et son pleur s'assécha, \\
  et la phrase sur mes lèvres expira.

  Je vais mon chemin; elle, un autre; \\
  mais en repensant à notre amour mutuel, \\
  je dis encore: \emph{Pourquoi n'ai-je rien dit ce jour-là?} \\
  et elle doit se dire: \emph{Pourquoi n'ai-je pas pleuré?}
\end{verse}

%\bigskip

\begin{center}
  \textbf{41 [LX]}
  \addcontentsline{toc}{section}{\emph{41.\ Ma vie est une friche}}
\end{center}

\begin{verse}
  Ma vie est une friche; \\
  fleur que je touche s'effeuille. \\
  Sur mon chemin fatal, \\
  on va semant le mal \\
  pour que moi je le recueille.
\end{verse}

%\bigskip

\begin{center}
  \textbf{42 [III]}
  \addcontentsline{toc}{section}{\emph{42.\ Secousse étrange qui
    agite les idées}}
\end{center}

\begin{verse}
  Secousse étrange \\
  qui agite les idées, \\
  comme ouragan qui pousse \\
  les vagues au galop;

  murmure qui dans l'âme \\
  s'élève et va croissant, \\
  comme volcan sourd qui \\
  annonce qu'il va s'embraser;

  silhouettes difformes \\
  d'êtres impossibles; \\
  paysages apparaissant \\
  comme au travers d'un tulle;

  couleurs qui se marient \\
  et imitent dans l'air \\
  les atomes de l'iris \\
  qui nagent dans la lumière;

  idées sans paroles, \\
  paroles insensées; \\
  cadences sans rythme \\
  ni mesure;

  souvenirs et désirs \\
  de ce qui n'existe pas; \\
  transports de joie, \\
  envies de pleurer;

  activité nerveuse \\
  qui erre sans emploi, \\
  sans rênes qui guident \\
  ce cheval ailé;

  folie que l'âme \\
  exalte et enflamme, \\
  ivresse divine \\
  du génie créateur...

  Telle est l'inspiration!

  Voix géante qui ordonne \\
  le chaos dans le cerveau, \\
  et, parmi les ombres, fait \\
  paraître la lumière;

  brillante rêne d'or \\
  qui, puissante, freine \\
  de l'esprit exalté \\
  le coursier volant;\footnote{Allusion à la mythologie grecque, où le
  héros Bellérophon reçoit de la déesse Athéna des rênes d'or pour
  dompter et monter Pégase, le cheval ailé.}

  fil de lumière qui noue \\
  les pensées en gerbes, \\
  soleil qui rompt les nuées \\
  et atteint le zénith;

  main intelligente \\
  qui parvient à réunir \\
  les mots indociles \\
  en un collier de perles;

  rythme harmonieux \\
  qui enserre dans la mesure \\
  les notes fugitives \\
  avec cadence et nombre;

  ciseau qui mord dans le bloc, \\
  modelant la statue, \\
  et la beauté plastique \\
  ajoute à l'idéale;

  atmosphère où tournent \\
  les idées en ordre, \\
  tels des atomes que réunit \\
  une attraction secrète;

  torrent où la fièvre
  éteint sa soif; \\
  oasis qui à l'esprit
  rend sa vigueur...

  Telle est notre raison!

  Avec ces deux toujours en lutte \\
  et des deux vainqueur, \\
  tant il n'est donné qu'au génie \\
  de les mettre sous le même joug.
\end{verse}

\pagebreak

\begin{center}
  \textbf{43 [XVI]}
  \addcontentsline{toc}{section}{\emph{43.\ Si, quand les
    clochettes bleues de ton balcon se bercent}}
\end{center}

\begin{verse}
  Si, quand les clochettes bleues de ton balcon \\
  se bercent, \\
  tu crois qu'en soupirant passe le vent \\
  qui murmure, \\
  sache que, caché parmi les feuilles vertes, \\
  moi je soupire.

  Si, quand résonne confusément derrière toi \\
  une vague rumeur, \\
  tu crois qu'une voix lointaine t'a appelé \\
  par ton nom, \\
  sache que, parmi les ombres qui t'entourent, \\
  moi je t'appelle.

  Si, quand ton cœur craintif se trouble \\
  en pleine nuit, \\
  tu sens sur tes lèvres une haleine \\
  qui embrase, \\
  sache que, bien que invisible à tes côtés, \\
  moi je respire.
\end{verse}

\bigskip

\begin{center}
  \textbf{44 [LXXVII]}
  \addcontentsline{toc}{section}{\emph{44.\ Tu dis que tu as un cœur}} \end{center}

\begin{verse}
  Tu dis que tu as un cœur, et tu le dis \\
  seulement parce que tu sens ses battements. \\
  Ce n'est pas un cœur... C'est une machine\footnote{Le thème du
  cœur-machine se retrouve à la rime~20.} \\
  qui, au rythme de son mouvement, fait du bruit.
\end{verse}

\pagebreak

\begin{center}
  \textbf{45 [LXI]}
  \addcontentsline{toc}{section}{\emph{45.\ En voyant mes heures
    lentes de fièvre}}
\end{center}

\begin{verse}
  En voyant mes heures lentes \\
  de fièvre et d'insomnie défiler: \\
  au bord de ma couche, \\
  qui s'assiéra?

  Quand ma main tremblante \\
  se tendra, prête à expirer: \\
  cherchant une main amie, \\
  qui la serrera?

  Quand la mort dépolira \\
  le cristal de mes yeux: \\
  mes paupières encore ouvertes, \\
  qui les clora?

  Quand la cloche sonnera \\
  (si elle sonne à mon enterrement): \\
  une prière en l'entendant, \\
  qui la murmurera?

  Quand mes pâles restes \\
  opprimeront la terre enfin: \\
  sur la fosse oubliée, \\
  qui viendra pleurer?

  Enfin, le jour suivant, \\
  quand le soleil brillera à nouveau: \\
  de mon passage de par le monde, \\
  qui se souviendra?
\end{verse}

\pagebreak

\begin{center}
  \textbf{46 [X]}
  \addcontentsline{toc}{section}{\emph{46.\ Les invisibles atomes de
    l'air alentour palpitent}}
\end{center}

\begin{verse}
  Les invisibles atomes de l'air \\
  alentour palpitent et s'enflamment, \\
  le ciel se défait en rayons d'or, \\
  la terre frémit de joie; \\
  j'entends, flottant sur des ondes d'harmonie, \\
  rumeurs de baisers et battements d'ailes, \\
  et mes paupières se closent... Qu'arrive-t-il? \\
  ---~C'est l'amour qui passe!
\end{verse}

\bigskip

\begin{center}
  \textbf{47 [LXV]}
  \addcontentsline{toc}{section}{\emph{47.\ Vint la nuit et point
    d'asile}}
\end{center}

\begin{verse}
  Vint la nuit et point d'asile; \\
  et j'eus soif!... Je bus mes larmes. \\
  Et j'eus faim!... Je fermai mes yeux enflés \\
  pour mourir.

  Étais-je dans un désert? Bien qu'à mon oreille \\
  parvenait le rauque bouillonnement des foules,\hspace*{-10pt} \\
  j'étais orphelin et pauvre. \\
  Le monde était un désert... pour moi!
\end{verse}

\bigskip

\begin{center}
  \textbf{48 [LXXVIII]}
  \addcontentsline{toc}{section}{\emph{48.\ Feignant des réalités
    avec l'ombre vaine}}
\end{center}

\begin{verse}
  Feignant des réalités avec l'ombre vaine, \\
  l'Espoir va, devançant le Désir.

  Et ses mensonges,
  comme le Phénix, \\
  renaissent de ses cendres.
\end{verse}

\pagebreak

\begin{center}
  \textbf{49 [LXIX]}
  \addcontentsline{toc}{section}{\emph{49.\ Nous naissons de l'éclair lorsqu'il brille}}
\end{center}

\begin{verse}
  Nous naissons de l'éclair lorsqu'il brille, \\
  et son éclat perdure quand nous mourons: \\
  si courte est la vie!

  Nous courons après gloire et amour, \\
  ombres d'un rêve que nous poursuivons: \\
  s'éveiller est mourir!\footnote{Référence à l'œuvre de Calderón de
  la Barca, \emph{La vida es sueño} (la vie est un rêve) (1635).}
\end{verse}

%\bigskip

\begin{center}
  \textbf{50 [XVII]}
  \addcontentsline{toc}{section}{\emph{50.\ Aujourd'hui la terre et les cieux me sourient}}
\end{center}

\begin{verse}
  Aujourd'hui la terre et les cieux me sourient, \\
  aujourd'hui le soleil atteint le fond de mon âme,\hspace*{-10pt} \\
  aujourd'hui je l'ai vue...,
  je l'ai vue et elle m'a regardé... \\
  Aujourd'hui je crois en Dieu!
\end{verse}

%\bigskip

\begin{center}
  \textbf{51 [XI]}
  \addcontentsline{toc}{section}{\emph{51.\ Je suis ardente, je
    suis brune}}
\end{center}

\begin{verse}
  ---~Je suis ardente, je suis brune, \\
  je suis le symbole de la passion; \\
  mon âme est pleine de désirs de jouissance. \\
  Est-ce moi que tu cherches?

  ---~Ce n'est pas toi, non.

  ---~Mon front est pâle, mes tresses d'or; \\
  je peux t'offrir des bonheurs sans fin; \\
  je garde un trésor de tendresse. \\
  Est-ce moi que tu appelles?

  ---~Ce n'est pas toi, non.

  ---~Je suis un songe, fantôme \\
  impossible et vain de brume et lumière; \\
  je suis incorporelle, je suis intangible, \\
  je ne puis t'aimer.

  ---~Oh! viens, toi, viens!
\end{verse}

%\bigskip

\begin{center}
  \textbf{52 [XIX]}
  \addcontentsline{toc}{section}{\emph{52.\ Quand sur ta poitrine
    tu penches un front mélancolique}}
\end{center}

\begin{verse}
  Quand sur ta poitrine tu penches \\
  un front mélancolique, \\
  tu me sembles
  un lys brisé,\footnote{Voir les rimes~83 et~85.}

  car, en te donnant la pureté, \\
  qui est un symbole céleste, \\
  comme lui te fit Dieu
  d'or et de neige.
\end{verse}

%\bigskip

\begin{center}
  \textbf{53 [XXIX]}
  \addcontentsline{toc}{section}{\emph{53.\ Sur sa jupe elle tenait
  le livre ouvert}}
\end{center}

\begin{flushright}
  \emph{La bocca mi baciò tutto tremante.}\footnote{Mise en abyme du
  chant V, vers~136, de l'\emph{Enfer} de Dante: «\,[celui qui ne sera
  plus jamais séparé de moi] me baisa la bouche, tout
  tremblant.\,». Dans ce passage, Francesca de Remini relate au poète
  comment son amoureux, Paolo Malatesta, l'embrassa alors qu'ils
  lisaient \emph{Lancelot du Lac}, où Lancelot embrasse Guenevièvre.} \\
  \textsc{Dante}
\end{flushright}

\begin{verse}
  Sur sa jupe elle tenait
  le livre ouvert, \\
  ses boucles noires
  touchaient ma joue:\\
  nous ne voyions pas les lettres, \\
  aucun des deux, je crois, \\
  mais nous gardions
  un profond silence.

  Combien cela dura? \\
  Ni alors je ne pus le savoir.

  Je sais seulement qu'on n'entendait \\
  rien d'autre que l'haleine \\
  pressée qui s'échappait des lèvres sèches,

  je sais seulement que nous nous tournâmes \\
  les deux en même temps, \\
  et nos yeux se trouvèrent,
  et retentit un baiser!
\end{verse}

%\[\star \ \ \ \star \ \ \ \star\]

\begin{verse}
  Le livre était l'œuvre de Dante,
  son \emph{Enfer}. \\
  Quand nous y baissâmes les yeux, \\
  je dis, tremblant: \\
  --- Comprends-tu maintenant qu'un poème \\
    tient tout entier dans un vers? \\
  Et elle répondit, enflammée: \\
  --- Je le comprends maintenant!
\end{verse}

%\bigskip

\begin{center}
  \textbf{54 [XXXVI]}
  \addcontentsline{toc}{section}{\emph{54.\ Si l'on écrivait dans un livre}}
\end{center}

\vspace*{-15pt}

\begin{verse}
  Si l'on écrivait dans un livre \\
  l'histoire de nos préjudices, \\
  et si l'on effaçait de nos âmes autant \\
  que l'on effacerait de ses pages... \\
  Je t'aime tant encore: ton amour laissa \\
  sur ma poitrine des traces si profondes \\
  que si tu n'en effaçais qu'une, \\
  je les effacerais toutes!
\end{verse}

%\pagebreak

\begin{center}
  \textbf{55\footnote{Ce poème ne fut pas publié dans \emph{Obras},
    car probablement considéré comme étant inspiré de la muse
    de l'auteur, Julia Espín, mariée lors de la parution.}}
  \addcontentsline{toc}{section}{\emph{55.\ Une femme
    m'a empoisonné l'âme}}
\end{center}

\vspace*{-15pt}

\begin{verse}
  Une femme m'a empoisonné l'âme, \\
  une autre m'a empoisonné le corps; \\
  aucune des deux ne vint me chercher; \\
  moi, d'aucune des deux je ne me plains.

  Comme le monde est rond, le monde tourne. \\
  Si demain, tournant, ce poison \\
  empoisonne à son tour, pourquoi m'accuser? \\
  Puis-je donner plus que ce que l'on me donna?
\end{verse}

\bigskip

\begin{center}
  \textbf{56 [LXII]}
  \addcontentsline{toc}{section}{\emph{56.\ D'abord une aube
    tremblante et vague}}
\end{center}

%\vspace*{-15pt}

\begin{verse}
  D'abord une aube tremblante et vague, \\
  un rai de lumière inquiète qui coupe la mer; \\
  puis elle étincelle et croît et se dilate \\
  en une ardente explosion de clarté.

  Le foyer brillant est la joie, \\
  l'ombre craintive est la peine; \\
  Hélas! Dans la nuit obscure de mon âme,\footnote{La «\,nuit obscure de
  l'âme\,» est une expression de Jean de la Croix, qui
  désigne l'épreuve de l'absence de Dieu chez le
  mytique. Cf. rime~11.} \\
  quand poindra le jour?
\end{verse}

\bigskip

\begin{center}
  \textbf{57 [VI]}
  \addcontentsline{toc}{section}{\emph{57.\ Comme la brise qui
    rafraîchit le sang}}
\end{center}

%\vspace*{-15pt}

\begin{verse}
  Comme la brise qui rafraîchit le sang \\
  sur le champ sombre des batailles, \\
  chargée de parfums et d'harmonies\footnote{Voir deuxième strophe de
  la rime~39.} \\
  dans le silence de la nuit, elle erre;

  symbole de la douleur et de la tendresse, \\
  dans l'horrible drame du barde anglais, \\
  la douce Ophélie,\footnote{Personnage au destin tragique dans la
  pièce de Shakespeare \emph{Hamlet}.} la  raison égarée, \\
  chante et cueille des fleurs en passant.
\end{verse}

\pagebreak

\begin{center}
  \textbf{58 [XXVIII]}
  \addcontentsline{toc}{section}{\emph{58.\ Quand, parmi l'ombre obscure}}
\end{center}

\begin{verse}
  Quand, parmi l'ombre obscure, \\
  une voix perdue murmure, \\
  troublant sa triste paix; \\
  si, au fond de mon âme, \\
  je l'entends résonner doucement,

  dis-moi: est-ce le vent virevoltant \\
  qui se plaint, ou bien tes soupirs \\
  me parlent-ils d'amour en passant?

  Quand le soleil à ma fenêtre \\
  brille rouge au matin, \\
  et mon amour évoque ton ombre; \\
  si sur ma bouche je crois sentir \\
  l'impression d'une autre bouche,

  dis-moi: est-ce que je délire aveuglément, \\
  ou bien un baiser m'envoie-t-il ton cœur \\
  dans un soupir?

  Et, dans le jour lumineux \\
  et la pleine nuit noire, \\
  si, dans tout ce qui entoure \\
  mon âme qui te désire, \\
  je crois te sentir et voir,

  dis-moi: est-ce que je touche et respire \\
  en rêve, ou est-ce que, dans un soupir, \\
  tu me donnes ton haleine à boire?
\end{verse}

%\pagebreak

\begin{center}
  \textbf{59 [LXX]}
  \addcontentsline{toc}{section}{\emph{59.\ Combien de fois, au
    pied des murs moussus qui la gardent}}
\end{center}

\begin{verse}
  Combien de fois, \\
  au pied des murs moussus qui la gardent, \\
  n'ai-je entendu la clochette au creux de la nuit \\
  convoquer aux matines?

  Combien de fois la lune argentée traça \\
  ma triste silhouette jointe à celle du cyprès \\
  se hissant par dessus la muraille de son verger?

  Quand l'église se drapait d'ombres, \\
  combien de fois n'ai-je vu trembler \\
  l'éclat de la lampe \\
  sur les vitraux de son ogive ajourée?

  Bien que le vent sifflât \\
  dans les angles obscurs de la tour, \\
  je percevais sa voix vibrante et claire \\
  parmi les voix du chœur.\footnote{Cette rime illustre le thème de
  la novice hors d'atteinte. Voir les rimes~9 et~24.}

  Dans les nuits d'hiver, si un poltron \\
  osait traverser la place déserte, \\
  il hâtait son pas
  quand il m'apercevait.

  Et il ne manqua pas une vieille qui ne racontât \\
  au matin suivant
  que j'étais l'âme \\
  de quelque sacristain mort en pécheur.

  À tâtons, je connaissais les recoins \\
  de l'atrium et de la façade; \\
  les orties qui poussent là-bas \\
  peut-être gardent les empreintes de mes pieds.

  Les hiboux effrayés, qui me suivaient \\
  de leurs yeux de flammes, \\
  finirent par me considérer \\
  comme un bon camarade, avec le temps. \\

  À mon côté, les reptiles sans peur \\
  avançaient en rampant: \\
  je crois que même les saints de granit muets \\
  me saluaient!
\end{verse}

%\bigskip

\begin{center}
  \textbf{60 [XV]}
  \addcontentsline{toc}{section}{\emph{60.\ Voile flottant de brume légère}}
\end{center}

\vspace*{-14pt}

\begin{verse}
  Voile flottant de brume légère, \\
  ruban plissé de blanche écume, \\
  rumeur sonore
  d'une harpe d'or, \\
  baiser du zéphyr\footnote{\emph{aura}, dans la poésie espagnole du XIX\ieme{}
  siècle, désignait un vent doux. Nous traduisons parfois par
  \emph{brise}. Voir rimes~5 et~27.}, onde de lumière, \\
  tu es cela.

  Toi, ombre aérienne qui t'évanouis \\
  quand je crois enfin te saisir. \\
  Comme la flamme, comme le son, \\
  comme la brume, \\
  comme le gémissement du lac bleu!

  En mer, onde sonore sans rivages; \\
  dans le vide, comète errante, \\
  longue complainte
  du vent rauque, \\
  soif perpétuelle de mieux, \\
  je suis cela.

  Moi, qui dans mon agonie, vers tes yeux \\
  tourne mes yeux jour et nuit; \\
  moi, qui, infatigable et dément, \\
  cours après une ombre, \\
  la fille ardente d'une vision!
\end{verse}

\pagebreak

\begin{center}
  \textbf{61 [LXVIII]}
  \addcontentsline{toc}{section}{\emph{61.\ Je ne sais ce que j'ai
    rêvé la nuit dernière}}
\end{center}

\begin{verse}
  Je ne sais ce que j'ai rêvé
  la nuit dernière. \\
  Triste, très triste dû être le rêve, \\
  car, éveillé, l'angoisse perdurait.

  En reprenant corps je notai
  l'oreiller humide \\
  et, pour la première fois, je sentis en le notant \\
  mon âme s'emplir d'un plaisir amer.

  Triste affaire qu'un rêve \\
  qui nous arrache des pleurs; \\
  mais j'ai une joie dans ma tristesse: \\
  je sais qu'il me reste encore des larmes.
\end{verse}

%\bigskip

\begin{center}
  \textbf{62 [V]}
  \addcontentsline{toc}{section}{\emph{62.\ Esprit sans nom,
    indéfinissable essence}}
\end{center}

\begin{verse}
  Esprit sans nom, \\
  indéfinissable essence, \\
  je vis avec la vie \\
  sans formes de l'idée.

  Je nage dans le vide, \\
  tremble dans le brasier solaire, \\
  je palpite parmi les ombres \\
  et flotte avec les brumes.

  Je suis la frange d'or \\
  de la lointaine étoile, \\
  je suis de la haute lune \\
  la lumière tiède et sereine.

  Je suis l'ardent nuage \\
  qui ondoie dans le couchant, \\
  je suis de l'astre errant \\
  le sillage lumineux.

  Je suis neige sur les cimes, \\
  je suis feu sur les sables, \\
  onde bleue sur les mers \\
  et écume sur les rivages.

  Dans le luth je suis note, \\
  parfum dans la violette, \\
  flamme fugace dans les tombes \\
  et lierre parmi les ruines.

  Je chante avec l'alouette \\
  et bourdonne avec l'abeille; \\
  j'imite les bruits \\
  qui résonnent en pleine nuit.\footnote{Ce quatrain ne
figure pas dans le manuscrit original, mais dans la publication dans
le journal \emph{El Museo Universal}, page~31, le~\oldstylenums{28}
janvier \oldstylenums{1866} (voir \url{prensahistorica.mcu.es}).}

  Je tonne dans le torrent, \\
  et siffle dans la foudre, \\
  et aveugle dans l'éclair, \\
  et rugis dans la tempête.

  Je ris sur les collines, \\
  susurre dans les herbes hautes, \\
  soupire dans l'onde pure, \\
  et pleure sur les feuilles sèches.

  J'ondule avec les atomes \\
  de la fumée qui s'élève \\
  et monte lentement au ciel \\
  en spirales immenses.

  Des fils dorés \\
  que les insectes suspendent \\
  aux arbres, je me berce \\
  d'une ardente sieste.

  Je cours après les nymphes \\
  qui, dans le courant frais\footnote{La publication dans
le journal \emph{El Museo Universal}, page~31, le~\oldstylenums{28} janvier \oldstylenums{1866}
(voir \url{prensahistorica.mcu.es}) recueille: «\,le courant inquiet\,».} \\
  de la rivière cristalline, \\
  s'ébattent nues.

  Dans des bois de coraux \\
  qui tapissent de blanches perles, \\
  je poursuis dans l'océan \\
  les naïades légères.

  Dans les cavernes concaves \\
  où le soleil ne pénètre jamais, \\
  me mêlant aux gnomes, \\
  je contemple leurs richesses.

  Je cherche des siècles \\
  les traces effacées, \\
  et je sais de ces empires \\
  dont il ne reste même pas le nom.\footnote{Variante dans le journal
  \emph{El Museo Universal}, page~31, le~\oldstylenums{28} janvier \oldstylenums{1866} (voir
  \url{prensahistorica.mcu.es}): «\,Je rencontre les traces effacées~/~de ces siècles,~/~dont il ne reste aucun souvenir~/~sur la face du globe.\,»}

  Je poursuis en un brusque vertige \\
  les mondes qui voltigent, \\
  et ma pupille embrasse \\
  la création entière.\footnote{Variante dans le journal
  \emph{El Museo Universal}, page~31, le~\oldstylenums{28} janvier \oldstylenums{1866} (voir
  \url{prensahistorica.mcu.es}): «\,J'embrasse du regard~/~la création
  entière,~/~et poursuis en un brusque vertige~/~les astres qui voltigent.\,»}

  Je sais de ces régions \\
  qu'une rumeur n'atteint pas, \\
  et où d'informes astres \\
  attendent un souffle de vie.

  Je suis sur l'abîme \\
  le pont qui traverse, \\
  et l'échelle inconnue \\
  qui unit le ciel à la terre.\footnote{Variante dans le journal
  \emph{El Museo Universal}, page~31, le~\oldstylenums{28} janvier \oldstylenums{1866} (voir
  \url{prensahistorica.mcu.es}): «\,Je suis l'échelle inconnue~/~qui unit
  le ciel à la terre,~/~et ouvre à la pensée~/~un chemin vers d'autres
  sphères.\,»}

  Je suis l'anneau invisible \\
  qui assujettit \\
  le monde de la forme \\
  au monde de l'idée.

  Enfin, je suis cet esprit, \\
  essence inconnue,\footnote{Variante dans le journal
  \emph{El Museo Universal}, page~31, le~\oldstylenums{28} janvier \oldstylenums{1866} (voir
  \url{prensahistorica.mcu.es}): «\,l'essence du sentiment,\,»} \\
  parfum mystérieux \\
  dont le vase est le poète.
\end{verse}

%\bigskip

\begin{center}
  \textbf{63 [XXVII]}
  \addcontentsline{toc}{section}{\emph{63.\ Éveillée, je tremble à ta vue}}
\end{center}

\begin{verse}
  Éveillée, je tremble à ta vue; \\
  assoupie, j'ose te regarder;\footnote{On retrouve la figure de la femme endormie à la rime~31.} \\
  c'est pour cela, âme de mon âme, \\
  que je veille pendant que tu dors.

  Éveillée, tu ris et, en riant, tes lèvres \\
  inquiètes me semblent \\
  des éclairs carmins qui serpentent \\
  sur un ciel enneigé.

  Assoupie, un léger sourire plisse \\
  les bords de ta bouche, \\
  suave comme le sillage brillant \\
  que laisse un soleil mourant...

  Dors!

  Éveillée, tu regardes et, en regardant, tes yeux \\
  humides resplendissent \\
  comme la vague bleue dont la crête \\
  est illuminée par un soleil étincelant.

  Au travers de tes paupières, assoupie, \\
  ils déversent un éclat calme, \\
  comme la lueur tiède que répand \\
  une lampe transparente...

  Dors!

  Éveillée, tu parles et, en parlant, \\
  tes paroles vibrantes semblent \\
  une pluie de perles se déversant à torrents \\
  dans une coupe dorée.

  Assoupie, dans le murmure de ton haleine \\
  rythmée et ténue, \\
  j'entends un poème que mon âme \\
  amoureuse comprend...

  Dors!

  J'ai posé une main sur mon cœur \\
  pour que son battement \\
  ne résonne et ne trouble \\
  le calme solennel de la nuit.

  J'ai fermé enfin les persiennes
  de ton balcon \\
  pour que le flamboiement fâcheux \\
  de l'aurore n'entre et ne t'éveille...

  Dors!
\end{verse}

%\bigskip

\begin{center}
  \textbf{64 [LXIV]}
  \addcontentsline{toc}{section}{\emph{64.\ Comme l'avare garde son
    trésor, je gardais ma douleur}}
\end{center}

\begin{verse}
  Comme l'avare garde son trésor, \\
  je gardais ma douleur; \\
  je voulais prouver que l'éternel existe \\
  à celle qui me jura un amour éternel.

  Mais aujourd'hui je l'appelle en vain et le Temps,\hspace*{-10pt} \\
  qui l'épuisa, me dit: \\
  \emph{Ah, boue misérable! \\
  Éternellement tu ne saurais même souffrir!}
\end{verse}

\bigskip

\begin{center}
  \textbf{65 [XXXIV]}
  \addcontentsline{toc}{section}{\emph{65.\ Muette, elle traverse
      et ses mouvements sont harmonie}}
\end{center}

\begin{verse}
  Muette, elle traverse et ses mouvements \\
  sont harmonie silencieuse; \\
  ses pas bruissent et, en bruissant, ils évoquent \\
  la cadence rythmée d'un hymne ailé.

  Elle entr'ouvre les yeux, ces yeux \\
  aussi clairs que le jour, \\
  et la terre et le ciel, ce qu'ils embrassent, \\
  flamboient d'un nouvel éclat dans ses pupilles.

  Elle rie, et ses éclats de rire ont les notes \\
  de l'eau fugitive; \\
  elle pleure, et chaque larme est un poème \\
  de tendresse infinie.

  Elle a la lumière, elle a le parfum, \\
  la couleur et la ligne, \\
  la forme qui engendre les désirs, \\
  l'expression, source éternelle de poésie.

  Qu'elle est stupide?\footnote{Cet éloge de la beauté plastique
  féminine qui prime se retrouve à la rime~75.} Bah! \\
  Tant qu'en se taisant elle garde l'énigme secrète, \\
  toujours vaudra ce que je crois qu'elle tait \\
  plus que ce qu'aucune autre ne me dirait.
\end{verse}

%\bigskip

\begin{center}
  \textbf{66 [XL]}
  \addcontentsline{toc}{section}{\emph{66.\ Sa main dans mes mains}}
\end{center}

\begin{verse}
  Sa main dans ma main,
  ses yeux dans mes yeux,\\
  la tête amoureuse
  appuyée sur mon épaule, \\
  Dieu sait combien de fois,
  d'un pas alangui, \\
  nous avons erré ensemble
  sous les grands ormes\hspace*{-10pt} \\
  qui prêtent mystère et ombre \\
  au porche de sa maison.

  Et hier..., un an à peine
  passé en coup de vent, \\
  avec quelle exquise grâce, \\
  avec quel admirable aplomb, \\
  elle me dit, me présentant
  quelque ami officieux:\hspace*{-10pt}\\
  « Je crois qu'en quelque endroit
  je vous ai vu.\,»

  Ah! Sots qui êtes des salons \\
  commères de bon ton \\
  et marchiez là en quête
  de galants imbroglios: \\
  quelle histoire vous avez manquée! \\
  Quelle ambroisie à dévorer \\
  \emph{sotto voce} parmi un cercle, \\
  derrière l'éventail de plumes et d'or!
\end{verse}

\[\star \ \ \ \star \ \ \ \star\]

\begin{verse}
  Lune discrète et chaste, \\
  ormes touffus et grands, \\
  murs de sa demeure, \\
  seuils de son porche, \\
  taisez-vous, et que le secret \\
  ne vous abandonne! \\
  Taisez-vous, pour ma part \\
  j'ai tout oublié; \\
  et elle..., elle, il n'y a de masque \\
  semblable à son visage!
\end{verse}

\bigskip

\begin{center}
  \textbf{67 [LXVI]}
  \addcontentsline{toc}{section}{\emph{67.\ D'où viens-je? Cherche
    le plus horrible et âpre des sentiers}}
\end{center}

\begin{verse}
  D'où viens-je? Cherche le plus \\
  horrible et âpre des sentiers; \\
  des empreintes de pieds ensanglantés \\
  sur la roche dure; \\
  les restes d'une âme en lambeaux \\
  dans les ronces acérées: \\
  ils te diront le chemin \\
  qui conduit à mon berceau.

  Où vais-je? Traverse le plus \\
  sombre et triste des plateaux, \\
  ou une vallée de neiges éternelles \\
  et de brumes mélancoliques. \\
  Où se trouve une pierre solitaire \\
  sans aucune inscription, \\
  où habite l'oubli: \\
  là se trouvera ma tombe.
\end{verse}

%\bigskip

\begin{center}
  \textbf{68 [LXIII]}
  \addcontentsline{toc}{section}{\emph{68.\ Comme un essaim
    d'abeilles irritées}}
\end{center}

\begin{verse}
  Comme un essaim d'abeilles irritées, \\
  les souvenirs des heures passées \\
  sortent d'un recoin sombre de la mémoire \\
  pour me poursuivre.

  Je veux les chasser. Effort inutile! \\
  Ils m'encerclent, me harcèlent, \\
  et, l'un après l'autre, ils viennent planter \\
  le fin aiguillon qui envenime l'âme.
\end{verse}

%\bigskip

\begin{center}
  \textbf{69 [XXXIII]}
  \addcontentsline{toc}{section}{\emph{69.\ C'est une question de mots, et pourtant}}
\end{center}

\begin{verse}
  C'est une question de mots, et pourtant \\
  ni toi ni moi, jamais, \\
  après ce qui advint, ne conviendra \\
  à qui la faute incombe.

  Quel dommage que l'Amour n'ait \\
  de dictionnaire à consulter \\
  quand l'orgueil est simplement orgueil \\
  et quand il est dignité!
\end{verse}

%\pagebreak

\begin{center}
  \textbf{70 [LI]}
  \addcontentsline{toc}{section}{\emph{70.\ Du peu de vie qu'il me
    reste}}
\end{center}

\begin{verse}
  Du peu de vie qu'il me reste, \\
  je donnerais volontiers les meilleures années \\
  pour savoir ce que tu as raconté \\
  de moi à d'autres.

  Et cette vie mortelle, et de l'éternelle \\
  ce qu'il me revienne ---~s'il m'en revient~--- \\
  pour savoir ce que, seule, \\
  de moi tu as pensé.
\end{verse}

%\bigskip

\begin{center}
  \textbf{71 [LXXIII]}
  \addcontentsline{toc}{section}{\emph{71.\ On ferma ses yeux
    qu'elle avait encore ouverts}}
\end{center}

\begin{verse}
  On ferma ses yeux
  qu'elle avait encore ouverts, \\
  on couvrit son visage
  d'une étoffe blanche, \\
  et d'aucuns sanglotant,
  d'autres silencieux, \\
  tous sortirent
  de la triste alcôve.

  La lumière, qui flamboyait
  dans un vase au sol, \\
  projetait sur le mur
  l'ombre de la couche, \\
  et parmi cette ombre
  on voyait, par intervalles, \\
  se dessiner, rigide,
  la forme du corps.

  Le jour s'éveillait,
  et à la première lueur, \\
  il réveillait le village
  de ses mille bruits. \\
  Devant ce contraste
  de vie et mystère, \\
  de lumière et ténèbres,
  je pensai un moment:

  \emph{Mon Dieu, oh combien seuls restent les morts!}

  Sur les épaules on la porta
  de la maison à l'église,\hspace*{-10pt} \\
  et on laissa le cercueil
  dans une chapelle. \\
  Là-bas on entoura
  sa pâle dépouille \\
  de cierges jaunes
  et d'étoffes noires.

  En sonnant des Âmes\footnote{Service nocturne pendant lequel les
  fidèles prient pour les âmes des défunts.}
  la dernière cloche, \\
  une vieille acheva
  ses ultimes prières; \\
  elle traversa la large nef,
  les portes gémirent \\
  et le saint lieu
  resta désert.

  D'une horloge tintait
  le balancier mesuré \\
  et, de quelques cierges,
  le crépitement. \\
  Tout était
  si craintif et triste, \\
  si obscur et transi,
  que je pensai un moment:

  \emph{Mon Dieu, oh combien seuls restent les morts!}

  La langue de fer
  de la haute cloche \\
  lui dédia une volée
  d'adieux plaintifs. \\
  Le deuil aux habits,
  amis et proches \\
  passèrent en file,
  formant le cortège.

  Le pic ouvrit la niche
  à une extrémité \\
  de l'ultime asile,
  obscur et étroit. \\
  Là, on la coucha
  et puis la mura, \\
  et, avec un salut,
  le cortège se retira.

  Le pic sur l'épaule,
  le fossoyeur, \\
  chantonnant dans sa barbe,
  se perdit au loin. \\
  La nuit s'avançait,
  le soleil s'était couché; \\
  perdu parmi les ombres,
  je pensai un moment:

  \emph{Mon Dieu, oh combien seuls restent les morts!}

  Dans les longues nuits
  de l'hiver glacé, \\
  quand le vent
  fait craquer les bois \\
  et la forte averse
  fouette les carreaux, \\
  je me souviens parfois
  de la pauvre enfant.

  Là-bas la pluie tombe
  d'un bruit éternel; \\
  là-bas le souffle de la bise
  la combat. \\
  Étendue dans le creux
  du mur humide, \\
  peut-être ses os se gèlent
  de froid...

La poussière retourne-t-elle à la poussière? \\
  L'âme s'envole-t-elle au ciel? \\
  Tout est-il sans âme,
  corruption et bourbe? \\
  Je ne sais; mais il y a
  quelque chose \\
  que je ne m'explique pas,
  quelque chose qui, \\
  bien qu'il soit courageux de le faire, \\
  répugne à laisser si tristes,
  si seuls, les morts!
\end{verse}



\begin{center}
  \textbf{72 [XIV]}
  \addcontentsline{toc}{section}{\emph{72.\ Je t'entrevis et l'image de tes yeux persista}}
\end{center}

\begin{verse}
  Je t'entrevis et l'image de tes yeux persista, \\
  flottant devant mes yeux \\
  comme la tâche sombre bordée de feu \\
  qui flotte et aveugle si l'on fixe le soleil.

  Et où que je pose le regard \\
  je revois tes iris flamboyer, \\
  mais tu n'es pas là; c'est ton regard, \\
  des yeux, les tiens; rien de plus.

  Dans l'angle de mon alcôve, je les regarde \\
  luire, détachés, fantastiques; \\
  quand je dors, je les sens m'examiner, \\
  grand ouverts sur moi.

  Je sais qu'il est des feux follets la nuit \\
  qui mènent le voyageur à sa perte; \\
  moi, je me sens entraîné par tes yeux, \\
  mais où ils m'entraînent, je ne le sais.
\end{verse}

%\bigskip

\begin{center}
  \textbf{73 [XXXII]}
  \addcontentsline{toc}{section}{\emph{73.\ Elle passait, irrésistible dans sa splendeur}}
\end{center}

\begin{verse}
  Elle passait, irrésistible dans sa splendeur, \\
  et je lui cédai le pas; \\
  je poursuivis sans me retourner, et pourtant \\
  j'entendis murmurer à mon oreille: \emph{«\,C'est elle.\,»}

  Qui unit le soir au matin? \\
  Je l'ignore: je sais seulement \\
  que lors d'une brève nuit d'été \\
  s'unirent les crépuscules et... \emph{ainsi fut-il}.\footnote{Le
  thème de l'amour d'un soir se retrouve aux rimes~8 et~9, qui offrent
  un contraste à l'image d'Épinal d'un poète transi d'amour.}
\end{verse}



\begin{center}
  \textbf{74 [LXXVI]}
  \addcontentsline{toc}{section}{\emph{74.\ Dans l'imposante nef de
    l'église romane}}
\end{center}

\begin{verse}
  Dans l'imposante nef
  de l'église romane,\footnote{Dans l'original figure \emph{templo bizantino}
  (temple byzantin), mais le \emph{Centro Virtual Cervantes}
  (\url{https://cvc.cervantes.es/obref/rimas/rimas/}), dans son
  commentaire de cette rime, indique que Bécquer, comme beaucoup de
  ses contemporains, utilisait \emph{bizantino} pour dire \emph{románico}
  (roman). Ailleurs, Bécquer utilise très souvent \emph{templo} (temple)
  en lieu de «\,église\,» ---~d'où notre traduction.} \\
  je vis la tombe gothique à la lueur \\
  indécise qui tremblait sur les vitraux.

  Les mains sur la poitrine, \\
  et dans les mains un livre, \\
  une belle femme reposait \\
  sur le sarcophage, prodige du ciseau.\footnote{Voir aussi l'image de
  la femme-statue à la rime~75.}

  La couche de granit \\
  ployait du poids doux \\
  de son corps abandonné, \\
  comme de tendre plume et satin.

  Son visage gardait le divin éclat \\
  de l'ultime sourire,
  comme le ciel garde \\
  du soleil qui meurt le rai fugitif.

  Assis sur le bord
  de l'oreiller de pierre, \\
  deux anges, le doigt sur la lèvre, \\
  imposaient silence à l'enceinte.

  Elle ne semblait pas morte: \\
  on l'aurait dit dormant \\
  dans la pénombre des arcs massifs \\
  et contemplant le paradis en songe.

  Je m'approchai
  de l'angle sombre de la nef, \\
  du pas retenu de qui vient \\
  au berceau d'un enfant endormi.

  Je la contemplai un moment, \\
  cet éclat tiède,
  ce lit de pierre qui offrait \\
  un autre lieu vide proche du mur.

  Dans l'âme s'avivèrent
  la soif de l'infini, \\
  le désir de cette vie de la mort, \\
  pour laquelle les siècles sont un instant...
\end{verse}

\[\star \ \ \ \star \ \ \ \star\]

\begin{verse}
  Fatigué du combat
  dans lequel je lutte, \\
  parfois je me souviens avec envie \\
  de ce recoin obscur et caché.

  De cette femme silencieuse et pâle \\
  je me souviens et dis: \\
  «\,Oh, quel amour sans paroles que celui de la mort! \\
    Quel sommeil, celui du sépulcre si calme!\,»
\end{verse}

%\pagebreak

\begin{center}
  \textbf{75 [XXXIX]}
  \addcontentsline{toc}{section}{\emph{75.\ Pourquoi me le dire?}}
\end{center}

\begin{verse}
  Pourquoi me le dire? Je sais: elle est changeante,\hspace*{-10pt} \\
  altière et vaine et capricieuse; \\
  l'eau jaillirait d'une roche stérile \\
  avant qu'un sentiment ne jaillisse de son âme.

  Je sais qu'en son cœur, nid de serpents, \\
  il n'y a de fibre qui réponde à l'amour; \\
  qu'elle est une statue inanimée...\footnote{Voir aussi l'image de la
  femme-statue à la rime~74.} \\
  mais elle est si belle!\footnote{Cet éloge de la beauté plastique
  féminine qui prime se retrouve à la rime~65.}
\end{verse}

\bigskip

\begin{center}
  \textbf{76 [LXXI]}
  \addcontentsline{toc}{section}{\emph{76.\ Je ne dormais pas,
    errant dans la limbe}}
\end{center}

\begin{verse}
  Je ne dormais pas, errant dans la limbe \\
  où les objets changent de forme, \\
  espaces mystérieux qui séparent \\
  la veille du sommeil.

  Les idées, qui en rondes silencieuses \\
  tournaient dans mon cerveau, \\
  bougeaient peu à peu en leur danse \\
  d'un rythme plus lent.

  Les paupières voilaient le reflet \\
  de la lumière qui parvient à l'âme par les yeux, \\
  mais le monde de visions \\
  allumait à l'intérieur une autre lumière.

  À ce moment résonna dans mon oreille \\
  une rumeur comme celle qui, à l'église, \\
  erre confusément quand les fidèles terminent \\
  leurs prières d'un \emph{Amen}.

  Et j'entendis comme une voix fine et triste \\
  qui m'appela de loin par mon nom, \\
  et je sentis une odeur de cierges éteints, \\
  d'humidité et d'encens.

  La nuit entra et, dans les bras de l'oubli, \\
  je tombai tel une pierre en son sein profond. \\
  Je dormis et au réveil je m'exclamai: \\
  «\,Quelqu'un que j'aimais est mort!\,».
\end{verse}

\bigskip

\begin{center}
  \textbf{77 [XLVI]}
  \addcontentsline{toc}{section}{\emph{77.\ Elle m'a blessé en
    se retirant dans l'ombre}}
\end{center}

%\vspace*{-10pt}

\begin{verse}
  Elle m'a blessé en se retirant dans l'ombre, \\
  scellant d'un baiser sa trahison. \\
  Elle se pendit à mon cou et, dans le dos, \\
  elle me brisa le cœur de sang froid.

  Et elle poursuit, joyeuse, son chemin, \\
  heureuse, gaie, impavide; et pourquoi? \\
  Parce que la blessure ne saigne pas, \\
  parce que le mort est debout.\footnote{Le thème de la blessure
  par arme blanche est récurrent chez Bécquer. Voir les rimes~1, 16
  et~28.}
\end{verse}

\bigskip

\begin{center}
  \textbf{78 [XXXV]}
  \addcontentsline{toc}{section}{\emph{78.\ Ton oubli ne m'admira pas!}}
\end{center}

%\vspace*{-10pt}

\begin{verse}
  Ton oubli ne m'admira pas, bien que \\
  ta tendresse m'admira bien plus qu'un jour, \\
  car ce qui en moi a de la valeur, \\
  cela... tu ne le soupçonnas même pas.
\end{verse}

\bigskip

\begin{center}
  \textbf{79 [XII]}
  \addcontentsline{toc}{section}{\emph{79.\ Petite, parce que tes
    yeux sont verts}}
\end{center}

%\vspace*{-10pt}

\begin{verse}
  Petite, parce que tes yeux \\
  sont verts comme la mer, tu te plains; \\
  verts sont ceux des naïades, \\
  verts les eut Minerve,\footnote{Déesse romaine assimilée au cours de
  l'Histoire à la déesse grecque Athéna qui avait les yeux pers,
  c'est-à-dire une couleur où le bleu domine, par exemple bleu-vert.} \\
  et verts sont les iris \\
  des houris\footnote{Beautés célestes que le Coran promet au musulman
  dans le paradis d'Allah. Elle ont de grands yeux noirs.} du
  Prophète.

  Le vert est gala et ornement \\
  de la forêt au printemps; \\
  parmi ses sept couleurs, \\
  l'iris brillant l'affiche; \\
  les émeraudes sont vertes, \\
  verte est la couleur de qui espère, \\
  et les ondes de l'océan, \\
  et le laurier des poètes.

  Ta joue est une rose matinale \\
  couverte de rosée congelée, \\
  où le carmin des pétales \\
  se voit à travers des perles.

  Et pourtant,
  je sais que tu te plains \\
  car tu crois que tes yeux
  l'enlaidissent: \\
  eh bien ne le crois pas,

  car tes iris humides,
  verts et inquiets, \\
  semblent de jeunes feuilles d'amandier \\
  tremblant dans la brise.

  Ta bouche pourpre-rubis \\
  est une grenade entrouverte \\
  qui, à l'été,
  à éteindre la soif en elle.

  Et pourtant,
  je sais que tu te plains \\
  car tu crois que tes yeux
  l'enlaidissent: \\
  eh bien ne le crois pas,

  car, si fâchée,
  tes iris scintillent, \\
  tes yeux ressemblent
  aux vagues se brisant \\
  sur les rochers cantabriques.

  Ton front, couronné \\
  de l'or crépu d'une large tresse, \\
  est une cime enneigée où le jour \\
  reflète sa première lueur.

  Et pourtant,
  je sais que tu te plains \\
  car tu crois que tes yeux
  l'enlaidissent: \\
  eh bien ne le crois pas,

  car parmi les cils blonds, \\
  proche des tempes, ils semblent \\
  des broches d'émeraude et or \\
  haussant une blanche hermine.

  Petite, parce que tes yeux \\
  sont verts comme la mer, tu te plains; \\
  peut-être, si noirs ou bleus \\
  ils devenaient, tu le regretterais.
\end{verse}
