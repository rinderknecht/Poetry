\emph{Las granadas} es el título de unos de los poemas de Paul Valéry
que recoge esta antología en versos rimados. Ilustra unos de los temas
característicos del autor, la introspección en búsqueda de la fuente
del pensamiento puro, y cómo aquél entiende el sentir que brota.

A la confluencia de los poetas simbolistas y clásicos, por sus
referencias a la cultura griega antigua y al mundo de las Ideas,
Valéry es sumamente perfeccionista en su estilo y las sonoridades de
sus versos. Su métrica refleja sin embargo modernas consideraciones en
que sirve el ritmo sin encerrar el poeta en una forma rígida.

Elegimos piezas cortas entre las menos cargadas con símbolos, para
alcanzar y cautivar un público moderno y joven.

El silencio y las exclamaciones, incluso en los diálogos, resuenan en
el interior del poeta y del lector. Cada uno de estos poemas se puede
concebir como una granada, un fruto de contemplación y de
descubrimiento.
