%%-*-latex-*-

\addcontentsline{toc}{subsection}{Érase una vez (\emph{Au bois dormant})}

\selectlanguage{spanish}

\poemtitle*{Al bosque durmiente}
%\poemtitle*{Érase una vez}

\settowidth{\versewidth}{flautas, cercenar el rumor de frases de cuerno.}

\bigskip

\begin{verse}[\versewidth]
  La princesa, en un palacio de pura rosa, \\
  en la móvil sombra duerme bajo los murmullos; \\
  de coral una palabra oscura esboza \\
  al picotear aves perdidas sus anillos.

  No escucha en largas caídas ni las gotas \\
  tañer el tesoro hueco del siglo hodierno, \\
  ni, sobre el bosque vago, un viento de notas, \\
  flautas, cercenar el rumor de frases de cuerno.

  Deja el eco dormir de nuevo la diana, \\
  oh cada vez más igual a la blanda liana \\
  que se mece y bate tus ojos sepultados.

  Cerca de tu mejilla ha llegado la rosa, \\
  no disipes esa delicia de los plegados \\
  sensibles en secreto al rayo que se posa.
\end{verse}

\newpage

\selectlanguage{french}

\poemtitle*{\emph{Au bois dormant}}

\settowidth{\versewidth}{Elle n'écoute ni les gouttes, dans leurs chutes,}

\bigskip

{\itshape
\begin{verse}[\versewidth]
  La princesse, dans un palais de rose pure, \\
  sous les murmures, sous la mobile ombre dort; \\
  et de corail ébauche une parole obscure \\
  quand les oiseaux perdus mordent ses bagues d'or.

  Elle n'écoute ni les gouttes, dans leurs chutes, \\
  tinter d'un siècle vide au lointain le trésor, \\
  ni, sur la forêt vague, un vent fondu de flûtes \\
  déchirer la rumeur d'une phrase de cor.

  Laisse, longue, l'écho rendormir la diane, \\
  ô toujours plus égale à la molle liane \\
  qui se balance et bat tes yeux ensevelis.

  Si proche de ta joue et si lente la rose \\
  ne vas pas dissiper ce délice de plis \\
  secrètement sensible au rayon qui s'y pose.
\end{verse}
}
