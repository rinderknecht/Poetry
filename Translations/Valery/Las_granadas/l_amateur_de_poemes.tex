%-*-latex-*-

\addcontentsline{toc}{subsection}{El aficionado de poesia (\emph{L{}'amateur de poèmes})}

\selectlanguage{spanish}

\poemtitle*{El aficionado de poesía}

\bigskip

Si miro de repente mi verdadero pensar, no me consuelo de deber sufrir
esta palabra interior sin persona y sin origen; esas figuras efímeras;
y esa infinidad de intentos interrumpidos por su propia facilidad, que
se transforman uno en otro, sin que nada cambie con ellos. Incoherente
sin parecerlo, nulo instantáneamente como es espontáneo, el
pensamiento, por su índole, carece de estilo.

Mas no tengo todos los días la potencia de proponer a mi atención
algunos seres necesarios, ni de fingir los obstáculos espirituales que
formarian una apariencia de comienzo, de plenitud y de fin, en lugar
de mi insoportable huida.

Un poema es una duración, durante la cual, lector, respiro una ley que
fue preparada; doy mi soplo y las máquinas de mi voz; o sólo su poder,
que se concilia con el silencio.

Yo me abandono al encantador paso: leer, vivir donde llevan las
palabras. Su aparición está escrita. Sus sonoridades concertadas. Su
estremecimiento se compone, según una meditación anterior, y se
precipitarán en grupos magníficos o puros, en la resonancia. Incluso
mis asombros son seguros: están escondidos de antemano, y forman parte
del número.

Movido por la escritura fatal, y si el metro siempre por venir
encadena sin vuelta mi memoria, siento cada palabra con toda su
fuerza, por haberla indefinidamente esperado. Esta medida que me
arrebata y que impregno, me preserva de lo verdadero y de lo falso. Ni
la duda me divide, ni la razón me agita. Ningún azar, mas una suerte
extraordinaria me fortifica. Encuentro sin esfuerzo el lenguage de esa
felicidad; y pienso por artificio, un pensamiento todo seguro,
maravillosamente previsor, --- de lagunas calculadas, sin tinieblas
involuntarias, cuyo movimiento me domina y la cantidad me colma: un
pensamiento singularmente acabado.

\newpage

\selectlanguage{french}

\poemtitle*{\emph{L{}'amateur de poèmes}}

\bigskip

{\itshape
SI je regarde tout à coup ma véritable pensée, je ne me console
pas de devoir subir cette parole intérieure sans personne et sans
origine; ces figures éphémères; et cette infinité
d'entreprises interrompues par leur propre facilité, qui se
transforment l'une dans l'autre, sans que rien ne change avec
elles. Incohérente sans le paraître, nulle instantanément
comme elle est spontanée, la pensée, par sa nature, manque de
style.

MAIS je n'ai pas tous les jours la puissance de proposer à mon
attention quelques êtres nécessaires, ni de feindre les obstacles
spirituels qui formeraient une apparence de commencement, de
plénitude et de fin, au lieu de mon insupportable fuite.

UN poème est une durée, pendant laquelle, lecteur, je respire une
loi qui fut préparée; je donne mon souffle et les machines de ma
voix; ou seulement leur pouvoir, qui se concilie avec le silence.

JE m'abandonne à l'adorable allure: lire, vivre où mènent les
mots. Leur apparition est écrite. Leurs sonorités concertées. Leur
ébranlement se compose, d'après une méditation antérieure, et
ils se précipiteront en groupes magnifiques ou purs, dans la
résonance. Même mes étonnements sont assurés: ils sont cachés
d'avance, et font partie du nombre.

MU par l'écriture fatale, et si le mètre toujours futur enchaîne sans
retour ma mémoire, je ressens chaque parole dans toute sa force, pour
l'avoir indéfiniment attendue. Cette mesure qui me transporte et que
je colore, me garde du vrai et du faux. Ni le doute ne me divise, ni
la raison ne me travaille. Nul hasard, mais une chance extraordinaire
se fortifie. Je trouve sans effort le langage de ce bonheur; et je
pense par artifice, une pensée toute certaine, merveilleusement
prévoyante, --- aux lacunes calculées, sans ténèbres involontaires,
dont le mouvement me commande et la quantité me comble: une pensée
singulièrement achevée.
}
