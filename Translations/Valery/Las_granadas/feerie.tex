%-*-latex-*-

\addcontentsline{toc}{subsection}{Hadada (\emph{Féerie})}

\selectlanguage{spanish}

\poemtitle*{Hadada}

\settowidth{\versewidth}{Con el umbral secreto de los ecos de cristal...}

\bigskip

\begin{verse}[\versewidth]
  La luna fina vierte su claridad risueña, \\
  falda ligera de visos de plata al moverla, \\
  sobre el mármol donde va la Sombra y sueña \\
  que una gasa nacarada sigue una perla.

  Para los cisnes, sedas que rozan los carrizos \\
  de carenas de pluma a medio luminosa, \\
  deshoja, infinita, una rosa nevosa \\
  cuyos pétalos redondean del agua los rizos...

  ¿Es vivir?... Oh desierto de delicia pasmada \\
  donde muere el latir del agua guarnecida, \\
  con el umbral secreto de los ecos de cristal...

  Y la carne confusa de las rosas comienza \\
  a temblar, si de un grito el diamante fatal \\
  cercena de un rayo la fábula immensa.
\end{verse}

\newpage

\selectlanguage{french}

\poemtitle*{\emph{Féerie}}

\settowidth{\versewidth}{Dont les pétales font des cercles sur les eaux...}

\bigskip

{\itshape
\begin{verse}[\versewidth]
  La lune mince verse une lueur sacrée, \\
  toute une jupe d'un tissu d'argent léger, \\
  sur les bases de marbre où vient l'Ombre songer \\
  que suit d'un char de perle une gaze nacrée.

  Pour les cygnes soyeux qui frôlent les roseaux \\
  de carènes de plume à demi lumineuse, \\
  elle effeuille infinie une rose neigeuse \\
  dont les pétales font des cercles sur les eaux...

  Est-ce vivre?... Ô désert de volupté pâmée \\
  où meurt le battement faible de l'eau lamée, \\
  usant le seuil secret des échos de cristal...

  La chair confuse des molles roses commence \\
  à frémir, si d'un cri le diamant fatal \\
  fêle d'un fil de jour toute la fable immense.
\end{verse}
}
