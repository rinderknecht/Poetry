%-*-latex-*-

\addcontentsline{toc}{subsection}{Coloquio (\emph{Colloque})}

\selectlanguage{spanish}

\poemtitle*{Coloquio\\ (para dos flautas)}

\settowidth{\versewidth}{Muere en nuestros brazos...}

\bigskip

\begin{flushright}
\scriptsize\emph{A Francis Poulenc, que hizo} \\
\emph{cantar este coloquio.}
\end{flushright}

\bigskip

\begin{verse}[\versewidth]
\qquad\qquad\textbf{A} \\ \

De una Rosa menguante \\
el tedio se inclina; \\
sabes, a pesar del guante \\
tu ser no desafina \\
a aquella flor menguante; \\
muere en nuestros brazos... \\
En ti se asemeja \\
aquella cuya oreja \\
jugaba bajo lazos, \\
aquella cuya oreja \\
nunca me escuchaba; \\
en ti se asemeja \\
hoy la que yo amaba: \\
pero jamás olvidaba \\
mis labios rojos y daba.
\\ \

\qquad\qquad\textbf{B} \\ \

¿Y por qué me comparas tú \\
alguna rosa marchita? \\
Sin Venus en su conchita \\
naciente ¿qué harías tú?... \\
Mi mirada zambullida \\
en ti no busca huida, \\
fíjate: ¡me veo desnuda! \\
Lágrimas como rocío \\
limpiaré si el hastío \\
con un recuerdo reanuda... \\
Si tu deseo es roca \\
qué muera en mi lecho; \\
únete a mi pecho \\
¡y me llevaré tu boca!...
\end{verse}

\newpage

\selectlanguage{french}

\poemtitle*{\emph{Colloque}\\ \emph{(pour deux flûtes)}}

\settowidth{\versewidth}{Que fraîche et spontanée...}

\begin{flushright}
\scriptsize\emph{À Francis Poulenc, qui a fait \\
chanter ce colloque.}
\end{flushright}

\bigskip

{\itshape
\begin{verse}[\versewidth]
\qquad\qquad \textbf{A} \\ \

D'une Rose mourante \\
l'ennui penche vers nous; \\
tu n'es pas différente \\
dans ton silence doux \\
de cette fleur mourante; \\
elle se meurt pour nous... \\
Tu me sembles pareille \\
à celle dont l'oreille \\
était sur mes genoux, \\
à celle dont l'oreille \\
ne m'écoutait jamais; \\
tu me sembles pareille \\
à l'autre que j'aimais: \\
mais de celle ancienne, \\
sa bouche était la mienne.
\\ \

\qquad\qquad\textbf{B} \\ \

Que me compares-tu \\
quelque rose fânée? \\
L'amour n'a de vertu \\
que fraîche et spontanée... \\
Mon regard dans le tien \\
ne trouve que son bien: \\
je m'y vois toute nue! \\
Mes yeux effaceront \\
tes larmes qui seront \\
d'un souvenir venues!... \\
Si ton désir naquit \\
qu'il meure sur ma couche \\
et sur mes lèvres qui \\
t'emporteront la bouche...
\end{verse}
}
