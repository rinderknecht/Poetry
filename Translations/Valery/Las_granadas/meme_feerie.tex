%-*-latex-*-

\addcontentsline{toc}{subsection}{Misma hadada (\emph{Même féerie})}

\selectlanguage{spanish}

\poemtitle*{Misma hadada}

\settowidth{\versewidth}{Por la luna cuenta sin fin sus ecos de cristal,}

\bigskip

\begin{verse}[\versewidth]
  La luna fina vierte su claridad risueña, \\
  falda ligera y plateada al moverla, \\
  en el mármol donde anda y cree que sueña \\
  alguna virgen de tul nacarado y perla.

  Para los cisnes, sedas que rozan los carrizos \\
  de carenas de pluma a medio luminosa, \\
  su mano dispensa alguna rosa nevosa \\
  cuyos pétalos redondean del agua los rizos...

  Oh delicioso desierto, soledad pasmada, \\
  si el remolino en el agua guarnecida \\
  por la luna cuenta ecos de cristal sin cese,

  de la noche brillante al firmamento fatal, \\
  ¿qué corazón sufre el hado mágico ese \\
  sin desenvainar un grito como un puñal?
\end{verse}

\newpage

\selectlanguage{french}

\poemtitle*{\emph{Même féerie}}

\bigskip

\settowidth{\versewidth}{Sur les masses de marbre où marche et croit songer}

{\itshape
\begin{verse}[\versewidth]
  La lune mince verse une lueur sacrée, \\
  comme une jupe d'un tissu d'argent léger, \\
  sur les masses de marbre où marche et croit songer \\
  quelque vierge de perle et de gaze nacrée.

  Pour les cygnes soyeux qui frôlent les roseaux \\
  de carènes de plume à demi lumineuse, \\
  sa main cueille et dispense une rose neigeuse \\
  dont les pétales font des cercles sur les eaux...

  Délicieux désert, solitude pâmée, \\
  quand le remous de l'eau par la lune lamée \\
  compte éternellement ses échos de cristal,

  quel cœur pourrait souffrir l'inexorable charme \\
  de la nuit éclatante au firmament fatal \\
  sans tirer de soi-même un cri pur comme une arme?
\end{verse}
}
