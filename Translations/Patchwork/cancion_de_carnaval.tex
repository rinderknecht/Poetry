%-*-latex-*-

\addcontentsline{toc}{subsection}{Chanson de carnaval (\emph{Canción de carnaval})}

\selectlanguage{french}

\poemtitle*{Chanson de carnaval}

\settowidth{\versewidth}{pour Guido, dans sa houppelande,}

\bigskip

\begin{verse}[\versewidth]
Muse, le masque apprête, \\
essaye un air jovial \\
et jouis et ris à la fête \\
du carnaval!

Dans la danse qui chavire, \\
que ta jambe rosée s'étire \\
et sonnent, comme d'une lyre, \\
tes éclats de rire.

Pour voler plus haut, ose: \\
vêts deux feuilles de rose, \\
comme fait ton compagnon \\
le papillon,

et que sur ta bouche adorée, \\
qui joint le chœur d'un joyeux air, \\
laisse l'abeille de Buenos Aires \\
son miel doré.

Joins la mascarade délurée, \\
pendant que grimace un \emph{clown} \\
au visage peinturluré \\
comme Franck Brown;

alors qu'Arlequin révéla \\
ses teintes au prisme dérobées; \\
quand apparaît Pulcinella \\
et sa bosse enrobée,

dit à Colombine la merveille \\
ce que je pense d'elle, tout haut, \\
et ouvre une bouteille \\
pour Pierrot.

Qu'il te conte comment riment \\
ses amours avec la lune \\
et te fasse un poème en une \\
pantomime...

Donne la sérénade fine, \\
dorée par la mandoline, \\
porte une cravache argentine \\
pour le spleen.

Soit l'alpha et l'oméga: \\
grecque avec la cithare, \\
gaucho avec la guitare \\
de Santos Vega.

Montre ton splendide dos \\
parmi les rues de mèche \\
et joue et décore le corso \\
de roses fraîches.

Verse des perles sur le trésor \\
d'Andrade où coulent les Andes; \\
pour Guido, dans sa houppelande, \\
de la poudre d'or.

Oublie peines, deuils et rides, \\
chante amours et douceurs, \\
cherche la fleur des fleurs \\
en Floride!

Avec l'harmonie tu l'enchantes \\
des rimes de cristal, \\
et tu effeuilles les plantes \\
d'un madrigal.

Pirouette, danse, inspire \\
vers fous et rimes joviales; \\
célèbre la joyeuse lyre \\
les carnavals,

ses cris et ses chansons, \\
ses masques et ses déguisements, \\
ses perles, teintes et ornements \\
et pompons.

Et souffle la brise, que vire, \\
sonore, argentine, burlesque, \\
la victoire de ton rire \\
funambulesque!
\end{verse}

\cleardoublepage

\selectlanguage{spanish}

\poemtitle*{Canción de carnaval}

\settowidth{\versewidth}{que al prisma sus tintes roba}

\bigskip

\begin{verse}[\versewidth]
Musa, la máscara apresta, \\
ensaya un aire jovial \\
y goza y ríe en la fiesta \\
del carnaval.

Ríe en la danza que gira, \\
muestra la pierna rosada, \\
y suene, como una lira, \\
tu carcajada.

Para volar más ligera \\
ponte dos hojas de rosa, \\
como hace tu compañera \\
la mariposa.

Y que en tu boca risueña, \\
que se une al alegre coro, \\
deje la abeja porteña \\
su miel de oro.

Únete a la mascarada, \\
y mientras muequea un {\em clown} \\
con la faz pintarrajeada \\
como Franck Brown;

mientras Arlequín revela \\
que al prisma sus tintes roba \\
y aparece Pulchinela \\
con su joroba,

di a Colombina la bella \\
lo que de ella pienso yo, \\
y descorcha una botella \\
para Pierrot.

Que él te cuente cómo rima \\
sus amores con la luna \\
y te haga un poema en una \\
pantomima.

Da al aire la serenata, \\
toca el áureo bandolín, \\
lleva un látigo de plata \\
para el {\em spleen}.

Sé lírica y sé bizarra; \\
con la cítara sé griega; \\
o gaucha, con la guitarra \\
de Santos Vega.

Mueve tu espléndido torso \\
por las calles pintorescas \\
y juega y adorna el corso \\
con rosas frescas.

De perlas riega un tesoro \\
de Andrade en el regio nido, \\
y en la hopalanda de Guido, \\
polvo de oro.

Penas y duelos olvida, \\
canta deleites y amores; \\
busca la flor de las flores \\
por Florida.

Con la armonía le encantas \\
de las rimas de cristal, \\
y deshojas a sus plantas \\
un madrigal.

Pirueta, baila, inspira \\
versos locos y joviales; \\
celebre la alegre lira \\
los carnavales.

Sus gritos y sus canciones, \\
sus comparsas y sus trajes, \\
sus perlas, tintes y encajes \\
y pompones.

Y lleve la rauda brisa, \\
sonora, argentina, fresca, \\
la victoria de tu risa \\
funambulesca.
\end{verse}

\bigskip

\qquad\textbf{Rubén Darío}. \emph{Prosas profanas y otros poemas.} (1896)
