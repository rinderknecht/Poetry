%-*-latex-*-

\addcontentsline{toc}{subsection}{Se taire ({\em Callar})}

\selectlanguage{french}

\poemtitle*{Se taire}

\settowidth{\versewidth}{vers l'envers, jusqu'au puits --- Oh se taire...}

\bigskip

\begin{verse}[\versewidth]
Lèvres immobiles qui se serrent: \\
c'est la peine qui me l'impose, \\
pour que la voix vive ne glose \\
mon silence profond, sincère.

Le vrai silence, luthier austère \\
qui entre deux musiciens compose \\
un pont dans l'air, que le muet l'ose \\
vers l'envers, jusqu'au puits --- Oh se taire...

J'aimerais tant ouvrir la cage d'or \\
avec les clés de tous les accords \\
pour que vole enfin l'oiseau qu'on loue,

sans craindre qu'il ne parte sans retour \\
et chante au lieu du cantique du Jaloux \\
le dur requiem des mots sans amour.
\end{verse}

\bigskip \bigskip

\selectlanguage{spanish}

\poemtitle*{Callar}

\settowidth{\versewidth}{hacia dentro, hasta el pozo, el salmo entero.}

\bigskip

\begin{verse}[\versewidth]
Callar, callar. No callo porque quiero, \\
callo porque la pena se me impone, \\
para que la palabra no destrone \\
mi más hondo silencio verdadero.

Reina el silencio, el obrador austero \\
que un puente entre dos músicos compone, \\
para que el labio enmudecido entone \\
hacia dentro, hasta el pozo, el salmo entero.

Yo bien quisiera abrir al sello el borde, \\
desligar a las aves del acorde \\
y en volador arpegio darles cielo

si no temiera que al soltar mi rama \\
en vez de dulce cántico del celo \\
sonara la palabra que no ama.
\end{verse}

\bigskip \bigskip

\hspace*{25mm} {\bf Gerardo Diego}. {\em Amor solo.} (1951)
