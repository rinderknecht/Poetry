%-*-latex-*-

\addcontentsline{toc}{subsection}{Le roman de boutique ({\em La novela de una tienda})}

\selectlanguage{french}

\begin{center}
LE ROMAN DE BOUTIQUE \\
Chapitre IV \\
Le chat
\end{center}

\settowidth{\versewidth}{à fixer --- Oiselle, le sais-tu? ---}

\bigskip

\begin{verse}[\versewidth]
Depuis le fond des âges \\
il y a un chat, le chat éternel, \\
le chien d'un griffonnage, \\
la lumière d'un miaulement de miel. \\
Perse, égyptien, sans bagages, \\
le chat est magnétisme, dynastie, \\
jungle, tigre qui s'engage \\
à rêver toujours de philosophie, \\
à fixer --- Oiselle, le sais-tu? --- \\
l'âme pâle dans ta statue.
\end{verse}

\bigskip \bigskip

\selectlanguage{spanish}

\begin{center}
LA NOVELA DE UNA TIENDA \\
Capítulo IV \\
El gato
\end{center}

\poemtitle*{}

\settowidth{\versewidth}{que era el gato, el gato eterno,}

\bigskip

\begin{verse}[\versewidth]
El gato. Siempro hubo un gato \\
que era el gato, el gato eterno, \\
la gracia de un garabato, \\
la luz de un maullido tierno. \\
El gato era Persia, Egipto, \\
magnetismo, dinastía, \\
la selva, el tigre conscripto \\
a soñar filosofía, \\
a coser --- tan siderales --- \\
sus ojos en tus ojales.
\end{verse}

\bigskip \bigskip

\hspace*{10mm} {\bf Gerardo Diego}. {\em Mi Santander, mi cuna, mi palabra.} (1946-1961)
