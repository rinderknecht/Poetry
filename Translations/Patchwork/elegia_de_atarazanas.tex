%-*latex-*-

\addcontentsline{toc}{subsection}{Élégie de cave ({\em Elegia de atarazanas})}

\selectlanguage{french}

\poemtitle*{Élégie de cave}

\settowidth{\versewidth}{C'est vrai. Tu n'es pas morte, ce n'est pas l'heure.}

\bigskip

\begin{verse}[\versewidth]
Ni braises, ni cendres; les flammèches s'effacent \\
dans l'espace ouvert du souvenir, dans l'air chanté \\
sur cette terrasse, ce trampoline hanté \\
par ce rêve entêté, lumineux, qui ne passe.

Et voilà l'aronde qui revient dans sa grâce \\
éternelle, poétise à mon ouïe enchantée \\
avec l'air léger du soir, complice serpenté \\
de hautes comètes que le nord-est embrasse.

Tout est néant? Le feu impunément peut-il \\
dérober la seule joie qui peuple notre île? \\
Qui donc jettera ce Barbe Rousse à la mer?

Néant est tout. Bien vivante est ma demeure. \\
C'est vrai. Tu n'es pas morte, ce n'est pas l'heure. \\
Un ange passe dans tes yeux bleus, mère, mère.
\end{verse}

\bigskip \bigskip

\cleardoublepage

\selectlanguage{spanish}

\poemtitle*{Elegía de atarazanas}

\settowidth{\versewidth}{A ese alado ladrón, ?`no hay quién le ladre?}

\bigskip

\begin{verse}[\versewidth]
Ni ascua ya, ni ceniza ni pavesa; \\
aire en el aire, luz en el sobrado \\
de la santa memoria. Aquel tejado, \\
trampolín de aquel sueño que no cesa;

vuelve la golondrina y embelesa \\
con su trovar mi oído enamorado, \\
y está el cielo del Alta serpeado \\
de altas cometas que el nordeste besa.

?`Todo es ya nada? El fuego ?`también puede \\
devorar la ilusión, lo que no cede? \\
A ese alado ladrón, ?`no hay quién le ladre?

Nada es ya todo. Viva está mi casa. \\
Es verdad. No te has muerto. Un ángel pasa \\
por tus ojos azules, madre, madre.
\end{verse}

\bigskip \bigskip

\hspace*{15mm} {\bf Gerardo Diego}. {\em Mi Santander, mi cuna, mi palabra.} (1946-1961)
