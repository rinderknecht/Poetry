%-*-latex-*-

\addcontentsline{toc}{subsection}{Syrinx ({\em Syrinx})}

\selectlanguage{french}

\poemtitle*{Syrinx\footnote{Nymphe qui, pour échapper aux
ardeurs de Pan, fut transformée en roseau. Le dieu fit en cette
matière la flûte qui porte son nom.}}

\settowidth{\versewidth}{tel Pan dans les champs je ferai dancer les chevreaux chétifs,}

\bigskip

\begin{verse}[\versewidth]
Syrinx, divine Syrinx! Vers toi mon c{\oe}ur s'allège, \\
vers le frêle roseau de ton sourire furtif; \\
j'en ferai ma flûte, j'inventerai des motifs \\
qui extasieront d'amour les cygnes de neige.

À mon chant éperdu le temps sera enfin un atout; \\
tel Pan dans les champs je ferai dancer les chevreaux chétifs, \\
comme jadis Orphée je retiendrai les lions captifs, \\
et je toucherai l'empire d'Amour qui touche à tout.

Et tout sera, Syrinx, grâce à la vertu secrète \\
qu'insuffle au roseau en un subtil ajout \\
avec la passion du dieu le rêve du poète;

parce que si ma bouche de la flûte joue \\
contre joue, son mystère de jonc doux interprète \\
et l'harmonie naît de ton baiser sur ma joue.
\end{verse}

\newpage

\selectlanguage{spanish}

\poemtitle*{Syrinx\footnote{Ninfa que, perseguida por Pan,
logr\'o transformarse en caña. De este material el dios hizo la
flauta que siempre lo acompañaba, llamada \emph{siringa} por su
origen.}}

\settowidth{\versewidth}{!`Syrinx, divina Syrinx! Buscar quiero la leve}

\bigskip

\begin{verse}[\versewidth]
!`Syrinx, divina Syrinx! Buscar quiero la leve \\
caña que corresponda a tus labios esquivos; \\
haré de ella mi flauta e inventaré motivos \\
que extasiarán de amor a los cisnes de nieve.

Al canto mío el tiempo parecerá más breve; \\
como Pan en el campo haré danzar los chivos; \\
como Orfeo tendré los leones cautivos, \\
y moveré el imperio de Amor que todo mueve.

Y todo será, Syrinx, por la virtud secreta \\
que en la fibra sutil de la caña coloca \\
con la pasión del dios el sueño del poeta;

porque si de la flauta la boca mía toca \\
el sonoro carrizo, su misterio interpreta \\
y la armonía nace del beso de tu boca.
\end{verse}

\bigskip \bigskip

\hspace*{10mm} {\bf Rubén Darío}. {\em Prosas profanas y otros poemas.} (1896)
