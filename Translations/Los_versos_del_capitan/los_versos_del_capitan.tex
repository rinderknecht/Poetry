%%-*-latex-*-

\documentclass[11pt,a4paper]{book}

\usepackage[T1]{fontenc}
\usepackage[utf8]{inputenc}
\usepackage[spanish,french]{babel}
\usepackage[charter]{mathdesign}
\usepackage{verse}
\usepackage{url}
\usepackage[nottoc]{tocbibind}

\begin{document}

\thispagestyle{empty}
\vspace*{70mm}
\begin{center}
\textbf{\Huge LES VERS DU CAPITAINE} \\
\vspace*{10mm}
{\LARGE Pablo Neruda} \\
\vspace*{10mm}
Traduction de Christian Rinderknecht \\
\today{}
\end{center}

\cleardoublepage

\noindent {\huge Explication} \\ \\

\addcontentsline{toc}{chapter}{Explication}

{\em
On a beaucoup débattu au sujet de l'anonymat de ce livre. Ce dont moi
je débattais dans mon for intérieur pendant ce temps était si je
devais ou non le tirer hors de son origine intime: révéler sa
progéniture était dénuder l'intimité de sa naissance. Et il ne me
semblait pas que cette action fusse loyale aux transports d'amour et
de furie, au climat inconsolable et ardent de l'exil qui lui donna
le jour. \\

D'autre part je pense que tous les livres devraient êtres
anonymes. Mais entre ôter à tous les miens mon nom et le donner au
plus mystérieux, je cédai, enfin, bien que sans beaucoup
d'enthousiasme. \\

Pourquoi conserva-t-il son mystère si longtemps? Parce que rien et
parce que tout, parce que ceci et parce que cela, à cause de joies
impropres, à cause de souffrances étrangères. Quand Paulo Ricci,
compagnon lumineux, l'imprima pour la première fois à Naples en 1952
nous pensâmes que ces rares exemplaires qu'il soigna et prépara avec
excellence disparaîtraient sans laisser de traces dans le sable du
sud. \\

Il n'en fut pas ainsi. Et la vie qui réclama son explosion secrète
aujourd'hui me l'impose comme présence de l'inébranlable amour. \\

Je livre donc cet ouvrage sans l'expliquer davantage, comme s'il
était mien et ne l'était pas: il suffit qu'il puisse marcher seul de
par le monde et croître de lui-même. Maintenant que je le reconnais
j'espère que son sang furieux me reconnaîtra aussi. \\ \\
}

\hspace*{80mm} {\sc Pablo Neruda}. \\ \\

\noindent {\em Île Noire, novembre 1963.}

\cleardoublepage

{\em
\hspace*{80mm} La Havane, 3 octobre 1951. \\ \\ \\

\noindent Cher monsieur\footnote{NDE. Nous reproduisons ici la
lettre prologue des éditions où le véritable auteur de ce livre se
maintint anonyme.}: \\ \\

Je me permet de vous envoyer ces papiers qui je crois vous
intéresseront et que je n'ai pu rendre publics jusqu'à présent. \\

Je possède tous les originaux de ces vers. Il furent écrits dans les
lieux les plus divers, comme des trains, des avions, des cafés et sur
de petits bouts de papier bizarres sur lesquels il n'y a presque pas
de corrections. Dans une de ses dernières lettres venait {\em La
lettre en chemin}. Beaucoup de ces papiers, étant froissés et coupés,
sont presque illisibles, mais je pense avoir réussi à les
déchiffrer. \\

Ma personne n'a pas d'importance, mais je suis la protagoniste de ce
livre et cela me rend fière et satisfaite de ma vie. \\

Cet amour, ce grand amour, naquit un août d'une année quelconque, lors
de mes tournées comme artiste, dans les villes et villages de la
frontière franco-espagnole. \\

Lui venait de la guerre d'Espagne. Il ne venait pas vaincu. Il était
du parti de Pasionaria, il était plein d'espoirs pour son petit et
lointain pays, en Amérique Centrale. \\

Je suis au regret de ne pouvoir vous donner son nom. Je n'ai jamais su
lequel était le véritable, si c'était Martínez, Ramírez ou
Sánchez. Moi je le nomme simplement mon Capitaine et celui-ci est le
nom que je veux conserver dans ce livre. \\

Ses vers sont comme lui-même: tendres, amoureux, passionnés, et
terribles dans leur colère. Il était fort et sa force tous ceux qui
l'approchaient la sentaient. C'était un homme privilégié, de ceux qui
naissent pour de grands destins. Moi je sentais sa force et mon
plaisir le plus grand était de me sentir petite à ses côtés. \\

Il entra dans ma vie, comme il le dit dans un vers, en abattant la
porte. Il ne frappa pas à la porte avec la timidité d'un amoureux. Dès
le premier instant il se sentit maître de mon corps et de mon âme. Il
me fit sentir que tout changeait dans ma vie, cette petite vie
d'artiste, de commodité, de mollesse qui était la mienne se transforma
comme tout ce qu'il touchait. \\

Il ne connaissait pas les petits sentiments, et ne les acceptait pas
non plus. Il me donna son amour avec toute la passion qu'il était
capable d'éprouver et je l'ai aimé comme jamais je ne m'avais cru
capable d'aimer. Tout se transforma dans ma vie. J'entrai dans un
monde dont je n'avais jamais rêvé l'existence auparavant. J'eus tout
d'abord peur, il y eut des moments de doute, mais l'amour ne me
laissa pas hésiter très longtemps. \\

Cet amour m'apportait tout. La tendresse douce et simple quand il
cherchait une fleur, un jouet, un galet de rivière et me l'offrait
avec ses yeux humides d'une tendresse infinie. Ses grandes mains
étaient, à ce moment, si douces et dans ses yeux pointait alors une
âme d'enfant. \\

Mais il y avait en moi un passé que lui ne connaissait pas et il y
avait des jalousies et des furies impossibles à contenir. Celles-ci
étaient comme des tempêtes furieuses qui battaient son âme et la
mienne, mais jamais elles n'eurent assez de force pour mettre en
pièces la chaîne qui nous unissait, qui était notre amour, et de
chaque tempête nous sortions plus unis, plus forts, plus sûrs de
nous-mêmes. \\

À chacun de ces moments, il écrivait ces vers, qui me faisaient
monter au ciel ou descendre à l'enfer même, avec la crudité de ses
mots qui me brûlaient comme des braises. \\

Il ne pouvait aimer d'une autre manière. Ces vers sont l'histoire de
notre amour, grand dans toutes ses manifestations. Il possédait la
même passion que lui mettait dans ses combats, dans ses luttes contre
les injustices. Il souffrait de la peine et de la misère, non
seulement de son peuple, mais de tous les peuples, toutes les luttes
parce qu'il les agitaient étaient siennes et il s'y livrait entier,
avec toute sa passion. \\

Moi je suis très peu littéraire et je ne peux rien dire de la valeur
de ces vers, en dehors de la valeur humaine qu'indiscutablement ils
possèdent. Peut-être le Capitaine ne pensa jamais que ces vers
seraient publiés, mais maintenant je crois qu'il est de mon devoir de
les donner au monde. \\

Vous salue avec empressement \\ \\
}

\hspace*{70mm} {\sc Rosario de La Cerda}

\cleardoublepage

\part{L'amour}

\cleardoublepage

{\huge En toi la terre} \\ \\
\addcontentsline{toc}{section}{En toi la terre}

\begin{verse}
Petite \\
rose, \\
rose petite, \\
parfois, \\
minuscule et nue, \\
on dirait \\
que dans une des mes mains \\
tu tiens, \\
qu'ainsi je vais te clôre \\
et te porter à ma bouche, \\
mais \\
soudain \\
mes pieds touchent tes pieds et ma bouche tes lèvres, \\
tu as grandi, \\
tes épaules s'élèvent comme deux collines, \\
tes seins se promènent sur ma poitrine, \\
mon bras parvient à peine à encercler la fine \\
ligne de nouvelle lune de ta ceinture: \\
dans l'amour comme eau de mer tu t'es défaite: \\
je mesure à peine les yeux les plus étendus du ciel \\
et je m'incline vers ta bouche pour baiser la terre.
\end{verse}

\newpage

{\huge La reine} \\ \\
\addcontentsline{toc}{section}{La reine}

\begin{verse}
Moi je t'ai nommée reine. \\
Il y en a de plus grandes que toi, de plus grandes. \\
Il y en a de plus pures que toi, de plus pures. \\
Il y en a de plus belles que toi, de plus belles. \\ \

Mais toi tu es la reine. \\ \

Quand tu vas dans les rues \\
personne ne te reconnait. \\
Personne ne voit ta couronne de cristal, personne ne regarde \\
le tapis d'or rouge \\
sur lequel tu marches où que tu passes, \\
le tapis qui n'existe pas. \\ \

Et quand tu parais \\
sonnent alors tous les fleuves \\
dans mon corps, les cloches \\
secouent le ciel, \\
et un hymne emplit le monde. \\ \

Seulement toi et moi. \\
Seulement toi et moi, mon amour, \\
l'entendont. \\
\end{verse}

\newpage

{\huge Le potier} \\ \\
\addcontentsline{toc}{section}{Le potier}

\begin{verse}
Tout ton corps possède \\
un sommet ou une douceur qui m'est destiné. \\ \

Quand je hausse la main \\
je trouve en chaque lieu une colombe \\
qui me cherchait, comme \\
si l'on t'avait, mon amour, fait d'argile \\
pour mes propres mains de potier. \\ \

Tes genoux, tes seins, \\
ta ceinture \\
manquent en moi comme dans le creux \\
d'une terre assoiffée \\
dont on a dégagé \\
une forme, \\
et ensemble \\
nous sommes complets comme un unique fleuve, \\
comme un unique sable.
\end{verse}

\newpage

{\huge 8 septembre} \\ \\
\addcontentsline{toc}{section}{8 septembre}

\begin{verse}
Aujourd'hui, ce jour fut une pleine coupe, \\
aujourd'hui, ce jour fut l'immense vague, \\
aujourd'hui fut toute la terre. \\ \

Aujourd'hui la mer tempétueuse \\
nous souleva dans un baiser \\
si haut que nous tremblâmes \\
à la lumière d'un éclair \\
et, enlacés, nous descendîmes \\
pour nous submerger sans nous dénouer. \\ \

Aujourd'hui nos corps se firent étendus, \\
ils crûrent jusqu'à la limite du monde \\
et ils roulèrent en se fondant \\
dans une unique goutte \\
de cire ou météore. \\ \

Entre toi et moi s'ouvrit une nouvelle porte \\
et quelqu'un, sans visage encore, \\
là nous attendait.
\end{verse}

\newpage

{\huge Tes pieds} \\ \\
\addcontentsline{toc}{section}{Tes pieds}

\begin{verse}
Quand je ne peux regarder ton visage \\
je regarde tes pieds. \\ \

Tes pieds d'os arqués, \\
tes petits pieds durs. \\ \

Je sais qu'ils te soutiennent, \\
et que ton doux poids \\
sur eux se lève. \\ \

Ta taille et ta poitrine, \\
la pourpre dupliquée \\
de tes mamelons, \\
la boîte de tes yeux \\
qui viennent de voler, \\
ta large bouche de fruit, \\
ta chevelure rouge, \\
ô ma petite tour. \\ \

Mais je n'aime tes pieds \\
que parce qu'ils marchèrent \\
sur la terre et sur \\
le vent et sur l'eau, \\
jusqu'à ce qu'ils me trouvèrent.
\end{verse}

\newpage

{\huge Tes mains} \\ \\
\addcontentsline{toc}{section}{Tes mains}

\begin{verse}
Quand tes mains s'en vont, \\
mon amour, vers les miennes, \\
que m'apportent-elles en un vol? \\
Pourquoi s'arrêtèrent-elles \\
sur ma bouche, soudain, \\
pourquoi les reconnais-je \\
comme si alors, avant, \\
je les avais touchées, \\
comme si avant d'être \\
elles avaient parcouru \\
mon front, ma taille? \\ \

Leur suavité venait \\
en volant au-dessus du temps, \\
de la mer, de la vapeur, \\
du printemps, \\
et quand tu posas \\
tes mains sur ma poitrine, \\
je reconnus ces ailes \\
de colombe dorée, \\
je reconnus cette glaise \\
et cette couleur de blé. \\ \

Durant les années de ma vie \\
j'ai cheminé en les cherchant. \\
Je montai les escaliers, \\
croisai les récifs, \\
les trains m'emportèrent, \\
les eaux m'apportèrent, \\
et en la peau des raisins \\
j'ai cru te toucher. \\
Le bois soudain \\
m'apporta ton contact, \\
l'amande m'annon\c{c}ait \\
ta suavité secrète, \\
jusqu'à ce que se refermèrent \\
tes mains sur ma poitrine \\
et là comme deux ailes \\
elles terminèrent leur voyage.
\end{verse}

\newpage

{\huge Ton rire} \\ \\
\addcontentsline{toc}{section}{Ton rire}

\begin{verse}
Ôte-moi le pain si tu veux, \\
ôte-moi l'air, mais \\
ne m'ôte pas ton rire. \\ \

Ne m'ôte pas la rose, \\
la lance qui égrène, \\
l'eau qui soudain \\
éclate dans ta joie, \\
la brusque vague \\
de plante qui t'enfante. \\ \

Ma lutte est dure et je reviens \\
avec les yeux fatigués \\
parfois d'avoir vu \\
la terre qui ne change pas, \\
mais en entrant ton rire \\
monte au ciel en me cherchant \\
et ouvre pour moi toutes \\
les portes de la vie. \\ \

Mon amour, à l'heure \\
la plus obscure égrène \\
ton rire, et si soudain \\
tu vois que mon sang tache \\
les pierres de la rue, \\
rie, parce que ton rire \\
sera pour mes mains \\
comme une épée fraîche. \\ \

Près de la mer en automne, \\
ton rire doit hisser \\
sa cascade d'écume, \\
et au printemps, mon amour, \\
je veux ton rire comme \\
la fleur que j'attendais, \\
la fleur azur, la rose \\
de ma patrie sonore. \\ \

\newpage

Ris de la nuit, \\
du jour, de la lune, \\
moque-toi des rues \\
tordues de l'île, \\
ris de cet enfant \\
maladroit qui t'aime, \\
mais quand j'ouvre \\
les yeux et les ferme, \\
quand mes pas vont, \\
quand reviennent mes pas, \\
refuse-moi le pain, l'air, \\
la lumière, le printemps, \\
mais ton rire jamais \\
car j'en mourrais.
\end{verse}

\newpage

{\huge L'inconstant} \\ \\
\addcontentsline{toc}{section}{L'inconstant}

\begin{verse}
Mes yeux s'en allèrent \\
derrière une brune \\
qui passa. \\ \

Elle était de nacre noire, \\
elle était de raisins violets, \\
et mon sang me fouetta \\
avec sa queue de feu. \\ \

Derrière toutes \\
je m'en vais. \\ \

Une claire blonde passa, \\
comme une plante d'or, \\
balan\c{c}ant ses dons. \\
Et ma bouche s'en alla \\
comme une vague \\
déchargeant sur sa poitrine \\
des éclairs de sang. \\ \

Derrière toutes \\
je m'en vais. \\ \

Mais vers toi, sans que je bouge, \\
sans te voir, toi distante, \\
vont mon sang et mes baisers, \\
ma brune et claire, \\
ma grande et petite, \\
ma large et fine, \\
ma laide, ma splendide, \\
faite de tout l'or \\
et de tout l'argent, \\
faite de tout le blé \\
et de toute la terre, \\
faite de toute l'eau \\
des vagues marines, \\
faite pour mes bras, \\
faite pour mes baisers, \\
faite pour mon âme.
\end{verse}

\newpage

{\huge La nuit sur l'île} \\ \\
\addcontentsline{toc}{section}{La nuit sur l'île}

\begin{verse}
Toute la nuit j'ai dormi contre toi \\
près de la mer, sur l'île. \\
Tu étais sauvage et douce entre le plaisir et le sommeil, \\
entre le feu et l'eau. \\ \

Peut-être très tard \\
nos rêves s'unirent \\
en haut ou au fond, \\
en haut comme les branches qu'un même vent agite, \\
en bas comme les rouges racines qui se touchent. \\ \

Peut-être ton rêve \\
se sépara du mien \\
et sur la mer obscure \\
me cherchait \\
comme avant, \\
quand tu n'existais pas encore, \\
quand sans te distinguer \\
je naviguai à ton côté, \\
et tes yeux cherchaient \\
ce que maintenant \\
--- pain, vin, amour et colère --- \\
je te donne à pleines mains \\
parce que tu es la coupe \\
qui attendait les dons de ma vie. \\ \

J'ai dormi contre toi \\
toute la nuit pendant \\
que l'obscure terre tourne \\
avec vivants et morts, \\
et au réveil soudain \\
au milieu de l'ombre \\
mon bras entourait ta taille. \\
Ni la nuit, ni le sommeil \\
ne purent nous séparer. \\ \

J'ai dormi contre toi \\
et au réveil ta bouche \\
sortie de ton rêve \\
me donna la saveur de terre, \\
d'eau marine, d'algues, \\
du fond de ta vie, \\
et je re\c{c}us ton baiser \\
mouillé par l'aurore \\
comme s'il me venait \\
de la mer qui nous entoure.
\end{verse}

\newpage

{\huge Le vent sur l'île} \\ \\
\addcontentsline{toc}{section}{Le vent sur l'île}

\begin{verse}
Le vent est un cheval: \\
écoute comme il court \\
à travers la mer, à travers le ciel. \\ \

Il veut m'emmener, écoute \\
comme il parcourt le monde \\
pour m'emmener au loin. \\ \

Cache-moi dans tes bras \\
cette nuit seulement, \\
pendant que la pluie éclate \\
contre la mer et la terre \\
sa bouche innombrable. \\ \

Entends comme le vent \\
m'appelle en galopant \\
pour m'emmener au loin. \\ \

Avec ton front contre mon front, \\
avec ta bouche contre ma bouche, \\
nos corps noués \\
à l'amour qui nous brûle, \\
laisse le vent passer \\
sans qu'il puisse m'emmener. \\ \

Laisse que le vent coure \\
couronné d'écume, \\
qu'il m'appelle et me cherche \\
en galopant dans l'ombre, \\
pendant que moi, immergé \\
sous tes grands yeux, \\
cette nuit seulement \\
je me reposerai, mon amour.
\end{verse}

\newpage

{\huge L'infinie} \\ \\
\addcontentsline{toc}{section}{L'infinie}

\begin{verse}
Vois-tu ces mains? Elles ont mesuré \\
la terre, elles ont séparé \\
les minéraux et les céréales, \\
elles ont fait la paix et la guerre, \\
elles ont abattu les distances \\
de tous les océans et fleuves, \\
et cependant \\
quand elles te parcourent, \\
toi, petite, \\
grain de blé, alouette, \\
elles ne parviennent pas à te saisir entière, \\
elles s'épuisent en atteignant \\
les colombes jumelles \\
qui reposent ou volent sur ta poitrine, \\
elles parcourent les distances de tes jambes, \\
elles s'enroulent dans la lumière de ta taille. \\
Pour moi tu es un trésor plus chargé \\
d'immensité que la mer et ses grappes \\
et tu es blanche et bleue et étendue comme \\
la terre en la vendange. \\
Dans ce territoire, \\
de tes pieds à ton front, \\
marchant, marchant, marchant, \\
je passerai ma vie.
\end{verse}

\newpage

{\huge Belle} \\ \\
\addcontentsline{toc}{section}{Belle}

\begin{verse}
Belle, \\
comme sur la pierre fraîche \\
de la source, l'eau \\
ouvre un large éclair d'écume, \\
ainsi est le sourire sur ton visage, \\
belle. \\ \

Belle, \\
avec de fines mains et minces pieds, \\
comme un petit cheval d'argent, \\
marchant, fleur du monde, \\
ainsi je te vois, \\
belle. \\ \

Belle, \\
avec un nid de cuivre emmêlé \\
sur ta tête, un nid \\
couleur de miel sombre \\
où mon c{\oe}ur flamboie et repose, \\
belle. \\ \

Belle, \\
tes yeux sont trop grands pour ta figure, \\
tes yeux sont trop grands pour la terre. \\
Il y a des pays, il y a des fleuves, \\
dans tes yeux, \\
ma patrie est dans tes yeux, \\
je chemine en eux, \\
ils donnent jour au monde \\
où je chemine, \\
belle. \\ \

Belle, \\
tes seins sont comme deux pains faits \\
de terre céréale et lune d'or, \\
belle. \\ \

Belle, \\
ta ceinture \\
la fit mon bras comme un cours d'eau quand \\
il passa mille ans sur ton corps doux, \\
belle. \\ \

\newpage

Belle, \\
rien n'égale tes hanches, \\
peut-être la terre possède \\
en quelque lieu occulte \\
la courbe et l'arôme de ton corps, \\
peut-être en quelque lieu, \\
belle. \\ \

Belle, ma belle, \\
ta voix, ta peau, tes ongles, \\
belle, ma belle, \\
ton être, ta lumière, ton ombre, \\
belle, \\
tout cela est à moi, belle, \\
tout cela est à moi, mienne, \\
quand tu marches ou reposes, \\
quand tu chantes ou dors, \\
quand tu souffres ou rêves, \\
toujours, \\
quand tu es proche ou loin, \\
toujours, \\
tu es mienne, ma belle, \\
toujours.
\end{verse}

\newpage

{\huge La branche dérobée} \\ \\
\addcontentsline{toc}{section}{La branche dérobée}

\begin{verse}
À la nuit tombée nous entrerons \\
pour dérober \\
une branche fleurie. \\ \

Nous passerons le mur, \\
dans les ténèbres du jardin d'autrui, \\
deux ombres dans l'ombre. \\ \

L'hiver n'est pas encore parti, \\
et le pommier apparaît \\
changé soudain \\
en cascade d'étoiles odorantes. \\ \

À la nuit tombée nous entrerons \\
jusqu'à son tremblant firmament, \\
et tes petites mains et les miennes \\
déroberont les étoiles. \\ \

Et secrètement, \\
chez nous, \\
dans la nuit et dans l'ombre, \\
le silencieux pas du parfum \\
entrera avec tes pas \\
et le corps clair du printemps \\
aux pieds étoilés.
\end{verse}

\newpage

{\huge Le fils} \\ \\
\addcontentsline{toc}{section}{Le fils}

\begin{verse}
Ah fils, sais-tu, sais-tu \\
d'où tu viens? \\ \

D'un lac aux mouettes \\
blanches et faméliques. \\ \

Au bord de l'eau d'hiver \\
elle et moi allumâmes \\
un feu de joie rouge \\
en nous usant les lèvres \\
à tant nous baiser l'âme, \\
jetant tout au brasier, \\
brûlant nos vies. \\ \

Ainsi tu vins au monde. \\ \

Mais elle pour me voir \\
et pour te voir un jour \\
traversa les mers \\
et moi pour embrasser \\
sa petite taille \\
je parcourus toute la terre, \\
avec des guerres et des montagnes, \\
avec des sables et des épines. \\ \

Ainsi tu vins au monde. \\ \

De tant de lieux tu viens, \\
de l'eau et de la terre, \\
du feu et de la neige, \\
de si loin tu chemines \\
vers nous deux, \\
depuis l'amour terrible \\
qui nous a enchaîné, \\
que nous voulons savoir \\
comment tu es, ce que tu nous dis, \\
parce que tu en sais plus \\
du monde que nous te donnâmes. \\ \

Comme une grande tornade \\
nous secouâmes \\
l'arbre de la vie \\
jusqu'aux plus occultes \\
fibres des racines \\
et tu apparais maintenant \\
chantant dans le feuillage, \\
à la plus haute branche \\
qu'avec toi nous atteignons.
\end{verse}

\newpage

{\huge La terre} \\ \\
\addcontentsline{toc}{section}{La terre}

\begin{verse}
La terre verte s'est livrée \\
à tout le jaune, or, récoltes, \\
mottes, feuilles, grains, \\
mais quand l'automne se lève \\
avec son vaste étendard \\
tu es celle que je vois, \\
ta chevelure est pour moi \\
celle qui distribue les épis. \\ \

Je vois les monuments \\
d'antique pierre brisée, \\
mais si je touche \\
la cicatrice de pierre \\
ton corps me répond, \\
mes doigts reconnaissent \\
soudain, frémissants, \\
ta chaude douceur. \\ \

Parmi les héros je vais \\
tout juste décoré \\
sur la terre et la poussière \\
et derrière eux, muette, \\
avec tes petits pas, \\
es-tu ou n'es-tu pas? \\ \

Hier quand on déracina, \\
pour le voir, \\
le vieil arbre nain, \\
je te vis sortir me regardant \\
depuis les racines \\
torturées et assoiffées. \\ \

Et quand vient le sommeil \\
pour m'étendre et m'enlever \\
vers mon propre silence \\
il y a un grand vent blanc \\
qui abat mon rêve \\
et les feuilles tombent de lui, \\
elles tombent comme des couteaux \\
sur moi saignant. \\ \

Et chaque blessure a \\
la forme de ta bouche.
\end{verse}

\newpage

{\huge Absence} \\ \\
\addcontentsline{toc}{section}{Absence}

\begin{verse}
À peine t'ai-je laissée, \\
tu vas en moi, cristalline \\
ou tremblante, \\
ou inquiète, blessée par moi-même \\
ou comblée d'amour, comme quand tes yeux \\
se ferment sur le don de la vie \\
que sans cesse je t'offre. \\ \

Mon amour, \\
nous nous sommes trouvés \\
assoiffés et nous avons \\
bu toute l'eau et le sang, \\
nous nous sommes trouvés \\
affamés \\
et nous nous mordîmes \\
comme le feu mord, \\
nous laissant des blessures. \\ \

Mais attends-moi, \\
garde-moi ta douceur. \\
Je te donnerai aussi \\
une rose.
\end{verse}

\cleardoublepage

\part{Le désir}

\cleardoublepage

{\huge Le tigre} \\ \\
\addcontentsline{toc}{section}{Le tigre}

\begin{verse}
Je suis le tigre. \\
Je te guette parmi les feuilles \\
larges comme des lingots \\
de minerai mouillé. \\ \

Le fleuve blanc croît \\
sous le brouillard. Tu arrives. \\ \

Nue tu t'immerges. \\
J'attends. \\ \

Alors d'un saut \\
de feu, sang, dents, \\
d'un coup de patte je jette à terre \\
ta poitrine, tes hanches. \\ \

Je bois ton sang, je brise \\
tes membres un par un. \\ \

Et je veille \\
des années dans la jungle \\
tes os, ta cendre, \\
immobile, loin \\
de la haine et de la colère, \\
désarmé dans ta mort, \\
croisé par les lianes, \\
immobile sous la pluie, \\
sentinelle implacable \\
de mon amour assassin.
\end{verse}

\newpage

{\huge Le condor} \\ \\
\addcontentsline{toc}{section}{Le condor}

\begin{verse}
Je suis le condor, je vole \\
au-dessus de toi qui chemines \\
et soudain dans un moulinet \\
de vent, plume, serres, \\
je t'attaque et t'enlève \\
dans un cyclone sifflant \\
de froid impétueux. \\ \

Et à ma tour de neige, \\
à mon repaire noir \\
je t'emporte et seule tu y vis, \\
et tu te couvres de plumes \\
et voles au-dessus du monde, \\
immobile, dans l'altitude. \\ \

Femelle condor, sautons \\
sur cette proie rouge, \\
déchirons la vie \\
qui passe en palpitant \\
et élevons ensemble \\
notre vol sauvage.
\end{verse}

\newpage

{\huge L'insecte} \\ \\
\addcontentsline{toc}{section}{L'insecte}

\begin{verse}
De tes hanches jusqu'à tes pieds \\
je veux faire un long voyage. \\ \

Je suis plus petit qu'un insecte. \\ \

Je vais par ces collines, \\
elles sont de couleur d'avoine, \\
elles ont de minces empreintes \\
que seul moi je connais, \\
centimètres brûlés, \\
pâles perspectives. \\ \

Ici il y a une montagne. \\
Je n'en sortirai jamais. \\
Oh quelle mousse géante! \\
Et un cratère, une rose \\
de feu humidifié! \\ \

Le long de tes jambes je descends \\
en filant une spirale \\
ou en dormant pendant le voyage \\
et je parviens à tes genoux \\
de ronde dureté \\
comme aux cimes dures \\
d'un clair continent. \\ \

Vers tes pieds je glisse, \\
vers les huit ouvertures \\
de tes doigts aigus, \\
lents, péninsulaires, \\
et d'eux jusqu'au vide \\
du drap blanc \\
je tombe, cherchant aveugle \\
et affamé ton contour \\
de pot brûlant!
\end{verse}

\cleardoublepage

\part{Les furies}

\cleardoublepage

{\huge L'amour} \\ \\
\addcontentsline{toc}{section}{L'amour}

\begin{verse}
Qu'as-tu, qu'avons-nous, \\
que nous arrive-t-il? \\
Ah notre amour est une corde dure \\
qui nous amarre en nous blessant \\
et si nous voulons \\
sortir de notre blessure, \\
nous séparer, \\
elle refait un n{\oe}ud et nous condamne \\
à saigner et brûler ensemble. \\ \

Qu'as-tu? Je te regarde \\
et je ne trouve rien en toi à part deux yeux \\
comme tous les yeux, une bouche \\
perdue parmi mille bouches que j'ai baisé, plus belles, \\
un corps pareil à ceux qui glissèrent \\
sous mon corps sans laisser de souvenirs. \\ \

Et si vide de par le monde tu allais \\
comme une jarre de couleur des blés \\
sans air, sans son, sans substance! \\
J'ai cherché en vain en toi \\
de la profondeur pour mes bras \\
qui creusent, sans cesse, sous la terre: \\
sous ta peau, sous tes yeux \\
rien, \\
sous ta double poitrine levée \\
à peine \\
un courant d'ordre cristallin \\
qui ne sait pourquoi il s'écoule en chantant. \\
Pourquoi, pourquoi, pourquoi, \\
mon amour, pourquoi?
\end{verse}

\newpage

{\huge Toujours} \\ \\
\addcontentsline{toc}{section}{Toujours}

\begin{verse}
Avant moi \\
je ne suis pas jaloux. \\ \

Viens avec un homme \\
dans ton dos, \\
viens avec cent hommes dans ta chevelure, \\
viens avec mille hommes entre ta poitrine et tes pieds, \\
viens comme un fleuve \\
plein de noyés \\
qui rencontre la mer furieuse, \\
l'écume éternelle, le temps! \\ \

Apporte-les tous \\
où je t'attends: \\
toujours nous serons seuls, \\
toujours nous serons toi et moi \\
seuls sur la terre \\
pour commencer la vie!
\end{verse}

\newpage

{\huge Le dévoiement} \\ \\
\addcontentsline{toc}{section}{Le dévoiement}

\begin{verse}
Si ton pied se dévoie à nouveau, \\
il sera tranché. \\ \

Si ta main t'emmène \\
vers un autre chemin \\
elle tombera putréfiée. \\ \

Si tu me prives de ta vie \\
tu mourras \\
bien que tu vives. \\ \

Tu poursuivras, morte ou ombre, \\
ta marche sans moi sur la terre.
\end{verse}

\newpage

{\huge La question} \\ \\
\addcontentsline{toc}{section}{La question}

\begin{verse}
Mon amour, une question \\
t'a détruite. \\ \

Je suis revenu vers toi \\
de l'incertitude aux épines. \\ \

Je te veux droite comme \\
l'épée ou le chemin. \\ \

Mais tu t'entêtes \\
à garder un détour \\
d'ombre que je ne veux pas. \\ \

Mon amour, \\
comprends-moi, \\
je te veux toute entière, \\
des yeux aux pieds, aux ongles, \\
dedans, \\
toute la clarté, celle que tu gardais. \\ \

C'est moi, mon amour, \\
qui cogne à ta porte. \\
Ce n'est pas le fantôme, ce n'est pas \\
celui qui avant s'arrêta \\
à ta fenêtre. \\
Moi j'abats la porte: \\
moi j'entre dans toute ta vie: \\
je viens vivre dans ton âme: \\
tu ne peux me résister. \\ \

Tu dois ouvrir porte à porte, \\
tu dois m'obéir, \\
tu dois ouvrir les yeux \\
pour que je cherche en eux, \\
tu dois voir comment je marche \\
d'un pas pesant \\
sur tous les chemins \\
qui, aveugles, m'attendaient. \\ \

\newpage

Ne me crains pas, \\
je suis à toi, \\
mais \\
je ne suis ni le passager ni le mendiant, \\
je suis ton maître, \\
celui que tu attendais, \\
et maintenant j'entre \\
dans ta vie, \\
pour ne plus en sortir, \\
mon amour, mon amour, mon amour, \\
pour y rester.
\end{verse}

\newpage

{\huge La prodigue} \\ \\
\addcontentsline{toc}{section}{La prodigue}

\begin{verse}
Je t'ai choisie parmi toutes les femmes \\
pour que tu répètes \\
sur la terrre \\
mon c{\oe}ur qui danse avec des épis \\
ou lutte sans caserne quand il le faut. \\ \

Je te demande: où est mon fils? \\ \

Ne m'attendait-il pas en toi, me reconnaissant, \\
et me disant: <<Appelle-moi pour paraître sur la terre \\
pour continuer tes luttes et tes chants>>? \\ \

Rends-moi mon fils! \\ \

Tu l'as oublié aux portes \\
du plaisir, oh prodigue \\
ennemie, \\
as-tu oublié que tu vins à ce rendez-vous, \\
le plus profond, celui \\
où tous les deux, unis, nous continuerons à parler \\
à travers sa bouche, mon amour, \\
ah, de tout ce \\
que nous ne pûmes nous dire? \\ \

Quand je te soulève en une vague \\
de feu et sang, et que se duplique \\
la vie entre nous, \\
souviens-toi \\
que quelqu'un nous appelle \\
comme jamais personne ne nous a appelé \\
et que nous ne répondons pas \\
et nous restons seuls et lâches \\
face à la vie que nous nions. \\ \

Prodigue, \\
ouvre les portes, \\
et que dans ton c{\oe}ur \\
le n{\oe}ud aveugle \\
se dénoue et vole \\
avec ton sang et le mien \\
dans le monde!
\end{verse}

\newpage

{\huge Le mal} \\ \\
\addcontentsline{toc}{section}{Le mal}

\begin{verse}
Je t'ai fait mal, mon c{\oe}ur, \\
j'ai déchiré ton âme. \\ \

Comprends-moi. \\
Tous savent qui je suis, \\
mais ce Je Suis \\
est en plus un homme \\
pour toi. \\ \

En toi j'hésite, je tombe \\
et me relève ardent. \\
Toi parmi les êtres \\
tu as le droit \\
de me voir faible. \\
Et ta petite main \\
de pain et de guitare \\
doit jouer\footnote{NDT. Le verbe {\em tocar} signifie
{\em jouer} (d'un instrument de musique) mais aussi {\em toucher}.} de ma poitrine \\
quand elle part combattre. \\ \

C'est pour cela que je cherche en toi la pierre ferme. \\
D'âpres mains je plante dans ton sang \\
en cherchant ta fermeté \\
et la profondeur dont j'ai besoin, \\
et si je ne parviens \\
à rien d'autre que ton rire de métal, si je ne trouve \\
rien pour soutenir mes durs pas, \\
adorée, re\c{c}ois \\
ma tristesse et ma colère, \\
mes mains ennemies \\
te détruisant un peu \\
pour que tu t'élèves de l'argile, \\
faite à nouveau pour mes combats.
\end{verse}

\newpage

{\huge Le puits} \\ \\
\addcontentsline{toc}{section}{Le puits}

\begin{verse}
Parfois tu t'enfonces, tu tombes \\
dans ton trou de silence, \\
dans ton abîme de colère fière, \\
et tu peux à peine \\
revenir, toujours avec des restes déchirés \\
de ce que tu trouvas \\
dans la profondeur de ton existence. \\ \

Mon amour, que trouves-tu \\
dans ton puits fermé? \\
Des algues, des marécages, des roches? \\
Que vois-tu avec des yeux aveugles, \\
rancunière et blessée? \\ \

Mon c{\oe}ur, tu ne trouveras pas \\
dans le puits dans lequel tu tombes \\
ce que moi je garde pour toi dans les hauteurs: \\
un bouquet de jasmins couvert de rosée \\
un baiser plus profond que ton abîme. \\ \

Ne me crains pas, ne tombe pas \\
dans ta ranc{\oe}ur à nouveau. \\
Bat le mot qui vint te blesser \\
et laisse-le s'envoler par la fenêtre ouverte. \\ \

Il reviendra me blesser \\
sans que tu le diriges \\
car il fut chargé d'un instant dur \\
et cet instant sera désarmé sur ma poitrine. \\ \

Souris-moi radieuse \\
si ma bouche te blesse. \\
Je ne suis pas un pasteur doux \\
comme dans les contes de fées, \\
mais un bon bûcheron qui partage avec toi \\
la terre, le vent et les épines des monts. \\ \

Aime-moi, toi, souris-moi, \\
aide-moi à être bon. \\
Ne te blesse pas sur moi, car ce sera inutile, \\
ne me blesse pas parce que tu te blesses.
\end{verse}

\newpage

{\huge Le songe} \\ \\
\addcontentsline{toc}{section}{Le songe}

\begin{verse}
Marchant sur le sable \\
je décidai de te quitter. \\ \

J'avan\c{c}ais sur une boue obscure \\
qui tremblait, \\
et m'y enfon\c{c}ant et en en sortant \\
je décidai que tu sortisses \\
de moi, que tu me pesais \\
comme une pierre coupante, \\
et j'élaborai ta perte \\
pas à pas: \\
te couper les racines, \\
te lâcher seule dans le vent. \\ \

Ah à ce moment, \\
mon c{\oe}ur, un songe \\
avec ses ailes terribles \\
te couvrait. \\ \

Tu te sentais avalée par la boue, \\
et tu m'appelais et je ne venais pas, \\
tu partais, immobile, \\
sans te défendre \\
jusqu'à te noyer dans la gueule de sable. \\ \

Après \\
ma décision rencontra ton songe, \\
et de la rupture qui nous fendait l'âme, \\
nous surgîmes propres à nouveau, nus, \\
nous aimant \\
sans rêve, sans sable, \\
complets et radieux, \\
scellés par le feu.
\end{verse}

\newpage

{\huge Si toi tu m'oublies} \\ \\
\addcontentsline{toc}{section}{Si toi tu m'oublies}

\begin{verse}
Je veux que tu saches \\
une chose. \\ \

Tu sais comment c'est: \\
si je regarde \\
la lune de cristal, la branche rouge \\
du lent automne à ma fenêtre, \\
si je touche \\
près du feu \\
l'impalpable cendre \\
ou le corps ridé de la bûche, \\
tout me conduit à toi, \\
comme si tout ce qui existe, \\
arômes, lumière, métaux \\
étaient de petits bateaux qui naviguent \\
vers tes îles qui m'attendent. \\ \

Bien, maintenant \\
si peu à peu tu cesses de m'aimer \\
je cesserai de t'aimer peu à peu. \\ \

Si soudain \\
tu m'oublies \\
ne me cherches pas \\
car je t'aurais déjà oublié. \\ \

Si tu juges long et fou \\
le vent de drapeaux \\
qui souffle dans ma vie \\
et que tu te décides \\
à me laisser au bord \\
du c{\oe}ur dans lequel j'ai des racines, \\
pense \\
que ce jour-là, \\
à cette heure-là \\
je lèverai les bras \\
et mes racines sortiront \\
pour chercher une autre terre. \\ \

\newpage

Mais \\
si chaque jour, \\
chaque heure \\
tu sens que tu m'es destinée \\
avec une douceur implacable. \\
Si chaque jour monte \\
me chercher une fleur à tes lèvres, \\
ah mon amour, ah mienne, \\
en moi tout ce feu se répète, \\
en moi rien ne s'éteint ni ne s'oublie, \\
mon amour se nourrit de ton amour, mon aimée, \\
et tant que tu vivras il sera dans tes bras \\
sans quitter les miens.
\end{verse}

\newpage

{\huge L'oubli} \\ \\
\addcontentsline{toc}{section}{L'oubli}

\begin{verse}
Tout l'amour dans une coupe \\
large comme la terre, tout \\
l'amour avec les étoiles et les épines \\
je te donnai, mais tu marchas \\
avec de petits pieds, aux talons sales, \\
sur le feu, en l'éteignant. \\ \

Ah grand amour, petite aimée! \\ \

Je ne m'arrêtai pas dans la lutte. \\
Je ne cessai pas de marcher vers la vie, \\
vers la paix, vers le pain pour tous, \\
mais je t'élevai dans mes bras \\
et te clouai à mes baisers \\
et te regardai comme jamais \\
ne regarderont à nouveau des yeux humains. \\ \

Ah grand amour, petite aimée! \\ \

Alors tu ne mesuras pas ma stature, \\
et l'homme qui pour toi mit de côté \\
le sang, le blé, l'eau \\
tu le confondis \\
avec le petit insecte qui tomba dans ta jupe. \\ \

Ah grand amour, petite aimée! \\ \

N'attends pas que je te regarde dans la distance \\
en arrière, reste \\
avec ce que je t'ai laissé, promène-toi \\
avec ma photographie trahie, \\
moi je continuerai à marcher, \\
ouvrant de larges chemins contre l'ombre, rendant \\
suave la terre, distribuant \\
l'étoile pour ceux qui viennent. \\ \

Reste sur le chemin. \\
La nuit est venue pour toi. \\
Peut-être au matin \\
nous reverrons-nous. \\ \

Ah grand amour, petite aimée!
\end{verse}

\newpage

{\huge Les jeunes filles} \\ \\
\addcontentsline{toc}{section}{Les jeunes filles}

\begin{verse}
Jeunes filles qui cherchiez \\
le grand amour, le grand amour terrible, \\
que s'est-il passé, jeunes filles? \\ \

Peut-être \\
le temps, le temps! \\ \

Parce que maintenant, \\
il est là, voyez comme il passe \\
traînant les pierres azur, \\
défaisant les fleurs et les feuilles, \\
avec un bruit d'écume battue \\
contre toutes les pierres de ton monde, \\
avec une odeur de sperme et de jasmins, \\
contre la lune sanglante! \\ \

Et maintenant \\
tu touches l'eau avec tes petits pieds, \\
avec ton petit c{\oe}ur \\
et tu ne sais que faire! \\ \

Ils sont meilleurs \\
certains voyages nocturnes, \\
certains appartements, \\
certaines promenades follement divertissantes, \\
certains bals sans plus de conséquences \\
que continuer le voyage! \\

Meurs de peur ou de froid, \\
ou de doute, \\
car moi avec mes grands pas \\
je la trouverai, \\
en toi, \\
ou loin de toi, \\
et elle me trouvera, \\
celle qui ne tremblera pas face à l'amour, \\
celle qui sera fondue \\
avec moi \\
dans la vie et la mort!
\end{verse}

\newpage

{\huge Tu venais} \\ \\
\addcontentsline{toc}{section}{Tu venais}

\begin{verse}
Tu ne m'as pas fait souffrir \\
mais attendre. \\ \

Ces heures \\
emmêlées, pleines \\
de serpents, \\
quand \\
s'abîmait mon âme et je me noyais, \\
tu venais en marchant, \\
tu venais nue et griffée, \\
tu arrivais sanglante jusqu'à ma couche, \\
ô ma fiancée, \\
et alors \\
toute la nuit nous marchâmes \\
en dormant \\
et quand nous nous éveillâmes \\
tu étais intacte et neuve, \\
comme si le grave vent des songes \\
avait encore embrasé \\
ta chevelure \\
et dans le blé et l'argent avait immergé \\
ton corps jusqu'à le laisser éblouissant. \\ \

Je n'ai pas souffert mon amour, \\
je t'attendais seulement. \\
Tu devais changer de c{\oe}ur \\
et de regard \\
après avoir touché la profonde \\
zone de mer que t'offrit ma poitrine. \\
Tu devais sortir de l'eau \\
pure comme une goutte soulevée \\
par une vague nocturne. \\ \

Ô ma fiancée, tu dus \\
mourir et naître, je t'attendais. \\
Je n'ai pas souffert en te cherchant, \\
je savais que tu viendrais, \\
une femme neuve avec ce que j'adore \\
de celle que je n'adorais pas, \\
avec tes yeux, tes mains et ta bouche \\
mais avec un autre c{\oe}ur \\
qui s'éveilla au matin à mon côté \\
comme s'il avait toujours été là \\
pour continuer avec moi pour toujours.
\end{verse}

\cleardoublepage

\part{Les vies}

\cleardoublepage

{\huge La montagne et la rivière} \\ \\
\addcontentsline{toc}{section}{La montagne et la rivière}

\begin{verse}
Dans ma patrie il y a une montagne. \\
Dans ma patrie il y a une rivière. \\ \

Viens avec moi. \\ \

La nuit gravit la montagne. \\
La faim descend la rivière. \\ \

Viens avec moi. \\ \

Qui sont ceux qui souffrent? \\
Je ne sais, mais ils sont des miens. \\ \

Viens avec moi. \\ \

Je ne sais, mais ils m'appellent \\
et me disent <<Nous souffrons>>. \\ \

Viens avec moi. \\ \

Et ils me disent: <<Ton peuple, \\
ton peuple malchanceux, \\
entre la montagne et la rivière, \\
affamé et endolori, \\
ne veut pas lutter seul, \\
il t'attend, mon ami>>. \\ \

Oh toi, celle que j'aime, \\
petite, rouge grain \\
de blé, \\ \

la lutte sera dure, \\
la vie sera dure, \\
mais tu viendras avec moi.
\end{verse}

\newpage

{\huge La pauvreté} \\ \\
\addcontentsline{toc}{section}{La pauvreté}

\begin{verse}
Ah tu ne veux pas, \\
la pauvreté \\
t'effraie, \\ \

tu ne veux pas \\
aller avec des souliers abîmés au marché \\
et revenir avec la même vieille robe. \\ \

Mon amour, nous n'aimons pas, \\
comme le veulent les riches, \\
la misère. Nous \\
l'extirperons comme dent gâtée \\
qui jusqu'à présent a mordu le c{\oe}ur de l'homme. \\ \

Mais je ne veux pas \\
que tu la craignes. \\
Si elle parvient par ma faute à ta demeure, \\
si la pauvreté expulse \\
tes souliers dorés, \\
qu'elle n'expulse pas ton rire qui est le pain de ma vie. \\
Si tu ne peux payer le loyer \\
pars travailler d'un pas fier, \\
et pense alors, mon amour, que je te regarde \\
et nous sommes ensemble la plus grande richesse \\
que l'on n'a jamais réunie sur la terre.
\end{verse}

\newpage

{\huge Les vies} \\ \\
\addcontentsline{toc}{section}{Les vies}

\begin{verse}
Ah si incommodée parfois \\
je te sens \\
avec moi, vainqueur parmi les hommes! \\ \

Parce que tu ne sais pas \\
qu'avec moi vainquirent \\
des milliers de visages que tu ne peux voir, \\
des milliers de pieds et poitrines qui marchèrent avec moi, \\
que je ne suis pas, \\
que je n'existe pas, \\
que je suis seulement le front de ceux qui vont avec moi, \\
que je suis plus fort \\
parce que je porte en moi \\
non ma petite vie \\
mais toutes les vies, \\
et je marche d'un pas sûr vers l'avant \\
parce que j'ai mille yeux, \\
je frappe du poids de la pierre \\
parce que j'ai mille mains \\
et ma voix s'entend sur les rives \\
de toutes les terres \\
parce qu'elle est la voix de tous \\
ceux qui ne parlèrent pas, \\
de ceux qui ne chantèrent pas \\
et chantent aujourd'hui avec cette bouche \\
qui te baise, toi.
\end{verse}

\newpage

{\huge Le drapeau} \\ \\
\addcontentsline{toc}{section}{Le drapeau}

\begin{verse}
Lève-toi avec moi. \\ \

Personne ne voudrait \\
autant que moi rester \\
sur l'oreiller sur lequel tes paupières \\
veulent clôre le monde pour moi. \\
Là aussi je voudrais \\
laisser dormir mon sang \\
entourant ta douceur. \\ \

Mais lève-toi, \\
toi, lève-toi, \\
mais lève-toi avec moi \\
et sortons réunis \\
pour lutter corps à corps \\
contre les toiles d'araignée du malfaisant, \\
contre le système qui distribue la faim, \\
contre l'organisation de la misère. \\ \

Allons, \\
et toi, mon étoile, près de moi, \\
nouveau-née de ma propre argile, \\
tu auras déjà trouvé la source que tu occultes \\
et au milieu du feu tu seras \\
près de moi, \\
avec tes yeux braves, \\
dressant mon drapeau.
\end{verse}

\newpage


{\huge L'amour du soldat} \\ \\
\addcontentsline{toc}{section}{L'amour du soldat}

\begin{verse}
En pleine guerre la vie t'amena \\
à être l'amour du soldat. \\ \

Avec ta pauvre robe de soie, \\
tes ongles de fausse pierre \\
il te revint de marcher sur le feu. \\ \

Viens ici, vagabonde, \\
viens boire sur ma poitrine \\
une rouge rosée. \\ \

Tu ne voulais pas savoir où tu marchais, \\
tu étais la compagne de bal, \\
tu n'avais ni parti ni patrie. \\ \

Et maintenant en marchant à mes côtés \\
tu vois qu'avec moi va la vie \\
et que derrière est la mort. \\ \

Tu ne peux danser à nouveau \\
avec ta robe de soie dans la salle. \\ \

Tu vas abîmer tes souliers, \\
mais tu vas grandir dans la marche. \\ \

Tu dois marcher sur les épines \\
en laissant des gouttelettes de sang. \\ \

Baise-moi à nouveau, chérie. \\ \

Nettoie ce fusil, camarade.
\end{verse}


\newpage

{\huge Non seulement le feu} \\ \\
\addcontentsline{toc}{section}{Non seulement le feu}

\begin{verse}
Ah oui je me souviens, \\
ah de tes yeux clos \\
comme pleins dedans de lumière noire, \\
de tout ton corps comme une main ouverte, \\
comme une grappe blanche de la lune, \\
et l'extase, \\
quand nous tue un rayon, \\
quand un poignard nous blesse aux racines \\
et une lumière nous brise la chevelure, \\
et quand \\
nous revenons peu à peu \\
à la vie, \\
comme si de l'océan nous sortions, \\
comme si du naufrage \\
nous revenions blessés \\
entre les pierres et les algues rouges. \\ \

Mais \\
il y a d'autres souvenirs, \\
non seulement des fleurs de l'incendie, \\
mais aussi de petits jaillissements \\
qui apparaissent soudain \\
quand je vais dans les trains \\
ou dans les rues. \\
Je te vois \\
lavant mes mouchoirs, \\
étendant à la fenêtre \\
mes chaussettes trouées, \\
ta silhouette dans laquelle tout, \\
tout le plaisir comme une flambée \\
tomba sans te détruire, \\
de nouveau, \\
petite femme \\
de tous les jours, \\
de nouveau être humain, \\
humblement humain, \\
fièrement pauvre, \\
comme tu dois être pour être \\
non pas la rapide rose \\
que la cendre de l'amour défait, \\
mais toute la vie, \\
toute la vie avec savon et aiguilles, \\
avec l'arôme que j'aime \\
de la cuisine que peut-être nous n'aurons pas \\
et dans laquelle ta main parmi les frites \\
et ta bouche chantant en hiver \\
pendant que vient le rôti \\
seraient pour moi le séjour \\
du bonheur sur la terre. \\ \

Ah mon aimée, \\
non seulement le feu entre nous brûle, \\
mais aussi toute la vie, \\
la simple histoire, \\
le simple amour \\
d'une femme et d'un homme \\
pareils aux autres.
\end{verse}

\newpage

{\huge La morte} \\ \\
\addcontentsline{toc}{section}{La morte}

\begin{verse}
Si soudain tu n'existes pas, \\
si soudain tu ne vis pas, \\
je continuerai à vivre. \\ \

Je n'ose \\
je n'ose pas l'écrire, \\
si tu meurs. \\ \

Je continuerai à vivre. \\ \

Parce que où un homme n'a pas de voix \\
là, ma voix. \\ \

Où que les noirs soient bastonnés, \\
je ne peux être mort. \\
Quand entreront en prison mes frères \\
j'entrerai avec eux. \\ \

Quand la victoire, \\
non ma victoire, \\
mais la grande victoire \\
viendra \\
même si j'étais muet je devrai parler: \\
je la verrai venir même si j'étais aveugle. \\ \

Non, pardonne-moi. \\
Si tu ne vis pas, \\
si \\
toi, chérie, mon amour, \\
si toi \\
tu es morte, \\
toutes les feuilles tomberont sur ma poitrine, \\
il pleuvra sur mon âme jour et nuit, \\
la neige brûlera mon c{\oe}ur, \\
je marcherai avec le froid et le feu et la mort et la neige, \\
mes pieds voudront marcher vers où tu dors, \\
mais \\
je continuerai à vivre, \\
parce que tu me voulus sur toutes les choses \\
intraitable, \\
et, mon amour, parce que tu sais que je suis non seulement un homme \\
mais tous les hommes.
\end{verse}

\newpage

{\huge Petite Amérique} \\ \\
\addcontentsline{toc}{section}{Petite Amérique}

\begin{verse}
Quand je regarde la forme \\
de l'Amérique sur la carte, \\
mon amour, c'est toi que je vois: \\
les hauteurs du cuivre sur ta tête, \\
tes seins, blé et neige, \\
ta taille mince, \\
de véloces rivières qui palpitent, de douces \\
collines et prairies \\
et dans le froid du sud tes pieds terminent \\
leur géographie d'or dupliqué. \\ \

Mon amour, quand je te touche \\
non seulement mes mains \\
ont parcouru tes délices, \\
mais aussi branches et terres, fruits et eau, \\
le printemps que j'aime, \\
la lune du désert, la poitrine \\
de la colombe sauvage, \\
la suavité des pierres usées \\
par les eaux de la mer ou des rivières \\
et la garrigue rouge \\
du maquis où \\
la soif et la faim guettent. \\
Et ainsi ma vaste patrie me re\c{c}oit, \\
petite Amérique, en ton corps. \\ \

Plus encore, quand je te vois penchée \\
je vois dans ta peau, dans ta couleur d'avoine, \\
la nationalité de mon c{\oe}ur. \\
Parce que de tes épaules \\
le coupeur de canne \\
de Cuba brûlante \\
me regarde, couvert de sueur obscure, \\
et de ta gorge \\
des pêcheurs qui tremblent \\
dans les humides maisons du rivage \\
me chantent leur secret. \\
Et ainsi le long de ton corps, \\
petite Amérique adorée, \\
les terres et les villes \\
interrompent mes baisers \\
et ta beauté alors \\
non seulement allume le feu \\
qui brûle sans se consumer entre nous, \\
mais aussi m'appelle avec ton amour \\
et au travers de ta vie \\
elle me donne la vie qui me manque \\
et la saveur de ton amour s'agrège à la boue, \\
le baiser de la terre qui m'attend.
\end{verse}

\cleardoublepage

\part{Ode et germinations}

\cleardoublepage

{\huge I} \\ \\
\addcontentsline{toc}{section}{I. \ La saveur de ta bouche et la couleur de ta peau}

\begin{verse}
La saveur de ta bouche et la couleur de ta peau, \\
peau, bouche, ô fruit de ces jours véloces, \\
dis-moi, étaient-ils sans cesse à tes côtés \\
durant des années et durant des voyages et durant des lunes et soleils
\\
et terre et pleur et pluie et joie \\
ou bien seulement maintenant, seulement \\
sortent-ils de tes racines \\
comme l'eau apporte à la terre sèche \\
des germinations qu'elle ne connaissait pas \\
ou bien aux lèvres de la cruche oubliée \\
est-ce que monte dans l'eau le goût de la terre? \\ \

Je ne sais, ne me le dis pas, tu ne sais. \\
Personne ne sait ces choses. \\
Mais approchant tous mes sens \\
à la lumière de ta peau, tu disparais, \\
tu fonds comme l'arôme \\
acide d'un fruit \\
et la chaleur d'un chemin, \\
l'odeur du maïs qui s'égrène, \\
le chèvrefeuille du soir pur, \\
les noms de la terre poussiéreuse, \\
le parfum infini de la patrie: \\
magnolia et maquis, sang et farine, \\
galops de chevaux, \\
la lune poussiéreuse du village, \\
le pain nouveau-né: \\
ah tout de ta peau revient à ma bouche, \\
revient à mon c{\oe}ur, revient à mon corps, \\
et je suis à nouveau avec toi \\
la terre que tu es: \\
tu es en mon profond printemps: \\
je sais à nouveau en toi comment je germe.
\end{verse}

\newpage

{\huge II} \\ \\
\addcontentsline{toc}{section}{II. \ Tes années que je dus sentir}

\begin{verse}
Tes années que je dus sentir \\
croître près de moi comme des grappes \\
jusqu'à ce que tu aies vu comment le soleil et la terre \\
à mes mains de pierre t'auraient destinée \\
jusqu'à ce que grain de raisin avec grain de raisin tu aies fait \\
chanter dans mes veines le vin. \\
Le vent ou le cheval \\
se dévoyant auraient pu \\
faire que je passasse par ton enfance, \\
le même ciel tu as vu chaque jour, \\
la même boue de l'hiver obscur, \\
la ramure sans fin des pruniers \\
et leur douceur de couleur mauve. \\
Seuls quelques kilomètres de nuit, \\
les distances mouillées \\
de l'aurore champêtre, \\
une poignée de terre nous sépara, les murs \\
transparents \\
que nous ne passâmes pas, pour que la vie, \\
après, mît toutes \\
les mers et la terre \\
entre nous, et que nous nous approchions \\
malgré l'espace, \\
pas à pas nous cherchant, \\
d'un océan à l'autre, \\
jusqu'à ce que je vis que le ciel s'incendiait \\
et que volait dans le ciel ta chevelure \\
et tu vins à mes baisers avec le feu \\
d'un météore déchaîné \\
et en te fondant dans mon sang, la douceur \\
de la prune sauvage \\
de notre enfance je re\c{c}us dans ma bouche, \\
et je te serrai contre ma poitrine comme \\
si la terre et la vie je recouvrais.
\end{verse}

\newpage

{\huge III} \\ \\
\addcontentsline{toc}{section}{III. \ Ma sauvageonne, nous dûmes}

\begin{verse}
Ma sauvageonne, nous dûmes \\
recouvrer le temps \\
et rebrousser chemin, dans la distance \\
de nos vies, baiser par baiser, \\
ramassant en un lieu ce que nous dîmes \\
sans joie, découvrant en un autre \\
le chemin secret \\
qui rapprochait peu à peu tes pas des miens, \\
et ainsi sous ma bouche \\
tu revois la plante insatisfaite \\
de ta vie allongeant ses racines \\
vers mon c{\oe}ur qui t'attendait. \\
Et une par une les nuits \\
entre nos villes séparées \\
s'agrègent à la nuit qui nous unit. \\
La lumière de chaque jour \\
nous offre sa flamme ou son repos, \\
hors du temps, \\
et ainsi se déterre \\
dans l'ombre ou la lumière notre trésor, \\
et ainsi nos baisers baisent la vie: \\
tout l'amour dans notre amour s'enferme: \\
toute la soif termine dans notre étreinte. \\
Nous voilà enfin face à face, \\
nous nous sommes trouvés, \\
nous n'avons rien perdu. \\
Nous nous sommes parcourus lèvre à lèvre, \\
nous avons changé mille fois, \\
entre nous la vie et la mort, \\
tout ce que nous apportions \\
comme de mortes médailles \\
nous le jetâmes au fond de la mer, \\
tout ce que nous apprîmes \\
ne nous servit à rien: \\
nous recommen\c{c}âmes à zéro, \\
nous terminâmes à nouveau\footnote{NDT. {\em de nuevo} peut signifier
{\em à nouveau} (répétition) mais aussi, avec le verbe {\em commencer},
l'idée d'un nouveau départ (vers précédent).} \\
la mort et la vie. \\
Et ici nous survécûmes, \\
purs, avec la pureté que nous créâmes, \\
plus larges que la terre qui ne put nous dévoyer, \\
éternels comme le feu qui brûlera \\
tant que la vie durera.
\end{verse}

\newpage

{\huge IV} \\ \\
\addcontentsline{toc}{section}{IV. \ Quand je suis parvenu ici ma main s'interrompt}

\begin{verse}
Quand je suis parvenu ici ma main s'interrompt. \\
Quelqu'un demande: --- Dis-moi pourquoi, comme les vagues \\
sur une même côte, tes mots \\
sans cesser vont et reviennent à son corps? \\
Est-elle la seule forme que tu aimes? \\
Je réponds: mes mains ne se rassasient pas \\
sur elle, mes baisers ne fatiguent pas, \\
pourquoi retirerais-je les mots \\
qui reproduisent l'empreinte de son contact aimé, \\
qui se referment en gardant \\
inutilement comme le filet garde l'eau, \\
la surface et la température \\
de la vague la plus pure de la vie? \\
Et, mon amour, ton corps n'est pas seulement la rose \\
qui dans l'ombre ou la lune se lève, \\
ou que je surprends ou poursuis. \\
Non seulement il est mouvement ou brûlure, \\
acte de sang ou pétale du feu, \\
mais pour moi tu m'as apporté \\
mon territoire, la boue de mon enfance, \\
les vagues de l'avoine, \\
la peau ronde du fruit obscur \\
que j'arrachai de la jungle, \\
arôme de bois et pommes, \\
couleur d'eau cachée où tombent \\
des fruits secrets et de profondes feuilles. \\
Oh mon amour ton corps s'élève \\
comme une ligne pure de pot \\
depuis la terre qui me reconnaît \\
et quand te rencontrèrent mes sens  \\
tu palpitas comme si tombaient \\
en toi la pluie et les graines! \\
Ah qu'on me dise comment \\
je pourrais t'abolir \\
et que mes mains sans ta forme \\
arrachent le feu de mes mots! \\
Ma douce, repose \\
ton corps sur ces lignes qui te doivent \\
plus que ce que tu me donnes à ton contact, \\
vis en ces mots et reproduis \\
en eux la douceur et l'incendie, \\
frisonne au milieu de leurs syllabes, \\
dors dans mon nom comme tu t'es endormie \\
sur mon c{\oe}ur, et ainsi demain \\
mes mots garderont \\
le creux de ta forme \\
et celui qui les entendra un jour recevra une rafale \\
de blé et de coquelicots: \\
il respirera encore \\
le corps de l'amour sur la terre!
\end{verse}

\newpage

{\huge V} \\ \\
\addcontentsline{toc}{section}{V. \ Fil de blé et eau}

\begin{verse}
Fil de blé et eau, \\
de cristal ou de feu, \\
la parole et la nuit, \\
le travail et l'ire, \\
l'ombre et la tendresse, \\
tout cela tu l'as peu à peu cousu \\
à mes poches trouées, \\
et non seulement tu m'attendis \\
dans la zone trépidante \\
où l'amour et le martyre sont jumeaux  \\
comme deux cloches d'incendie, \\
mon amour, \\
mais aussi dans les plus petites \\
obligations douces. \\
L'huile d'olive d'Italie fit ton nimbe, \\
sainte de la cuisine et la couture, \\
et ta toute petite coquetterie, \\
qui s'attardait tant dans le miroir, \\
avec tes mains qui ont \\
des pétales que le jasmin envierait \\
lava les ustensiles et mes habits, \\
désinfecta les plaies. \\
Mon amour, tu vins \\
à ma vie préparée \\
comme coquelicot et comme guérilléro: \\
la splendeur de soie que je parcours \\
avec la faim et la soif \\
que j'apportai pour toi seulement à ce monde, \\
et derrière la soie \\
la jeune fille de fer \\
qui luttera à mes côtés. \\
Mon amour, mon amour, ici nous nous trouvâmes. \\
Soie et métal, approche-toi de ma bouche.
\end{verse}

\newpage

{\huge VI} \\ \\
\addcontentsline{toc}{section}{VI. \ Et parce que Amour combat}

\begin{verse}
Et parce que Amour combat \\
non seulement dans sa brûlante agriculture, \\
mais aussi dans la bouche des hommes et femmes, \\
je finirai par sortir du chemin \\
ceux qui entre ma poitrine et ta fragrance \\
voudraient interposer leur plante obscure. \\
De moi rien de mauvais en plus \\
ils ne te diront, mon amour, \\
de ce que je t'ai dit. \\
J'ai vécu dans les prairies \\
avant de te connaître \\
et je n'ai pas attendu l'amour mais \\
je guettais et je sautai sur la rose. \\
Quoi de plus peuvent-ils te dire? \\
Je ne suis ni bon ni mauvais mais un homme, \\
et ils ajouteront alors le danger \\
de ma vie, que tu connais \\
et qu'avec ta passion tu as partagé. \\
Eh bien, ce danger \\
est danger d'amour, d'amour complet \\
envers toute la vie, \\
envers toutes les vies, \\
et si cet amour nous apporte \\
la mort ou les prisons, \\
je suis certain que tes grands yeux, \\
comme quand je les baise \\
se fermeront alors avec fierté, \\
avec double fierté, mon amour, \\
avec ta fierté et la mienne. \\
Mais vers mes oreilles ils viendront avant \\
pour saper la tour \\
de l'amour doux et dur qui nous lie, \\
et ils me diront: --- <<Celle \\
que tu aimes, \\
elle n'est pas une femme pour toi, \\
pourquoi l'aimes-tu? Je crois \\
que tu pourrais en trouver une plus belle, \\
plus sérieuse, plus profonde, \\
plus autre, tu me comprends, regarde la si fluette, \\
et quelle tête elle a, \\
et regarde comme elle s'habille \\
et {\em et cetera} et {\em et cetera}>>. \\
Et moi dans ces lignes je dis: \\
comme cela je t'aime, mon amour, \\
mon amour, comme cela je t'aime, \\
comme tu t'habilles \\
et comme se lève \\
ta chevelure et comme \\
ta bouche sourit, \\
légère comme l'eau \\
de la source sur les pierres pures, \\
comme cela je t'aime ma chérie. \\
Je ne demande pas au pain qu'il m'enseigne \\
mais qu'il ne me vienne jamais à manquer \\
tous les jours de la vie. \\
Moi je ne sais rien de la lumière, d'où \\
elle vient, où elle va, \\
moi je veux seulement que la lumière allume, \\
moi je ne demande pas à la nuit \\
des explications, \\
moi je l'attends et elle m'enveloppe, \\
et comme cela toi, pain et lumière \\
et ombre tu es. \\
Tu es venue à ma vie \\
avec ce que tu apportais \\
faite \\
de lumière et pain et ombre je t'attendais, \\
et comme cela j'ai besoin de toi, \\
et comme cela je t'aime, \\
et à ceux qui voudront écouter demain \\
ce que je ne leur dirai pas, qu'ils le lisent ici, \\
et qu'ils reculent aujourd'hui car il est tôt \\
pour ces arguments. \\
Demain seulement nous leur donnerons \\
une feuille de notre amour, une feuille \\
qui tombera sur la terre, \\
comme si nos lèvres l'avaient faite, \\
comme un baiser qui tombe \\
depuis nos hauteurs invincibles \\
pour montrer le feu et la tendresse \\
d'un amour véritable.
\end{verse}

\cleardoublepage

\part{Épithalame}

\cleardoublepage

\begin{verse}
Te souviens-tu quand \\
en hiver nous arrivâmes à l'île? \\
La mer vers nous soulevait \\
une coupe de froid. \\
Sur les murs les liserons \\
susurraient en laissant \\
tomber d'obscures feuilles \\
sur notre passage. \\
Tu étais aussi une petite feuille \\
qui tremblait sur ma poitrine. \\
Le vent de la vie là-bas te poussa. \\
Au début je ne te vis pas: je ne sus \\
que tu marchais avec moi, \\
jusqu'à ce que tes racines \\
percent ma poitrine, \\
elles s'unirent aux fils de mon sang, \\
elles parlèrent par ma bouche, \\
elles fleurirent avec moi. \\
Ainsi fut ta présence inaper\c{c}ue, \\
feuille ou branche invisible \\
et mon c{\oe}ur se peupla soudain \\
de fruits et de sons. \\
Tu habitas la maison \\
qui t'attendait obscure \\
et tu allumas les lampes alors. \\
Te souviens-tu, mon amour, \\
de nos premiers pas sur l'île? \\
Les pierres grises nous reconnurent, \\
les averses, \\
les cris du vent dans l'ombre. \\
Mais le feu a été \\
notre unique ami, \\
près de lui nous serrâmes \\
le doux amour d'hiver \\
à quatre bras. \\
Le feu vit croître notre baiser nu \\
jusqu'à toucher des étoiles cachées, \\
et il vit naître et mourir la douleur \\
comme une épée brisée \\
contre l'amour invincible. \\
Te souviens-tu, \\
oh dormeuse dans mon ombre, \\
comment de toi croissait \\
le songe, \\
de ta poitrine nue \\
ouverte avec ses coupoles jumelles \\
vers la mer, vers le vent de l'île \\
et comment dans ton songe je naviguais \\
libre, sur la mer et dans le vent \\
attaché et immergé pourtant \\
dans le volume bleu de ta douceur? \\
Oh douce, oh ma douce, \\
le printemps changea \\
les murs de l'île. \\
Une fleur apparut comme une goutte \\
de sang orangé, \\
et ensuite les couleurs déchargèrent \\
tout leur poids pur. \\
La mer reconquérit sa transparence, \\
la nuit dans le ciel \\
apporta ses grappes \\
et alors toutes les choses susurrèrent \\
notre nom d'amour, pierre par pierre \\
elles dirent notre nom et notre baiser. \\
L'île de pierre et mousse \\
résonna dans le secret de ses grottes \\
comme dans ta bouche le chant, \\
et la fleur qui naissait \\
parmi les interstices de la pierre \\
avec sa secrète syllabe \\
dit au passage ton nom \\
de plante brûlante, \\
et la roche escarpée, levée \\
comme le mur du monde \\
reconnut mon chant, ma bien-aimée, \\
et toutes les choses dirent \\
ton amour, mon amour, mon aimée, \\
parce que la terre, le temps, la mer, l'île, \\
la vie, la marée, \\
le germe qui entrouvre \\
ses lèvres dans la terre, \\
la fleur dévoreuse, \\
le mouvement du printemps, \\
tout nous reconnait. \\
Notre amour est né \\
hors des murs, \\
dans le vent, \\
dans la nuit, \\
sur la terre, \\
et c'est pourquoi l'argile et la corolle, \\
la boue et les racines \\
savent comment tu t'appelles, \\
et elles savent que ma bouche \\
se joignit à la tienne \\
parce que sur la terre on nous sema ensemble \\
sans que seulement nous le sûmes, \\
et que nous croissons ensemble \\
et fleurissons ensemble \\
et c'est pourquoi \\
quand nous passons, \\
ton nom est dans les pétales \\
de la rose qui croît sur la pierre, \\
mon nom est dans les grottes. \\
Elles savent tout, \\
nous n'avons pas de secrets, \\
nous avons crû ensemble \\
mais nous ne le savions pas. \\
La mer connaît notre amour, les pierres \\
de la hauteur rocheuse \\
savent que nos baisers fleurirent \\
avec une pureté infinie, \\
comment parmi ses interstices une bouche \\
écarlate voit poindre le jour: \\
ainsi elles connaissent notre amour et le baiser \\
qui réunit ta bouche et la mienne \\
en une fleur éternelle. \\
Mon amour, \\
le doux printemps, \\
fleur et mer, nous entourent. \\
Nous ne l'échangeâmes pas \\
contre notre hiver, \\
quand le vent \\
commen\c{c}a à déchiffrer ton nom \\
qu'aujourd'hui à toute heure il répète, \\
quand \\
les feuilles ne savaient pas \\
que tu étais une feuille, \\
quand \\
les racines \\
ne savaient pas que tu me cherchais \\
sur ma poitrine. \\
Mon amour, mon amour, \\
le printemps nous offre le ciel, \\
mais la terre obscure \\
est notre nom, \\
notre amour appartient \\
à tout le temps et la terre. \\
En nous aimant, mon bras \\
sous ton cou de sable \\
nous attendrons \\
comment changent la terre et le temps \\
sur l'île, \\
comment tombent les feuilles \\
des liserons taciturnes, \\
comment s'en va l'automne \\
par la fenêtre cassée. \\
Mais nous \\
nous allons attendre \\
notre ami, \\
notre ami aux yeux rouges, \\
le feu, \\
quand à nouveau le vent \\
secouera les frontières de l'île \\
et ignorera le nom \\
de tous, \\
l'hiver \\
nous cherchera, mon amour, \\
toujours, \\
il nous cherchera, parce que nous le connaissons, \\
parce que nous ne le craignons pas, \\
parce que nous avons \\
avec nous \\
le feu \\
pour toujours. \\
Nous avons \\
la terre avec nous \\
pour toujours, \\
le printemps avec nous \\
pour toujours, \\
et quand se détachera \\
des liserons \\
une feuille \\
tu sauras, mon amour, \\
quel nom est inscrit \\
sur cette feuille, \\
un nom qui est le tien et est le mien, \\
notre nom d'amour, un seul \\
être, la flèche \\
qui traversa l'hiver, \\
l'amour invincible, \\
le feu des journées, \\
une feuille \\
qui tomba sur ma poitrine, \\
une feuille de l'arbre \\
de la vie \\
qui fit son nid et chanta, \\
qui fit des racines, \\
qui donna des fleurs et des fruits. \\
Et ainsi tu vois, mon amour, \\
comment il marcha \\
à travers l'île, \\
à travers le monde, \\
sûr au milieu du printemps, \\
fou de lumière dans le froid, \\
marchant calme dans le feu, \\
soulevant ton poids \\
de pétale dans mes bras, \\
comme s'il n'avait toujours cheminé \\
qu'avec toi, mon c{\oe}ur, \\
comme s'il ne savait cheminer \\
qu'avec toi, \\
comme s'il ne savait chanter \\
que quand tu chantes.
\end{verse}

\cleardoublepage

\part{La lettre en chemin}

\cleardoublepage

\begin{verse}
Adieu, mais avec moi \\
tu seras, tu iras dans \\
une goutte de sang qui circulera dans mes veines \\
ou dehors, baiser qui embrase le visage \\
ou ceinture de feu sur ma taille. \\
Ma douce, re\c{c}oit \\
le grand amour qui sortit de ma vie \\
et qui en toi ne trouvait pas de territoire \\
comme l'explorateur perdu \\
dans les îles du pain et du miel. \\
Je te trouvai après \\
l'orage, \\
la pluie lava l'air \\
et dans l'eau \\
tes doux pieds brillèrent comme des poissons. \\ \

Mon adorée, je vais à mes combats. \\ \

Je grifferai la terre pour te faire une caverne \\
et là ton Capitaine \\
t'attendra avec des fleurs sur la couche. \\
Ne pense plus, ma douce, \\
à la tourmente \\
qui passa entre nous \\
comme un éclair de phosphore \\
en nous laissant peut-être sa brûlure. \\
La paix arriva aussi parce que je reviens \\
lutter sur ma terre, \\
et comme j'ai le c{\oe}ur complet \\
avec la part de sang que tu me donnas \\
pour toujours, \\
et comme \\
j'ai \\
les mains pleines de ton être nu, \\
regarde-moi, \\
regarde-moi, \\
regarde-moi sur la mer, car je vais radieux, \\
regarde-moi dans la nuit car je navigue, \\
et mer et nuit sont tes yeux. \\
Je ne suis pas sorti de toi quand je m'éloigne. \\
Maintenant je vais te raconter: \\
ma terre sera la tienne, \\
je vais la conquérir, \\
non seulement pour te la donner, \\
mais aussi pour tous, \\
pour tout mon peuple. \\
Un jour le voleur sortira de sa tour. \\
Et l'envahisseur sera bouté. \\
Tous les fruits de la vie \\
croîtront dans mes mains \\
habituées avant à la poudre. \\
Et je saurai caresser les nouvelles fleurs \\
parce que tu m'enseignas la tendresse. \\
Ma douce, mon adorée, \\
tu viendras avec moi lutter corps à corps \\
parce que dans mon c{\oe}ur vivent tes baisers \\
comme de rouges drapeaux, \\
et si je tombe, non seulement \\
la terre me couvrira \\
mais aussi ce grand amour que tu m'apportas \\
et qui vécut circulant dans mon sang. \\
Tu viendras avec moi, \\
en cette heure je t'attends, \\
en cette heure et en toutes les heures, \\
en toutes les heures je t'attends. \\
Et quand viendra la tristesse que je hais \\
cogner à ta porte, \\
dis-lui que je t'attends \\
et quand la solitude voudra que tu changes \\
le pot sur lequel mon nom est écrit, \\
dis à la solitude qu'elle parle avec moi, \\
que je dus partir \\
parce que je suis un soldat \\
et que là où je suis, \\
sous la pluie ou sous \\
le feu, \\
mon amour, je t'attends. \\
Je t'attends dans le désert le plus dur \\
et près du citronnier en fleurs, \\
partout où se trouve la vie, \\
où le printemps nait, \\
mon amour, je t'attends. \\
Quand on te dira: <<Cet homme \\
ne t'aime pas>>, souviens-toi \\
que mes pieds sont seuls dans cette nuit, et cherchent \\
les doux et petits pieds que j'adore. \\
Mon amour, quand on te dira \\
que je t'ai oubliée, et quand bien même \\
ce serait moi qui le dirait, \\
quand même je te le dirais, \\
ne me crois pas, \\
qui et comment pourrait \\
te couper de ma poitrine \\
et qui recevrait \\
mon sang \\
si alors vers toi je saignais? \\
Mais je ne peux non plus \\
oublier mon peuple. \\
Je vais lutter dans chaque rue, \\
derrière chaque pierre. \\
Ton amour aussi m'aide: \\
il est une fleur close \\
qui à chaque fois m'emplit de son arôme \\
et qui s'ouvre soudain \\
en moi comme une grande étoile. \\ \

Mon amour, c'est la nuit. \\ \

L'eau noire, le monde \\
endormi, m'entourent. \\
Plus tard viendra l'aurore, \\
et pendant ce temps je t'écris \\
pour te dire: <<Je t'aime>>. \\
Pour te dire <<Je t'aime>>, soigne, \\
nettoie, élève, \\
défends \\
notre amour, mon c{\oe}ur. \\
Je te le laisse comme si je te laissais \\
une poignée de terre avec des graines. \\
De notre amour naîtront des vies. \\
En notre amour elles boieront de l'eau. \\
Peut-être viendra un jour \\
où un homme \\
et une femme, pareils \\
à nous, \\
toucheront cet amour et il aura encore la force \\
de brûler les mains qui le toucheront. \\
Qui avons-nous été? Qu'importe? \\
Ils toucheront cet amour \\
et le feu, ma douce, dira ton simple nom \\
et le mien, le nom \\
que toi seule tu sus parce que toi seule \\
sur la terre sait \\
qui je suis, et parce que personne ne me connut comme une, \\
une seule de tes mains, \\
parce que personne \\
ne sut comment, ni quand \\
mon c{\oe}ur brûlait, \\
seulement \\
tes grands yeux sombres le surent, \\
ta large bouche, \\
ta peau, tes seins, \\
ton ventre, tes entrailles \\
et ton âme que je réveillai \\
pour qu'elle chante \\
jusqu'à la fin de la vie. \\ \

Mon amour, je t'attends. \\ \

Adieu, mon amour, je t'attends. \\ \

Mon amour, mon amour, je t'attends. \\ \

\newpage

Et ainsi cette lettre s'achève \\
sans aucune tristesse: \\
mes pieds sont fermes sur la terre, \\
ma main écrit cette lettre en chemin, \\
et au milieu de la vie je serai \\
toujours \\
près de l'ami, face à l'ennemi, \\
avec ton nom aux lèvres \\
et un baiser qui jamais \\
ne quitta les tiennes.
\end{verse}

\cleardoublepage

\tableofcontents

\end{document}
