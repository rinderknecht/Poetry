%%-*-latex-*-

\documentclass[11pt,a4paper]{book}

\usepackage[T1]{fontenc}
\usepackage[utf8]{inputenc}
\usepackage[spanish,french]{babel}
\usepackage[charter]{mathdesign}
\usepackage{verse}
\usepackage{url}

\frenchspacing  % Follow French conventions after a period

\begin{document}

\thispagestyle{empty}
\vspace*{70mm}
\begin{center}
  \textbf{\Huge VINGT POÈMES D'AMOUR \\
    ET UNE CHANSON \\
    DÉSESPÉRÉE} \\
\vspace*{10mm}
{\LARGE Pablo Neruda} \\
\vspace*{10mm}
Traduction de Christian Rinderknecht \\
\date{Noël 1996}
\end{center}

\cleardoublepage

\thispagestyle{empty}

\vspace*{80mm}
\begin{center}
{\huge LES VINGT POÈMES D'AMOUR}
\end{center}

\cleardoublepage

\addcontentsline{toc}{chapter}{LES VINGT POÈMES D'AMOUR}

\addcontentsline{toc}{subsection}{\emph{I.\ Corps de femme, blanches
collines, cuisses blanches...}}

\begin{center} \textbf{I} \end{center}

\bigskip

\begin{verse}
  Corps de femme, blanches collines, cuisses blanches, \\
  tu ressembles au monde par tes dons.\\
  Mon corps de laboureur sauvage te sape \\
  et fait bondir le fils du fond de la terre.

  Je fus seul comme un tunnel. Les oiseaux me fuyaient \\
  et en moi la nuit pénétrait de son invasion puissante. \\
  Pour me survivre je t'ai forgée comme une arme, \\
  comme une flèche à mon arc, comme une pierre à ma fronde.

  Mais l'heure de la vengeance tombe à pic, et je t'aime. \\
  Corps de peau, de mousse, de lait avide et ferme. \\
  Ah les vases de la poitrine! Ah les yeux de l'absence! \\
  Ah les roses du pubis! Ah ta voix lente et triste!

  Corps de femme mienne, je persisterai en ta grâce. \\
  Ma soif, mon désir sans bornes, mon chemin indécis! \\
  Lits de rivières obscurs où la soif éternelle continue, \\
  et la fatigue continue, et la douleur infinie.
\end{verse}

\newpage

\addcontentsline{toc}{subsection}{\emph{II.\ Dans sa flamme mortelle la
lumière t'enveloppe....}}

\begin{center} \textbf{II} \end{center}

\bigskip

\begin{verse}
  Dans sa flamme mortelle la lumière t'enveloppe. \\
  Absorbée, pâle dolente, ainsi située \\
  contre les vieilles hélices du crépuscule \\
  qui tourne autour de toi.

  Muette, mon amie, \\
  seule dans la solitude de cette heure de morts \\
  et pleine des vies du feu, \\
  pure héritière du jour détruit.

  Du soleil tombe une grappe sur ta robe noire. \\
  Les grandes racines de la nuit\\
  croissent subitement de ton âme, \\
  et les choses en toi occultes s'en retournent au dehors, \\
  de telle sorte qu'un peuple pâle et bleu \\
  de toi nouveau-né s'en alimente.

  Oh grandiose et féconde et magnétique esclave \\
  du cercle qui le noir et le doré alterne: \\
  dressée, elle tente et obtient une création si vive \\
  que succombent ses fleurs, et est pleine de tristesse.
\end{verse}

\newpage

\addcontentsline{toc}{subsection}{\emph{III.\ Ah vastitude de pins, rumeur de
vagues se brisant...}}

\begin{center} \textbf{III} \end{center}

\bigskip

\begin{verse}
  Ah vastitude de pins, rumeur de vagues se brisant, \\
  lent jeu de lumières, cloche solitaire, \\
  crépuscule tombant dans tes  yeux, poupée, \\
  conque terrestre, en toi la terre chante!

  En toi les rivières chantent et mon âme sur elles s'enfuit \\
  comme tu le désirerais et vers où tu le voudrais. \\
  Trace mon chemin sur ton arc d'espérance \\
  et je lâcherai en délire ma volée de flèches.

  Autour de moi je vois maintenant ta ceinture de brume \\
  et ton silence harcèle mes heures poursuivies, \\
  et c'est toi avec tes bras de pierre transparente \\
  où mes baisers jettent l'ancre et mon humide désir niche.

  Ah ta voix mystérieuse que l'amour teinte et redouble \\
  dans le soir résonnant et mourant! \\
  Ainsi dans les heures profondes sur les champs j'ai vu \\
  se plier les épis dans la gueule du vent.
\end{verse}

\newpage

\addcontentsline{toc}{subsection}{\emph{IV.\ C'est le matin plein de
tempête au c{\oe}ur de l'été...}}

\begin{center} \textbf{IV} \end{center}

\bigskip

\begin{verse}
  C'est le matin plein de tempête \\
  au cœur de l'été.

  Les nuages voyagent tels de blancs mouchoirs d'adieu, \\
  le vent les agite de ses mains voyageuses.

  Innombrable cœur du vent \\
  battant sur notre silence amoureux.

  Bourdonnant entre les arbres, orchestral et divin, \\
  comme une langue pleine de guerres et de chants.

  Vent qui dérobe en vol rapide les feuilles mortes \\
  et dévie les flèches palpitantes des oiseaux.

  Vent qui la détrône en vague sans écume \\
  et substance sans poids, et feux inclinés.

  Se brise et se submerge son volume de baisers \\
  combattu à la porte du vent de l'été.
\end{verse}

\newpage

\addcontentsline{toc}{subsection}{\emph{V.\ Pour que tu m'entendes mes
paroles s'amenuisent parfois...}}

\begin{center} \textbf{V} \end{center}

\bigskip

\begin{verse}
  Pour que tu m'entendes \\
  mes mots \\
  s'amenuisent parfois \\
  comme les empreintes des mouettes sur les plages.

  Collier, grelot ivre \\
  pour tes mains douces comme le raisin.

  Et je les regarde lointains mes mots. \\
  Plus que miens ils sont tiens. \\
  Ils vont grimpant sur ma vieille douleur comme le lierre.

  Ils grimpent ainsi sur les murs humides. \\
  C'est toi la coupable de ce jeu sanglant.

  Ils s'enfuient de mon antre obscur. \\
  C'est toi qui emplis tout, tu emplis tout.

  Avant toi ils peuplèrent la solitude que tu occupes, \\
  et ils sont plus habitués que toi à ma tristesse.

  Maintenant je veux qu'ils disent ce que je veux te dire \\
  pour que tu les entendes comme je veux que tu m'entendes.

  Le vent de l'angoisse encore et toujours les traîne. \\
  Des ouragans de songes encore et parfois les couchent.

  Tu écoutes d'autres voix dans ma voix endolorie. \\
  Sanglot d'anciennes bouches, sang d'anciennes suppliques. \\
  Aime-moi, compagne. Ne m'abandonne pas. Suis-moi. \\
  Suis-moi, compagne, sur cette vague de nausée.

  Mais peu à peu mes mots se teintent de ton amour. \\
  C'est toi qui occupes tout, tu occupes tout.

  Je fais d'eux un collier infini \\
  pour tes blanches mains, douces comme le raisin.
\end{verse}

\newpage

\addcontentsline{toc}{subsection}{\emph{VI.\ Je me souviens de toi comme tu
étais au dernier automne...}}

\begin{center} \textbf{VI} \end{center}

\bigskip

\begin{verse}
  Je me souviens de toi telle que tu étais au dernier automne. \\
  Tu étais le béret gris et le cœur en paix. \\
  Dans tes yeux se battaient les flammes du crépuscule. \\
  Et les feuilles tombaient sur l'eau de ton âme.

  Serrée contre mes bras comme du liseron, \\
  les feuilles recueillaient ta voix lente et en paix. \\
  Foyer de stupeur dans lequel ma soif flambait. \\
  Douce jacinthe bleue incurvée sur mon âme.

  Je sens tes yeux voyager et l'automne est distant: \\
  béret gris, voix d'oiseau et cœur de logis \\
  vers lesquels émigraient mes profonds désirs \\
  et tombaient mes baisers joyeux comme des braises.

  Ciel depuis un navire. Champ depuis les collines. \\
  Ton souvenir est de lumière, de fumée, d'étang en paix! \\
  Au-delà de tes yeux flambaient les crépuscules. \\
  Des feuilles mortes d'automne tournoyaient dans ton âme.
\end{verse}

\newpage

\addcontentsline{toc}{subsection}{\emph{VII.\ Penché dans les soirs je jette
mes tristes filets...}}

\begin{center} \textbf{VII} \end{center}

\bigskip

\begin{verse}
  Penché dans les soirs je jette mes tristes filets \\
  à tes yeux océaniques.

  Là s'étire et flambe dans le plus haut brasier \\
  ma solitude qui tourne les bras comme un naufragé.

  Je fais de rouges signaux sur tes yeux absents \\
  qui palpitent comme la mer au pied d'un phare.

  Tu ne retiens que ténèbres, femme distante et mienne, \\
  de ton regard émerge parfois la côte de l'effroi.

  Penché dans les soirs je tends mes tristes filets \\
  à cette mer qui bat tes yeux océaniques.

  Les oiseaux nocturnes picorent les premières étoiles \\
  qui scintillent comme mon âme quand je t'aime.

  La nuit galope sur sa sombre jument, \\
  répandant des épis bleus sur la campagne.
\end{verse}

\newpage

\addcontentsline{toc}{subsection}{\emph{VIII.\ Blanche abeille tu bourdonnes
--- ivre de miel ---...}}

\begin{center} \textbf{VIII} \end{center}

\bigskip

\begin{verse}
  Blanche abeille tu bourdonnes --- ivre de miel~--- dans mon âme \\
  et tu te tords en lentes spirales de fumée.

  Je suis le désespéré, la parole sans échos, \\
  celui qui perdit tout, et celui qui posséda tout.

  Ultime amarre, en toi craque mon anxiété ultime. \\
  En ma terre déserte tu es l'ultime rose.

  Ah silencieuse!

  Clos tes yeux profonds. Là bat des ailes la nuit. \\
  Ah dénude ton corps de statue craintive.

  Tu as des yeux profonds où la nuit bat des ailes. \\
  De frais bras de fleur et giron de rose.

  Tes seins ressemblent aux escargots blancs. \\
  Un papillon d'ombre est venu s'endormir sur ton ventre.

  Ah silencieuse!

  Voici la solitude d'où tu es absente. \\
  Il pleut. Le vent marin chasse d'errantes mouettes.

  L'eau marche pieds nus dans les rues trempées. \\
  De cet arbre geignent, comme des malades, les feuilles.

  Blanche abeille, absente, encore tu bourdonnes dans mon âme. \\
  Tu revis dans le temps, fine et silencieuse.

  Ah silencieuse!
\end{verse}

\newpage

\addcontentsline{toc}{subsection}{\emph{IX.\ Ivre de térébenthine et de
longs baisers...}}

\begin{center} \textbf{IX} \end{center}

\bigskip

\begin{verse}
  Ivre de térébenthine et de longs baisers, \\
  estival, je dirige le voilier des roses, \\
  tordu vers la mort du mince jour, \\
  cimenté dans la solide frénésie marine.

  Pâle et amarré à mon eau dévorante \\
  je croise dans l'aigre odeur de la région découverte, \\
  encore vêtu de gris et de sons amers, \\
  et un cimier triste d'écume abandonnée.

  Je vais, endurci de passions, montant mon unique vague, \\
  lunaire, solaire, ardent et froid, soudain, \\
  endormi dans la gorge des fortunées \\
  îles blanches et douces comme des hanches fraîches.

  Dans la nuit humide tremble mon habit de baisers \\
  follement chargé d'électriques manœuvres, \\
  héroïquement composé de songes \\
  et d'enivrantes roses se déployant en moi.

  Eaux debout, au milieu des lames externes, \\
  ton corps parallèle s'agrippe à mes bras \\
  comme un poisson infiniment collé à mon âme \\
  rapide et lent dans l'énergie subcéleste.
\end{verse}

\newpage

\addcontentsline{toc}{subsection}{\emph{X.\ Nous avons perdu même ce
crépuscule...}}

\begin{center} \textbf{X} \end{center}

\bigskip

\begin{verse}
  Nous avons perdu même ce crépuscule. \\
  Personne ne nous vit ce soir les mains jointes \\
  pendant que la nuit bleue tombait sur le monde.

  J'ai vu de ma fenêtre \\
  la fête du couchant sur les coteaux lointains.

  Parfois comme une monnaie \\
  s'allumait un morceau de soleil dans mes mains.

  Je me souvenais de toi avec l'âme enserrée \\
  de cette tristesse que toi tu me connais.

  Où étais-tu alors? \\
  Parmi quelles gens? \\
  Disant quelles paroles? \\
  Pourquoi me vient tout l'amour d'un coup \\
  lorsque je me sens triste, et que je te sens lointaine?

  Le livre que l'on emporte toujours dans le crépuscule tomba, \\
  et comme un chien blessé ma cape roula à mes pieds.

  Toujours, toujours tu t'éloignes dans les soirs \\
  vers où le crépuscule court en effaçant des statues.
\end{verse}

\newpage

\addcontentsline{toc}{subsection}{\emph{XI.\ Presque hors du ciel jette
l'ancre entre deux montagnes...}}

\begin{center} \textbf{XI} \end{center}

\bigskip

\begin{verse}
  Presque hors du ciel jette l'ancre entre deux montagnes \\
  la moitié de la lune. \\
  Tournante, errante nuit, la terrassière des yeux. \\
  Que d'étoiles en morceaux à voir dans la flaque.

  Elle fait une croix de deuil entre mes sourcils, elle fuit. \\
  Forge de métaux bleus, nuits des luttes silencieuses, \\
  mon cœur tournoie comme un volant fou. \\
  Petite venue de si loin, amenée de si loin, \\
  parfois fulgure son regard sous le ciel. \\
  Gémissement, tempête, tourbillon de furie, \\
  traverse sur mon cœur, sans t'arrêter. \\
  Ô vent des sépulcres charrie, détruis, disperse ta racine somnolente. \\
  Déracine les grands arbres de l'autre côté d'elle. \\
  Mais toi, claire petite, question de fumée, épi. \\
  Elle était celle que formait peu à peu le vent avec des feuilles illuminées. \\
  Derrière les montagnes nocturnes, blanc lys d'incendie, \\
  ah je ne peux rien dire! Elle était faite de toutes les choses. \\ \

  Désir violent qui me fendit la poitrine à coups de couteau, \\
  il est l'heure de suivre un autre chemin, où elle ne sourira pas.\\
  Tempête qui enterra les cloches, trouble et nouvel essor des tourments \\
  pourquoi la toucher maintenant, pourquoi l'attrister.

  Suivre hélas le chemin qui s'éloigne de tout, \\
  où ne taillade pas l'angoisse, la mort, l'hiver, \\
  avec ses yeux ouverts parmi la rosée.
\end{verse}

\newpage

\addcontentsline{toc}{subsection}{\emph{XII.\ Pour mon cœur suffit ta
poitrine...}}

\begin{center} \textbf{XII} \end{center}

\bigskip

\begin{verse}
  Pour mon cœur suffit ta poitrine, \\
  pour ta liberté suffisent mes ailes. \\
  De ma bouche parviendra jusqu'au ciel \\
  ce qui était endormi sur ton âme.

  Sont en toi les promesses de chaque jour. \\
  Tu viens comme la rosée aux corolles. \\
  Tu sapes l'horizon par ton absence. \\
  Éternellement en fugue comme la vague.

  J'ai dit que tu chantais dans le vent \\
  comme les pins et comme les mâts. \\
  Comme eux tu es haute et taciturne. \\
  Et tu t'attristes soudain, comme un voyage.

  Accueillante comme un vieux chemin. \\
  Tu es peuplée d'échos et de voix nostalgiques. \\
  Je me suis éveillé et parfois émigrent et fuient \\
  des oiseaux qui dormaient sur ton âme.
\end{verse}

\newpage

\addcontentsline{toc}{subsection}{\emph{XIII.\ J'ai marqué au fur et à mesure
avec des croix de feu...}}

\begin{center} \textbf{XIII} \end{center}

\bigskip

\begin{verse}
  J'ai marqué au fur et à mesure avec des croix de feu \\
  l'atlas blanc de ton corps. \\
  Ma bouche était une araignée qui traversait en tapinois. \\
  En toi, derrière toi, craintive, assoiffée.

  Des histoires à te conter à l'orée du crépuscule, \\
  poupée triste et douce, pour que tu ne fusses pas triste. \\
  Un cygne, un arbre, quelque chose lointaine et joyeuse. \\
  Le temps des raisins, le temps mature et fruitier.

  Moi qui vécus dans un port depuis lequel je t'aimais. \\
  La solitude traversée de songe et de silence. \\
  Acculé entre la mer et la tristesse. \\
  Silencieux, délirant, entre deux gondoliers immobiles.

  Entre les lèvres et la voix, quelque chose se meurt. \\
  Quelque chose avec des ailes d'oiseau, quelque chose d'angoisse et d'oubli. \\
  Tout comme les filets ne retiennent pas l'eau. \\
  Ma poupée, à peine reste-t-il des gouttes tremblotantes. \\
  Pourtant, quelque chose chante parmi ces paroles fugaces. \\
  Quelque chose chante, quelque chose monte jusqu'à mon avide bouche. \\
  Oh pouvoir te célébrer avec toutes les paroles de joie. \\
  Chanter, flamber, fuir, comme un clocher aux mains d'un fou. \\
  Ma triste tendresse, que deviens-tu soudain? \\
  Quand je suis parvenu au sommet le plus osé et froid \\
  mon cœur se referme comme une fleur nocturne.
\end{verse}

\newpage

\addcontentsline{toc}{subsection}{\emph{XIV.\ Tu joues tous les jours avec la
lumière de l'univers...}}

\begin{center} \textbf{XIV} \end{center}

\bigskip

\begin{verse}
  Tu joues tous les jours avec la lumière de l'univers. \\
  Subtile visiteuse, tu viens sur la fleur et dans l'eau. \\
  Tu es plus que cette blanche et petite tête que je presse \\
  comme une grappe entre mes mains chaque jour.

  Tu ne ressembles à personne depuis que je t'aime. \\
  Laisse-moi t'étendre parmi les guirlandes jaunes. \\
  Qui inscrit ton nom avec des lettres de fumée \\
  parmi les étoiles du sud? \\
  Ah laisse-moi me souvenir de celle que tu étais alors, \\
  quand tu n'existais pas encore.

  Soudain le vent hurle et cogne ma fenêtre close. \\
  Le ciel est un filet chargé de sombres poissons. \\
  Ici viennent frapper tous les vents, tous. \\
  La pluie se dévêt.

  Les oiseaux passent en fuite. \\
  Le vent. Le vent. \\
  Je ne peux lutter que contre la force des hommes. \\
  La tempête entourbillonne d'obscures feuilles \\
  et libère toutes les barques qu'hier soir on amarra au ciel.

  Toi tu es ici. Ah toi tu ne fuis pas. \\
  Toi tu me répondras jusqu'au dernier cri. \\
  Blottis-toi à mon côté comme si tu avais peur. \\
  Pourtant une ombre étrange a parfois traversé tes yeux.

  Maintenant, maintenant aussi, petite, tu m'apportes du chèvrefeuille \\
  et jusqu'à tes seins en sont parfumés. \\
  Pendant que le vent triste galope en tuant des papillons \\
  moi je t'aime, et ma joie mord ta bouche de prune.

  Ce qu'il t'en aura coûté de t'habituer à moi, \\
  à mon âme esseulée et sauvage, à mon nom que tous chassent, \\
  tant de fois nous avons vu s'embraser l'étoile du Berger en nous baisant les yeux \\
  et sur nos têtes se détordre les crépuscules en éventails tournants. \\
  Mes paroles ont plu sur toi en te caressant. \\
  Depuis longtemps j'ai aimé ton corps de nacre ensoleillée. \\
  Je te crois même reine de l'univers. \\
  Je t'apporterai des fleurs joyeuses des montagnes, des \emph{copihues}, \\
  des noisettes foncées, et des paniers sylvestres de baisers.

  Je veux faire avec toi \\
  ce que le printemps fait avec les cerisiers.
\end{verse}

\newpage

\addcontentsline{toc}{subsection}{\emph{XV.\ Tu me plais quand tu te tais
car tu es comme absente...}}

\begin{center} \textbf{XV} \end{center}

\bigskip

\begin{verse}
  Tu me plais quand tu te tais car tu es comme absente, \\
  et tu m'entends de loin, et ma voix point ne te touche. \\
  On dirait que tes yeux se seraient envolés \\
  et on dirait qu'un baiser t'aurait scellé la bouche.

  Comme toutes les choses sont emplies de mon âme \\
  tu émerges des choses, de toute mon âme emplie. \\
  Papillon de sommeil, tu ressembles à mon âme, \\
  et tu ressembles au mot mélancolie.

  Tu me plais quand tu te tais et sembles distante. \\
  Et tu sembles gémir, papillon dans la berceuse. \\
  Et tu m'entends de loin, et ma voix ne t'atteint pas: \\
  laisse-moi me taire avec ton silence.

  Laisse-moi aussi te parler avec ton silence \\
  clair comme une lampe, simple comme un anneau. \\
  Tu es comme la nuit, muette et constellée. \\
  Ton silence est d'étoile, si lointain et simple.

  Tu me plais quand tu te tais car tu es comme absente. \\
  Distante et endolorie comme si tu étais morte. \\
  Un mot alors, un sourire suffisent. \\
  Et je suis joyeux, joyeux que ce ne soit pas vrai.
\end{verse}

\newpage

\addcontentsline{toc}{subsection}{\emph{XVI.\ Dans mon ciel au crépuscule tu
es comme un nuage...}}

\begin{center} \textbf{XVI} \end{center}

\smallskip
\hspace*{60mm}{\scriptsize Paraphrase à R.~Tagore.}

\bigskip

\begin{verse}
  Dans mon ciel au crépuscule tu es comme un nuage \\
  et ta couleur et forme sont comme moi je les veux. \\
  Tu es mienne, tu es mienne, femme aux douces lèvres, \\
  et vivent dans ta vie mes rêves infinis.

  La lampe de mon âme te rosit les pieds, \\
  mon aigre vin est plus doux sur tes lèvres: \\
  ô moissonneuse de ma chanson du soir tombant, \\
  comme te sentent mienne mes songes solitaires!

  Tu es mienne, tu es mienne, vais-je criant dans la brise \\
  du soir, et le vent emporte ma voix veuve. \\
  Chasseresse du fond de mes yeux, ton larcin \\
  retient comme l'eau ton regard nocturne.

  Dans le filet de ma musique tu es captive, mon amour, \\
  et mes filets de musique sont larges comme le ciel. \\
  Mon âme naît au bord de tes yeux de deuil. \\
  Dans tes yeux de deuil commence le pays du rêve.
\end{verse}

\newpage

\addcontentsline{toc}{subsection}{\emph{XVII.\ Pensant, prenant dans mes
filets des ombres...}}

\begin{center} \textbf{XVII} \end{center}

\bigskip

\begin{verse}
  Pensant, prenant dans mes filets des ombres en la profonde solitude. \\
  Toi aussi tu es loin, ah plus loin que personne. \\
  Pensant, lâchant des oiseaux, dissipant des images, \\
  enterrant des lampes. \\
  Clocher de brumes, si loin, là-haut! \\
  Noyant des lamentations, moulant de sombres espoirs, \\
  meunier taciturne, \\
  la nuit vient à toi à plat ventre, loin de la ville.

  Ta présence est étrangère, extérieure à moi comme une chose. \\
  Je pense, parcours longuement, ma vie avant toi. \\
  Ma vie d'avant quiconque, mon âpre vie. \\
  Le cri face à la mer, parmi les pierres, \\
  courant libre, fou, dans la vapeur de la mer. \\
  La furie triste, le cri, la solitude de la mer. \\
  Emporté, violent, étiré vers le ciel.

  Toi, femme, qu'étais-tu là-bas, quel pli, quelle branche \\
  de cet éventail immense? Tu étais loin comme maintenant. \\
  Incendie dans le bois! Il flambe en croix bleues. \\
  Il flambe, flambe, flamboie, étincelle en arbres de lumière. \\
  Il s'écroule, crépite. Incendie. Incendie.

  Et mon âme danse blessée de copeaux de feu. \\
  Qui appelle? Quel silence peuplé d'échos? \\
  Heure de la nostalgie, heure de la joie, heure de la solitude, \\
  heure mienne entre toutes!

  Corne dans laquelle le vent passe en chantant. \\
  Tant de passion de pleurs nouée à mon corps.

  Secousse de toutes les racines, \\
  assaut de toutes les vagues! \\
  Roulait, joyeuse, triste, interminable, mon âme.

  Pensant, enterrant des lampes en la profonde solitude. \\
  Qui es-tu toi, qui es-tu?
\end{verse}

\newpage

\addcontentsline{toc}{subsection}{\emph{XVIII.\ Ici je t'aime...}}

\begin{center} \textbf{XVIII} \end{center}

\bigskip

\begin{verse}
  Ici je t'aime. \\
  Dans les obscurs pins se démêle le vent. \\
  La lune phosphorescente sur les eaux errantes. \\
  Des jours égaux passent en se poursuivant.

  La brume défait sa ceinture en dansantes figures. \\
  Une mouette d'argent se décroche du soleil couchant. \\
  Parfois une voile. Hautes, hautes étoiles.

  Oh la croix noire d'un bateau. \\
  Seul. \\
  Parfois je m'éveille au matin, et jusqu'à mon âme est humide. \\
  Sonne, résonne la mer lointaine. \\
  Voici un port. \\
  Ici je t'aime.

  Ici je t'aime et l'horizon en vain t'occulte. \\
  Je t'aime néanmoins parmi ces choses froides. \\
  Parfois mes baisers vont sur ces bateaux graves, \\
  qui vont par les mers vers où ils n'arrivent pas.

  Déjà je me vois oublié comme ces vieilles ancres. \\
  Les quais sont plus tristes quand le soir jette les amarres. \\
  Ma vie inutilement affamée se fatigue. \\
  J'aime ce que je n'ai pas. Toi tu es si distante.

  Mon ennui lutte avec les lents crépuscules. \\
  Mais la nuit vient et commence à chanter pour moi. \\
  La lune fait tourner ses rouages de songe.

  Les étoiles les plus grandes me regardent avec tes yeux. \\
  Et puisque je t'aime, les pins dans le vent \\
  veulent chanter ton nom avec leurs feuilles de fil de fer.
\end{verse}

\newpage

\addcontentsline{toc}{subsection}{\emph{XIX.\ Petite brune et agile, le
soleil qui fait les fruits...}}

\begin{center} \textbf{XIX} \end{center}

\bigskip

\begin{verse}
  Petite brune et agile, le soleil qui fait les fruits, \\
  celui qui charge les blés, celui qui tord les algues, \\
  il a fait ton corps joyeux, tes yeux lumineux \\
  et ta bouche qui a le sourire de l'eau.

  Un soleil noir et avide s'enroule dans les mèches \\
  de ta noire chevelure, quand tu étires les bras. \\
  Toi tu joues avec le soleil comme avec un marais \\
  et il laisse dans tes yeux deux obscures mares.

  Petite brune et agile, rien ne me rapproche de toi. \\
  Tout m'éloigne de toi, comme du plein midi. \\
  Tu es la délirante jeunesse de l'abeille, \\
  l'ivresse de la vague, la force de l'épi.

  Mon cœur sombre te cherche, pourtant, \\
  et j'aime ton corps joyeux, ta voix libre et fine. \\
  Papillon brun, doux et définitif \\
  comme le champ de blé et le soleil, le coquelicot et l'eau.
\end{verse}

\newpage

\addcontentsline{toc}{subsection}{\emph{XX.\ Je peux écrire les vers les
plus tristes cette nuit...}}

\begin{center} \textbf{XX} \end{center}

\bigskip

\begin{verse}
  Je peux écrire les vers les plus tristes cette nuit.

  Écrire, par exemple: «~La nuit est étoilée, \\
  et grelottent, bleus, les astres, au lointain.~»

  Le vent de la nuit tourne dans le ciel et chante.

  Je peux écrire les vers les plus tristes cette nuit. \\
  Je l'ai aimée, et parfois elle aussi m'aima.

  Dans les nuits comme celle-ci je l'ai tenue dans mes bras. \\
  Je l'ai embrassée tant de fois sous le ciel infini.

  Elle m'aima, parfois moi aussi je l'aimais. \\
  Comment ne pas avoir aimé ses grands yeux fixes.

  Je peux écrire les vers les plus tristes cette nuit. \\
  Songer que je ne l'ai pas. Sentir que je l'ai perdue.

  Entendre la nuit immense, plus immense sans elle. \\
  Et le vers tombe sur l'âme comme la rosée sur l'herbe.

  Qu'importe que mon amour ne pouvait la garder. \\
  La nuit est étoilée et elle n'est pas avec moi.

  C'est tout. Au loin quelqu'un chante. Au loin. \\
  Mon âme n'est pas satisfaite, l'ayant perdue.

  Comme pour la rapprocher mon regard la cherche. \\
  Mon cœur la cherche, et elle n'est pas avec moi.

  La même nuit qui fait blanchir les mêmes arbres. \\
  Nous autres, ceux d'alors, déjà ne sommes plus les mêmes.

  Déjà je ne l'aime plus, c'est vrai, mais combien l'ai-je aimée. \\
  Ma voix recherchait le vent pour toucher son oreille.

  À un autre. Elle sera à un autre. Comme avant mes baisers. \\
  Sa voix, son corps clair. Ses yeux infinis.

  Déjà je ne l'aime plus, c'est vrai, mais peut-être que je l'aime. \\
  L'amour est si court, et l'oubli est si long.

  Parce qu'en des nuits comme celle-ci je l'ai tenue dans mes bras, \\
  mon âme n'est pas satisfaite, l'ayant perdue.

  Bien que celle-ci soit l'ultime douleur qu'elle m'inflige, \\
  et ceux-ci soient les ultimes vers que je lui écris.
\end{verse}

\cleardoublepage

\thispagestyle{empty}
\vspace*{80mm}
\begin{center}
{\huge LA CHANSON DÉSESPÉRÉE}
\end{center}

\cleardoublepage

\addcontentsline{toc}{chapter}{LA CHANSON DÉSESPÉRÉE}

\addcontentsline{toc}{subsection}{\emph{Ton souvenir émerge de la nuit
où je suis...}}

\begin{verse}
  Ton souvenir émerge de la nuit où je suis. \\
  Le fleuve noue sa lamentation obstinée à la mer.

  Abandonné comme les quais dans l'aube. \\
  C'est l'heure de partir, oh abandonné!

  Sur mon cœur pleuvent de froides corolles. \\
  Ô sentine de décombres, féroce grotte de naufragés!

  En toi s'accumulèrent les guerres et les envols. \\
  De toi éployèrent leurs ailes les oiseaux du chant.

  Tu as tout englouti, comme le lointain. \\
  Comme la mer, comme le temps. Tout en toi fut naufrage!

  C'était l'heure joyeuse de l'assaut et le baiser. \\
  L'heure de la stupeur ardente comme un phare.

  Anxiété de pilote, furie de plongeur aveugle, \\
  trouble ivresse d'amour, tout en toi fut naufrage!

  Dans l'enfance de brouillard mon âme ailée et blessée. \\
  Découvreur perdu, tout en toi fut naufrage!

  Tu enlaças la douleur, tu t'agrippas au désir. \\
  La tristesse te coucha, tout en toi fut naufrage!

  J'ai fait reculer la muraille d'ombre, \\
  j'ai marché au-delà du désir et de l'acte.

  Ô chair, ma chair, femme que j'ai aimée et perdue, \\
  c'est toi dans cette heure humide que j'évoque et fais chant.

  Comme un vase tu abritas l'infinie tendresse, \\
  et l'oubli infini te pulvérisa comme un vase.

  J'étais la noire, noire solitude des îles, \\
  et là, femme d'amour, m'accueillirent tes bras.

  J'étais la soif et la faim, et toi tu fus le fruit. \\
  J'étais le deuil et les ruines, et toi tu fus le miracle.

  Ah femme, je ne sais comment tu pus me contenir \\
  dans la terre de ton âme, et dans la croix de tes bras!

  Mon désir de toi fut le plus terrible et court, \\
  le plus désordonné et soûl, le plus tendu et avide.

  \newpage

  Cimetière de baisers, il y a encore du feu dans tes tombes, \\
  les grappes resplendissent encore picorées d'oiseaux.

  Oh la bouche mordue, oh les membres baisés, \\
  oh les dents affamées, oh les corps tressés.

  Oh l'accouplement fou d'espoir et d'effort \\
  en lequel nous nous sommes noués et désespérés.

  Et la tendresse, légère comme l'eau et la farine. \\
  Et le mot à peine commencé sur les lèvres.

  Cela fut mon destin et en lui voyagea mon désir ardent, \\
  et en lui chuta mon désir ardent, tout en toi fut naufrage!

  Ô sentine de décombres, en toi tout chutait, \\
  quelle douleur n'exprimas-tu pas, \\
  quelles vagues ne te noyèrent pas!

  De cahot en cahot tu continuas pourtant à flamboyer et à chanter. \\
  Debout comme un marin à la proue d'un bateau.

  Pourtant tu fleuris en chants, pourtant tu te brisas en courants. \\
  Ô sentine de décombres, puits ouvert et amer.

  Pâle plongeur aveugle, infortuné frondeur, \\
  découvreur perdu, tout en toi fut naufrage!

  C'est l'heure de partir, l'heure dure et froide \\
  que la nuit fixe aux petites aiguilles des montres.

  La ceinture bruyante de la mer enserre la côte. \\
  Surgissent de froides étoiles, émigrent de noirs oiseaux.

  Abandonné comme les quais dans l'aube. \\
  Seule l'ombre tremblante se contorsionne dans mes mains.

  Ah au-delà de tout. Ah au-delà de tout.

  C'est l'heure de partir. Oh abandonné!
\end{verse}

%\cleardoublepage

%\tableofcontents

\end{document}
